\documentclass{report}

\usepackage[utf8]{inputenc}
\usepackage{eumat}
\usepackage[Conny]{fncychap}
\usepackage[bahasa]{babel}

% Rename Contents
\addto\captionsenglish{\renewcommand{\contentsname}{\vspace{-0.5cm} \textbf{Daftar Isi} \vspace{-2cm}}}

\begin{document}

% Cover Page
\begin{titlepage}
    \begin{center}
        \vspace*{1cm}
        
        \vspace{0.5cm}
        
        \LARGE
        Tugas Latex Aplikasi Komputer  
        
        \vspace{1cm}
        
        \includegraphics[width=0.5\textwidth]{Logo UNY.png}

        \vspace{1cm}
        
        \textbf{Vikram Zaky Ardianto}\\
        22305144028\\
        Matematika E 2022
        
        \vspace{2cm}
        
        \Large
        \textbf{PRODI MATEMATIKA}\\
        \textbf{DEPARTEMEN PENDIDIKAN MATEMATIKA}\\
        \textbf{FAKULTAS MATEMATIKA DAN ILMU PENGETAHUAN ALAM}
        \textbf{UNIVERSITAS NEGERI YOGYAKARTA}\\
        \textbf{2022}
        
    \end{center}
\end{titlepage}

\newpage
\tableofcontents

\chapter{KB Pekan 2 (Belajar Menggunakan Software EMT)}
\documentclass{article}

\usepackage{eumat}

\begin{document}
\begin{eulernotebook}
\eulerheading{Pendahuluan dan Pengenalan Cara Kerja EMT}
\begin{eulercomment}
Selamat datang! Ini adalah pengantar pertama ke Euler Math Toolbox
(disingkat EMT atau Euler). EMT adalah sistem terintegrasi yang
merupakan perpaduan kernel numerik Euler dan program komputer aljabar
Maxima.

- Bagian numerik, GUI, dan komunikasi dengan Maxima telah dikembangkan
oleh R. Grothmann, seorang profesor matematika di Universitas
Eichstätt, Jerman. Banyak algoritma numerik dan pustaka software open
source yang digunakan di dalamnya.

- Maxima adalah program open source yang matang dan sangat kaya untuk
perhitungan simbolik dan aritmatika tak terbatas. Software ini
dikelola oleh sekelompok pengembang di internet.

- Beberapa program lain (LaTeX, Povray, Tiny C Compiler, Python) dapat
digunakan di Euler untuk memungkinkan perhitungan yang lebih cepat
maupun tampilan atau grafik yang lebih baik.

Yang sedang Anda baca (jika dibaca di EMT) ini adalah berkas notebook
di EMT. Notebook aslinya bawaan EMT (dalam bahasa Inggris) dapat
dibuka melalui menu File, kemudian pilih "Open Tutorias and Example",
lalu pilih file "00 First Steps.en". Perhatikan, file notebook EMT
memiliki ekstensi ".en". Melalui notebook ini Anda akan belajar
menggunakan software Euler untuk menyelesaikan berbagai masalah
matematika.
\end{eulercomment}
\begin{eulercomment}
Panduan ini ditulis dengan Euler dalam bentuk notebook Euler, yang
berisi teks (deskriptif), baris-baris perintah, tampilan hasil
perintah (numerik, ekspresi matematika, atau gambar/plot), dan gambar
yang disisipkan dari file gambar.

Untuk menambah jendela EMT, Anda dapat menekan [F11]. EMT akan
menampilkan jendela grafik di layar desktop Anda. Tekan [F11] lagi
untuk kembali ke tata letak favorit Anda. Tata letak disimpan untuk
sesi berikutnya.

Anda juga dapat menggunakan [Ctrl]+[G] untuk menyembunyikan jendela
grafik. Selanjutnya Anda dapat beralih antara grafik dan teks dengan
tombol [TAB].

Seperti yang Anda baca, notebook ini berisi tulisan (teks) berwarna
hijau, yang dapat Anda edit dengan mengklik kanan teks atau tekan menu
Edit -\textgreater{} Edit Comment atau tekan [F5], dan juga baris perintah EMT yang
ditandai dengan "\textgreater{}" dan berwarna merah. Anda dapat menyisipkan baris
perintah baru dengan cara menekan tiga tombol bersamaan:
[Shift]+[Ctrl]+[Enter].

\end{eulercomment}
\eulersubheading{Komentar (Teks Uraian)}
\begin{eulercomment}
Komentar atau teks penjelasan dapat berisi beberapa "markup" dengan
sintaks sebagai berikut.

\end{eulercomment}
\begin{eulerttcomment}
   - * Judul
   - ** Sub-Judul
   - latex: F (x) = \(\backslash\)int_a^x f (t) \(\backslash\), dt
   - mathjax: \(\backslash\)frac\{x^2-1\}\{x-1\} = x + 1
   - maxima: 'integrate(x^3,x) = integrate(x^3,x) + C
   - http://www.euler-math-toolbox.de
   - See: http://www.google.de | Google
   - image: MU.jpg
   - ---
\end{eulerttcomment}
\begin{eulercomment}

Hasil sintaks-sintaks di atas (tanpa diawali tanda strip) adalah
sebagai berikut.

\begin{eulercomment}
\eulerheading{Judul}
\begin{eulercomment}
\end{eulercomment}
\eulersubheading{Sub-Judul}
\begin{eulercomment}
\end{eulercomment}
\begin{eulerformula}
\[
F(x) = \int_a^x f(t) \, dt
\]
\end{eulerformula}
\begin{eulerformula}
\[
\frac{x^2-1}{x-1} = x + 1
\]
\end{eulerformula}
\begin{eulerformula}
\[
\int {x^3}{\;dx}=\frac{x^4}{4}+\mbox{ C }
\]
\end{eulerformula}
\begin{eulercomment}
http://www.euler-math-toolbox.de\\
See: http://www.google.de \textbar{} Google\\
\end{eulercomment}
\eulerimg{17}{images/Vikram Zaky Ardianto_22305144028-004.png}
\eulersubheading{}
\begin{eulercomment}
Gambar diambil dari folder images di tempat file notebook berada dan
tidak dapat dibaca dari Web. Untuk "See:", tautan (URL)web lokal dapat
digunakan.

Paragraf terdiri atas satu baris panjang di editor. Pergantian baris
akan memulai baris baru. Paragraf harus dipisahkan dengan baris
kosong.
\end{eulercomment}
\begin{eulerprompt}
>// baris perintah diawali dengan >, komentar (keterangan) diawali dengan //
\end{eulerprompt}
\eulerheading{Baris Perintah}
\begin{eulercomment}
Mari kita tunjukkan cara menggunakan EMT sebagai kalkulator yang
sangat canggih.

EMT berorientasi pada baris perintah. Anda dapat menuliskan satu atau
lebih perintah dalam satu baris perintah. Setiap perintah harus
diakhiri dengan koma atau titik koma.

- Titik koma menyembunyikan output (hasil) dari perintah.\\
- Sebuah koma mencetak hasilnya.\\
- Setelah perintah terakhir, koma diasumsikan secara otomatis (boleh
tidak ditulis).

Dalam contoh berikut, kita mendefinisikan variabel r yang diberi nilai
1,25. Output dari definisi ini adalah nilai variabel. Tetapi karena
tanda titik koma, nilai ini tidak ditampilkan. Pada kedua perintah di
belakangnya, hasil kedua perhitungan tersebut ditampilkan.
\end{eulercomment}
\begin{eulerprompt}
>r=1.25; pi*r^2, 2*pi*r
\end{eulerprompt}
\begin{euleroutput}
  4.90873852123
  7.85398163397
\end{euleroutput}
\eulersubheading{Latihan untuk Anda}
\begin{eulercomment}
- Sisipkan beberapa baris perintah baru\\
- Tulis perintah-perintah baru untuk melakukan suatu perhitungan yang
Anda inginkan, boleh menggunakan variabel, boleh tanpa variabel.\\
\end{eulercomment}
\eulersubheading{}
\eulersubheading{Jawab :}
\begin{eulerprompt}
>x=36; y=5; z=20; (x+y)*z
\end{eulerprompt}
\begin{euleroutput}
  820
\end{euleroutput}
\begin{eulerprompt}
>x=14; y=22; z=30; (x+y)/z
\end{eulerprompt}
\begin{euleroutput}
  1.2
\end{euleroutput}
\begin{eulerprompt}
>v=19; z=150; a=6; (v*z+56)/a
\end{eulerprompt}
\begin{euleroutput}
  484.333333333
\end{euleroutput}
\begin{eulerprompt}
>a=7; b=45; c=30; (b-a)*c*2
\end{eulerprompt}
\begin{euleroutput}
  2280
\end{euleroutput}
\begin{eulerprompt}
>p=9; q=25; (p+q)^3
\end{eulerprompt}
\begin{euleroutput}
  39304
\end{euleroutput}
\eulersubheading{}
\begin{eulercomment}
Beberapa catatan yang harus Anda perhatikan tentang penulisan sintaks
perintah EMT.

- Pastikan untuk menggunakan titik desimal, bukan koma desimal untuk
bilangan!\\
- Gunakan * untuk perkalian dan \textasciicircum{} untuk eksponen (pangkat).\\
- Seperti biasa, * dan / bersifat lebih kuat daripada + atau -.\\
- \textasciicircum{} mengikat lebih kuat dari *, sehingga pi * r \textasciicircum{} 2 merupakan rumus
luas lingkaran.\\
- Jika perlu, Anda harus menambahkan tanda kurung, seperti pada 2 \textasciicircum{} (2
\textasciicircum{} 3).

Perintah r = 1.25 adalah menyimpan nilai ke variabel di EMT. Anda juga
dapat menulis r: = 1.25 jika mau. Anda dapat menggunakan spasi sesuka
Anda.

Anda juga dapat mengakhiri baris perintah dengan komentar yang diawali
dengan dua garis miring (//).
\end{eulercomment}
\begin{eulerprompt}
>r := 1.25 // Komentar: Menggunakan  := sebagai ganti =
\end{eulerprompt}
\begin{euleroutput}
  1.25
\end{euleroutput}
\begin{eulercomment}
Argumen atau input untuk fungsi ditulis di dalam tanda kurung.
\end{eulercomment}
\begin{eulerprompt}
>sin(45°), cos(pi), log(sqrt(E))
\end{eulerprompt}
\begin{euleroutput}
  0.707106781187
  -1
  0.5
\end{euleroutput}
\begin{eulercomment}
Seperti yang Anda lihat, fungsi trigonometri bekerja dengan radian, dan derajat
dapat diubah dengan °. Jika keyboard Anda tidak memiliki karakter derajat tekan
[F7], atau gunakan fungsi deg() untuk mengonversi.

EMT menyediakan banyak sekali fungsi dan operator matematika.Hampir semua fungsi
matematika sudah tersedia di EMT. Anda dapat melihat daftar lengkap fungsi-fungsi
matematika di EMT pada berkas Referensi (klik menu Help -\textgreater{} Reference)

Untuk membuat rangkaian komputasi lebih mudah, Anda dapat merujuk ke hasil
sebelumnya dengan "\%". Cara ini sebaiknya hanya digunakan untuk merujuk hasil
perhitungan dalam baris perintah yang sama.
\end{eulercomment}
\begin{eulerprompt}
>(sqrt(5)+1)/2, %^2-%+1 // Memeriksa solusi x^2-x+1=0
\end{eulerprompt}
\begin{euleroutput}
  1.61803398875
  2
\end{euleroutput}
\eulersubheading{Latihan untuk Anda}
\begin{eulercomment}
- Buka berkas Reference dan baca fungsi-fungsi matematika yang
tersedia di EMT.\\
- Sisipkan beberapa baris perintah baru.\\
- Lakukan contoh-contoh perhitungan menggunakan fungsi-fungsi
matematika di EMT.\\
\end{eulercomment}
\eulersubheading{}
\eulersubheading{Jawab :}
\begin{eulerprompt}
>sqrt(1296)/sqrt(36)+7
\end{eulerprompt}
\begin{euleroutput}
  13
\end{euleroutput}
\begin{eulerprompt}
>(cos(90°)+sin(90°))/cos(45°)
\end{eulerprompt}
\begin{euleroutput}
  1.41421356237
\end{euleroutput}
\begin{eulerprompt}
>sqrt(1764)+sqrt(49)
\end{eulerprompt}
\begin{euleroutput}
  49
\end{euleroutput}
\begin{eulerprompt}
> sqrt(256)/4, ((%*4)^2)*2 // Memeriksa solusi ((x*4)^2)*2=0
\end{eulerprompt}
\begin{euleroutput}
  4
  512
\end{euleroutput}
\eulersubheading{}
\eulerheading{Satuan}
\begin{eulercomment}
EMT dapat mengubah unit satuan menjadi sistem standar internasional
(SI). Tambahkan satuan di belakang angka untuk konversi sederhana.
\end{eulercomment}
\begin{eulerprompt}
>1miles  // 1 mil = 1609,344 m
\end{eulerprompt}
\begin{euleroutput}
  1609.344
\end{euleroutput}
\begin{eulercomment}
Beberapa satuan yang sudah dikenal di dalam EMT adalah sebagai
berikut. Semua unit diakhiri dengan tanda dolar (\textdollar{}), namun boleh tidak
perlu ditulis dengan mengaktifkan easyunits.

kilometer\textdollar{}:=1000;\\
km\textdollar{}:=kilometer\textdollar{};\\
cm\textdollar{}:=0.01;\\
mm\textdollar{}:=0.001;\\
minute\textdollar{}:=60;\\
min\textdollar{}:=minute\textdollar{};\\
minutes\textdollar{}:=minute\textdollar{};\\
hour\textdollar{}:=60*minute\textdollar{};\\
h\textdollar{}:=hour\textdollar{};\\
hours\textdollar{}:=hour\textdollar{};\\
day\textdollar{}:=24*hour\textdollar{};\\
days\textdollar{}:=day\textdollar{};\\
d\textdollar{}:=day\textdollar{};\\
year\textdollar{}:=365.2425*day\textdollar{};\\
years\textdollar{}:=year\textdollar{};\\
y\textdollar{}:=year\textdollar{};\\
inch\textdollar{}:=0.0254;\\
in\textdollar{}:=inch\textdollar{};\\
feet\textdollar{}:=12*inch\textdollar{};\\
foot\textdollar{}:=feet\textdollar{};\\
ft\textdollar{}:=feet\textdollar{};\\
yard\textdollar{}:=3*feet\textdollar{};\\
yards\textdollar{}:=yard\textdollar{};\\
yd\textdollar{}:=yard\textdollar{};\\
mile\textdollar{}:=1760*yard\textdollar{};\\
miles\textdollar{}:=mile\textdollar{};\\
kg\textdollar{}:=1;\\
sec\textdollar{}:=1;\\
ha\textdollar{}:=10000;\\
Ar\textdollar{}:=100;\\
Tagwerk\textdollar{}:=3408;\\
Acre\textdollar{}:=4046.8564224;\\
pt\textdollar{}:=0.376mm;

Untuk konversi ke dan antar unit, EMT menggunakan operator khusus,
yakni -\textgreater{}.
\end{eulercomment}
\begin{eulerprompt}
>4km -> miles, 4inch -> " mm"
\end{eulerprompt}
\begin{euleroutput}
  2.48548476895
  101.6 mm
\end{euleroutput}
\eulerheading{Format Tampilan Nilai}
\begin{eulercomment}
Akurasi internal untuk nilai bilangan di EMT adalah standar IEEE,
sekitar 16 digit desimal. Aslinya, EMT tidak mencetak semua digit
suatu bilangan. Ini untuk menghemat tempat dan agar terlihat lebih
baik. Untuk mengatrtamilan satu bilangan, operator berikut dapat
digunakan.

\end{eulercomment}
\begin{eulerprompt}
>pi
\end{eulerprompt}
\begin{euleroutput}
  3.14159265359
\end{euleroutput}
\begin{eulerprompt}
>longest pi
\end{eulerprompt}
\begin{euleroutput}
        3.141592653589793 
\end{euleroutput}
\begin{eulerprompt}
>long pi
\end{eulerprompt}
\begin{euleroutput}
  3.14159265359
\end{euleroutput}
\begin{eulerprompt}
>short pi
\end{eulerprompt}
\begin{euleroutput}
  3.1416
\end{euleroutput}
\begin{eulerprompt}
>shortest pi
\end{eulerprompt}
\begin{euleroutput}
     3.1 
\end{euleroutput}
\begin{eulerprompt}
>fraction pi
\end{eulerprompt}
\begin{euleroutput}
  312689/99532
\end{euleroutput}
\begin{eulerprompt}
>short 1200*1.03^10, long E, longest pi
\end{eulerprompt}
\begin{euleroutput}
  1612.7
  2.71828182846
        3.141592653589793 
\end{euleroutput}
\begin{eulercomment}
Format aslinya untuk menampilkan nilai menggunakan sekitar 10 digit.
Format tampilan nilai dapat diatur secara global atau hanya untuk satu
nilai.

Anda dapat mengganti format tampilan bilangan untuk semua perintah
selanjutnya. Untuk mengembalikan ke format aslinya dapat digunakan
perintah "defformat" atau "reset".
\end{eulercomment}
\begin{eulerprompt}
>longestformat; pi, defformat; pi
\end{eulerprompt}
\begin{euleroutput}
  3.141592653589793
  3.14159265359
\end{euleroutput}
\begin{eulercomment}
Kernel numerik EMT bekerja dengan bilangan titik mengambang (floating
point) dalam presisi ganda IEEE (berbeda dengan bagian simbolik EMT).
Hasil numerik dapat ditampilkan dalam bentuk pecahan.
\end{eulercomment}
\begin{eulerprompt}
>1/7+1/4, fraction %
\end{eulerprompt}
\begin{euleroutput}
  0.392857142857
  11/28
\end{euleroutput}
\eulerheading{Perintah Multibaris}
\begin{eulercomment}
Perintah multi-baris membentang di beberapa baris yang terhubung
dengan "..." di setiap akhir baris, kecuali baris terakhir. Untuk
menghasilkan tanda pindah baris tersebut, gunakan tombol
[Ctrl]+[Enter]. Ini akan menyambung perintah ke baris berikutnya dan
menambahkan "..." di akhir baris sebelumnya. Untuk menggabungkan suatu
baris ke baris sebelumnya, gunakan [Ctrl]+[Backspace].

Contoh perintah multi-baris berikut dapat dijalankan setiap kali
kursor berada di salah satu barisnya. Ini juga menunjukkan bahwa ...
harus berada di akhir suatu baris meskipun baris tersebut memuat
komentar.
\end{eulercomment}
\begin{eulerprompt}
>a=4; b=15; c=2; // menyelesaikan a*x^2+b*x+c=0 secara manual ...
>D=sqrt(b^2/(a^2*4)-c/a); ...
>-b/(2*a) + D, ...
>-b/(2*a) - D
\end{eulerprompt}
\begin{euleroutput}
  -0.138444501319
  -3.61155549868
\end{euleroutput}
\eulerheading{Menampilkan Daftar Variabe}
\begin{eulercomment}
Untuk menampilkan semua variabel yang sudah pernah Anda definisikan
sebelumnya (dan dapat dilihat kembali nilainya), gunakan perintah
"listvar".
\end{eulercomment}
\begin{eulerprompt}
>listvar
\end{eulerprompt}
\begin{euleroutput}
  r                   1.25
  a                   4
  b                   15
  c                   2
  D                   1.73655549868123
\end{euleroutput}
\begin{eulercomment}
Perintah listvar hanya menampilkan variabel buatan pengguna.
Dimungkinkan untuk menampilkan variabel lain, dengan menambahkan
string  termuat di dalam nama variabel yang diinginkan.

Perlu Anda perhatikan, bahwa EMT membedakan huruf besar dan huruf
kecil. Jadi variabel "d" berbeda dengan variabel "D".

Contoh berikut ini menampilkan semua unit yang diakhiri dengan "m"
dengan mencari semua variabel yang berisi "m\textdollar{}".
\end{eulercomment}
\begin{eulerprompt}
>listvar m$
\end{eulerprompt}
\begin{euleroutput}
  km$                 1000
  cm$                 0.01
  mm$                 0.001
  nm$                 1853.24496
  gram$               0.001
  m$                  1
  hquantum$           6.62606957e-34
  atm$                101325
\end{euleroutput}
\begin{eulercomment}
Untuk menghapus variabel tanpa harus memulai ulang EMT gunakan
perintah "remvalue".
\end{eulercomment}
\begin{eulerprompt}
>remvalue a,b,c,D
>D
\end{eulerprompt}
\begin{euleroutput}
  Variable D not found!
  Error in:
  D ...
   ^
\end{euleroutput}
\eulerheading{Menampilkan Panduan}
\begin{eulercomment}
Untuk mendapatkan panduan tentang penggunaan perintah atau fungsi di EMT, buka
jendela panduan dengan menekan [F1] dan cari fungsinya. Anda juga dapat
mengklik dua kali pada fungsi yang tertulis di baris perintah atau di teks
untuk membuka jendela panduan.

Coba klik dua kali pada perintah "intrandom" berikut ini!
\end{eulercomment}
\begin{eulerprompt}
>intrandom(10,6)
\end{eulerprompt}
\begin{euleroutput}
  [4,  2,  6,  2,  4,  2,  3,  2,  2,  6]
\end{euleroutput}
\begin{eulercomment}
Di jendela panduan, Anda dapat mengklik kata apa saja untuk menemukan
referensi atau fungsi.

Misalnya, coba klik kata "random" di jendela panduan. Kata tersebut
boleh ada dalam teks atau di bagian "See:" pada panduan. Anda akan
menemukan penjelasan fungsi "random", untuk menghasilkan bilangan acak
berdistribusi uniform antara 0,0 dan 1,0. Dari panduan untuk "random"
Anda dapat menampilkan panduan untuk fungsi "normal", dll.
\end{eulercomment}
\begin{eulerprompt}
>random(10)
\end{eulerprompt}
\begin{euleroutput}
  [0.270906,  0.704419,  0.217693,  0.445363,  0.308411,  0.914541,  0.193585,
  0.463387,  0.095153,  0.595017]
\end{euleroutput}
\begin{eulerprompt}
>normal(10)
\end{eulerprompt}
\begin{euleroutput}
  [-0.495418,  1.6463,  -0.390056,  -1.98151,  3.44132,  0.308178,  -0.733427,
  -0.526167,  1.10018,  0.108453]
\end{euleroutput}
\eulerheading{Matriks dan Vektor}
\begin{eulercomment}
EMT merupakan suatu aplikasi matematika yang mengerti "bahasa matriks". Artinya,
EMT menggunakan vektor dan matriks untuk perhitungan-perhitungan tingkat lanjut.
Suatu vektor atau matriks dapat didefinisikan dengan tanda kurung siku.
Elemen-elemennya dituliskan di dalam tanda kurung siku, antar elemen dalam satu
baris dipisahkan oleh koma(,), antar baris dipisahkan oleh titik koma (;).

Vektor dan matriks dapat diberi nama seperti variabel biasa.
\end{eulercomment}
\begin{eulerprompt}
>v=[4,5,6,3,2,1]
\end{eulerprompt}
\begin{euleroutput}
  [4,  5,  6,  3,  2,  1]
\end{euleroutput}
\begin{eulerprompt}
>A=[1,2,3;4,5,6;7,8,9]
\end{eulerprompt}
\begin{euleroutput}
              1             2             3 
              4             5             6 
              7             8             9 
\end{euleroutput}
\begin{eulercomment}
Karena EMT mengerti bahasa matriks, EMT memiliki kemampuan yang sangat canggih
untuk melakukan perhitungan matematis untuk masalah-masalah aljabar linier,
statistika, dan optimisasi.

Vektor juga dapat didefinisikan dengan menggunakan rentang nilai dengan interval
tertentu menggunakan tanda titik dua (:),seperti contoh berikut ini.
\end{eulercomment}
\begin{eulerprompt}
>c=1:5
\end{eulerprompt}
\begin{euleroutput}
  [1,  2,  3,  4,  5]
\end{euleroutput}
\begin{eulerprompt}
>w=0:0.1:1
\end{eulerprompt}
\begin{euleroutput}
  [0,  0.1,  0.2,  0.3,  0.4,  0.5,  0.6,  0.7,  0.8,  0.9,  1]
\end{euleroutput}
\begin{eulerprompt}
>mean(w^2)
\end{eulerprompt}
\begin{euleroutput}
  0.35
\end{euleroutput}
\eulerheading{Bilangan Kompleks}
\begin{eulercomment}
EMT juga dapat menggunakan bilangan kompleks. Tersedia banyak fungsi
untuk bilangan kompleks di EMT. Bilangan imaginer

\end{eulercomment}
\begin{eulerformula}
\[
i = \sqrt{-1}
\]
\end{eulerformula}
\begin{eulercomment}
dituliskan dengan huruf I (huruf besar I), namun akan ditampilkan
dengan huruf i (i kecil).

\end{eulercomment}
\begin{eulerttcomment}
  re(x) : bagian riil pada bilangan kompleks x.
  im(x) : bagian imaginer pada bilangan kompleks x.
  complex(x) : mengubah bilangan riil x menjadi bilangan kompleks.
  conj(x) : Konjugat untuk bilangan bilangan komplkes x.
  arg(x) : argumen (sudut dalam radian) bilangan kompleks x.
  real(x) : mengubah x menjadi bilangan riil.
\end{eulerttcomment}
\begin{eulercomment}

Apabila bagian imaginer x terlalu besar, hasilnya akan menampilkan
pesan kesalahan.

\end{eulercomment}
\begin{eulerttcomment}
  >sqrt(-1) // Error!
  >sqrt(complex(-1))
\end{eulerttcomment}
\begin{eulerprompt}
>z=2+3*I, re(z), im(z), conj(z), arg(z), deg(arg(z)), deg(arctan(3/2))
\end{eulerprompt}
\begin{euleroutput}
  2+3i
  2
  3
  2-3i
  0.982793723247
  56.309932474
  56.309932474
\end{euleroutput}
\begin{eulerprompt}
>deg(arg(I)) // 90°
\end{eulerprompt}
\begin{euleroutput}
  90
\end{euleroutput}
\begin{eulerprompt}
>sqrt(-1)
\end{eulerprompt}
\begin{euleroutput}
  Floating point error!
  Error in sqrt
  Error in:
  sqrt(-1) ...
          ^
\end{euleroutput}
\begin{eulerprompt}
>sqrt(complex(-1))
\end{eulerprompt}
\begin{euleroutput}
  0+1i
\end{euleroutput}
\begin{eulercomment}
EMT selalu menganggap semua hasil perhitungan berupa bilangan riil dan
tidak akan secara otomatis mengubah ke bilangan kompleks.

Jadi akar kuadrat -1 akan menghasilkan kesalahan, tetapi akar kuadrat
kompleks didefinisikan untuk bidang koordinat dengan cara seperti
biasa. Untuk mengubah bilangan riil menjadi kompleks, Anda dapat
menambahkan 0i atau menggunakan fungsi "complex".
\end{eulercomment}
\begin{eulerprompt}
>complex(-1), sqrt(%)
\end{eulerprompt}
\begin{euleroutput}
  -1+0i 
  0+1i
\end{euleroutput}
\eulerheading{Matematika Simbolik}
\begin{eulercomment}
EMT dapat melakukan perhitungan matematika simbolis (eksak) dengan
bantuan software Maxima. Software Maxima otomatis sudah terpasang di
komputer Anda ketika Anda memasang EMT. Meskipun demikian, Anda dapat
juga memasang software Maxima tersendiri (yang terpisah dengan
instalasi Maxima di EMT).

Pengguna Maxima yang sudah mahir harus memperhatikan bahwa terdapat
sedikit perbedaan dalam sintaks antara sintaks asli Maxima dan sintaks
ekspresi simbolik di EMT.

Untuk melakukan perhitungan matematika simbolis di EMT, awali perintah
Maxima dengan tanda "\&". Setiap ekspresi yang dimulai dengan "\&"
adalah ekspresi simbolis dan dikerjakan oleh Maxima.
\end{eulercomment}
\begin{eulerprompt}
>&(a+b)^2
\end{eulerprompt}
\begin{euleroutput}
  
                                             2
                                      (b + a)
  
\end{euleroutput}
\begin{eulerprompt}
>&expand((a+b)^2), &factor(x^2+5*x+6)
\end{eulerprompt}
\begin{euleroutput}
  
                                    2            2
                                   b  + 2 a b + a
  
  
                                   (x + 2) (x + 3)
  
\end{euleroutput}
\begin{eulerprompt}
>&solve(a*x^2+b*x+c,x) // rumus abc
\end{eulerprompt}
\begin{euleroutput}
  
                            2                          2
                   (- sqrt(b  - 4 a c)) - b      sqrt(b  - 4 a c) - b
              [x = ------------------------, x = --------------------]
                             2 a                         2 a
  
\end{euleroutput}
\begin{eulerprompt}
>&(a^2-b^2)/(a+b), &ratsimp(%) // ratsimp menyederhanakan bentuk pecahan
\end{eulerprompt}
\begin{euleroutput}
  
                                        2    2
                                       a  - b
                                       -------
                                        b + a
  
  
                                        a - b
  
\end{euleroutput}
\begin{eulerprompt}
>10! // nilai faktorial (modus EMT)
\end{eulerprompt}
\begin{euleroutput}
  3628800
\end{euleroutput}
\begin{eulerprompt}
>&10! //nilai faktorial (simbolik dengan Maxima)
\end{eulerprompt}
\begin{euleroutput}
  
                                       3628800
  
\end{euleroutput}
\begin{eulercomment}
Untuk menggunakan perintah Maxima secara langsung (seperti perintah
pada layar Maxima) awali perintahnya dengan tanda "::" pada baris
perintah EMT. Sintaks Maxima disesuaikan dengan sintaks EMT (disebut
"modus kompatibilitas").
\end{eulercomment}
\begin{eulerprompt}
>factor(1000) // mencari semua faktor 1000 (EMT)
\end{eulerprompt}
\begin{euleroutput}
  [2,  2,  2,  5,  5,  5]
\end{euleroutput}
\begin{eulerprompt}
>:: factor(1000) // faktorisasi prima 1000 (dengan Maxima) 
\end{eulerprompt}
\begin{euleroutput}
  
                                         3  3
                                        2  5
  
\end{euleroutput}
\begin{eulerprompt}
>:: factor(20!)
\end{eulerprompt}
\begin{euleroutput}
  
                               18  8  4  2
                              2   3  5  7  11 13 17 19
  
\end{euleroutput}
\begin{eulercomment}
Jika Anda sudah mahir menggunakan Maxima, Anda dapat menggunakan
sintaks asli perintah Maxima dengan menggunakan tanda ":::" untuk
mengawali setiap perintah Maxima di EMT. Perhatikan, harus ada spasi
antara ":::" dan perintahnya.
\end{eulercomment}
\begin{eulerprompt}
>::: binomial(5,2); // nilai C(5,2)
\end{eulerprompt}
\begin{euleroutput}
  
                                         10
  
\end{euleroutput}
\begin{eulerprompt}
>::: binomial(m,4); // C(m,4)=m!/(4!(m-4)!)
\end{eulerprompt}
\begin{euleroutput}
  
                              (m - 3) (m - 2) (m - 1) m
                              -------------------------
                                         24
  
\end{euleroutput}
\begin{eulerprompt}
>::: trigexpand(cos(x+y)); // rumus cos(x+y)=cos(x) cos(y)-sin(x)sin(y) 
\end{eulerprompt}
\begin{euleroutput}
  
                            cos(x) cos(y) - sin(x) sin(y)
  
\end{euleroutput}
\begin{eulerprompt}
>::: trigexpand(sin(x+y));
\end{eulerprompt}
\begin{euleroutput}
  
                            cos(x) sin(y) + sin(x) cos(y)
  
\end{euleroutput}
\begin{eulerprompt}
>::: trigsimp(((1-sin(x)^2)*cos(x))/cos(x)^2+tan(x)*sec(x)^2) //menyederhanakan fungsi trigonometri
\end{eulerprompt}
\begin{euleroutput}
  
                                              4
                                  sin(x) + cos (x)
                                  ----------------
                                         3
                                      cos (x)
  
\end{euleroutput}
\begin{eulercomment}
Untuk menyimpan ekspresi simbolik ke dalam suatu variabel digunakan
tanda "\&=".
\end{eulercomment}
\begin{eulerprompt}
>p1 &= (x^3+1)/(x+1)
\end{eulerprompt}
\begin{euleroutput}
  
                                        3
                                       x  + 1
                                       ------
                                       x + 1
  
\end{euleroutput}
\begin{eulerprompt}
>&ratsimp(p1)
\end{eulerprompt}
\begin{euleroutput}
  
                                      2
                                     x  - x + 1
  
\end{euleroutput}
\begin{eulercomment}
Untuk mensubstitusikan suatu nilai ke dalam variabel dapat digunakan
perintah "with".
\end{eulercomment}
\begin{eulerprompt}
>&p1 with x=3 // (3^3+1)/(3+1)
\end{eulerprompt}
\begin{euleroutput}
  
                                          7
  
\end{euleroutput}
\begin{eulerprompt}
>&p1 with x=a+b, &ratsimp(%) //substitusi dengan variabel baru
\end{eulerprompt}
\begin{euleroutput}
  
                                           3
                                    (b + a)  + 1
                                    ------------
                                     b + a + 1
  
  
                             2                  2
                            b  + (2 a - 1) b + a  - a + 1
  
\end{euleroutput}
\begin{eulerprompt}
>&diff(p1,x) //turunan p1 terhadap x
\end{eulerprompt}
\begin{euleroutput}
  
                                     2      3
                                  3 x      x  + 1
                                  ----- - --------
                                  x + 1          2
                                          (x + 1)
  
\end{euleroutput}
\begin{eulerprompt}
>&integrate(p1,x) // integral p1 terhadap x
\end{eulerprompt}
\begin{euleroutput}
  
                                     3      2
                                  2 x  - 3 x  + 6 x
                                  -----------------
                                          6
  
\end{euleroutput}
\eulerheading{Tampilan Matematika Simbolik dengan LaTeX}
\begin{eulercomment}
Anda dapat menampilkan hasil perhitunagn simbolik secara lebih bagus
menggunakan LaTeX. Untuk melakukan hal ini, tambahkan tanda dolar (\textdollar{})
di depan tanda \& pada setiap perintah Maxima.\\
Perhatikan, hal ini hanya dapat menghasilkan tampilan yang diinginkan
apabila komputer Anda sudah terpasang software LaTeX.
\end{eulercomment}
\begin{eulerprompt}
>$&(a+b)^2
>$&expand((a+b)^2), $&factor(x^2+5*x+6)
>$&solve(a*x^2+b*x+c,x) // rumus abc
>$&(a^2-b^2)/(a+b), $&ratsimp(%)
\end{eulerprompt}
\eulerheading{Selamat Belajar dan Berlatih!}
\begin{eulercomment}
Baik, itulah sekilas pengantar penggunaan software EMT. Masih banyak
kemampuan EMT yang akan Anda pelajari dan praktikkan.

Sebagai latihan untuk memperlancar penggunaan perintah-perintah EMT
yang sudah dijelaskan di atas, silakan Anda lakukan hal-hal sebagai
berikut.

- Carilah soal-soal matematika dari buku-buku Matematika.\\
- Tambahkan beberapa baris perintah EMT pada notebook ini.\\
- Selesaikan soal-soal matematika tersebut dengan menggunakan EMT.\\
Pilih soal-soal yang sesuai dengan perintah-perintah yang sudah
dijelaskan dan dicontohkan di atas.

Soal 1\\
\end{eulercomment}
\eulerimg{6}{images/Vikram Zaky Ardianto_22305144028-005.png}
\begin{eulerprompt}
>-5*(7+9) 
\end{eulerprompt}
\begin{euleroutput}
  -80
\end{euleroutput}
\begin{eulercomment}
Soal 2\\
\end{eulercomment}
\eulerimg{6}{images/Vikram Zaky Ardianto_22305144028-006.png}
\begin{eulerprompt}
>72/(-36)/(-2)
\end{eulerprompt}
\begin{euleroutput}
  1
\end{euleroutput}
\begin{eulercomment}
Soal 3\\
\end{eulercomment}
\eulerimg{6}{images/Vikram Zaky Ardianto_22305144028-007.png}
\begin{eulerprompt}
>(sqrt(2)+sqrt(3))^2
\end{eulerprompt}
\begin{euleroutput}
  9.89897948557
\end{euleroutput}
\begin{eulercomment}
Soal 4\\
\end{eulercomment}
\eulerimg{6}{images/Vikram Zaky Ardianto_22305144028-008.png}
\begin{eulerprompt}
>(sqrt(5)+sqrt(3))*(sqrt(5)-sqrt(3))
\end{eulerprompt}
\begin{euleroutput}
  2
\end{euleroutput}
\end{eulernotebook}
\end{document}


\newpage
\chapter{KB Pekan 3: Menggunakan EMT untuk menyelesaikan masalah-masalah Aljabar}
\documentclass{article}

\usepackage{eumat}

\begin{document}
\begin{eulernotebook}
\begin{eulercomment}
Nama  : Vikram Zaky Ardianto\\
Kelas : Matematika E\\
NIM   : 22305144028

\begin{eulercomment}
\eulerheading{EMT untuk Perhitungan Aljabar}
\begin{eulercomment}
Pada notebook ini Anda belajar menggunakan EMT untuk melakukan
berbagai perhitungan terkait dengan materi atau topik dalam Aljabar.
Kegiatan yang harus Anda lakukan adalah sebagai berikut:

- Membaca secara cermat dan teliti notebook ini;\\
- Menerjemahkan teks bahasa Inggris ke bahasa Indonesia;\\
- Mencoba contoh-contoh perhitungan (perintah EMT) dengan cara meng
ENTER setiap perintah EMT yang ada (pindahkan kursor ke baris
perintah)\\
- Jika perlu Anda dapat memodifikasi perintah yang ada dan memberikan
keterangan/penjelasan tambahan terkait hasilnya.\\
- Menyisipkan baris-baris perintah baru untuk mengerjakan soal-soal
Aljabar dari file PDF yang saya berikan;\\
- Memberi catatan hasilnya.\\
- Jika perlu tuliskan soalnya pada teks notebook (menggunakan format
LaTeX).\\
- Gunakan tampilan hasil semua perhitungan yang eksak atau simbolik
dengan format LaTeX. (Seperti contoh-contoh pada notebook ini.)

\end{eulercomment}
\eulersubheading{Contoh pertama}
\begin{eulercomment}
Menyederhanakan bentuk aljabar:

\end{eulercomment}
\begin{eulerformula}
\[
6x^{-3}y^5\times -7x^2y^{-9}
\]
\end{eulerformula}
\begin{eulercomment}
\end{eulercomment}
\begin{eulerprompt}
>$&6*x^(-3)*y^5*-7*x^2*y^(-9)
\end{eulerprompt}
\begin{eulercomment}
Menyederhanakan fungsi :\\
\end{eulercomment}
\begin{eulerformula}
\[
2y^2+2x^2-3y^2+2x^2
\]
\end{eulerformula}
\begin{eulerprompt}
>$&2*y^2+2*x^2-3*y^2+2*x^2 
\end{eulerprompt}
\begin{eulercomment}
Menjabarkan:

\end{eulercomment}
\begin{eulerformula}
\[
(6x^{-3}+y^5)(-7x^2-y^{-9})
\]
\end{eulerformula}
\begin{eulerprompt}
>$&showev('expand((6*x^(-3)+y^5)*(-7*x^2-y^(-9))))
\end{eulerprompt}
\eulersubheading{Baris Perintah}
\begin{eulercomment}
Baris perintah Euler terdiri dari satu atau beberapa perintah Euler
diikuti dengan titik koma ";" atau koma ",". Titik koma mencegah
pencetakan hasil. Koma setelah perintah terakhir dapat dihilangkan.

Baris perintah berikut hanya akan mencetak hasil ekspresi, bukan tugas
atau perintah format.
\end{eulercomment}
\begin{eulerprompt}
>r:=8; h:=5; pi*r^2*h^2
\end{eulerprompt}
\begin{euleroutput}
  5026.54824574
\end{euleroutput}
\begin{eulercomment}
Perintah harus dipisahkan dengan yang kosong. Baris perintah berikut
mencetak dua hasilnya.
\end{eulercomment}
\begin{eulerprompt}
>pi*r*2*h, %+2*pi*r*h^3 // Ingat tanda % menyatakan hasil perhitungan terakhir sebelumnya
\end{eulerprompt}
\begin{euleroutput}
  251.327412287
  6534.51271947
\end{euleroutput}
\begin{eulercomment}
Baris perintah dieksekusi dalam urutan yang ditekan pengguna kembali.
Jadi Anda mendapatkan nilai baru setiap kali Anda menjalankan baris
kedua.
\end{eulercomment}
\begin{eulerprompt}
>x := 7;
>x := cos(x) // nilai cosinus (x dalam radian)
\end{eulerprompt}
\begin{euleroutput}
  0.753902254343
\end{euleroutput}
\begin{eulerprompt}
>x := cos(x)
\end{eulerprompt}
\begin{euleroutput}
  0.729023376891
\end{euleroutput}
\begin{eulercomment}
Jika dua garis terhubung dengan "..." kedua garis akan selalu
dieksekusi secara bersamaan.
\end{eulercomment}
\begin{eulerprompt}
>x := 5.5; ...
>x := (x+15/x^2)/2, x := (x+15/x^2)/2, x := (x+15/x^2)/2, 
\end{eulerprompt}
\begin{euleroutput}
  2.9979338843
  2.33344930442
  2.54413874904
\end{euleroutput}
\begin{eulercomment}
Ini juga merupakan cara yang baik untuk menyebarkan perintah panjang
pada dua atau lebih baris. Anda dapat menekan Ctrl+Return untuk
membagi garis menjadi dua pada posisi kursor saat ini, atau Ctrl+Back
untuk menggabungkan garis.

Untuk melipat semua multi-garis tekan Ctrl + L. Kemudian garis-garis
berikutnya hanya akan terlihat, jika salah satunya memiliki fokus.
Untuk melipat satu multi-baris, mulailah baris pertama dengan "\%+".
\end{eulercomment}
\begin{eulerprompt}
>%+ x=7+7; ...
\end{eulerprompt}
\begin{eulercomment}
Garis yang dimulai dengan \%\% tidak akan terlihat sama sekali.
\end{eulercomment}
\begin{euleroutput}
  196
\end{euleroutput}
\begin{eulerprompt}
> 
\end{eulerprompt}
\begin{eulercomment}
Euler mendukung loop di baris perintah, selama mereka masuk ke dalam
satu baris atau multi-baris. Dalam program, pembatasan ini tidak
berlaku, tentu saja. Untuk informasi lebih lanjut lihat pengantar
berikut.
\end{eulercomment}
\begin{eulerprompt}
>y=9; for i=1 to 10; y := (y+2/x^3)/y, end; // menghitung akar 2
\end{eulerprompt}
\begin{euleroutput}
  1.00008098477
  1.00072880395
  1.00072833216
  1.00072833251
  1.00072833251
  1.00072833251
  1.00072833251
  1.00072833251
  1.00072833251
  1.00072833251
\end{euleroutput}
\begin{eulercomment}
Tidak apa-apa untuk menggunakan multi-line. Pastikan baris diakhiri
dengan "...".
\end{eulercomment}
\begin{eulerprompt}
>x := 5.5; // comments go here before the ...
>repeat xnew:=(x/x+2)/2; until xnew~=x; ...
>   x := xnew; ...
>end; ...
>x,
\end{eulerprompt}
\begin{euleroutput}
  1.5
\end{euleroutput}
\begin{eulercomment}
Struktur bersyarat juga berfungsi.
\end{eulercomment}
\begin{eulerprompt}
>if E^pi>pi^E; then "Thought so!", endif;
\end{eulerprompt}
\begin{euleroutput}
  Thought so!
\end{euleroutput}
\begin{eulercomment}
Saat Anda menjalankan perintah, kursor dapat berada di posisi mana pun
di baris perintah. Anda dapat kembali ke perintah sebelumnya atau
melompat ke perintah berikutnya dengan tombol panah. Atau Anda dapat
mengklik ke bagian komentar di atas perintah untuk menuju ke perintah.

Saat Anda menggerakkan kursor di sepanjang garis, pasangan tanda
kurung atau kurung buka dan tutup akan disorot. Juga, perhatikan baris
status. Setelah kurung buka fungsi sqrt(), baris status akan
menampilkan teks bantuan untuk fungsi tersebut. Jalankan perintah
dengan tombol kembali.
\end{eulercomment}
\begin{eulerprompt}
>sqrt(cos(45°)/cos(0°))
\end{eulerprompt}
\begin{euleroutput}
  0.840896415254
\end{euleroutput}
\begin{eulercomment}
Untuk melihat bantuan untuk perintah terbaru, buka jendela bantuan
dengan F1. Di sana, Anda dapat memasukkan teks untuk dicari. Pada
baris kosong, bantuan untuk jendela bantuan akan ditampilkan. Anda
dapat menekan escape untuk menghapus garis, atau untuk menutup jendela
bantuan.

Anda dapat mengklik dua kali pada perintah apa pun untuk membuka
bantuan untuk perintah ini. Coba klik dua kali perintah exp di bawah
ini di baris perintah.
\end{eulercomment}
\begin{eulerprompt}
>exp(log(67.8))
\end{eulerprompt}
\begin{euleroutput}
  67.8
\end{euleroutput}
\begin{eulercomment}
Anda dapat menyalin dan menempel di Euler ke. Gunakan Ctrl-C dan
Ctrl-V untuk ini. Untuk menandai teks, seret mouse atau gunakan shift
bersama dengan tombol kursor apa pun. Selain itu, Anda dapat menyalin
tanda kurung yang disorot.
\end{eulercomment}
\begin{eulercomment}

\end{eulercomment}
\eulersubheading{Sintaks Dasar}
\begin{eulercomment}
Euler tahu fungsi matematika biasa. Seperti yang Anda lihat di atas,
fungsi trigonometri bekerja dalam radian atau derajat. Untuk
mengonversi ke derajat, tambahkan simbol derajat (dengan tombol F7) ke
nilainya, atau gunakan fungsi rad(x). Fungsi akar kuadrat disebut
kuadrat dalam Euler. Tentu saja, x\textasciicircum{}(1/2) juga dimungkinkan.

Untuk menyetel variabel, gunakan "=" atau ":=". Demi kejelasan,
pengantar ini menggunakan bentuk yang terakhir. Spasi tidak masalah.
Tapi ruang antara perintah diharapkan.

Beberapa perintah dalam satu baris dipisahkan dengan "," atau ";".
Titik koma menekan output dari perintah. Di akhir baris perintah ","
diasumsikan, jika ";" hilang.
\end{eulercomment}
\begin{eulerprompt}
>z:=10.1; y:=9.5; 1/2*z^3*y
\end{eulerprompt}
\begin{euleroutput}
  4893.92975
\end{euleroutput}
\begin{eulercomment}
EMT menggunakan sintaks pemrograman untuk ekspresi. Memasuki

\end{eulercomment}
\begin{eulerformula}
\[
e^2 \cdot \left( \frac{1}{3+4 \log(0.6)}+\frac{1}{7} \right)
\]
\end{eulerformula}
\begin{eulercomment}
Anda harus mengatur tanda kurung yang benar dan menggunakan / untuk
pecahan. Perhatikan tanda kurung yang disorot untuk bantuan.
Perhatikan bahwa konstanta Euler e diberi nama E dalam EMT.
\end{eulercomment}
\begin{eulerprompt}
>E^2*(1/(8+9*log(0.8))+3/9)
\end{eulerprompt}
\begin{euleroutput}
  3.69623234287
\end{euleroutput}
\begin{eulercomment}
Untuk menghitung ekspresi rumit seperti

\end{eulercomment}
\begin{eulerformula}
\[
\left(\frac{\frac17 + \frac18 + 2}{\frac13 + \frac12}\right)^2 \pi
\]
\end{eulerformula}
\begin{eulercomment}
Anda harus memasukkannya dalam bentuk baris.
\end{eulercomment}
\begin{eulerprompt}
>((8/7 + 1/3 + 2 + 1/9) / (1/2 + 1/8))^3 * pi^2
\end{eulerprompt}
\begin{euleroutput}
  1866.22224762
\end{euleroutput}
\begin{eulercomment}
Letakkan tanda kurung dengan hati-hati di sekitar sub-ekspresi yang
perlu dihitung terlebih dahulu. EMT membantu Anda dengan menyorot
ekspresi bahwa braket penutup selesai. Anda juga harus memasukkan nama
"pi" untuk huruf Yunani pi.

Hasil dari perhitungan ini adalah bilangan floating point. Secara
default dicetak dengan akurasi sekitar 12 digit. Di baris perintah
berikut, kita juga belajar bagaimana kita bisa merujuk ke hasil
sebelumnya dalam baris yang sama.
\end{eulercomment}
\begin{eulerprompt}
>2/4+3/8, fraction %
\end{eulerprompt}
\begin{euleroutput}
  0.875
  7/8
\end{euleroutput}
\begin{eulercomment}
Perintah Euler dapat berupa ekspresi atau perintah primitif. Ekspresi
terbuat dari operator dan fungsi. Jika perlu, itu harus berisi tanda
kurung untuk memaksa urutan eksekusi yang benar. Jika ragu, memasang
braket adalah ide yang bagus. Perhatikan bahwa EMT menunjukkan tanda
kurung buka dan tutup saat mengedit baris perintah.
\end{eulercomment}
\begin{eulerprompt}
>(sin(pi/4)^2)*2*(cos(pi/4)^2)*2
\end{eulerprompt}
\begin{euleroutput}
  1
\end{euleroutput}
\begin{eulercomment}
Operator numerik Euler meliputi

\end{eulercomment}
\begin{eulerttcomment}
  + unary atau operator plus
  - unary atau operator minus
  *, /
  . produk matriks
  a^b daya untuk positif a atau bilangan bulat b (a**b juga berfungsi)
  n! operator faktorial
\end{eulerttcomment}
\begin{eulercomment}

dan masih banyak lagi.

Berikut adalah beberapa fungsi yang mungkin Anda butuhkan. Ada banyak
lagi.

\end{eulercomment}
\begin{eulerttcomment}
  sin,cos,tan,atan,asin,acos,rad,deg
  log,exp,log10,sqrt,logbase
  bin,logbin,logfac,mod,lantai,ceil,bulat,abs,tanda
  conj,re,im,arg,conj,nyata,kompleks
  beta,betai,gamma,complexgamma,ellrf,ellf,ellrd,elle
  bitand, bitor, bitxor, bitnot
\end{eulerttcomment}
\begin{eulercomment}

Beberapa perintah memiliki alias, mis. Untuk log.
\end{eulercomment}
\begin{eulerprompt}
>ln(E^2), arctan(tan(0.)), logbase(60,30)
\end{eulerprompt}
\begin{euleroutput}
  2
  0
  1.20379504709
\end{euleroutput}
\begin{eulerprompt}
>sin(90°)
\end{eulerprompt}
\begin{euleroutput}
  1
\end{euleroutput}
\begin{eulercomment}
Pastikan untuk menggunakan tanda kurung (kurung bulat), setiap kali
ada keraguan tentang urutan eksekusi! Berikut ini tidak sama dengan
(2\textasciicircum{}3)\textasciicircum{}4, yang merupakan default untuk 2\textasciicircum{}3\textasciicircum{}4 di EMT (beberapa sistem
numerik melakukannya dengan cara lain).
\end{eulercomment}
\begin{eulerprompt}
>8^3^4, (9^3)^5, 2^(2^4)
\end{eulerprompt}
\begin{euleroutput}
  1.41347765182e+73
  2.05891132095e+14
  65536
\end{euleroutput}
\eulersubheading{Bilangan Asli}
\begin{eulercomment}
Tipe data utama dalam Euler adalah bilangan real. Real
direpresentasikan dalam format IEEE dengan akurasi sekitar 16 digit
desimal.
\end{eulercomment}
\begin{eulerprompt}
>longest 4/3
\end{eulerprompt}
\begin{euleroutput}
        1.333333333333333 
\end{euleroutput}
\begin{eulerprompt}
>longest 1/8
\end{eulerprompt}
\begin{euleroutput}
                    0.125 
\end{euleroutput}
\begin{eulercomment}
Representasi ganda internal membutuhkan 8 byte.
\end{eulercomment}
\begin{eulerprompt}
>printdual(13/3)
\end{eulerprompt}
\begin{euleroutput}
  1.0001010101010101010101010101010101010101010101010101*2^2
\end{euleroutput}
\begin{eulerprompt}
>printhex(1/2)
\end{eulerprompt}
\begin{euleroutput}
  8.0000000000000*16^-1
\end{euleroutput}
\begin{eulerprompt}
>printdual(2/9)
\end{eulerprompt}
\begin{euleroutput}
  1.1100011100011100011100011100011100011100011100011100*2^-3
\end{euleroutput}
\begin{eulerprompt}
>printhex(7/9)
\end{eulerprompt}
\begin{euleroutput}
  C.71C71C71C71C8*16^-1
\end{euleroutput}
\begin{eulerprompt}
> 
\end{eulerprompt}
\eulersubheading{String}
\begin{eulercomment}
Sebuah string dalam Euler didefinisikan dengan "...".
\end{eulercomment}
\begin{eulerprompt}
>"A string can contain anything."
\end{eulerprompt}
\begin{euleroutput}
  A string can contain anything.
\end{euleroutput}
\begin{eulercomment}
String dapat digabungkan dengan \textbar{} atau dengan +. Ini juga berfungsi
dengan angka, yang dikonversi menjadi string dalam kasus itu.
\end{eulercomment}
\begin{eulerprompt}
>"The area of the circle with radius " + 2 + " cm is " + pi*4 + " cm^2."
\end{eulerprompt}
\begin{euleroutput}
  The area of the circle with radius 2 cm is 12.5663706144 cm^2.
\end{euleroutput}
\begin{eulercomment}
Fungsi print juga mengonversi angka menjadi string. Ini dapat
mengambil sejumlah digit dan sejumlah tempat (0 untuk keluaran padat),
dan secara optimal satu unit.
\end{eulercomment}
\begin{eulerprompt}
>"Golden Ratio : " + print((1+sqrt(5))/2,5,0)
\end{eulerprompt}
\begin{euleroutput}
  Golden Ratio : 1.61803
\end{euleroutput}
\begin{eulercomment}
Ada string khusus tidak ada, yang tidak dicetak. Itu dikembalikan oleh
beberapa fungsi, ketika hasilnya tidak masalah. (Ini dikembalikan
secara otomatis, jika fungsi tidak memiliki pernyataan pengembalian.)
\end{eulercomment}
\begin{eulerprompt}
>none
\end{eulerprompt}
\begin{eulercomment}
Untuk mengonversi string menjadi angka, cukup evaluasi saja. Ini juga
berfungsi untuk ekspresi (lihat di bawah).
\end{eulercomment}
\begin{eulerprompt}
>"1234.5"()
\end{eulerprompt}
\begin{euleroutput}
  1234.5
\end{euleroutput}
\begin{eulercomment}
Untuk mendefinisikan vektor string, gunakan notasi vektor [...].
\end{eulercomment}
\begin{eulerprompt}
>v:=["affe","charlie","bravo"]
\end{eulerprompt}
\begin{euleroutput}
  affe
  charlie
  bravo
\end{euleroutput}
\begin{eulercomment}
Vektor string kosong dilambangkan dengan [none]. Vektor string dapat
digabungkan.
\end{eulercomment}
\begin{eulerprompt}
>w:=[none]; w|v|v
\end{eulerprompt}
\begin{euleroutput}
  affe
  charlie
  bravo
  affe
  charlie
  bravo
\end{euleroutput}
\begin{eulercomment}
String dapat berisi karakter Unicode. Secara internal, string ini
berisi kode UTF-8. Untuk menghasilkan string seperti itu, gunakan
u"..." dan salah satu entitas HTML.

String Unicode dapat digabungkan seperti string lainnya.
\end{eulercomment}
\begin{eulerprompt}
>u"&alpha; = " + 45 + u"&deg;" // pdfLaTeX mungkin gagal menampilkan secara benar
\end{eulerprompt}
\begin{euleroutput}
  α = 45°
\end{euleroutput}
\begin{eulercomment}
I
\end{eulercomment}
\begin{eulercomment}
Dalam komentar, entitas yang sama seperti alpha;, beta; dll dapat
digunakan. Ini mungkin alternatif cepat untuk Lateks. (Lebih detail di
komentar di bawah).
\end{eulercomment}
\begin{eulercomment}
Ada beberapa fungsi untuk membuat atau menganalisis string unicode.
Fungsi strtochar() akan mengenali string Unicode, dan menerjemahkannya
dengan benar.
\end{eulercomment}
\begin{eulerprompt}
>v=strtochar(u"&Auml; is a German letter")
\end{eulerprompt}
\begin{euleroutput}
  [196,  32,  105,  115,  32,  97,  32,  71,  101,  114,  109,  97,  110,
  32,  108,  101,  116,  116,  101,  114]
\end{euleroutput}
\begin{eulercomment}
Hasilnya adalah vektor angka Unicode. Fungsi kebalikannya adalah
chartoutf().
\end{eulercomment}
\begin{eulerprompt}
>v[1]=strtochar(u"&Uuml;")[1]; chartoutf(v)
\end{eulerprompt}
\begin{euleroutput}
  Ü is a German letter
\end{euleroutput}
\begin{eulercomment}
Fungsi utf() dapat menerjemahkan string dengan entitas dalam variabel
menjadi string Unicode.
\end{eulercomment}
\begin{eulerprompt}
>s="We have &alpha;=&beta;."; utf(s) // pdfLaTeX mungkin gagal menampilkan secara benar
\end{eulerprompt}
\begin{euleroutput}
  We have α=β.
\end{euleroutput}
\begin{eulercomment}
Dimungkinkan juga untuk menggunakan entitas numerik.
\end{eulercomment}
\begin{eulerprompt}
>u"&#196;hnliches"
\end{eulerprompt}
\begin{euleroutput}
  Ähnliches
\end{euleroutput}
\eulersubheading{Nilai Boolean}
\begin{eulercomment}
Nilai Boolean direpresentasikan dengan 1=true atau 0=false dalam
Euler. String dapat dibandingkan, seperti halnya angka.
\end{eulercomment}
\begin{eulerprompt}
>7<8, "biji anggur"<"biji nangka"
\end{eulerprompt}
\begin{euleroutput}
  1
  1
\end{euleroutput}
\begin{eulercomment}
"dan" adalah operator "\&\&" dan "atau" adalah operator "\textbar{}\textbar{}", seperti
dalam bahasa C. (Kata-kata "dan" dan "atau" hanya dapat digunakan
dalam kondisi untuk "jika".)
\end{eulercomment}
\begin{eulerprompt}
>2<E && E<3
\end{eulerprompt}
\begin{euleroutput}
  1
\end{euleroutput}
\begin{eulercomment}
Operator Boolean mematuhi aturan bahasa matriks.
\end{eulercomment}
\begin{eulerprompt}
>(1:10)>5, nonzeros(%)
\end{eulerprompt}
\begin{euleroutput}
  [0,  0,  0,  0,  0,  1,  1,  1,  1,  1]
  [6,  7,  8,  9,  10]
\end{euleroutput}
\begin{eulercomment}
Anda dapat menggunakan fungsi bukan nol() untuk mengekstrak elemen
tertentu dari vektor. Dalam contoh, kami menggunakan isprima
bersyarat(n).
\end{eulercomment}
\begin{eulerprompt}
>N=2|3:2:99 // N berisi elemen 2 dan bilangan2 ganjil dari 3 s.d. 99
\end{eulerprompt}
\begin{euleroutput}
  [2,  3,  5,  7,  9,  11,  13,  15,  17,  19,  21,  23,  25,  27,  29,
  31,  33,  35,  37,  39,  41,  43,  45,  47,  49,  51,  53,  55,  57,
  59,  61,  63,  65,  67,  69,  71,  73,  75,  77,  79,  81,  83,  85,
  87,  89,  91,  93,  95,  97,  99]
\end{euleroutput}
\begin{eulerprompt}
>N[nonzeros(isprime(N))] //pilih anggota2 N yang prima
\end{eulerprompt}
\begin{euleroutput}
  [2,  3,  5,  7,  11,  13,  17,  19,  23,  29,  31,  37,  41,  43,  47,
  53,  59,  61,  67,  71,  73,  79,  83,  89,  97]
\end{euleroutput}
\eulersubheading{Format Keluaran}
\begin{eulercomment}
Format output default EMT mencetak 12 digit. Untuk memastikan bahwa
kami melihat default, kami mengatur ulang format.
\end{eulercomment}
\begin{eulerprompt}
>defformat; pi
\end{eulerprompt}
\begin{euleroutput}
  3.14159265359
\end{euleroutput}
\begin{eulercomment}
Secara internal, EMT menggunakan standar IEEE untuk bilangan ganda
dengan sekitar 16 digit desimal. Untuk melihat jumlah digit penuh,
gunakan perintah "format terpanjang", atau kita gunakan operator
"terpanjang" untuk menampilkan hasil dalam format terpanjang.
\end{eulercomment}
\begin{eulerprompt}
>longest pi
\end{eulerprompt}
\begin{euleroutput}
        3.141592653589793 
\end{euleroutput}
\begin{eulercomment}
Berikut adalah representasi heksadesimal internal dari bilangan ganda.
\end{eulercomment}
\begin{eulerprompt}
>printhex(pi)
\end{eulerprompt}
\begin{euleroutput}
  3.243F6A8885A30*16^0
\end{euleroutput}
\begin{eulercomment}
Format output dapat diubah secara permanen dengan perintah format.
\end{eulercomment}
\begin{eulerprompt}
>format(12,5); 1/3, pi, sin(1)
\end{eulerprompt}
\begin{euleroutput}
      0.33333 
      3.14159 
      0.84147 
\end{euleroutput}
\begin{eulercomment}
Standarnya adalah format (12).
\end{eulercomment}
\begin{eulerprompt}
>format(12); 1/3
\end{eulerprompt}
\begin{euleroutput}
  0.333333333333
\end{euleroutput}
\begin{eulercomment}
Fungsi seperti "shortestformat", "shortformat", "longformat" bekerja
untuk vektor dengan cara berikut.
\end{eulercomment}
\begin{eulerprompt}
>shortestformat; random(3,8)
\end{eulerprompt}
\begin{euleroutput}
    0.66    0.2   0.89   0.28   0.53   0.31   0.44    0.3 
    0.28   0.88   0.27    0.7   0.22   0.45   0.31   0.91 
    0.19   0.46  0.095    0.6   0.43   0.73   0.47   0.32 
\end{euleroutput}
\begin{eulercomment}
Format default untuk skalar adalah format (12). Tapi ini bisa diubah.
\end{eulercomment}
\begin{eulerprompt}
>setscalarformat(5); pi
\end{eulerprompt}
\begin{euleroutput}
  3.1416
\end{euleroutput}
\begin{eulercomment}
Fungsi "format terpanjang" mengatur format skalar juga.
\end{eulercomment}
\begin{eulerprompt}
>longestformat; pi
\end{eulerprompt}
\begin{euleroutput}
  3.141592653589793
\end{euleroutput}
\begin{eulercomment}
Untuk referensi, berikut adalah daftar format output yang paling
penting.

\end{eulercomment}
\begin{eulerttcomment}
  format terpendek format pendek format panjang, format terpanjang
  format(panjang,digit) format baik(panjang)
  fracformat (panjang)
  mengubah bentuk
\end{eulerttcomment}
\begin{eulercomment}

Akurasi internal EMT adalah sekitar 16 tempat desimal, yang merupakan
standar IEEE. Angka disimpan dalam format internal ini.

Tetapi format output EMT dapat diatur dengan cara yang fleksibel.
\end{eulercomment}
\begin{eulerprompt}
>longestformat; pi,
\end{eulerprompt}
\begin{euleroutput}
  3.141592653589793
\end{euleroutput}
\begin{eulerprompt}
>format(10,5); pi
\end{eulerprompt}
\begin{euleroutput}
    3.14159 
\end{euleroutput}
\begin{eulercomment}
Standarnya adalah defformat().
\end{eulercomment}
\begin{eulerprompt}
>defformat; // default
\end{eulerprompt}
\begin{eulercomment}
Ada operator pendek yang hanya mencetak satu nilai. Operator
"terpanjang" akan mencetak semua digit angka yang valid.
\end{eulercomment}
\begin{eulerprompt}
>longest pi^2/2
\end{eulerprompt}
\begin{euleroutput}
        4.934802200544679 
\end{euleroutput}
\begin{eulercomment}
Ada juga operator pendek untuk mencetak hasil dalam format pecahan.
Kami sudah menggunakannya di atas.
\end{eulercomment}
\begin{eulerprompt}
>fraction 1+1/2+1/3+1/4
\end{eulerprompt}
\begin{euleroutput}
  25/12
\end{euleroutput}
\begin{eulercomment}
Karena format internal menggunakan cara biner untuk menyimpan angka,
nilai 0,1 tidak akan direpresentasikan dengan tepat. Kesalahan
bertambah sedikit, seperti yang Anda lihat dalam perhitungan berikut.
\end{eulercomment}
\begin{eulerprompt}
>longest 0.1+0.1+0.1+0.1+0.1+0.1+0.1+0.1+0.1+0.1-1
\end{eulerprompt}
\begin{euleroutput}
   -1.110223024625157e-16 
\end{euleroutput}
\begin{eulercomment}
Tetapi dengan "format panjang" default Anda tidak akan melihat ini.
Untuk kenyamanan, output dari angka yang sangat kecil adalah 0.
\end{eulercomment}
\begin{eulerprompt}
>0.1+0.1+0.1+0.1+0.1+0.1+0.1+0.1+0.1+0.1-1
\end{eulerprompt}
\begin{euleroutput}
  0
\end{euleroutput}
\eulerheading{Ekspresi}
\begin{eulercomment}
String atau nama dapat digunakan untuk menyimpan ekspresi matematika,
yang dapat dievaluasi oleh EMT. Untuk ini, gunakan tanda kurung
setelah ekspresi. Jika Anda bermaksud menggunakan string sebagai
ekspresi, gunakan konvensi untuk menamakannya "fx" atau "fxy" dll.
Ekspresi lebih diutamakan daripada fungsi.

Variabel global dapat digunakan dalam evaluasi.
\end{eulercomment}
\begin{eulerprompt}
>r:=3; fx:="pi*2*r^2"; longest fx()
\end{eulerprompt}
\begin{euleroutput}
        56.54866776461628 
\end{euleroutput}
\begin{eulercomment}
Parameter ditetapkan ke x, y, dan z dalam urutan itu. Parameter
tambahan dapat ditambahkan menggunakan parameter yang ditetapkan.
\end{eulercomment}
\begin{eulerprompt}
>fx:="2*a*sin(x)^2"; fx(4,a=3)
\end{eulerprompt}
\begin{euleroutput}
  3.43650010143
\end{euleroutput}
\begin{eulercomment}
Perhatikan bahwa ekspresi akan selalu menggunakan variabel global,
bahkan jika ada variabel dalam fungsi dengan nama yang sama. (Jika
tidak, evaluasi ekspresi dalam fungsi dapat memberikan hasil yang
sangat membingungkan bagi pengguna yang memanggil fungsi tersebut.)
\end{eulercomment}
\begin{eulerprompt}
>at:=4; function f(expr,x,at) := expr(x); ...
>f("at*x^2",3,5) // computes 4*3^2 not 5*3^2
\end{eulerprompt}
\begin{euleroutput}
  36
\end{euleroutput}
\begin{eulercomment}
Jika Anda ingin menggunakan nilai lain untuk "at" daripada nilai
global, Anda perlu menambahkan "at=value".
\end{eulercomment}
\begin{eulerprompt}
>at:=4; function f(expr,x,a) := expr(x,at=a); ...
>f("at*x^2",3,5)
\end{eulerprompt}
\begin{euleroutput}
  45
\end{euleroutput}
\begin{eulercomment}
Untuk referensi, kami berkomentar bahwa koleksi panggilan (dibahas di
tempat lain) dapat berisi ekspresi. Jadi kita bisa membuat contoh di
atas sebagai berikut.
\end{eulercomment}
\begin{eulerprompt}
>at:=4; function f(expr,x) := expr(x); ...
>f(\{\{"at*x^2",at=5\}\},3)
\end{eulerprompt}
\begin{euleroutput}
  45
\end{euleroutput}
\begin{eulercomment}
Ekspresi dalam x sering digunakan seperti fungsi.\\
Perhatikan bahwa mendefinisikan fungsi dengan nama yang sama seperti
ekspresi simbolik global menghapus variabel ini untuk menghindari
kebingungan antara ekspresi simbolik dan fungsi.
\end{eulercomment}
\begin{eulerprompt}
>f &= 6*x;
>f(15)
\end{eulerprompt}
\begin{euleroutput}
  90
\end{euleroutput}
\begin{eulercomment}
Dengan cara konvensi, ekspresi simbolik atau numerik harus diberi nama
fx, fxy dll. Skema penamaan ini tidak boleh digunakan untuk fungsi.
\end{eulercomment}
\begin{eulerprompt}
>fx &= diff(x^x,x); $&fx
\end{eulerprompt}
\begin{eulercomment}
Bentuk khusus dari ekspresi memungkinkan variabel apa pun sebagai
parameter tanpa nama untuk evaluasi ekspresi, bukan hanya "x", "y"
dll. Untuk ini, mulai ekspresi dengan "@(variabel) ...".
\end{eulercomment}
\begin{eulerprompt}
>"@(a,b) a^2+b^2", %(4,5)
\end{eulerprompt}
\begin{euleroutput}
  @(a,b) a^2+b^2
  41
\end{euleroutput}
\begin{eulercomment}
Ini memungkinkan untuk memanipulasi ekspresi dalam variabel lain untuk
fungsi EMT yang membutuhkan ekspresi dalam "x".

Cara paling dasar untuk mendefinisikan fungsi sederhana adalah dengan
menyimpan rumusnya dalam ekspresi simbolis atau numerik. Jika variabel
utama adalah x, ekspresi dapat dievaluasi seperti fungsi.

Seperti yang Anda lihat dalam contoh berikut, variabel global terlihat
selama evaluasi.
\end{eulercomment}
\begin{eulerprompt}
>fx &= x^3-a*x;  ...
>a=1.2; fx(0.5)
\end{eulerprompt}
\begin{euleroutput}
  -0.475
\end{euleroutput}
\begin{eulercomment}
Semua variabel lain dalam ekspresi dapat ditentukan dalam evaluasi
menggunakan parameter yang ditetapkan.
\end{eulercomment}
\begin{eulerprompt}
>fx(0.5,a=1.1)
\end{eulerprompt}
\begin{euleroutput}
  -0.425
\end{euleroutput}
\begin{eulercomment}
Sebuah ekspresi tidak perlu simbolis. Ini diperlukan, jika ekspresi
berisi fungsi, yang hanya diketahui di kernel numerik, bukan di
Maxima.

\begin{eulercomment}
\eulerheading{Matematika Simbolik}
\begin{eulercomment}
EMT melakukan matematika simbolis dengan bantuan Maxima. Untuk
detailnya, mulailah dengan tutorial berikut, atau telusuri referensi
untuk Maxima. Para ahli di Maxima harus mencatat bahwa ada perbedaan
sintaks antara sintaks asli Maxima dan sintaks default ekspresi
simbolik di EMT.

Matematika simbolik terintegrasi dengan mulus ke dalam Euler dengan \&.
Ekspresi apa pun yang dimulai dengan \& adalah ekspresi simbolis. Itu
dievaluasi dan dicetak oleh Maxima.

Pertama-tama, Maxima memiliki aritmatika "tak terbatas" yang dapat
menangani angka yang sangat besar.
\end{eulercomment}
\begin{eulerprompt}
>$&44!
\end{eulerprompt}
\begin{eulercomment}
Dengan cara ini, Anda dapat menghitung hasil yang besar dengan tepat.
Mari kita hitung

lateks: C(44,10) = \textbackslash{}frac\{44!\}\{34! \textbackslash{}cdot 10!\}
\end{eulercomment}
\begin{eulerprompt}
>$& 44!/(34!*10!) // nilai C(44,10)
\end{eulerprompt}
\begin{eulercomment}
Tentu saja, Maxima memiliki fungsi yang lebih efisien untuk ini
(seperti halnya bagian numerik dari EMT).
\end{eulercomment}
\begin{eulerprompt}
>$binomial(44,10) //menghitung C(44,10) menggunakan fungsi binomial()
\end{eulerprompt}
\begin{eulercomment}
Untuk mempelajari lebih lanjut tentang fungsi tertentu klik dua kali
di atasnya. Misalnya, coba klik dua kali pada "\&binomial" di baris
perintah sebelumnya. Ini membuka dokumentasi Maxima seperti yang
disediakan oleh penulis program itu.

Anda akan belajar bahwa yang berikut ini juga berfungsi.

\end{eulercomment}
\begin{eulerformula}
\[
C(x,3)=\frac{x!}{(x-3)!3!}=\frac{(x-2)(x-1)x}{6}
\]
\end{eulerformula}
\begin{eulerprompt}
>$binomial(x,3) // C(x,3)
\end{eulerprompt}
\begin{eulercomment}
Jika Anda ingin mengganti x dengan nilai tertentu, gunakan "dengan".
\end{eulercomment}
\begin{eulerprompt}
>$&binomial(x,3) with x=10 // substitusi x=10 ke C(x,3)
\end{eulerprompt}
\begin{eulercomment}
Dengan begitu Anda dapat menggunakan solusi persamaan dalam persamaan
lain.

Ekspresi simbolik dicetak oleh Maxima dalam bentuk 2D. Alasan untuk
ini adalah bendera simbolis khusus dalam string.

Seperti yang akan Anda lihat pada contoh sebelumnya dan berikut, jika
Anda telah menginstal LaTeX, Anda dapat mencetak ekspresi simbolis
dengan Lateks. Jika tidak, perintah berikut akan mengeluarkan pesan
kesalahan.

Untuk mencetak ekspresi simbolis dengan LaTeX, gunakan \textdollar{} di depan \&
(atau Anda dapat menghilangkan \&) sebelum perintah. Jangan menjalankan
perintah Maxima dengan \textdollar{}, jika Anda tidak menginstal LaTeX.
\end{eulercomment}
\begin{eulerprompt}
>$(3+x)/(x^2+1)
\end{eulerprompt}
\begin{eulercomment}
Ekspresi simbolik diuraikan oleh Euler. Jika Anda membutuhkan sintaks
yang kompleks dalam satu ekspresi, Anda dapat menyertakan ekspresi
dalam "...". Untuk menggunakan lebih dari ekspresi sederhana adalah
mungkin, tetapi sangat tidak disarankan.
\end{eulercomment}
\begin{eulerprompt}
>&"v := 5; v^2"
\end{eulerprompt}
\begin{euleroutput}
  
                                    25
  
\end{euleroutput}
\begin{eulercomment}
Untuk kelengkapan, kami menyatakan bahwa ekspresi simbolik dapat
digunakan dalam program, tetapi perlu diapit dalam tanda kutip. Selain
itu, jauh lebih efektif untuk memanggil Maxima pada waktu kompilasi
jika memungkinkan.
\end{eulercomment}
\begin{eulerprompt}
>$&expand((1+x)^4), $&factor(diff(%,x)) // diff: turunan, factor: faktor
\end{eulerprompt}
\begin{eulercomment}
Sekali lagi, \% mengacu pada hasil sebelumnya.

Untuk mempermudah, kami menyimpan solusi ke variabel simbolik.
Variabel simbolik didefinisikan dengan "\&=".
\end{eulercomment}
\begin{eulerprompt}
>fx &= (x+1)/(x^4+1); $&fx
\end{eulerprompt}
\begin{eulercomment}
Ekspresi simbolik dapat digunakan dalam ekspresi simbolik lainnya.
\end{eulercomment}
\begin{eulerprompt}
>$&factor(diff(fx,x))
\end{eulerprompt}
\begin{eulercomment}
Masukan langsung dari perintah Maxima juga tersedia. Mulai baris
perintah dengan "::". Sintaks Maxima disesuaikan dengan sintaks EMT
(disebut "mode kompatibilitas").
\end{eulercomment}
\begin{eulerprompt}
>&factor(20!)
\end{eulerprompt}
\begin{euleroutput}
  
                           2432902008176640000
  
\end{euleroutput}
\begin{eulerprompt}
>::: factor(10!)
\end{eulerprompt}
\begin{euleroutput}
  
                                 8  4  2
                                2  3  5  7
  
\end{euleroutput}
\begin{eulerprompt}
>:: factor(20!)
\end{eulerprompt}
\begin{euleroutput}
  
                          18  8  4  2
                         2   3  5  7  11 13 17 19
  
\end{euleroutput}
\begin{eulercomment}
Jika Anda ahli dalam Maxima, Anda mungkin ingin menggunakan sintaks
asli Maxima. Anda dapat melakukannya dengan ":::".
\end{eulercomment}
\begin{eulerprompt}
>::: av:g$ av^2;
\end{eulerprompt}
\begin{euleroutput}
  
                                     2
                                    g
  
\end{euleroutput}
\begin{eulerprompt}
>fx &= x^3*exp(x), $fx
\end{eulerprompt}
\begin{euleroutput}
  
                                   3  x
                                  x  E
  
\end{euleroutput}
\begin{eulercomment}
Variabel tersebut dapat digunakan dalam ekspresi simbolik lainnya.
Perhatikan, bahwa dalam perintah berikut sisi kanan \&= dievaluasi
sebelum penugasan ke Fx.
\end{eulercomment}
\begin{eulerprompt}
>&(fx with x=5), $%, &float(%)
\end{eulerprompt}
\begin{euleroutput}
  
                                       5
                                  125 E
  
  
                            18551.64488782208
  
\end{euleroutput}
\begin{eulerprompt}
>fx(5)
\end{eulerprompt}
\begin{euleroutput}
  18551.6448878
\end{euleroutput}
\begin{eulercomment}
Untuk evaluasi ekspresi dengan nilai variabel tertentu, Anda dapat
menggunakan operator "with".

Baris perintah berikut juga menunjukkan bahwa Maxima dapat
mengevaluasi ekspresi secara numerik dengan float().
\end{eulercomment}
\begin{eulerprompt}
>&(fx with x=10)-(fx with x=5), &float(%)
\end{eulerprompt}
\begin{euleroutput}
  
                                  10        5
                            1000 E   - 125 E
  
  
                           2.20079141499189e+7
  
\end{euleroutput}
\begin{eulerprompt}
>$factor(diff(fx,x,2))
\end{eulerprompt}
\begin{eulercomment}
Untuk mendapatkan kode Lateks untuk ekspresi, Anda dapat menggunakan
perintah tex.
\end{eulercomment}
\begin{eulerprompt}
>tex(fx)
\end{eulerprompt}
\begin{euleroutput}
  x^3\(\backslash\),e^\{x\}
\end{euleroutput}
\begin{eulercomment}
Ekspresi simbolik dapat dievaluasi seperti ekspresi numerik.
\end{eulercomment}
\begin{eulerprompt}
>fx(0.5)
\end{eulerprompt}
\begin{euleroutput}
  0.206090158838
\end{euleroutput}
\begin{eulercomment}
Dalam ekspresi simbolis, ini tidak berfungsi, karena Maxima tidak
mendukungnya. Sebagai gantinya, gunakan sintaks "with" (bentuk yang
lebih bagus dari perintah at(...) dari Maxima).
\end{eulercomment}
\begin{eulerprompt}
>$&fx with x=1/2
\end{eulerprompt}
\begin{eulercomment}
Penugasan juga bisa bersifat simbolis.
\end{eulercomment}
\begin{eulerprompt}
>$&fx with x=1+t
\end{eulerprompt}
\begin{eulercomment}
Perintah solve memecahkan ekspresi simbolik untuk variabel di Maxima.
Hasilnya adalah vektor solusi.
\end{eulercomment}
\begin{eulerprompt}
>$&solve(x^2+x=4,x)
\end{eulerprompt}
\begin{eulercomment}
Bandingkan dengan perintah numerik "selesaikan" di Euler, yang
membutuhkan nilai awal, dan secara opsional nilai target.
\end{eulercomment}
\begin{eulerprompt}
>solve("x^2+x",1,y=4)
\end{eulerprompt}
\begin{euleroutput}
  1.56155281281
\end{euleroutput}
\begin{eulercomment}
Nilai numerik dari solusi simbolik dapat dihitung dengan evaluasi
hasil simbolis. Euler akan membaca tugas x= dll. Jika Anda tidak
memerlukan hasil numerik untuk perhitungan lebih lanjut, Anda juga
dapat membiarkan Maxima menemukan nilai numerik.
\end{eulercomment}
\begin{eulerprompt}
>sol &= solve(x^2+2*x=4,x); $&sol, sol(), $&float(sol)
\end{eulerprompt}
\begin{euleroutput}
  [-3.23607,  1.23607]
\end{euleroutput}
\begin{eulercomment}
Untuk mendapatkan solusi simbolis tertentu, seseorang dapat
menggunakan "with" dan index.
\end{eulercomment}
\begin{eulerprompt}
>$&solve(x^2+x=1,x), x2 &= x with %[2]; $&x2
\end{eulerprompt}
\begin{eulercomment}
Untuk menyelesaikan sistem persamaan, gunakan vektor persamaan.
Hasilnya adalah vektor solusi.
\end{eulercomment}
\begin{eulerprompt}
>sol &= solve([x+y=3,x^2+y^2=5],[x,y]); $&sol, $&x*y with sol[1]
\end{eulerprompt}
\begin{eulercomment}
Ekspresi simbolis dapat memiliki bendera, yang menunjukkan perlakuan
khusus di Maxima. Beberapa flag dapat digunakan sebagai perintah juga,
yang lain tidak. Bendera ditambahkan dengan "\textbar{}" (bentuk yang lebih
bagus dari "ev(...,flags)")
\end{eulercomment}
\begin{eulerprompt}
>$& diff((x^3-1)/(x+1),x) //turunan bentuk pecahan
>$& diff((x^3-1)/(x+1),x) | ratsimp //menyederhanakan pecahan
>$&factor(%)
\end{eulerprompt}
\eulerheading{Fungsi}
\begin{eulercomment}
Dalam EMT, fungsi adalah program yang didefinisikan dengan perintah
"fungsi". Ini bisa berupa fungsi satu baris atau fungsi multibaris.\\
Fungsi satu baris dapat berupa numerik atau simbolis. Fungsi satu
baris numerik didefinisikan oleh ":=".
\end{eulercomment}
\begin{eulerprompt}
>function f(x) := x*sqrt(x^2+1)
\end{eulerprompt}
\begin{eulercomment}
Untuk gambaran umum, kami menunjukkan semua kemungkinan definisi untuk
fungsi satu baris. Suatu fungsi dapat dievaluasi sama seperti fungsi
Euler bawaan lainnya.
\end{eulercomment}
\begin{eulerprompt}
>f(2)
\end{eulerprompt}
\begin{euleroutput}
  4.472135955
\end{euleroutput}
\begin{eulercomment}
Fungsi ini akan bekerja untuk vektor juga, dengan mematuhi bahasa
matriks Euler, karena ekspresi yang digunakan dalam fungsi
divektorkan.
\end{eulercomment}
\begin{eulerprompt}
>f(0:0.1:1)
\end{eulerprompt}
\begin{euleroutput}
  [0,  0.100499,  0.203961,  0.313209,  0.430813,  0.559017,  0.699714,
  0.854459,  1.0245,  1.21083,  1.41421]
\end{euleroutput}
\begin{eulercomment}
Fungsi dapat diplot. Alih-alih ekspresi, kita hanya perlu memberikan
nama fungsi.

Berbeda dengan ekspresi simbolik atau numerik, nama fungsi harus
diberikan dalam string.
\end{eulercomment}
\begin{eulerprompt}
>solve("f",1,y=1)
\end{eulerprompt}
\begin{euleroutput}
  0.786151377757
\end{euleroutput}
\begin{eulercomment}
Secara default, jika Anda perlu menimpa fungsi bawaan, Anda harus
menambahkan kata kunci "menimpa". Menimpa fungsi bawaan berbahaya dan
dapat menyebabkan masalah untuk fungsi lain tergantung pada fungsi
tersebut.

Anda masih dapat memanggil fungsi bawaan sebagai "\_...", jika itu
adalah fungsi di inti Euler.
\end{eulercomment}
\begin{eulerprompt}
>function overwrite sin (x) := _sin(x°) // redine sine in degrees
>sin(45)
\end{eulerprompt}
\begin{euleroutput}
  0.707106781187
\end{euleroutput}
\begin{eulercomment}
Lebih baik kita menghapus redefinisi dosa ini.
\end{eulercomment}
\begin{eulerprompt}
>forget sin; sin(pi/4)
\end{eulerprompt}
\begin{euleroutput}
  0.707106781187
\end{euleroutput}
\eulersubheading{Parameter Default}
\begin{eulercomment}
Fungsi numerik dapat memiliki parameter default.
\end{eulercomment}
\begin{eulerprompt}
>function f(x,a=1) := a*x^2
\end{eulerprompt}
\begin{eulercomment}
Menghilangkan parameter ini menggunakan nilai default.
\end{eulercomment}
\begin{eulerprompt}
>f(4)
\end{eulerprompt}
\begin{euleroutput}
  16
\end{euleroutput}
\begin{eulercomment}
Menyetelnya akan menimpa nilai default.
\end{eulercomment}
\begin{eulerprompt}
>f(4,5)
\end{eulerprompt}
\begin{euleroutput}
  80
\end{euleroutput}
\begin{eulercomment}
Parameter yang ditetapkan menimpanya juga. Ini digunakan oleh banyak
fungsi Euler seperti plot2d, plot3d.
\end{eulercomment}
\begin{eulerprompt}
>f(4,a=1)
\end{eulerprompt}
\begin{euleroutput}
  16
\end{euleroutput}
\begin{eulercomment}
Jika suatu variabel bukan parameter, itu harus global. Fungsi satu
baris dapat melihat variabel global.
\end{eulercomment}
\begin{eulerprompt}
>function f(x) := a*x^2
>a=6; f(2)
\end{eulerprompt}
\begin{euleroutput}
  24
\end{euleroutput}
\begin{eulercomment}
Tetapi parameter yang ditetapkan menimpa nilai global.

Jika argumen tidak ada dalam daftar parameter yang telah ditentukan
sebelumnya, argumen tersebut harus dideklarasikan dengan ":="!
\end{eulercomment}
\begin{eulerprompt}
>f(2,a:=5)
\end{eulerprompt}
\begin{euleroutput}
  20
\end{euleroutput}
\begin{eulercomment}
Fungsi simbolis didefinisikan dengan "\&=". Mereka didefinisikan dalam
Euler dan Maxima, dan bekerja di kedua dunia. Ekspresi yang
mendefinisikan dijalankan melalui Maxima sebelum definisi.
\end{eulercomment}
\begin{eulerprompt}
>function g(x) &= x^3-x*exp(-x); $&g(x)
\end{eulerprompt}
\begin{eulercomment}
Fungsi simbolik dapat digunakan dalam ekspresi simbolik.
\end{eulercomment}
\begin{eulerprompt}
>$&diff(g(x),x), $&% with x=4/3
\end{eulerprompt}
\begin{eulercomment}
Mereka juga dapat digunakan dalam ekspresi numerik. Tentu saja, ini
hanya akan berfungsi jika EMT dapat menginterpretasikan semua yang ada
di dalam fungsi tersebut.
\end{eulercomment}
\begin{eulerprompt}
>g(5+g(1))
\end{eulerprompt}
\begin{euleroutput}
  178.635099908
\end{euleroutput}
\begin{eulercomment}
Mereka dapat digunakan untuk mendefinisikan fungsi atau ekspresi
simbolik lainnya.
\end{eulercomment}
\begin{eulerprompt}
>function G(x) &= factor(integrate(g(x),x)); $&G(c) // integrate: mengintegralkan
>solve(&g(x),0.5)
\end{eulerprompt}
\begin{euleroutput}
  0.703467422498
\end{euleroutput}
\begin{eulercomment}
Berikut ini juga berfungsi, karena Euler menggunakan ekspresi simbolis
dalam fungsi g, jika tidak menemukan variabel simbolik g, dan jika ada
fungsi simbolis g.
\end{eulercomment}
\begin{eulerprompt}
>solve(&g,0.5)
\end{eulerprompt}
\begin{euleroutput}
  0.703467422498
\end{euleroutput}
\begin{eulerprompt}
>function P(x,n) &= (2*x-1)^n; $&P(x,n)
>function Q(x,n) &= (x+2)^n; $&Q(x,n)
>$&P(x,4), $&expand(%)
>P(3,4)
\end{eulerprompt}
\begin{euleroutput}
  625
\end{euleroutput}
\begin{eulerprompt}
>$&P(x,4)+ Q(x,3), $&expand(%)
>$&P(x,4)-Q(x,3), $&expand(%), $&factor(%)
>$&P(x,4)*Q(x,3), $&expand(%), $&factor(%)
>$&P(x,4)/Q(x,1), $&expand(%), $&factor(%)
>function f(x) &= x^3-x; $&f(x)
\end{eulerprompt}
\begin{eulercomment}
Dengan \&= fungsinya simbolis, dan dapat digunakan dalam ekspresi
simbolik lainnya.
\end{eulercomment}
\begin{eulerprompt}
>$&integrate(f(x),x)
\end{eulerprompt}
\begin{eulercomment}
Dengan := fungsinya numerik. Contoh yang baik adalah integral tak
tentu seperti

\end{eulercomment}
\begin{eulerformula}
\[
f(x) = \int_1^x t^t \, dt,
\]
\end{eulerformula}
\begin{eulercomment}
yang tidak dapat dinilai secara simbolis.

Jika kita mendefinisikan kembali fungsi dengan kata kunci "peta" dapat
digunakan untuk vektor x. Secara internal, fungsi dipanggil untuk
semua nilai x satu kali, dan hasilnya disimpan dalam vektor.
\end{eulercomment}
\begin{eulerprompt}
>function map f(x) := integrate("x^x",1,x)
>f(0:0.5:2)
\end{eulerprompt}
\begin{euleroutput}
  [-0.783431,  -0.410816,  0,  0.676863,  2.05045]
\end{euleroutput}
\begin{eulercomment}
Fungsi dapat memiliki nilai default untuk parameter.
\end{eulercomment}
\begin{eulerprompt}
>function mylog (x,base=10) := ln(x)/ln(base);
\end{eulerprompt}
\begin{eulercomment}
Sekarang fungsi dapat dipanggil dengan atau tanpa parameter "basis".
\end{eulercomment}
\begin{eulerprompt}
>mylog(100), mylog(2^6.7,2)
\end{eulerprompt}
\begin{euleroutput}
  2
  6.7
\end{euleroutput}
\begin{eulercomment}
Selain itu, dimungkinkan untuk menggunakan parameter yang ditetapkan.
\end{eulercomment}
\begin{eulerprompt}
>mylog(E^2,base=E)
\end{eulerprompt}
\begin{euleroutput}
  2
\end{euleroutput}
\begin{eulercomment}
Seringkali, kita ingin menggunakan fungsi untuk vektor di satu tempat,
dan untuk elemen individual di tempat lain. Ini dimungkinkan dengan
parameter vektor.
\end{eulercomment}
\begin{eulerprompt}
>function f([a,b]) &= a^2+b^2-a*b+b; $&f(a,b), $&f(x,y)
\end{eulerprompt}
\begin{eulercomment}
Fungsi simbolik seperti itu dapat digunakan untuk variabel simbolik.

Tetapi fungsinya juga dapat digunakan untuk vektor numerik.
\end{eulercomment}
\begin{eulerprompt}
>v=[3,4]; f(v)
\end{eulerprompt}
\begin{euleroutput}
  17
\end{euleroutput}
\begin{eulercomment}
Ada juga fungsi simbolis murni, yang tidak dapat digunakan secara
numerik.
\end{eulercomment}
\begin{eulerprompt}
>function lapl(expr,x,y) &&= diff(expr,x,2)+diff(expr,y,2)//turunan parsial kedua
\end{eulerprompt}
\begin{euleroutput}
  
                   diff(expr, y, 2) + diff(expr, x, 2)
  
\end{euleroutput}
\begin{eulerprompt}
>$&realpart((x+I*y)^4), $&lapl(%,x,y)
\end{eulerprompt}
\begin{eulercomment}
Tetapi tentu saja, mereka dapat digunakan dalam ekspresi simbolik atau
dalam definisi fungsi simbolik.
\end{eulercomment}
\begin{eulerprompt}
>function f(x,y) &= factor(lapl((x+y^2)^5,x,y)); $&f(x,y)
\end{eulerprompt}
\begin{eulercomment}
Untuk meringkas

- \&= mendefinisikan fungsi simbolis,\\
- := mendefinisikan fungsi numerik,\\
- \&\&= mendefinisikan fungsi simbolis murni.

\begin{eulercomment}
\eulerheading{Memecahkan Ekspresi}
\begin{eulercomment}
Ekspresi dapat diselesaikan secara numerik dan simbolis.

Untuk menyelesaikan ekspresi sederhana dari satu variabel, kita dapat
menggunakan fungsi solve(). Perlu nilai awal untuk memulai pencarian.
Secara internal, solve() menggunakan metode secant.
\end{eulercomment}
\begin{eulerprompt}
>solve("x^2-2",1)
\end{eulerprompt}
\begin{euleroutput}
  1.41421356237
\end{euleroutput}
\begin{eulercomment}
Ini juga berfungsi untuk ekspresi simbolis. Ambil fungsi berikut.
\end{eulercomment}
\begin{eulerprompt}
>$&solve(x^2=2,x)
>$&solve(x^2-2,x)
>$&solve(a*x^2+b*x+c=0,x)
>$&solve([a*x+b*y=c,d*x+e*y=f],[x,y])
>px &= 4*x^8+x^7-x^4-x; $&px
\end{eulerprompt}
\begin{eulercomment}
Sekarang kita mencari titik, di mana polinomialnya adalah 2. Dalam
solve(), nilai target default y=0 dapat diubah dengan variabel yang
ditetapkan.\\
Kami menggunakan y=2 dan memeriksa dengan mengevaluasi polinomial pada
hasil sebelumnya.
\end{eulercomment}
\begin{eulerprompt}
>solve(px,1,y=2), px(%)
\end{eulerprompt}
\begin{euleroutput}
  0.966715594851
  2
\end{euleroutput}
\begin{eulercomment}
Memecahkan ekspresi simbolis dalam bentuk simbolis mengembalikan
daftar solusi. Kami menggunakan pemecah simbolik solve() yang
disediakan oleh Maxima.
\end{eulercomment}
\begin{eulerprompt}
>sol &= solve(x^2-x-1,x); $&sol
\end{eulerprompt}
\begin{eulercomment}
Cara termudah untuk mendapatkan nilai numerik adalah dengan
mengevaluasi solusi secara numerik seperti ekspresi.
\end{eulercomment}
\begin{eulerprompt}
>longest sol()
\end{eulerprompt}
\begin{euleroutput}
      -0.6180339887498949       1.618033988749895 
\end{euleroutput}
\begin{eulercomment}
Untuk menggunakan solusi secara simbolis dalam ekspresi lain, cara
termudah adalah "dengan".
\end{eulercomment}
\begin{eulerprompt}
>$&x^2 with sol[1], $&expand(x^2-x-1 with sol[2])
\end{eulerprompt}
\begin{eulercomment}
Memecahkan sistem persamaan secara simbolis dapat dilakukan dengan
vektor persamaan dan solver simbolis solve(). Jawabannya adalah daftar
daftar persamaan.
\end{eulercomment}
\begin{eulerprompt}
>$&solve([x+y=2,x^3+2*y+x=4],[x,y])
\end{eulerprompt}
\begin{eulercomment}
Fungsi f() dapat melihat variabel global. Namun seringkali kita ingin
menggunakan parameter lokal.

lateks: a\textasciicircum{}x-x\textasciicircum{}a = 0.1

dengan a=3.
\end{eulercomment}
\begin{eulerprompt}
>function f(x,a) := x^a-a^x;
\end{eulerprompt}
\begin{eulercomment}
Salah satu cara untuk meneruskan parameter tambahan ke f() adalah
dengan menggunakan daftar dengan nama fungsi dan parameter (sebaliknya
adalah parameter titik koma).
\end{eulercomment}
\begin{eulerprompt}
>solve(\{\{"f",3\}\},2,y=0.1)
\end{eulerprompt}
\begin{euleroutput}
  2.54116291558
\end{euleroutput}
\begin{eulercomment}
Ini juga bekerja dengan ekspresi. Tapi kemudian, elemen daftar bernama
harus digunakan. (Lebih lanjut tentang daftar di tutorial tentang
sintaks EMT).
\end{eulercomment}
\begin{eulerprompt}
>solve(\{\{"x^a-a^x",a=3\}\},2,y=0.1)
\end{eulerprompt}
\begin{euleroutput}
  2.54116291558
\end{euleroutput}
\eulerheading{Menyelesaikan Pertidaksamaan}
\begin{eulercomment}
Untuk menyelesaikan pertidaksamaan, EMT tidak akan dapat melakukannya,
melainkan dengan bantuan Maxima, artinya secara eksak (simbolik).
Perintah Maxima yang digunakan adalah fourier\_elim(), yang harus
dipanggil dengan perintah "load(fourier\_elim)" terlebih dahulu.
\end{eulercomment}
\begin{eulerprompt}
>&load(fourier_elim)
\end{eulerprompt}
\begin{euleroutput}
  
          C:/Program Files/Euler x64/maxima/share/maxima/5.35.1/share/f\(\backslash\)
  ourier_elim/fourier_elim.lisp
  
\end{euleroutput}
\begin{eulerprompt}
>$&fourier_elim([x^2 - 1>0],[x]) // x^2-1 > 0
>$&fourier_elim([x^2 - 1<0],[x]) // x^2-1 < 0
>$&fourier_elim([x^2 - 1 # 0],[x]) // x^-1 <> 0
>$&fourier_elim([x # 6],[x])
>$&fourier_elim([x < 1, x > 1],[x]) // tidak memiliki penyelesaian
>$&fourier_elim([minf < x, x < inf],[x]) // solusinya R
>$&fourier_elim([x^3 - 1 > 0],[x])
>$&fourier_elim([cos(x) < 1/2],[x]) // ??? gagal
>$&fourier_elim([y-x < 5, x - y < 7, 10 < y],[x,y]) // sistem pertidaksamaan
>$&fourier_elim([y-x < 5, x - y < 7, 10 < y],[y,x])
>$&fourier_elim((x + y < 5) and (x - y >8),[x,y])
>$&fourier_elim(((x + y < 5) and x < 1) or  (x - y >8),[x,y])
>&fourier_elim([max(x,y) > 6, x # 8, abs(y-1) > 12],[x,y])
\end{eulerprompt}
\begin{euleroutput}
  
          [6 < x, x < 8, y < - 11] or [8 < x, y < - 11]
   or [x < 8, 13 < y] or [x = y, 13 < y] or [8 < x, x < y, 13 < y]
   or [y < x, 13 < y]
  
\end{euleroutput}
\begin{eulerprompt}
>$&fourier_elim([(x+6)/(x-9) <= 6],[x])
\end{eulerprompt}
\eulerheading{Bahasa Matriks}
\begin{eulercomment}
Dokumentasi inti EMT berisi diskusi terperinci tentang bahasa matriks
Euler.

Vektor dan matriks dimasukkan dengan tanda kurung siku, elemen
dipisahkan dengan koma, baris dipisahkan dengan titik koma.
\end{eulercomment}
\begin{eulerprompt}
>A=[1,2;3,4]
\end{eulerprompt}
\begin{euleroutput}
              1             2 
              3             4 
\end{euleroutput}
\begin{eulercomment}
Produk matriks dilambangkan dengan titik.
\end{eulercomment}
\begin{eulerprompt}
>b=[3;4]
\end{eulerprompt}
\begin{euleroutput}
              3 
              4 
\end{euleroutput}
\begin{eulerprompt}
>b' // transpose b
\end{eulerprompt}
\begin{euleroutput}
  [3,  4]
\end{euleroutput}
\begin{eulerprompt}
>inv(A) //inverse A
\end{eulerprompt}
\begin{euleroutput}
             -2             1 
            1.5          -0.5 
\end{euleroutput}
\begin{eulerprompt}
>A.b //perkalian matriks
\end{eulerprompt}
\begin{euleroutput}
             11 
             25 
\end{euleroutput}
\begin{eulerprompt}
>A.inv(A)
\end{eulerprompt}
\begin{euleroutput}
              1             0 
              0             1 
\end{euleroutput}
\begin{eulercomment}
Poin utama dari bahasa matriks adalah bahwa semua fungsi dan operator
bekerja elemen untuk elemen.
\end{eulercomment}
\begin{eulerprompt}
>A.A
\end{eulerprompt}
\begin{euleroutput}
              7            10 
             15            22 
\end{euleroutput}
\begin{eulerprompt}
>A^2 //perpangkatan elemen2 A
\end{eulerprompt}
\begin{euleroutput}
              1             4 
              9            16 
\end{euleroutput}
\begin{eulerprompt}
>A.A.A
\end{eulerprompt}
\begin{euleroutput}
             37            54 
             81           118 
\end{euleroutput}
\begin{eulerprompt}
>power(A,3) //perpangkatan matriks
\end{eulerprompt}
\begin{euleroutput}
             37            54 
             81           118 
\end{euleroutput}
\begin{eulerprompt}
>A/A //pembagian elemen-elemen matriks yang seletak
\end{eulerprompt}
\begin{euleroutput}
              1             1 
              1             1 
\end{euleroutput}
\begin{eulerprompt}
>A/b //pembagian elemen2 A oleh elemen2 b kolom demi kolom (karena b vektor kolom)
\end{eulerprompt}
\begin{euleroutput}
       0.333333      0.666667 
           0.75             1 
\end{euleroutput}
\begin{eulerprompt}
>A\(\backslash\)b // hasilkali invers A dan b, A^(-1)b 
\end{eulerprompt}
\begin{euleroutput}
             -2 
            2.5 
\end{euleroutput}
\begin{eulerprompt}
>inv(A).b
\end{eulerprompt}
\begin{euleroutput}
             -2 
            2.5 
\end{euleroutput}
\begin{eulerprompt}
>A\(\backslash\)A   //A^(-1)A
\end{eulerprompt}
\begin{euleroutput}
              1             0 
              0             1 
\end{euleroutput}
\begin{eulerprompt}
>inv(A).A
\end{eulerprompt}
\begin{euleroutput}
              1             0 
              0             1 
\end{euleroutput}
\begin{eulerprompt}
>A*A //perkalin elemen-elemen matriks seletak
\end{eulerprompt}
\begin{euleroutput}
              1             4 
              9            16 
\end{euleroutput}
\begin{eulercomment}
Ini bukan produk matriks, tetapi perkalian elemen demi elemen. Hal
yang sama berlaku untuk vektor.
\end{eulercomment}
\begin{eulerprompt}
>b^2 // perpangkatan elemen-elemen matriks/vektor
\end{eulerprompt}
\begin{euleroutput}
              9 
             16 
\end{euleroutput}
\begin{eulercomment}
Jika salah satu operan adalah vektor atau skalar, itu diperluas secara
alami.
\end{eulercomment}
\begin{eulerprompt}
>2*A
\end{eulerprompt}
\begin{euleroutput}
              2             4 
              6             8 
\end{euleroutput}
\begin{eulercomment}
Misalnya, jika operan adalah vektor kolom, elemennya diterapkan ke
semua baris A.
\end{eulercomment}
\begin{eulerprompt}
>[1,2]*A
\end{eulerprompt}
\begin{euleroutput}
              1             4 
              3             8 
\end{euleroutput}
\begin{eulercomment}
Jika itu adalah vektor baris, itu diterapkan ke semua kolom A.
\end{eulercomment}
\begin{eulerprompt}
>A*[2,3]
\end{eulerprompt}
\begin{euleroutput}
              2             6 
              6            12 
\end{euleroutput}
\begin{eulercomment}
Seseorang dapat membayangkan perkalian ini seolah-olah vektor baris v
telah digandakan untuk membentuk matriks dengan ukuran yang sama
dengan A.
\end{eulercomment}
\begin{eulerprompt}
>dup([1,2],2) // dup: menduplikasi/menggandakan vektor [1,2] sebanyak 2 kali (baris)
\end{eulerprompt}
\begin{euleroutput}
              1             2 
              1             2 
\end{euleroutput}
\begin{eulerprompt}
>A*dup([1,2],2) 
\end{eulerprompt}
\begin{euleroutput}
              1             4 
              3             8 
\end{euleroutput}
\begin{eulercomment}
Ini juga berlaku untuk dua vektor di mana satu adalah vektor baris dan
yang lainnya adalah vektor kolom. Kami menghitung i*j untuk i,j dari 1
hingga 5. Caranya adalah dengan mengalikan 1:5 dengan transposnya.
Bahasa matriks Euler secara otomatis menghasilkan tabel nilai.
\end{eulercomment}
\begin{eulerprompt}
>(1:5)*(1:5)' // hasilkali elemen-elemen vektor baris dan vektor kolom
\end{eulerprompt}
\begin{euleroutput}
              1             2             3             4             5 
              2             4             6             8            10 
              3             6             9            12            15 
              4             8            12            16            20 
              5            10            15            20            25 
\end{euleroutput}
\begin{eulercomment}
Sekali lagi, ingat bahwa ini bukan produk matriks!
\end{eulercomment}
\begin{eulerprompt}
>(1:5).(1:5)' // hasilkali vektor baris dan vektor kolom
\end{eulerprompt}
\begin{euleroutput}
  55
\end{euleroutput}
\begin{eulerprompt}
>sum((1:5)*(1:5)) // sama hasilnya
\end{eulerprompt}
\begin{euleroutput}
  55
\end{euleroutput}
\begin{eulercomment}
Bahkan operator seperti \textless{} atau == bekerja dengan cara yang sama.
\end{eulercomment}
\begin{eulerprompt}
>(1:10)<6 // menguji elemen-elemen yang kurang dari 6
\end{eulerprompt}
\begin{euleroutput}
  [1,  1,  1,  1,  1,  0,  0,  0,  0,  0]
\end{euleroutput}
\begin{eulercomment}
Misalnya, kita dapat menghitung jumlah elemen yang memenuhi kondisi
tertentu dengan fungsi sum().
\end{eulercomment}
\begin{eulerprompt}
>sum((1:10)<6) // banyak elemen yang kurang dari 6
\end{eulerprompt}
\begin{euleroutput}
  5
\end{euleroutput}
\begin{eulercomment}
Euler memiliki operator perbandingan, seperti "==", yang memeriksa
kesetaraan.

Kami mendapatkan vektor 0 dan 1, di mana 1 berarti benar.
\end{eulercomment}
\begin{eulerprompt}
>t=(1:10)^2; t==25 //menguji elemen2 t yang sama dengan 25 (hanya ada 1)
\end{eulerprompt}
\begin{euleroutput}
  [0,  0,  0,  0,  1,  0,  0,  0,  0,  0]
\end{euleroutput}
\begin{eulercomment}
Dari vektor seperti itu, "bukan nol" memilih elemen bukan nol.

Dalam hal ini, kami mendapatkan indeks semua elemen lebih besar dari
50.
\end{eulercomment}
\begin{eulerprompt}
>nonzeros(t>50) //indeks elemen2 t yang lebih besar daripada 50
\end{eulerprompt}
\begin{euleroutput}
  [8,  9,  10]
\end{euleroutput}
\begin{eulercomment}
Tentu saja, kita dapat menggunakan vektor indeks ini untuk mendapatkan
nilai yang sesuai dalam t.
\end{eulercomment}
\begin{eulerprompt}
>t[nonzeros(t>50)] //elemen2 t yang lebih besar daripada 50
\end{eulerprompt}
\begin{euleroutput}
  [64,  81,  100]
\end{euleroutput}
\begin{eulercomment}
Sebagai contoh, mari kita cari semua kuadrat dari angka 1 hingga 1000,
yaitu 5 modulo 11 dan 3 modulo 13.
\end{eulercomment}
\begin{eulerprompt}
>t=1:1000; nonzeros(mod(t^2,11)==5 && mod(t^2,13)==3)
\end{eulerprompt}
\begin{euleroutput}
  [4,  48,  95,  139,  147,  191,  238,  282,  290,  334,  381,  425,
  433,  477,  524,  568,  576,  620,  667,  711,  719,  763,  810,  854,
  862,  906,  953,  997]
\end{euleroutput}
\begin{eulercomment}
EMT tidak sepenuhnya efektif untuk perhitungan bilangan bulat. Ini
menggunakan titik mengambang presisi ganda secara internal. Namun,
seringkali sangat berguna.

Kita dapat memeriksa keutamaan. Mari kita cari tahu, berapa banyak
kuadrat ditambah 1 adalah bilangan prima.
\end{eulercomment}
\begin{eulerprompt}
>t=1:1000; length(nonzeros(isprime(t^2+1)))
\end{eulerprompt}
\begin{euleroutput}
  112
\end{euleroutput}
\begin{eulercomment}
Fungsi bukan nol() hanya berfungsi untuk vektor. Untuk matriks, ada
mnonzeros().
\end{eulercomment}
\begin{eulerprompt}
>seed(2); A=random(3,4)
\end{eulerprompt}
\begin{euleroutput}
       0.765761      0.401188      0.406347      0.267829 
        0.13673      0.390567      0.495975      0.952814 
       0.548138      0.006085      0.444255      0.539246 
\end{euleroutput}
\begin{eulercomment}
Ini mengembalikan indeks elemen, yang bukan nol.
\end{eulercomment}
\begin{eulerprompt}
>k=mnonzeros(A<0.4) //indeks elemen2 A yang kurang dari 0,4
\end{eulerprompt}
\begin{euleroutput}
              1             4 
              2             1 
              2             2 
              3             2 
\end{euleroutput}
\begin{eulercomment}
Indeks ini dapat digunakan untuk mengatur elemen ke beberapa nilai.
\end{eulercomment}
\begin{eulerprompt}
>mset(A,k,0) //mengganti elemen2 suatu matriks pada indeks tertentu
\end{eulerprompt}
\begin{euleroutput}
       0.765761      0.401188      0.406347             0 
              0             0      0.495975      0.952814 
       0.548138             0      0.444255      0.539246 
\end{euleroutput}
\begin{eulercomment}
Fungsi mset() juga dapat mengatur elemen pada indeks ke entri dari
beberapa matriks lainnya.
\end{eulercomment}
\begin{eulerprompt}
>mset(A,k,-random(size(A)))
\end{eulerprompt}
\begin{euleroutput}
       0.765761      0.401188      0.406347     -0.126917 
      -0.122404     -0.691673      0.495975      0.952814 
       0.548138     -0.483902      0.444255      0.539246 
\end{euleroutput}
\begin{eulercomment}
Dan dimungkinkan untuk mendapatkan elemen dalam vektor.
\end{eulercomment}
\begin{eulerprompt}
>mget(A,k)
\end{eulerprompt}
\begin{euleroutput}
  [0.267829,  0.13673,  0.390567,  0.006085]
\end{euleroutput}
\begin{eulercomment}
Fungsi lain yang berguna adalah ekstrem, yang mengembalikan nilai
minimal dan maksimal di setiap baris matriks dan posisinya.
\end{eulercomment}
\begin{eulerprompt}
>ex=extrema(A)
\end{eulerprompt}
\begin{euleroutput}
       0.267829             4      0.765761             1 
        0.13673             1      0.952814             4 
       0.006085             2      0.548138             1 
\end{euleroutput}
\begin{eulercomment}
Kita dapat menggunakan ini untuk mengekstrak nilai maksimal di setiap
baris.
\end{eulercomment}
\begin{eulerprompt}
>ex[,3]'
\end{eulerprompt}
\begin{euleroutput}
  [0.765761,  0.952814,  0.548138]
\end{euleroutput}
\begin{eulercomment}
Ini, tentu saja, sama dengan fungsi max().
\end{eulercomment}
\begin{eulerprompt}
>max(A)'
\end{eulerprompt}
\begin{euleroutput}
  [0.765761,  0.952814,  0.548138]
\end{euleroutput}
\begin{eulercomment}
Tetapi dengan mget(), kita dapat mengekstrak indeks dan menggunakan
informasi ini untuk mengekstrak elemen pada posisi yang sama dari
matriks lain.
\end{eulercomment}
\begin{eulerprompt}
>j=(1:rows(A))'|ex[,4], mget(-A,j)
\end{eulerprompt}
\begin{euleroutput}
              1             1 
              2             4 
              3             1 
  [-0.765761,  -0.952814,  -0.548138]
\end{euleroutput}
\eulerheading{Fungsi Matriks Lainnya (Membangun Matriks)}
\begin{eulercomment}
Untuk membangun matriks, kita dapat menumpuk satu matriks di atas yang
lain. Jika keduanya tidak memiliki jumlah kolom yang sama, kolom yang
lebih pendek akan diisi dengan 0.
\end{eulercomment}
\begin{eulerprompt}
>v=1:3; v_v
\end{eulerprompt}
\begin{euleroutput}
              1             2             3 
              1             2             3 
\end{euleroutput}
\begin{eulercomment}
Demikian juga, kita dapat melampirkan matriks ke yang lain secara
berdampingan, jika keduanya memiliki jumlah baris yang sama.
\end{eulercomment}
\begin{eulerprompt}
>A=random(3,4); A|v'
\end{eulerprompt}
\begin{euleroutput}
       0.032444     0.0534171      0.595713      0.564454             1 
        0.83916      0.175552      0.396988       0.83514             2 
      0.0257573      0.658585      0.629832      0.770895             3 
\end{euleroutput}
\begin{eulercomment}
Jika mereka tidak memiliki jumlah baris yang sama, matriks yang lebih
pendek diisi dengan 0.

Ada pengecualian untuk aturan ini. Bilangan real yang dilampirkan pada
matriks akan digunakan sebagai kolom yang diisi dengan bilangan real
tersebut.
\end{eulercomment}
\begin{eulerprompt}
>A|1
\end{eulerprompt}
\begin{euleroutput}
       0.032444     0.0534171      0.595713      0.564454             1 
        0.83916      0.175552      0.396988       0.83514             1 
      0.0257573      0.658585      0.629832      0.770895             1 
\end{euleroutput}
\begin{eulercomment}
Dimungkinkan untuk membuat matriks vektor baris dan kolom.
\end{eulercomment}
\begin{eulerprompt}
>[v;v]
\end{eulerprompt}
\begin{euleroutput}
              1             2             3 
              1             2             3 
\end{euleroutput}
\begin{eulerprompt}
>[v',v']
\end{eulerprompt}
\begin{euleroutput}
              1             1 
              2             2 
              3             3 
\end{euleroutput}
\begin{eulercomment}
Tujuan utama dari ini adalah untuk menafsirkan vektor ekspresi untuk
vektor kolom.
\end{eulercomment}
\begin{eulerprompt}
>"[x,x^2]"(v')
\end{eulerprompt}
\begin{euleroutput}
              1             1 
              2             4 
              3             9 
\end{euleroutput}
\begin{eulercomment}
Untuk mendapatkan ukuran A, kita dapat menggunakan fungsi berikut.
\end{eulercomment}
\begin{eulerprompt}
>C=zeros(2,4); rows(C), cols(C), size(C), length(C)
\end{eulerprompt}
\begin{euleroutput}
  2
  4
  [2,  4]
  4
\end{euleroutput}
\begin{eulercomment}
Untuk vektor, ada panjang().
\end{eulercomment}
\begin{eulerprompt}
>length(2:10)
\end{eulerprompt}
\begin{euleroutput}
  9
\end{euleroutput}
\begin{eulercomment}
Ada banyak fungsi lain, yang menghasilkan matriks.
\end{eulercomment}
\begin{eulerprompt}
>ones(2,2)
\end{eulerprompt}
\begin{euleroutput}
              1             1 
              1             1 
\end{euleroutput}
\begin{eulercomment}
Ini juga dapat digunakan dengan satu parameter. Untuk mendapatkan
vektor dengan angka selain 1, gunakan yang berikut ini.
\end{eulercomment}
\begin{eulerprompt}
>ones(5)*6
\end{eulerprompt}
\begin{euleroutput}
  [6,  6,  6,  6,  6]
\end{euleroutput}
\begin{eulercomment}
Juga matriks bilangan acak dapat dihasilkan dengan acak (distribusi
seragam) atau normal (distribusi Gau).
\end{eulercomment}
\begin{eulerprompt}
>random(2,2)
\end{eulerprompt}
\begin{euleroutput}
        0.66566      0.831835 
          0.977      0.544258 
\end{euleroutput}
\begin{eulercomment}
Berikut adalah fungsi lain yang berguna, yang merestrukturisasi elemen
matriks menjadi matriks lain.
\end{eulercomment}
\begin{eulerprompt}
>redim(1:9,3,3) // menyusun elemen2 1, 2, 3, ..., 9 ke bentuk matriks 3x3
\end{eulerprompt}
\begin{euleroutput}
              1             2             3 
              4             5             6 
              7             8             9 
\end{euleroutput}
\begin{eulercomment}
Dengan fungsi berikut, kita dapat menggunakan ini dan fungsi dup untuk
menulis fungsi rep(), yang mengulang vektor n kali.
\end{eulercomment}
\begin{eulerprompt}
>function rep(v,n) := redim(dup(v,n),1,n*cols(v))
\end{eulerprompt}
\begin{eulercomment}
Mari kita uji.
\end{eulercomment}
\begin{eulerprompt}
>rep(1:3,5)
\end{eulerprompt}
\begin{euleroutput}
  [1,  2,  3,  1,  2,  3,  1,  2,  3,  1,  2,  3,  1,  2,  3]
\end{euleroutput}
\begin{eulercomment}
Fungsi multdup() menduplikasi elemen vektor.
\end{eulercomment}
\begin{eulerprompt}
>multdup(1:3,5), multdup(1:3,[2,3,2])
\end{eulerprompt}
\begin{euleroutput}
  [1,  1,  1,  1,  1,  2,  2,  2,  2,  2,  3,  3,  3,  3,  3]
  [1,  1,  2,  2,  2,  3,  3]
\end{euleroutput}
\begin{eulercomment}
Fungsi flipx() dan flipy() mengembalikan urutan baris atau kolom
matriks. Yaitu, fungsi flipx() membalik secara horizontal.
\end{eulercomment}
\begin{eulerprompt}
>flipx(1:5) //membalik elemen2 vektor baris
\end{eulerprompt}
\begin{euleroutput}
  [5,  4,  3,  2,  1]
\end{euleroutput}
\begin{eulercomment}
Untuk rotasi, Euler memiliki rotleft() dan rotright().
\end{eulercomment}
\begin{eulerprompt}
>rotleft(1:5) // memutar elemen2 vektor baris
\end{eulerprompt}
\begin{euleroutput}
  [2,  3,  4,  5,  1]
\end{euleroutput}
\begin{eulercomment}
Sebuah fungsi khusus adalah drop(v,i), yang menghilangkan elemen
dengan indeks di i dari vektor v.
\end{eulercomment}
\begin{eulerprompt}
>drop(10:20,3)
\end{eulerprompt}
\begin{euleroutput}
  [10,  11,  13,  14,  15,  16,  17,  18,  19,  20]
\end{euleroutput}
\begin{eulercomment}
Perhatikan bahwa vektor i di drop(v,i) mengacu pada indeks elemen di
v, bukan nilai elemen. Jika Anda ingin menghapus elemen, Anda harus
menemukan elemennya terlebih dahulu. Fungsi indexof(v,x) dapat
digunakan untuk mencari elemen x dalam vektor terurut v.
\end{eulercomment}
\begin{eulerprompt}
>v=primes(50), i=indexof(v,10:20), drop(v,i)
\end{eulerprompt}
\begin{euleroutput}
  [2,  3,  5,  7,  11,  13,  17,  19,  23,  29,  31,  37,  41,  43,  47]
  [0,  5,  0,  6,  0,  0,  0,  7,  0,  8,  0]
  [2,  3,  5,  7,  23,  29,  31,  37,  41,  43,  47]
\end{euleroutput}
\begin{eulercomment}
Seperti yang Anda lihat, tidak ada salahnya untuk memasukkan indeks di
luar rentang (seperti 0), indeks ganda, atau indeks yang tidak
diurutkan.
\end{eulercomment}
\begin{eulerprompt}
>drop(1:10,shuffle([0,0,5,5,7,12,12]))
\end{eulerprompt}
\begin{euleroutput}
  [1,  2,  3,  4,  6,  8,  9,  10]
\end{euleroutput}
\begin{eulercomment}
Ada beberapa fungsi khusus untuk mengatur diagonal atau untuk
menghasilkan matriks diagonal.

Kita mulai dengan matriks identitas.
\end{eulercomment}
\begin{eulerprompt}
>A=id(5) // matriks identitas 5x5
\end{eulerprompt}
\begin{euleroutput}
              1             0             0             0             0 
              0             1             0             0             0 
              0             0             1             0             0 
              0             0             0             1             0 
              0             0             0             0             1 
\end{euleroutput}
\begin{eulercomment}
Kemudian kita atur diagonal bawah (-1) menjadi 1:4.
\end{eulercomment}
\begin{eulerprompt}
>setdiag(A,-1,1:4) //mengganti diagonal di bawah diagonal utama
\end{eulerprompt}
\begin{euleroutput}
              1             0             0             0             0 
              1             1             0             0             0 
              0             2             1             0             0 
              0             0             3             1             0 
              0             0             0             4             1 
\end{euleroutput}
\begin{eulercomment}
Perhatikan bahwa kami tidak mengubah matriks A. Kami mendapatkan
matriks baru sebagai hasil dari setdiag().

Berikut adalah fungsi, yang mengembalikan matriks tri-diagonal.
\end{eulercomment}
\begin{eulerprompt}
>function tridiag (n,a,b,c) := setdiag(setdiag(b*id(n),1,c),-1,a); ...
>tridiag(5,1,2,3)
\end{eulerprompt}
\begin{euleroutput}
              2             3             0             0             0 
              1             2             3             0             0 
              0             1             2             3             0 
              0             0             1             2             3 
              0             0             0             1             2 
\end{euleroutput}
\begin{eulercomment}
Diagonal suatu matriks juga dapat diekstraksi dari matriks tersebut.
Untuk mendemonstrasikan ini, kami merestrukturisasi vektor 1:9 menjadi
matriks 3x3.
\end{eulercomment}
\begin{eulerprompt}
>A=redim(1:9,3,3)
\end{eulerprompt}
\begin{euleroutput}
              1             2             3 
              4             5             6 
              7             8             9 
\end{euleroutput}
\begin{eulercomment}
Sekarang kita dapat mengekstrak diagonal.
\end{eulercomment}
\begin{eulerprompt}
>d=getdiag(A,0)
\end{eulerprompt}
\begin{euleroutput}
  [1,  5,  9]
\end{euleroutput}
\begin{eulercomment}
Misalnya. Kita dapat membagi matriks dengan diagonalnya. Bahasa
matriks memperhatikan bahwa vektor kolom d diterapkan ke matriks baris
demi baris.
\end{eulercomment}
\begin{eulerprompt}
>fraction A/d'
\end{eulerprompt}
\begin{euleroutput}
          1         2         3 
        4/5         1       6/5 
        7/9       8/9         1 
\end{euleroutput}
\eulerheading{Vektorisasi}
\begin{eulercomment}
Hampir semua fungsi di Euler juga berfungsi untuk input matriks dan
vektor, kapan pun ini masuk akal.

Misalnya, fungsi sqrt() menghitung akar kuadrat dari semua elemen
vektor atau matriks.
\end{eulercomment}
\begin{eulerprompt}
>sqrt(1:3)
\end{eulerprompt}
\begin{euleroutput}
  [1,  1.41421,  1.73205]
\end{euleroutput}
\begin{eulercomment}
Jadi Anda dapat dengan mudah membuat tabel nilai. Ini adalah salah
satu cara untuk memplot suatu fungsi (alternatifnya menggunakan
ekspresi).
\end{eulercomment}
\begin{eulerprompt}
>x=1:0.01:5; y=log(x)/x^2; // terlalu panjang untuk ditampikan
\end{eulerprompt}
\begin{eulercomment}
Dengan ini dan operator titik dua a:delta:b, vektor nilai fungsi dapat
dihasilkan dengan mudah.

Pada contoh berikut, kita membangkitkan vektor nilai t[i] dengan spasi
0,1 dari -1 hingga 1. Kemudian kita membangkitkan vektor nilai fungsi

lateks: s = t\textasciicircum{}3-t
\end{eulercomment}
\begin{eulerprompt}
>t=-1:0.1:1; s=t^3-t
\end{eulerprompt}
\begin{euleroutput}
  [0,  0.171,  0.288,  0.357,  0.384,  0.375,  0.336,  0.273,  0.192,
  0.099,  0,  -0.099,  -0.192,  -0.273,  -0.336,  -0.375,  -0.384,
  -0.357,  -0.288,  -0.171,  0]
\end{euleroutput}
\begin{eulercomment}
EMT memperluas operator untuk skalar, vektor, dan matriks dengan cara
yang jelas.

Misalnya, vektor kolom dikalikan vektor baris menjadi matriks, jika
operator diterapkan. Berikut ini, v' adalah vektor yang
ditransposisikan (vektor kolom).
\end{eulercomment}
\begin{eulerprompt}
>shortest (1:5)*(1:5)'
\end{eulerprompt}
\begin{euleroutput}
       1      2      3      4      5 
       2      4      6      8     10 
       3      6      9     12     15 
       4      8     12     16     20 
       5     10     15     20     25 
\end{euleroutput}
\begin{eulercomment}
Perhatikan, bahwa ini sangat berbeda dari produk matriks. Produk
matriks dilambangkan dengan titik "." di EMT.
\end{eulercomment}
\begin{eulerprompt}
>(1:5).(1:5)'
\end{eulerprompt}
\begin{euleroutput}
  55
\end{euleroutput}
\begin{eulercomment}
Secara default, vektor baris dicetak dalam format yang ringkas.
\end{eulercomment}
\begin{eulerprompt}
>[1,2,3,4]
\end{eulerprompt}
\begin{euleroutput}
  [1,  2,  3,  4]
\end{euleroutput}
\begin{eulercomment}
Untuk matriks operator khusus . menunjukkan perkalian matriks, dan A'
menunjukkan transpos. Matriks 1x1 dapat digunakan seperti bilangan
real.
\end{eulercomment}
\begin{eulerprompt}
>v:=[1,2]; v.v', %^2
\end{eulerprompt}
\begin{euleroutput}
  5
  25
\end{euleroutput}
\begin{eulercomment}
Untuk mentranspos matriks kita menggunakan apostrof.
\end{eulercomment}
\begin{eulerprompt}
>v=1:4; v'
\end{eulerprompt}
\begin{euleroutput}
              1 
              2 
              3 
              4 
\end{euleroutput}
\begin{eulercomment}
Jadi kita dapat menghitung matriks A kali vektor b.
\end{eulercomment}
\begin{eulerprompt}
>A=[1,2,3,4;5,6,7,8]; A.v'
\end{eulerprompt}
\begin{euleroutput}
             30 
             70 
\end{euleroutput}
\begin{eulercomment}
Perhatikan bahwa v masih merupakan vektor baris. Jadi v'.v berbeda
dari v.v'.
\end{eulercomment}
\begin{eulerprompt}
>v'.v
\end{eulerprompt}
\begin{euleroutput}
              1             2             3             4 
              2             4             6             8 
              3             6             9            12 
              4             8            12            16 
\end{euleroutput}
\begin{eulercomment}
v.v' menghitung norma v kuadrat untuk vektor baris v. Hasilnya adalah
vektor 1x1, yang bekerja seperti bilangan real.
\end{eulercomment}
\begin{eulerprompt}
>v.v'
\end{eulerprompt}
\begin{euleroutput}
  30
\end{euleroutput}
\begin{eulercomment}
Ada juga fungsi norma (bersama dengan banyak fungsi lain dari Aljabar
Linier).
\end{eulercomment}
\begin{eulerprompt}
>norm(v)^2
\end{eulerprompt}
\begin{euleroutput}
  30
\end{euleroutput}
\begin{eulercomment}
Operator dan fungsi mematuhi bahasa matriks Euler.

Berikut ringkasan aturannya.

- Fungsi yang diterapkan ke vektor atau matriks diterapkan ke setiap
elemen.

- Operator yang beroperasi pada dua matriks dengan ukuran yang sama
diterapkan berpasangan ke elemen matriks.

- Jika kedua matriks memiliki dimensi yang berbeda, keduanya diperluas
dengan cara yang masuk akal, sehingga memiliki ukuran yang sama.

Misalnya, nilai skalar kali vektor mengalikan nilai dengan setiap
elemen vektor. Atau matriks kali vektor (dengan *, bukan .) memperluas
vektor ke ukuran matriks dengan menduplikasinya.

Berikut ini adalah kasus sederhana dengan operator \textasciicircum{}.
\end{eulercomment}
\begin{eulerprompt}
>[1,2,3]^2
\end{eulerprompt}
\begin{euleroutput}
  [1,  4,  9]
\end{euleroutput}
\begin{eulercomment}
Berikut adalah kasus yang lebih rumit. Vektor baris dikalikan dengan
vektor kolom mengembang keduanya dengan menduplikasi.
\end{eulercomment}
\begin{eulerprompt}
>v:=[1,2,3]; v*v'
\end{eulerprompt}
\begin{euleroutput}
              1             2             3 
              2             4             6 
              3             6             9 
\end{euleroutput}
\begin{eulercomment}
Perhatikan bahwa produk skalar menggunakan produk matriks, bukan *!
\end{eulercomment}
\begin{eulerprompt}
>v.v'
\end{eulerprompt}
\begin{euleroutput}
  14
\end{euleroutput}
\begin{eulercomment}
Ada banyak fungsi matriks. Kami memberikan daftar singkat. Anda harus
berkonsultasi dengan dokumentasi untuk informasi lebih lanjut tentang
perintah ini.

\end{eulercomment}
\begin{eulerttcomment}
   sum,prod menghitung jumlah dan produk dari baris
   cumsum,cumprod melakukan hal yang sama secara kumulatif
   menghitung nilai ekstrem dari setiap baris
   extrema mengembalikan vektor dengan informasi ekstrim
   diag(A,i) mengembalikan diagonal ke-i
   setdiag(A,i,v) mengatur diagonal ke-i
   id(n) matriks identitas
   det(A) penentu
   charpoly(A) polinomial karakteristik
   nilai eigen(A) nilai eigen
\end{eulerttcomment}
\begin{eulerprompt}
>v*v, sum(v*v), cumsum(v*v)
\end{eulerprompt}
\begin{euleroutput}
  [1,  4,  9]
  14
  [1,  5,  14]
\end{euleroutput}
\begin{eulercomment}
Operator : menghasilkan vektor baris spasi yang sama, opsional dengan
ukuran langkah.
\end{eulercomment}
\begin{eulerprompt}
>1:4, 1:2:10
\end{eulerprompt}
\begin{euleroutput}
  [1,  2,  3,  4]
  [1,  3,  5,  7,  9]
\end{euleroutput}
\begin{eulercomment}
Untuk menggabungkan matriks dan vektor ada operator "\textbar{}" dan "\_".
\end{eulercomment}
\begin{eulerprompt}
>[1,2,3]|[4,5], [1,2,3]_1
\end{eulerprompt}
\begin{euleroutput}
  [1,  2,  3,  4,  5]
              1             2             3 
              1             1             1 
\end{euleroutput}
\begin{eulercomment}
Unsur-unsur matriks disebut dengan "A[i,j]".
\end{eulercomment}
\begin{eulerprompt}
>A:=[1,2,3;4,5,6;7,8,9]; A[2,3]
\end{eulerprompt}
\begin{euleroutput}
  6
\end{euleroutput}
\begin{eulercomment}
Untuk vektor baris atau kolom, v[i] adalah elemen ke-i dari vektor.
Untuk matriks, ini mengembalikan baris ke-i lengkap dari matriks.
\end{eulercomment}
\begin{eulerprompt}
>v:=[2,4,6,8]; v[3], A[3]
\end{eulerprompt}
\begin{euleroutput}
  6
  [7,  8,  9]
\end{euleroutput}
\begin{eulercomment}
Indeks juga bisa menjadi vektor baris dari indeks. : menunjukkan semua
indeks.
\end{eulercomment}
\begin{eulerprompt}
>v[1:2], A[:,2]
\end{eulerprompt}
\begin{euleroutput}
  [2,  4]
              2 
              5 
              8 
\end{euleroutput}
\begin{eulercomment}
Bentuk singkat untuk : adalah menghilangkan indeks sepenuhnya.
\end{eulercomment}
\begin{eulerprompt}
>A[,2:3]
\end{eulerprompt}
\begin{euleroutput}
              2             3 
              5             6 
              8             9 
\end{euleroutput}
\begin{eulercomment}
Untuk tujuan vektorisasi, elemen matriks dapat diakses seolah-olah
mereka adalah vektor.
\end{eulercomment}
\begin{eulerprompt}
>A\{4\}
\end{eulerprompt}
\begin{euleroutput}
  4
\end{euleroutput}
\begin{eulercomment}
Matriks juga dapat diratakan, menggunakan fungsi redim(). Ini
diimplementasikan dalam fungsi flatten().
\end{eulercomment}
\begin{eulerprompt}
>redim(A,1,prod(size(A))), flatten(A)
\end{eulerprompt}
\begin{euleroutput}
  [1,  2,  3,  4,  5,  6,  7,  8,  9]
  [1,  2,  3,  4,  5,  6,  7,  8,  9]
\end{euleroutput}
\begin{eulercomment}
Untuk menggunakan matriks untuk tabel, mari kita reset ke format
default, dan menghitung tabel nilai sinus dan kosinus. Perhatikan
bahwa sudut dalam radian secara default.
\end{eulercomment}
\begin{eulerprompt}
>defformat; w=0°:45°:360°; w=w'; deg(w)
\end{eulerprompt}
\begin{euleroutput}
              0 
             45 
             90 
            135 
            180 
            225 
            270 
            315 
            360 
\end{euleroutput}
\begin{eulercomment}
Sekarang kita menambahkan kolom ke matriks.
\end{eulercomment}
\begin{eulerprompt}
>M = deg(w)|w|cos(w)|sin(w)
\end{eulerprompt}
\begin{euleroutput}
              0             0             1             0 
             45      0.785398      0.707107      0.707107 
             90        1.5708             0             1 
            135       2.35619     -0.707107      0.707107 
            180       3.14159            -1             0 
            225       3.92699     -0.707107     -0.707107 
            270       4.71239             0            -1 
            315       5.49779      0.707107     -0.707107 
            360       6.28319             1             0 
\end{euleroutput}
\begin{eulercomment}
Dengan menggunakan bahasa matriks, kita dapat menghasilkan beberapa
tabel dari beberapa fungsi sekaligus.

Dalam contoh berikut, kita menghitung t[j]\textasciicircum{}i untuk i dari 1 hingga n.
Kami mendapatkan matriks, di mana setiap baris adalah tabel t\textasciicircum{}i untuk
satu i. Yaitu, matriks memiliki elemen lateks: a\_\{i,j\} = t\_j\textasciicircum{}i, \textbackslash{}quad
1 \textbackslash{}le j \textbackslash{}le 101, \textbackslash{}quad 1 \textbackslash{}le i \textbackslash{}le n

Fungsi yang tidak berfungsi untuk input vektor harus "divektorkan".
Ini dapat dicapai dengan kata kunci "peta" dalam definisi fungsi.
Kemudian fungsi tersebut akan dievaluasi untuk setiap elemen dari
parameter vektor.

Integrasi numerik terintegrasi() hanya berfungsi untuk batas interval
skalar. Jadi kita perlu membuat vektor.
\end{eulercomment}
\begin{eulerprompt}
>function map f(x) := integrate("x^x",1,x)
\end{eulerprompt}
\begin{eulercomment}
Kata kunci "peta" membuat vektor fungsi. Fungsinya sekarang akan
bekerja\\
untuk vektor bilangan.
\end{eulercomment}
\begin{eulerprompt}
>f([1:5])
\end{eulerprompt}
\begin{euleroutput}
  [0,  2.05045,  13.7251,  113.336,  1241.03]
\end{euleroutput}
\eulerheading{Sub-Matriks dan Matriks-Elemen}
\begin{eulercomment}
Untuk mengakses elemen matriks, gunakan notasi braket.
\end{eulercomment}
\begin{eulerprompt}
>A=[1,2,3;4,5,6;7,8,9], A[2,2]
\end{eulerprompt}
\begin{euleroutput}
              1             2             3 
              4             5             6 
              7             8             9 
  5
\end{euleroutput}
\begin{eulercomment}
Kita dapat mengakses satu baris matriks yang lengkap.
\end{eulercomment}
\begin{eulerprompt}
>A[2]
\end{eulerprompt}
\begin{euleroutput}
  [4,  5,  6]
\end{euleroutput}
\begin{eulercomment}
Dalam kasus vektor baris atau kolom, ini mengembalikan elemen vektor.
\end{eulercomment}
\begin{eulerprompt}
>v=1:3; v[2]
\end{eulerprompt}
\begin{euleroutput}
  2
\end{euleroutput}
\begin{eulercomment}
Untuk memastikan, Anda mendapatkan baris pertama untuk matriks 1xn dan
mxn, tentukan semua kolom menggunakan indeks kedua kosong.
\end{eulercomment}
\begin{eulerprompt}
>A[2,]
\end{eulerprompt}
\begin{euleroutput}
  [4,  5,  6]
\end{euleroutput}
\begin{eulercomment}
Jika indeks adalah vektor indeks, Euler akan mengembalikan baris
matriks yang sesuai.

Di sini kita ingin baris pertama dan kedua dari A.
\end{eulercomment}
\begin{eulerprompt}
>A[[1,2]]
\end{eulerprompt}
\begin{euleroutput}
              1             2             3 
              4             5             6 
\end{euleroutput}
\begin{eulercomment}
Kita bahkan dapat menyusun ulang A menggunakan vektor indeks.
Tepatnya, kami tidak mengubah A di sini, tetapi menghitung versi A
yang disusun ulang.
\end{eulercomment}
\begin{eulerprompt}
>A[[3,2,1]]
\end{eulerprompt}
\begin{euleroutput}
              7             8             9 
              4             5             6 
              1             2             3 
\end{euleroutput}
\begin{eulercomment}
Trik indeks bekerja dengan kolom juga.

Contoh ini memilih semua baris A dan kolom kedua dan ketiga.
\end{eulercomment}
\begin{eulerprompt}
>A[1:3,2:3]
\end{eulerprompt}
\begin{euleroutput}
              2             3 
              5             6 
              8             9 
\end{euleroutput}
\begin{eulercomment}
Untuk singkatan ":" menunjukkan semua indeks baris atau kolom.
\end{eulercomment}
\begin{eulerprompt}
>A[:,3]
\end{eulerprompt}
\begin{euleroutput}
              3 
              6 
              9 
\end{euleroutput}
\begin{eulercomment}
Atau, biarkan indeks pertama kosong.
\end{eulercomment}
\begin{eulerprompt}
>A[,2:3]
\end{eulerprompt}
\begin{euleroutput}
              2             3 
              5             6 
              8             9 
\end{euleroutput}
\begin{eulercomment}
Kita juga bisa mendapatkan baris terakhir dari A.
\end{eulercomment}
\begin{eulerprompt}
>A[-1]
\end{eulerprompt}
\begin{euleroutput}
  [7,  8,  9]
\end{euleroutput}
\begin{eulercomment}
Sekarang mari kita ubah elemen A dengan menetapkan submatriks A ke
beberapa nilai. Ini sebenarnya mengubah matriks A yang disimpan.
\end{eulercomment}
\begin{eulerprompt}
>A[1,1]=4
\end{eulerprompt}
\begin{euleroutput}
              4             2             3 
              4             5             6 
              7             8             9 
\end{euleroutput}
\begin{eulercomment}
Kami juga dapat menetapkan nilai ke baris A.
\end{eulercomment}
\begin{eulerprompt}
>A[1]=[-1,-1,-1]
\end{eulerprompt}
\begin{euleroutput}
             -1            -1            -1 
              4             5             6 
              7             8             9 
\end{euleroutput}
\begin{eulercomment}
Kami bahkan dapat menetapkan sub-matriks jika memiliki ukuran yang
tepat.
\end{eulercomment}
\begin{eulerprompt}
>A[1:2,1:2]=[5,6;7,8]
\end{eulerprompt}
\begin{euleroutput}
              5             6            -1 
              7             8             6 
              7             8             9 
\end{euleroutput}
\begin{eulercomment}
Selain itu, beberapa jalan pintas diperbolehkan.
\end{eulercomment}
\begin{eulerprompt}
>A[1:2,1:2]=0
\end{eulerprompt}
\begin{euleroutput}
              0             0            -1 
              0             0             6 
              7             8             9 
\end{euleroutput}
\begin{eulercomment}
Peringatan: Indeks di luar batas mengembalikan matriks kosong, atau
pesan kesalahan, tergantung pada pengaturan sistem. Standarnya adalah
pesan kesalahan. Ingat, bagaimanapun, bahwa indeks negatif dapat
digunakan untuk mengakses elemen matriks yang dihitung dari akhir.
\end{eulercomment}
\begin{eulerprompt}
>A[4]
\end{eulerprompt}
\begin{euleroutput}
  Row index 4 out of bounds!
  Error in:
  A[4] ...
      ^
\end{euleroutput}
\eulerheading{Menyortir dan Mengacak}
\begin{eulercomment}
Fungsi sort() mengurutkan vektor baris.
\end{eulercomment}
\begin{eulerprompt}
>sort([5,6,4,8,1,9])
\end{eulerprompt}
\begin{euleroutput}
  [1,  4,  5,  6,  8,  9]
\end{euleroutput}
\begin{eulercomment}
Seringkali perlu untuk mengetahui indeks dari vektor yang diurutkan
dalam vektor aslinya. Ini dapat digunakan untuk menyusun ulang vektor
lain dengan cara yang sama.

Mari kita mengocok vektor.
\end{eulercomment}
\begin{eulerprompt}
>v=shuffle(1:10)
\end{eulerprompt}
\begin{euleroutput}
  [4,  5,  10,  6,  8,  9,  1,  7,  2,  3]
\end{euleroutput}
\begin{eulercomment}
Indeks berisi urutan yang tepat dari v.
\end{eulercomment}
\begin{eulerprompt}
>\{vs,ind\}=sort(v); v[ind]
\end{eulerprompt}
\begin{euleroutput}
  [1,  2,  3,  4,  5,  6,  7,  8,  9,  10]
\end{euleroutput}
\begin{eulercomment}
Ini bekerja untuk vektor string juga.
\end{eulercomment}
\begin{eulerprompt}
>s=["a","d","e","a","aa","e"]
\end{eulerprompt}
\begin{euleroutput}
  a
  d
  e
  a
  aa
  e
\end{euleroutput}
\begin{eulerprompt}
>\{ss,ind\}=sort(s); ss
\end{eulerprompt}
\begin{euleroutput}
  a
  a
  aa
  d
  e
  e
\end{euleroutput}
\begin{eulercomment}
Seperti yang Anda lihat, posisi entri ganda agak acak.
\end{eulercomment}
\begin{eulerprompt}
>ind
\end{eulerprompt}
\begin{euleroutput}
  [4,  1,  5,  2,  6,  3]
\end{euleroutput}
\begin{eulercomment}
Fungsi unik mengembalikan daftar elemen unik vektor yang diurutkan.
\end{eulercomment}
\begin{eulerprompt}
>intrandom(1,10,10), unique(%)
\end{eulerprompt}
\begin{euleroutput}
  [4,  4,  9,  2,  6,  5,  10,  6,  5,  1]
  [1,  2,  4,  5,  6,  9,  10]
\end{euleroutput}
\begin{eulercomment}
Ini bekerja untuk vektor string juga.
\end{eulercomment}
\begin{eulerprompt}
>unique(s)
\end{eulerprompt}
\begin{euleroutput}
  a
  aa
  d
  e
\end{euleroutput}
\eulerheading{Aljabar linier}
\begin{eulercomment}
EMT memiliki banyak fungsi untuk menyelesaikan sistem linier, sistem
sparse, atau masalah regresi.

Untuk sistem linier Ax=b, Anda dapat menggunakan algoritma Gauss,
matriks invers atau kecocokan linier. Operator A\textbackslash{}b menggunakan versi
algoritma Gauss.
\end{eulercomment}
\begin{eulerprompt}
>A=[1,2;3,4]; b=[5;6]; A\(\backslash\)b
\end{eulerprompt}
\begin{euleroutput}
             -4 
            4.5 
\end{euleroutput}
\begin{eulercomment}
Untuk contoh lain, kami membuat matriks 200x200 dan jumlah barisnya.
Kemudian kita selesaikan Ax=b menggunakan matriks invers. Kami
mengukur kesalahan sebagai deviasi maksimal semua elemen dari 1, yang
tentu saja merupakan solusi yang benar.
\end{eulercomment}
\begin{eulerprompt}
>A=normal(200,200); b=sum(A); longest totalmax(abs(inv(A).b-1))
\end{eulerprompt}
\begin{euleroutput}
    8.790745908981989e-13 
\end{euleroutput}
\begin{eulercomment}
Jika sistem tidak memiliki solusi, kecocokan linier meminimalkan norma
kesalahan Ax-b.
\end{eulercomment}
\begin{eulerprompt}
>A=[1,2,3;4,5,6;7,8,9]
\end{eulerprompt}
\begin{euleroutput}
              1             2             3 
              4             5             6 
              7             8             9 
\end{euleroutput}
\begin{eulercomment}
Determinan matriks ini adalah 0.
\end{eulercomment}
\begin{eulerprompt}
>det(A)
\end{eulerprompt}
\begin{euleroutput}
  0
\end{euleroutput}
\eulerheading{Matriks Simbolik}
\begin{eulercomment}
Maxima memiliki matriks simbolis. Tentu saja, Maxima dapat digunakan
untuk masalah aljabar linier sederhana seperti itu. Kita dapat
mendefinisikan matriks untuk Euler dan Maxima dengan \&:=, dan kemudian
menggunakannya dalam ekspresi simbolis. Bentuk [...] biasa untuk
mendefinisikan matriks dapat digunakan di Euler untuk mendefinisikan
matriks simbolik.
\end{eulercomment}
\begin{eulerprompt}
>A &= [a,1,1;1,a,1;1,1,a]; $A
>$&det(A), $&factor(%)
>$&invert(A) with a=0
>A &= [1,a;b,2]; $A
\end{eulerprompt}
\begin{eulercomment}
Seperti semua variabel simbolik, matriks ini dapat digunakan dalam
ekspresi simbolik lainnya.
\end{eulercomment}
\begin{eulerprompt}
>$&det(A-x*ident(2)), $&solve(%,x)
\end{eulerprompt}
\begin{eulercomment}
Nilai eigen juga dapat dihitung secara otomatis. Hasilnya adalah
vektor dengan dua vektor nilai eigen dan multiplisitas.
\end{eulercomment}
\begin{eulerprompt}
>$&eigenvalues([a,1;1,a])
\end{eulerprompt}
\begin{eulercomment}
Untuk mengekstrak vektor eigen tertentu perlu pengindeksan yang
cermat.
\end{eulercomment}
\begin{eulerprompt}
>$&eigenvectors([a,1;1,a]), &%[2][1][1]
\end{eulerprompt}
\begin{euleroutput}
  
                                 [1, - 1]
  
\end{euleroutput}
\begin{eulercomment}
Matriks simbolik dapat dievaluasi dalam Euler secara numerik seperti
ekspresi simbolik lainnya.
\end{eulercomment}
\begin{eulerprompt}
>A(a=4,b=5)
\end{eulerprompt}
\begin{euleroutput}
              1             4 
              5             2 
\end{euleroutput}
\begin{eulercomment}
Dalam ekspresi simbolik, gunakan dengan.
\end{eulercomment}
\begin{eulerprompt}
>$&A with [a=4,b=5]
\end{eulerprompt}
\begin{eulercomment}
Akses ke baris matriks simbolik bekerja seperti halnya dengan matriks
numerik.
\end{eulercomment}
\begin{eulerprompt}
>$&A[1]
\end{eulerprompt}
\begin{eulercomment}
Ekspresi simbolis dapat berisi tugas. Dan itu mengubah matriks A.
\end{eulercomment}
\begin{eulerprompt}
>&A[1,1]:=t+1; $&A
\end{eulerprompt}
\begin{eulercomment}
Ada fungsi simbolik di Maxima untuk membuat vektor dan matriks. Untuk
ini, lihat dokumentasi Maxima atau tutorial tentang Maxima di EMT.
\end{eulercomment}
\begin{eulerprompt}
>v &= makelist(1/(i+j),i,1,3); $v
\end{eulerprompt}
\begin{eulerttcomment}
 
\end{eulerttcomment}
\begin{eulerprompt}
>B &:= [1,2;3,4]; $B, $&invert(B)
\end{eulerprompt}
\begin{eulercomment}
Hasilnya dapat dievaluasi secara numerik dalam Euler. Untuk informasi
lebih lanjut tentang Maxima, lihat pengantar Maxima.
\end{eulercomment}
\begin{eulerprompt}
>$&invert(B)()
\end{eulerprompt}
\begin{euleroutput}
             -2             1 
            1.5          -0.5 
\end{euleroutput}
\begin{eulercomment}
Euler juga memiliki fungsi xinv() yang kuat, yang membuat upaya lebih
besar dan mendapatkan hasil yang lebih tepat.

Perhatikan, bahwa dengan \&:= matriks B telah didefinisikan sebagai
simbolik dalam ekspresi simbolik dan sebagai numerik dalam ekspresi
numerik. Jadi kita bisa menggunakannya di sini.
\end{eulercomment}
\begin{eulerprompt}
>longest B.xinv(B)
\end{eulerprompt}
\begin{euleroutput}
                        1                       0 
                        0                       1 
\end{euleroutput}
\begin{eulercomment}
Misalnya. nilai eigen dari A dapat dihitung secara numerik.
\end{eulercomment}
\begin{eulerprompt}
>A=[1,2,3;4,5,6;7,8,9]; real(eigenvalues(A))
\end{eulerprompt}
\begin{euleroutput}
  [16.1168,  -1.11684,  0]
\end{euleroutput}
\begin{eulercomment}
Atau secara simbolis. Lihat tutorial tentang Maxima untuk detailnya.
\end{eulercomment}
\begin{eulerprompt}
>$&eigenvalues(@A)
\end{eulerprompt}
\eulerheading{Nilai Numerik dalam Ekspresi simbolis}
\begin{eulercomment}
Ekspresi simbolis hanyalah string yang berisi ekspresi. Jika kita
ingin mendefinisikan nilai baik untuk ekspresi simbolik maupun
ekspresi numerik, kita harus menggunakan "\&:=".
\end{eulercomment}
\begin{eulerprompt}
>A &:= [1,pi;4,5]
\end{eulerprompt}
\begin{euleroutput}
              1       3.14159 
              4             5 
\end{euleroutput}
\begin{eulercomment}
Masih ada perbedaan antara bentuk numerik dan simbolik. Saat
mentransfer matriks ke bentuk simbolis, pendekatan fraksional untuk
real akan digunakan.
\end{eulercomment}
\begin{eulerprompt}
>$&A
\end{eulerprompt}
\begin{eulercomment}
Untuk menghindarinya, ada fungsi "mxmset(variable)".
\end{eulercomment}
\begin{eulerprompt}
>mxmset(A); $&A
\end{eulerprompt}
\begin{eulercomment}
Maxima juga dapat menghitung dengan angka floating point, dan bahkan
dengan angka floating besar dengan 32 digit. Namun, evaluasinya jauh
lebih lambat.
\end{eulercomment}
\begin{eulerprompt}
>$&bfloat(sqrt(2)), $&float(sqrt(2))
\end{eulerprompt}
\begin{eulercomment}
Ketepatan angka floating point besar dapat diubah.
\end{eulercomment}
\begin{eulerprompt}
>&fpprec:=100; &bfloat(pi)
\end{eulerprompt}
\begin{euleroutput}
  
          3.14159265358979323846264338327950288419716939937510582097494\(\backslash\)
  4592307816406286208998628034825342117068b0
  
\end{euleroutput}
\begin{eulercomment}
Variabel numerik dapat digunakan dalam ekspresi simbolis apa pun
menggunakan "@var".

Perhatikan bahwa ini hanya diperlukan, jika variabel telah
didefinisikan dengan ":=" atau "=" sebagai variabel numerik.
\end{eulercomment}
\begin{eulerprompt}
>B:=[1,pi;3,4]; $&det(@B)
\end{eulerprompt}
\eulerheading{Demo - Suku Bunga}
\begin{eulercomment}
Di bawah ini, kami menggunakan Euler Math Toolbox (EMT) untuk
perhitungan suku bunga. Kami melakukannya secara numerik dan simbolis
untuk menunjukkan kepada Anda bagaimana Euler dapat digunakan untuk
memecahkan masalah kehidupan nyata.

Asumsikan Anda memiliki modal awal 5000 (katakanlah dalam dolar).
\end{eulercomment}
\begin{eulerprompt}
>K=5000
\end{eulerprompt}
\begin{euleroutput}
  5000
\end{euleroutput}
\begin{eulercomment}
Sekarang kita asumsikan tingkat bunga 3\% per tahun. Mari kita
tambahkan satu tarif sederhana dan hitung hasilnya.
\end{eulercomment}
\begin{eulerprompt}
>K*1.03
\end{eulerprompt}
\begin{euleroutput}
  5150
\end{euleroutput}
\begin{eulercomment}
Euler akan memahami sintaks berikut juga.
\end{eulercomment}
\begin{eulerprompt}
>K+K*3%
\end{eulerprompt}
\begin{euleroutput}
  5150
\end{euleroutput}
\begin{eulercomment}
Tetapi lebih mudah menggunakan faktornya
\end{eulercomment}
\begin{eulerprompt}
>q=1+3%, K*q
\end{eulerprompt}
\begin{euleroutput}
  1.03
  5150
\end{euleroutput}
\begin{eulercomment}
Selama 10 tahun, kita cukup mengalikan faktornya dan mendapatkan nilai
akhir dengan suku bunga majemuk.
\end{eulercomment}
\begin{eulerprompt}
>K*q^10
\end{eulerprompt}
\begin{euleroutput}
  6719.58189672
\end{euleroutput}
\begin{eulercomment}
Untuk tujuan kita, kita dapat mengatur format menjadi 2 digit setelah
titik desimal.
\end{eulercomment}
\begin{eulerprompt}
>format(12,2); K*q^10
\end{eulerprompt}
\begin{euleroutput}
      6719.58 
\end{euleroutput}
\begin{eulercomment}
Mari kita cetak yang dibulatkan menjadi 2 digit dalam kalimat lengkap.
\end{eulercomment}
\begin{eulerprompt}
>"Starting from " + K + "$ you get " + round(K*q^10,2) + "$."
\end{eulerprompt}
\begin{euleroutput}
  Starting from 5000$ you get 6719.58$.
\end{euleroutput}
\begin{eulercomment}
Bagaimana jika kita ingin mengetahui hasil antara dari tahun 1 sampai
tahun 9? Untuk ini, bahasa matriks Euler sangat membantu. Anda tidak
harus menulis loop, tetapi cukup masukkan
\end{eulercomment}
\begin{eulerprompt}
>K*q^(0:10)
\end{eulerprompt}
\begin{euleroutput}
  Real 1 x 11 matrix
  
      5000.00     5150.00     5304.50     5463.64     ...
\end{euleroutput}
\begin{eulercomment}
Bagaimana keajaiban ini bekerja? Pertama ekspresi 0:10 mengembalikan
vektor bilangan bulat.
\end{eulercomment}
\begin{eulerprompt}
>short 0:10
\end{eulerprompt}
\begin{euleroutput}
  [0,  1,  2,  3,  4,  5,  6,  7,  8,  9,  10]
\end{euleroutput}
\begin{eulercomment}
Kemudian semua operator dan fungsi dalam Euler dapat diterapkan pada
elemen vektor untuk elemen. Jadi
\end{eulercomment}
\begin{eulerprompt}
>short q^(0:10)
\end{eulerprompt}
\begin{euleroutput}
  [1,  1.03,  1.0609,  1.0927,  1.1255,  1.1593,  1.1941,  1.2299,
  1.2668,  1.3048,  1.3439]
\end{euleroutput}
\begin{eulercomment}
adalah vektor faktor q\textasciicircum{}0 sampai q\textasciicircum{}10. Ini dikalikan dengan K, dan kami
mendapatkan vektor nilai.
\end{eulercomment}
\begin{eulerprompt}
>VK=K*q^(0:10);
\end{eulerprompt}
\begin{eulercomment}
Tentu saja, cara realistis untuk menghitung suku bunga ini adalah
dengan membulatkan ke sen terdekat setelah setiap tahun. Mari kita
tambahkan fungsi untuk ini.
\end{eulercomment}
\begin{eulerprompt}
>function oneyear (K) := round(K*q,2)
\end{eulerprompt}
\begin{eulercomment}
Mari kita bandingkan dua hasil, dengan dan tanpa pembulatan.
\end{eulercomment}
\begin{eulerprompt}
>longest oneyear(1234.57), longest 1234.57*q
\end{eulerprompt}
\begin{euleroutput}
                  1271.61 
                1271.6071 
\end{euleroutput}
\begin{eulercomment}
Sekarang tidak ada rumus sederhana untuk tahun ke-n, dan kita harus
mengulang selama bertahun-tahun. Euler memberikan banyak solusi untuk
ini.

Cara termudah adalah iterasi fungsi, yang mengulangi fungsi tertentu
beberapa kali.
\end{eulercomment}
\begin{eulerprompt}
>VKr=iterate("oneyear",5000,10)
\end{eulerprompt}
\begin{euleroutput}
  Real 1 x 11 matrix
  
      5000.00     5150.00     5304.50     5463.64     ...
\end{euleroutput}
\begin{eulercomment}
Kami dapat mencetaknya dengan cara yang ramah, menggunakan format kami
dengan tempat desimal tetap.
\end{eulercomment}
\begin{eulerprompt}
>VKr'
\end{eulerprompt}
\begin{euleroutput}
      5000.00 
      5150.00 
      5304.50 
      5463.64 
      5627.55 
      5796.38 
      5970.27 
      6149.38 
      6333.86 
      6523.88 
      6719.60 
\end{euleroutput}
\begin{eulercomment}
Untuk mendapatkan elemen tertentu dari vektor, kami menggunakan indeks
dalam tanda kurung siku.
\end{eulercomment}
\begin{eulerprompt}
>VKr[2], VKr[1:3]
\end{eulerprompt}
\begin{euleroutput}
      5150.00 
      5000.00     5150.00     5304.50 
\end{euleroutput}
\begin{eulercomment}
Anehnya, kita juga bisa menggunakan vektor indeks. Ingat bahwa 1:3
menghasilkan vektor [1,2,3].

Mari kita bandingkan elemen terakhir dari nilai yang dibulatkan dengan
nilai penuh.
\end{eulercomment}
\begin{eulerprompt}
>VKr[-1], VK[-1]
\end{eulerprompt}
\begin{euleroutput}
      6719.60 
      6719.58 
\end{euleroutput}
\begin{eulercomment}
Perbedaannya sangat kecil.

\begin{eulercomment}
\eulerheading{Memecahkan Persamaan}
\begin{eulercomment}
Sekarang kita mengambil fungsi yang lebih maju, yang menambahkan
tingkat uang tertentu setiap tahun.
\end{eulercomment}
\begin{eulerprompt}
>function onepay (K) := K*q+R
\end{eulerprompt}
\begin{eulercomment}
Kita tidak perlu menentukan q atau R untuk definisi fungsi. Hanya jika
kita menjalankan perintah, kita harus mendefinisikan nilai-nilai ini.
Kami memilih R=200.
\end{eulercomment}
\begin{eulerprompt}
>R=200; iterate("onepay",5000,10)
\end{eulerprompt}
\begin{euleroutput}
  Real 1 x 11 matrix
  
      5000.00     5350.00     5710.50     6081.82     ...
\end{euleroutput}
\begin{eulercomment}
Bagaimana jika kita menghapus jumlah yang sama setiap tahun?
\end{eulercomment}
\begin{eulerprompt}
>R=-200; iterate("onepay",5000,10)
\end{eulerprompt}
\begin{euleroutput}
  Real 1 x 11 matrix
  
      5000.00     4950.00     4898.50     4845.45     ...
\end{euleroutput}
\begin{eulercomment}
Kami melihat bahwa uang berkurang. Jelas, jika kita hanya mendapatkan
150 bunga di tahun pertama, tetapi menghapus 200, kita kehilangan uang
setiap tahun.

Bagaimana kita bisa menentukan berapa tahun uang itu akan bertahan?
Kita harus menulis loop untuk ini. Cara termudah adalah dengan iterasi
cukup lama.
\end{eulercomment}
\begin{eulerprompt}
>VKR=iterate("onepay",5000,50)
\end{eulerprompt}
\begin{euleroutput}
  Real 1 x 51 matrix
  
      5000.00     4950.00     4898.50     4845.45     ...
\end{euleroutput}
\begin{eulercomment}
Dengan menggunakan bahasa matriks, kita dapat menentukan nilai negatif
pertama dengan cara berikut.
\end{eulercomment}
\begin{eulerprompt}
>min(nonzeros(VKR<0))
\end{eulerprompt}
\begin{euleroutput}
        48.00 
\end{euleroutput}
\begin{eulercomment}
Alasan untuk ini adalah bahwa bukan nol(VKR\textless{}0) mengembalikan vektor
indeks i, di mana VKR[i]\textless{}0, dan min menghitung indeks minimal.

Karena vektor selalu dimulai dengan indeks 1, jawabannya adalah 47
tahun.

Fungsi iterate() memiliki satu trik lagi. Itu bisa mengambil kondisi
akhir sebagai argumen. Kemudian akan mengembalikan nilai dan jumlah
iterasi.
\end{eulercomment}
\begin{eulerprompt}
>\{x,n\}=iterate("onepay",5000,till="x<0"); x, n,
\end{eulerprompt}
\begin{euleroutput}
       -19.83 
        47.00 
\end{euleroutput}
\begin{eulercomment}
Mari kita coba menjawab pertanyaan yang lebih ambigu. Asumsikan kita
tahu bahwa nilainya adalah 0 setelah 50 tahun. Apa yang akan menjadi
tingkat bunga?

Ini adalah pertanyaan yang hanya bisa dijawab dengan angka. Di bawah
ini, kita akan mendapatkan formula yang diperlukan. Kemudian Anda akan
melihat bahwa tidak ada formula yang mudah untuk tingkat bunga. Tapi
untuk saat ini, kami bertujuan untuk solusi numerik.

Langkah pertama adalah mendefinisikan fungsi yang melakukan iterasi
sebanyak n kali. Kami menambahkan semua parameter ke fungsi ini.
\end{eulercomment}
\begin{eulerprompt}
>function f(K,R,P,n) := iterate("x*(1+P/100)+R",K,n;P,R)[-1]
\end{eulerprompt}
\begin{eulercomment}
Iterasinya sama seperti di atas

\end{eulercomment}
\begin{eulerformula}
\[
x_{n+1} = x_n \cdot \left(1+ \frac{P}{100}\kanan) + R
\]
\end{eulerformula}
\begin{eulercomment}
Tapi kami tidak lagi menggunakan nilai global R dalam ekspresi kami.
Fungsi seperti iterate() memiliki trik khusus di Euler. Anda dapat
meneruskan nilai variabel dalam ekspresi sebagai parameter titik koma.
Dalam hal ini P dan R.

Selain itu, kami hanya tertarik pada nilai terakhir. Jadi kita ambil
indeks [-1].

Mari kita coba tes.
\end{eulercomment}
\begin{eulerprompt}
>f(5000,-200,3,47)
\end{eulerprompt}
\begin{euleroutput}
       -19.83 
\end{euleroutput}
\begin{eulercomment}
Sekarang kita bisa menyelesaikan masalah kita.
\end{eulercomment}
\begin{eulerprompt}
>solve("f(5000,-200,x,50)",3)
\end{eulerprompt}
\begin{euleroutput}
         3.15 
\end{euleroutput}
\begin{eulercomment}
Rutin memecahkan memecahkan ekspresi=0 untuk variabel x. Jawabannya
adalah 3,15\% per tahun. Kami mengambil nilai awal 3\% untuk algoritma.
Fungsi solve() selalu membutuhkan nilai awal.

Kita dapat menggunakan fungsi yang sama untuk menyelesaikan pertanyaan
berikut: Berapa banyak yang dapat kita keluarkan per tahun sehingga
modal awal habis setelah 20 tahun dengan asumsi tingkat bunga 3\% per
tahun.
\end{eulercomment}
\begin{eulerprompt}
>solve("f(5000,x,3,20)",-200)
\end{eulerprompt}
\begin{euleroutput}
      -336.08 
\end{euleroutput}
\begin{eulercomment}
Perhatikan bahwa Anda tidak dapat memecahkan jumlah tahun, karena
fungsi kami mengasumsikan n sebagai nilai integer.

\end{eulercomment}
\eulersubheading{Solusi Simbolik untuk Masalah Suku Bunga}
\begin{eulercomment}
Kita dapat menggunakan bagian simbolik dari Euler untuk mempelajari
masalah tersebut. Pertama kita mendefinisikan fungsi onepay() kita
secara simbolis.
\end{eulercomment}
\begin{eulerprompt}
>function op(K) &= K*q+R; $&op(K)
\end{eulerprompt}
\begin{eulercomment}
Kita sekarang dapat mengulangi ini.
\end{eulercomment}
\begin{eulerprompt}
>$&op(op(op(op(K)))), $&expand(%)
\end{eulerprompt}
\begin{eulercomment}
Kami melihat sebuah pola. Setelah n periode yang kita miliki

lateks: K\_n = q\textasciicircum{}n K + R (1+q+\textbackslash{}ldots+q\textasciicircum{}\{n-1\}) = q\textasciicircum{}n K +
\textbackslash{}frac\{q\textasciicircum{}n-1\}\{q-1\} R

Rumusnya adalah rumus untuk jumlah geometri, yang diketahui Maxima.
\end{eulercomment}
\begin{eulerprompt}
>&sum(q^k,k,0,n-1); $& % = ev(%,simpsum)
\end{eulerprompt}
\begin{eulercomment}
Ini agak rumit. Jumlahnya dievaluasi dengan bendera "simpsum" untuk
menguranginya menjadi hasil bagi.

Mari kita membuat fungsi untuk ini.
\end{eulercomment}
\begin{eulerprompt}
>function fs(K,R,P,n) &= (1+P/100)^n*K + ((1+P/100)^n-1)/(P/100)*R; $&fs(K,R,P,n)
\end{eulerprompt}
\begin{eulercomment}
Fungsi tersebut melakukan hal yang sama seperti fungsi f kita
sebelumnya. Tapi itu lebih efektif.
\end{eulercomment}
\begin{eulerprompt}
>longest f(5000,-200,3,47), longest fs(5000,-200,3,47)
\end{eulerprompt}
\begin{euleroutput}
       -19.82504734650985 
       -19.82504734652684 
\end{euleroutput}
\begin{eulercomment}
Kita sekarang dapat menggunakannya untuk menanyakan waktu n. Kapan
modal kita habis? Dugaan awal kami adalah 30 tahun.
\end{eulercomment}
\begin{eulerprompt}
>solve("fs(5000,-330,3,x)",30)
\end{eulerprompt}
\begin{euleroutput}
        20.51 
\end{euleroutput}
\begin{eulercomment}
Jawaban ini mengatakan bahwa itu akan menjadi negatif setelah 21
tahun.

Kita juga dapat menggunakan sisi simbolis Euler untuk menghitung
formula pembayaran.

Asumsikan kita mendapatkan pinjaman sebesar K, dan membayar n
pembayaran sebesar R (dimulai setelah tahun pertama) meninggalkan sisa
hutang sebesar Kn (pada saat pembayaran terakhir). Rumus untuk ini
jelas
\end{eulercomment}
\begin{eulerprompt}
>equ &= fs(K,R,P,n)=Kn; $&equ
\end{eulerprompt}
\begin{eulercomment}
Biasanya rumus ini diberikan dalam bentuk

\end{eulercomment}
\begin{eulerformula}
\[
i = \frac{P}{100}
\]
\end{eulerformula}
\begin{eulerprompt}
>equ &= (equ with P=100*i); $&equ
\end{eulerprompt}
\begin{eulercomment}
Kita dapat memecahkan tingkat R secara simbolis.
\end{eulercomment}
\begin{eulerprompt}
>$&solve(equ,R)
\end{eulerprompt}
\begin{eulercomment}
Seperti yang Anda lihat dari rumus, fungsi ini mengembalikan kesalahan
titik mengambang untuk i=0. Euler tetap merencanakannya.

Tentu saja, kami memiliki batas berikut.
\end{eulercomment}
\begin{eulerprompt}
>$&limit(R(5000,0,x,10),x,0)
\end{eulerprompt}
\begin{eulercomment}
Jelas, tanpa bunga kita harus membayar kembali 10 tarif 500.

Persamaan juga dapat diselesaikan untuk n. Kelihatannya lebih bagus,
jika kita menerapkan beberapa penyederhanaan untuk itu.
\end{eulercomment}
\begin{eulerprompt}
>fn &= solve(equ,n) | ratsimp; $&fn
\end{eulerprompt}
\end{eulernotebook}
\end{document}


\newpage
\chapter{KB Pekan 4: Menggunakan EMT untuk mengambar grafik 2 dimensi (2D)}
\documentclass{article}

\usepackage{eumat}

\begin{document}
\begin{eulernotebook}
\eulersubheading{Vikram Zaky Ardianto}
\eulersubheading{22305144028}
\eulersubheading{Matematika E}
\begin{eulercomment}
\begin{eulercomment}
\eulerheading{Menggambar Grafik Fungsi Satu Variabel}
\begin{eulercomment}
\begin{eulercomment}
\eulerheading{dalam Bentuk Ekspresi Langsung Ekspresi tunggal}
\begin{eulercomment}
Di dalam program numerik EMT, ekspresi adalah string. Jika ditandai
sebagai simbolis, mereka akan mencetak melalui Maxima, jika tidak
melalui EMT. Ekspresi dalam string digunakan untuk membuat plot dan
banyak fungsi numerik. Untuk ini, variabel dalam ekspresi harus "x".

expresi dalam string
\end{eulercomment}
\begin{eulerprompt}
>expr := "x^5-x^2-3"
\end{eulerprompt}
\begin{euleroutput}
  x^5-x^2-3
\end{euleroutput}
\begin{eulercomment}
plot ekspresi
\end{eulercomment}
\begin{eulerprompt}
>plot2d(expr,-2,2) :
\end{eulerprompt}
\eulerimg{27}{images/Vikram Zaky Ardianto_22305144028_Plot 2d-001.png}
\begin{eulercomment}
contoh 1
\end{eulercomment}
\begin{eulerprompt}
>expr := "sin (x-5)"
\end{eulerprompt}
\begin{euleroutput}
  sin (x-5)
\end{euleroutput}
\begin{eulerprompt}
>aspect (1) ; plot2d(expr,-2,2):
\end{eulerprompt}
\eulerimg{27}{images/Vikram Zaky Ardianto_22305144028_Plot 2d-002.png}
\begin{eulercomment}
contoh 2 dan penggunaan grid
\end{eulercomment}
\begin{eulerprompt}
>aspect(1)plot2d("log(x) + 3",-0.1,2, grid=6):
\end{eulerprompt}
\begin{euleroutput}
  Commands must be separated by semicolon or comma!
  Found: plot2d("log(x) + 3",-0.1,2, grid=6): (character 112)
  You can disable this in the Options menu.
  Error in:
  aspect(1)plot2d("log(x) + 3",-0.1,2, grid=6): ...
           ^
\end{euleroutput}
\begin{eulercomment}
contoh 3 dan penggunaan parameter square (atau \textgreater{}square) untuk memilih
y-range secara otomatis 
\end{eulercomment}
\begin{eulerprompt}
>aspect(1,1) ; plot2d("x^4-2",-5,5, >square); insimg(15)
\end{eulerprompt}
\eulerimg{14}{images/Vikram Zaky Ardianto_22305144028_Plot 2d-003.png}
\begin{eulerprompt}
>aspect(2) ; plot2d("x^4-2", -5,5 ):
\end{eulerprompt}
\eulerimg{13}{images/Vikram Zaky Ardianto_22305144028_Plot 2d-004.png}
\begin{eulercomment}
contoh 4 dan memberikan nama atau label pada garis sumbu
\end{eulercomment}
\begin{eulerprompt}
>plot2d("cos(x)", -4, 6, xl="x",yl="y") :
\end{eulerprompt}
\eulerimg{13}{images/Vikram Zaky Ardianto_22305144028_Plot 2d-005.png}
\eulerheading{Menggambar Grafik Fungsi Satu Variabel}
\begin{eulercomment}
* yang Rumusnya Disimpan dalam Variabel Ekspresi

ekspresi
\end{eulercomment}
\begin{eulerprompt}
>expr &= x^5-1
\end{eulerprompt}
\begin{euleroutput}
  
                                   5
                                  x  - 1
  
\end{euleroutput}
\begin{eulercomment}
plot dari ekspresi diatas 
\end{eulercomment}
\begin{eulerprompt}
>aspect(2); plot2d(expr,-1,1):
\end{eulerprompt}
\eulerimg{13}{images/Vikram Zaky Ardianto_22305144028_Plot 2d-006.png}
\begin{eulercomment}
contoh 1
\end{eulercomment}
\begin{eulerprompt}
>expr := "x^10-x-5"
\end{eulerprompt}
\begin{euleroutput}
  x^10-x-5
\end{euleroutput}
\begin{eulerprompt}
>aspect(2) ; plot2d(expr,-1,1):
\end{eulerprompt}
\eulerimg{13}{images/Vikram Zaky Ardianto_22305144028_Plot 2d-007.png}
\begin{eulercomment}
menggunakan variabel lokal
\end{eulercomment}
\begin{eulercomment}
Ekspresi dapat dievaluasi secara numerik. Variabel x,y,z ditetapkan
secara otomatis. Variabel lain dapat ditetapkan berdasarkan parameter
yang ditetapkan( variabel lokal ) atau melalui variabel global.
variabel global adalah variabel yang selalu bisa diakses kapan pun dan
di mana pun.
\end{eulercomment}
\begin{eulerprompt}
>expr &= a*x^5
\end{eulerprompt}
\begin{euleroutput}
  
                                      5
                                   a x
  
\end{euleroutput}
\begin{eulercomment}
menggunakan variabel global 
\end{eulercomment}
\begin{eulerprompt}
>a=6; expr(2.5)
\end{eulerprompt}
\begin{euleroutput}
  585.9375
\end{euleroutput}
\begin{eulercomment}
menggunakan variabel lokal
\end{eulercomment}
\begin{eulerprompt}
>expr(2.5,a=6)
\end{eulerprompt}
\begin{euleroutput}
  585.9375
\end{euleroutput}
\begin{eulercomment}
evaluasi langsung
\end{eulercomment}
\begin{eulerprompt}
>"a*x^5"(3,4)
\end{eulerprompt}
\begin{euleroutput}
  1458
\end{euleroutput}
\begin{eulercomment}
Oleh karena itu, banyak algoritma EMT yang dapat menggunakan ekspresi
dalam x, bukan fungsi. Namun jika parameter tambahan yang tidak
bersifat global dilibatkan, fungsi harus diutamakan.

menggunakan variabel  global "a"
\end{eulercomment}
\begin{eulerprompt}
>a=5; plot2d("a*x^3-x",0,1):
\end{eulerprompt}
\eulerimg{13}{images/Vikram Zaky Ardianto_22305144028_Plot 2d-008.png}
\begin{eulerprompt}
>function f(x,a) := a*x^3-x
\end{eulerprompt}
\begin{eulercomment}
gunakan "a=6" sebagai parameter
\end{eulercomment}
\begin{eulerprompt}
>plot2d("f",0,1;6):
\end{eulerprompt}
\eulerimg{13}{images/Vikram Zaky Ardianto_22305144028_Plot 2d-009.png}
\begin{eulercomment}
alternatif lain
\end{eulercomment}
\begin{eulerprompt}
>plot2d(\{\{"f",6\}\},0,1):
\end{eulerprompt}
\eulerimg{13}{images/Vikram Zaky Ardianto_22305144028_Plot 2d-010.png}
\begin{eulercomment}
alternatif lain 
\end{eulercomment}
\begin{eulerprompt}
>plot2d("f(x,6)",0,1):
\end{eulerprompt}
\eulerimg{13}{images/Vikram Zaky Ardianto_22305144028_Plot 2d-011.png}
\eulerheading{Menggambar Fungsi Simbolik}
\begin{eulercomment}
Fungsi Plot yang paling penting untuk plot planar adalah plot2d().
Fungsi ini diimplementasikan dalam bahasa Euler dalam file "plot.e",
yang dimuat diawal program.

plot2d() menerima ekspresi, fungsi, dan data.

Rentang plot diatur dengan parameter yang ditetapkan ssbagai berikut\\
- a,b: rentang x (default -2,2)\\
- -c,d: rentang y (default: skala dengan nilai)\\
- r: alternatifnya radius di sekitar pusat plot\\
- cx,cy: koordinat pusat plot (default 0,0)

Keterangan:(menggambar grafik fungsi satu variabel yang fungsinya
didefinisikan sebagai fungsi simbolik)\\
- \&: untuk menampilkan variabel pada teks

Berikut adalah beberapa contoh menggunakan fungsi. Seperti biasa di
EMT, fungsi yang berfungsi untuk fungsi atau ekspresi lain, jadi kita
dapat meneruskan parameter tambahan (selain x) yang bukan variabel
global ke fungsi dengan parameter titik koma atau dengan koleksi
panggilan.
\end{eulercomment}
\begin{eulerprompt}
>plot2d("f",0,1;0.4): // plot with a=0.4
\end{eulerprompt}
\eulerimg{13}{images/Vikram Zaky Ardianto_22305144028_Plot 2d-012.png}
\begin{eulerprompt}
>plot2d(\{\{"f",0.2\}\},0,1); 
>plot2d(\{\{"f(x,b)",b=0.1\}\},0,1):
\end{eulerprompt}
\eulerimg{13}{images/Vikram Zaky Ardianto_22305144028_Plot 2d-013.png}
\begin{eulerprompt}
>function f(x) := x^3-x;...
>plot2d("f",r=1):
\end{eulerprompt}
\eulerimg{13}{images/Vikram Zaky Ardianto_22305144028_Plot 2d-014.png}
\begin{eulerprompt}
>plot2d("exp(-a*x^2)/a"):
\end{eulerprompt}
\eulerimg{13}{images/Vikram Zaky Ardianto_22305144028_Plot 2d-015.png}
\begin{eulercomment}
Berikut merupakan ringkasan dari fungsi yang diterima\\
- ekspresi atau ekspresi simbolik dalam x\\
- fungsi atau fungsi simbolis dengan nama sebagai "f"\\
- fungsi simbolis hanya dengan nama f\\
\end{eulercomment}
\begin{eulerttcomment}
 
\end{eulerttcomment}
\begin{eulercomment}
Fungsi plot2d() juga menerima fungsi simbolis. Untuk fungsi simbolis,
hanya nama saja yang berfungsi.
\end{eulercomment}
\begin{eulerprompt}
>function f(x) &= diff(x^x,x)
\end{eulerprompt}
\begin{euleroutput}
  
                              x
                             x  (log(x) + 1)
  
\end{euleroutput}
\begin{eulerprompt}
>plot2d(f,0,2):
\end{eulerprompt}
\eulerimg{13}{images/Vikram Zaky Ardianto_22305144028_Plot 2d-016.png}
\begin{eulerprompt}
>$&expr = sin (x)*exp(-x)
\end{eulerprompt}
\begin{eulerformula}
\[
a\,x^5=e^ {- x }\,\sin x
\]
\end{eulerformula}
\begin{eulerprompt}
>plot2d(expr,0,3pi):
\end{eulerprompt}
\eulerimg{13}{images/Vikram Zaky Ardianto_22305144028_Plot 2d-018.png}
\begin{eulerprompt}
>plot2d("cos(x)","sin(3*x)"):
\end{eulerprompt}
\eulerimg{13}{images/Vikram Zaky Ardianto_22305144028_Plot 2d-019.png}
\eulerheading{Menggambar Fungsi Numerik}
\begin{eulercomment}
Fungsi Numerik adalah sebuah fungsi dengan himpunan bilangan cacah
sebagai domain dan himpunan mendasar yang melibatkan hubungan
matematis antara bilangan yang menjadi domain dan bilangan sebagai
kodomain.
\end{eulercomment}
\begin{eulerprompt}
> 
\end{eulerprompt}
\begin{eulercomment}
Fungsi numerik  memiliki  1  atau  lebih  variabel  independen, yang
sering dilambangkan sebagai "X". Variabel X adalah nilai atau
parameter yang dapat berubah, dan fungsi numerik menggambarkan
bagaimana variabel ini memengaruhi variabel dependen. Variabel
dependen adalah hasil perhitungan atau keluaran dari fungsi numerik
yang bergantung pada nilai atau perubahan dalam variabel independen.

\end{eulercomment}
\begin{eulercomment}
Dalam EMT cara mendefinisikan fungsi menggunakan syntak function.
untuk mendefinisikan fungsi numerik menggunakan tanda ":="

Fungsi  numerik  menjelaskan bagaimana bilangan  dalam  domain
berhubungan dengan bilangan sebagai kodomain, biasanya diberikan dalam
bentuk rumus matematik(persamaan) atau aturan yang memetakan setiap
domain kedalam kodomain yang sesuai. contoh:

f(x)=2x+3
\end{eulercomment}
\begin{eulerprompt}
> 
\end{eulerprompt}
\begin{eulercomment}
(x)(variabel dependen) adalah fungsi yang memetakan setiap nilai
x(variabel independen)kedalam nilai 2x+3. Terdapat berbagai jenis
fungsi yang termasuk ke dalam fungsi numerik, diantaranya:

Fungsi linier dengan bentuk umum\\
f (x) = ax + b
\end{eulercomment}
\begin{eulercomment}
Fungsi kuadrat dengan bentuk umum

f (x) = ax2 + bx + c
\end{eulercomment}
\begin{eulercomment}
Fungsi eksponensial dengan bentuk umum

f (x) = ax
\end{eulercomment}
\begin{eulercomment}
Fungsi logaritma dengan bentuk umum

f (x) = log a(x)

\end{eulercomment}
\begin{eulercomment}
Fungsi trigonometri dengan bentuk umum

f (x) = sin(x), f (x) = cos(x)

\end{eulercomment}
\begin{eulercomment}
Salah satu  cara  yang  umum  digunakan  untuk  memvisualisasikan
fungsi numerik adalah dengan menggambar grafiknya. Grafik ini
menggambarkan bagaimana variabel dependen berubah seiring perubahan
variabel independen dan membantu dalam memahami sifat-sifat fungsi,
seperti titik ekstrim
\end{eulercomment}
\eulersubheading{Contoh soal}
\begin{eulerprompt}
>function r(x):= abs(x-10)
>function s(x):= abs(sin(x))
>r(-5)
\end{eulerprompt}
\begin{euleroutput}
  15
\end{euleroutput}
\begin{eulerprompt}
>function t(x):=log(x*(2+sin(x/1000)))
>function u(x):=integrate("(sin(x)*exp(-x^2)"0,x)
>function v(x):=logbase((x^2),2)
>plot2d("v"):
\end{eulerprompt}
\eulerimg{13}{images/Vikram Zaky Ardianto_22305144028_Plot 2d-020.png}
\begin{eulerprompt}
>plot2d("s"):
\end{eulerprompt}
\eulerimg{13}{images/Vikram Zaky Ardianto_22305144028_Plot 2d-021.png}
\begin{eulerprompt}
>plot2d("t",-2,2):
\end{eulerprompt}
\eulerimg{13}{images/Vikram Zaky Ardianto_22305144028_Plot 2d-022.png}
\begin{eulerprompt}
>function P(x):=x*cos(x)
>plot2d("P",-2*pi,2*pi):
\end{eulerprompt}
\eulerimg{13}{images/Vikram Zaky Ardianto_22305144028_Plot 2d-023.png}
\begin{eulercomment}
Fungsi plot2d() adalah fungsi serbaguna untuk membuat grafik dalam
bidang (grafik 2D). Fungsi ini dapat digunakan untuk membuat grafik
fungsi-fungsi satu variabel, grafik data,  kurva-kurva  dalam  bidang,
grafik batang (bar plots), grid dari bilangan kompleks, dan grafik
implisit dari fungsi dua variabel.

Parameter\\
x,y : persamaan, fungsi, atau vektor data a,b,c,d : area plot (default
a=-2, b=2)\\
r  :  jika  r  diatur,  maka  a=cx-r,  b=cx+r,  c=cy-r,  d=cy+r r bisa
berupa vektor [rx,ry] atau vektor [rx1,rx2,ry1,ry2]. xmin,xmax :
rentang parameter untuk kurva\\
auto : tentukan rentang y secara otomatis (default)\\
square : jika benar, mencoba menjaga rentang x-y tetap persegi n :
jumlah interval (default adalah adaptif)\\
grid : 0 = tanpa grid dan label, 1 = hanya sumbu,\\
2 = grid normal (lihat di bawah untuk jumlah garis grid) 3 = di dalam
sumbu\\
4 = tanpa grid\\
5 = grid penuh termasuk margin 6 = tanda di pinggiran\\
7 = hanya sumbu\\
8 = hanya sumbu, sub-ticks frame : 0 = tanpa bingkai\\
framecolor: warna bingkai dan grid\\
margin : angka antara 0 dan 0,4 untuk margin di sekitar plot color :
Warna kurva. Jika ini adalah vektor warna,akan digunakan untuk setiap
baris matriks plot. Dalam  hal grafik titik, harus berupa vektor
kolom. Jika vektor baris atau matriks penuh warna digunakan untuk
grafik titik, akan digunakan untuk setiap titik data.\\
thickness : ketebalan garis untuk kurva

Nilai ini dapat lebih kecil dari 1 untuk garis yang sangat tipis. \\
style: Gaya plot untuk garis, penanda, dan isian.

Untuk titik gunakan\\
"[]", "\textless{}\textgreater{}", ".", "..", "...", "*", "+", " ", "-", "o"\\
"[]", "\textless{}\textgreater{}", "o" (bentuk terisi)\\
"[]w", "\textless{}\textgreater{}w", "ow" (tidak transparan)

Untuk garis gunakan\\
"-", "-", "-.", ".", ".-.", "-.-", "-\textgreater{}"

Untuk poligon terisi atau plot batang gunakan\\
"", "O", "O", "/", "", "/","+", " ", "-", "t"

points : plot titik tunggal sebagai gantinya garis segmen addpoints :
jika benar, plot segmen garis dan titik\\
add : tambahkan plot ke plot yang ada\\
user : aktifkan interaksi pengguna untuk fungsi delta : ukuran langkah
untuk interaksi pengguna\\
bar : plot batang (x adalah batas interval, y adalah nilai interval)
histogram : plot frekuensi x dalam n subinterval\\
distribusi=n : plot distribusi x dengan n subinterval even : gunakan
nilai antar untuk histogram otomatis. steps : plot fungsi sebagai
fungsi langkah (steps=1,2)\\
adaptive : gunakan plot adaptif (n adalah jumlah minimal langkah)
level : plot garis level dari fungsi implisit dua variabel\\
outline : menggambar batas rentang level.
\end{eulercomment}
\begin{eulerprompt}
>function s(x):=(x-10)
>function r(x):=abs(sin(x))
>s(-5)
\end{eulerprompt}
\begin{euleroutput}
  -15
\end{euleroutput}
\begin{eulerprompt}
>function t(x):=log(x*(2+sin(x/1000)))
>function u(x):=integrate("(sin(x)*exp(-x^2)"),0,x)
>function v(x):=logbase((x^2),2)
>plot2d("v"):
\end{eulerprompt}
\eulerimg{13}{images/Vikram Zaky Ardianto_22305144028_Plot 2d-024.png}
\begin{eulerprompt}
>plot2d("s"):
\end{eulerprompt}
\eulerimg{13}{images/Vikram Zaky Ardianto_22305144028_Plot 2d-025.png}
\begin{eulerprompt}
>function P(x):=x*cos(x)
>plot2d("P", -2*pi,2*pi):
\end{eulerprompt}
\eulerimg{13}{images/Vikram Zaky Ardianto_22305144028_Plot 2d-026.png}
\eulerheading{Menggambar Beberapa Kurva Sekaligus}
\begin{eulercomment}
Dalam subtopik ini, kita akan membahas mengenai cara menggambar
beberapa kurva sekaligus. Dalam hal ini kita dapat menggambar beberapa
kurva dalam jendela grafik yang berbeda secara bersama-sama. Untuk
membuat ini kita dapat menggunakan perintah figure(). Berikut contoh
dari menggambar beberapa kurva sekaligus

Menggambar plot fungsi\\
\end{eulercomment}
\begin{eulerformula}
\[
x^n, 1 \leq n \leq 4
\]
\end{eulerformula}
\begin{eulerprompt}
>reset;
>figure(2,2);...
>for n=1 to 4; figure(n); plot2d("x^"+n); end;...
>figure(0):
\end{eulerprompt}
\eulerimg{27}{images/Vikram Zaky Ardianto_22305144028_Plot 2d-028.png}
\begin{eulercomment}
Penjelasan sintaks dari plot fungsi

\end{eulercomment}
\begin{eulerformula}
\[
x^n,  1 \leq n \leq 4
\]
\end{eulerformula}
\begin{eulercomment}
- reset;\\
Perintah ini berguna untuk menghapus grafik yang telah ada sebelumnya,
sehingga kita dapat memulai dari awal untuk menggambar grafik\\
- figure(2x2);\\
Perintah figure() digunakan untuk membuat jendela grafik dengan ukuran\\
axb. Dalam kasus ini perintah figure(2,2) memiliki makna bahwa jendela
grafik yang dibuat berukuran 2x2. Artinya, akan ada empat jendela
grafik yang akan ditampilkan dengan tata letak 2 baris dan 2 kolom.\\
- for n=1 to 4;\\
Perintah ini digunakan untuk melakukan pengulangan (looping) perintah
sebanyak empat kali, yaitu dari 1 hingga 4.\\
- figure(n);\\
Perintah ini digunakan untuk beralih dari jendela grafik satu ke
jendela grafik lainnya (jendela grafik ke-n).\\
- plot2d("x\textasciicircum{}"+n);\\
Perintah plot2d() digunakan untuk membuat plot fungsi matematika.\\
Dalam hal ini fungsi yang diplot adalah x\textasciicircum{}n, di mana n adalah nilai
dari variabel yang sedang diulang. Dengan kata lain, ini akan membuat\\
plot dari x\textasciicircum{}1, x\textasciicircum{}2, x\textasciicircum{}3, dan x\textasciicircum{}4 dalam jendela grafik yang sesuai\\
- end;\\
Perintah ini menandakan akhir dari looping.\\
- figure(0);\\
Perintah ini digunakan untuk beralih kembali ke jendela grafik utama.
\end{eulercomment}
\begin{eulercomment}
Dari sini dapat kita perhatikan untuk membuat kurva fungsi x\textasciicircum{}n (x
pangkat n) perintahnya tidak ditulis dengan (x\textasciicircum{}n) melainkan ditulis
dengan ("x\textasciicircum{}"+n). Tanda petik dua ("...") digunakan untuk
mengidentifikasi bahwa teks tersebut merupakan ekspresi matematika.\\
Sedangkan tanda (+) digunakan untuk menggabungkan string dengan nilai
yang berubah-ubah atau variabel.

Contoh lain:\\
Menggambar plot fungsi\\
\end{eulercomment}
\begin{eulerformula}
\[
f(x)=x^3-x, -2<x<2
\]
\end{eulerformula}
\begin{eulerprompt}
>reset;
>figure(3,3);...
>for k=1:9; figure(k); plot2d("x^3-x",-2,2,grid=k); end;...
>figure(0):
\end{eulerprompt}
\eulerimg{27}{images/Vikram Zaky Ardianto_22305144028_Plot 2d-029.png}
\begin{eulerttcomment}
 Penjelasan sintaks dari plot fungsi
\end{eulerttcomment}
\begin{eulerformula}
\[
f(x)=x^3-x, -2<x<2
\]
\end{eulerformula}
\begin{eulercomment}
- reset;\\
Perintah ini berguna untuk menghapus grafik yang telah ada sebelumnya,
sehingga kita dapat memulai dari awal untuk menggambar grafik\\
- figure (3,3);\\
Perintah ini digunakan untuk membuat jendela grafik dengan ukuran 3x3.
Artinya, akan ada empat jendela grafik yang akan ditampilkan dengan
tata letak 3 baris dan 3 kolom.\\
- for k=1:9;\\
Perintah ini digunakan untuk melakukan pengulangan (looping) perintah
sebanyak sembilan kali.\\
- figure(n);\\
Perintah ini digunakan untuk beralih dari jendela grafik satu ke\\
\end{eulercomment}
\begin{eulerttcomment}
 jendela grafik lainnya (jendela grafik ke-n).
\end{eulerttcomment}
\begin{eulercomment}
- plot2d("x\textasciicircum{}3-x",-2,2,grid=k);\\
Perintah plot2d() digunakan untuk membuat plot fungsi matematika.\\
Dalam hal ini fungsi yang diplot adalah x\textasciicircum{}3-x, dengan batas sumbu x
dari -2 hingga 2. Argumen grid=k digunakan untuk mengaktifkan grid
pada jendela grafik ke-k.\\
- end;\\
Perintah ini menandakan akhir dari looping.\\
- figure(0);\\
Perintah ini digunakan untuk beralih kembali ke jendela grafik utama.

Dari contoh diatas dapat kita perhatikan bahwa tampilan plot dari yang
ke-1 hingga ke-9 memiliki tampilan yang berbeda-beda. Dalam EMT
memiliki berbagai gaya plot 2D yang dapat dijalankan menggunakan
perintah grid=n dimana n adalah jumlah langkah minimal. Setiap nilai n
memiliki tampilan plot adaptif yang berbeda dalam plot 2D, diantaranya
yaitu:\\
0 : tidak ada grid (kisi), frame, sumbu, dan label, hanya kurva saja\\
1 : dengan sumbu, label-label sumbu di luar frame jendela grafik\\
2 : tampilan default\\
3 : dengan grid pada sumbu x dan y, label-label sumbu berada di dalam
jendela grafik\\
4 : tidak ada grid (kisi), sumbu x dan y, dan label berada di luar
frame jendela grafik\\
5 : tampilan default tanpa margin di sekitar plot\\
6 : hanya dengan sumbu x y dan label, tanpa grid\\
7 : hanya dengan sumbu x y dan tanda-tanda pada sumbu.\\
8 : hanya dengan sumbu dan tanda-tanda pada sumbu, dengan tanda-tanda
yang lebih halus pada sumbu.\\
9 : tampilan default dengan tanda-tanda kecil di dalam jendela\\
10: hanya dengan sumbu-sumbu, tanpa tanda

Contoh lain:\\
Menggambar plot fungsi\\
\end{eulercomment}
\begin{eulerformula}
\[
g(x)=2x^3-x
\]
\end{eulerformula}
\begin{eulerprompt}
>reset;
>aspect(1.2);
>figure(3,4); ...
> figure(2); plot2d("2x^3-x",grid=1); ... // x-y-axis
> figure(3); plot2d("2x^3-x",grid=2); ... // default ticks
>figure(4); plot2d("2x^3-x",grid=3); ... // x-y- axis with labels inside
> figure(5); plot2d("2x^3-x",grid=4); ... // no ticks, only labels
>figure(6); plot2d("2x^3-x",grid=5); ... // default, but no margin
>figure(7); plot2d("2x^3-x",grid=6); ... // axes only
>figure(8); plot2d("2x^3-x",grid=7); ... // axes only, ticks at axis
>figure(9); plot2d("2x^3-x",grid=8); ... // axes only, finer ticks at axis
>figure(10); plot2d("2x^3-x",grid=9); ... // default, small ticks inside
>figure(11); plot2d("2x^3-x",grid=10); ...// no ticks, axes only
>figure(0):
\end{eulerprompt}
\eulerimg{22}{images/Vikram Zaky Ardianto_22305144028_Plot 2d-030.png}
\begin{eulercomment}
Penjelasan sintaks dari plot fungsi\\
\end{eulercomment}
\begin{eulerformula}
\[
g(x)=2x^3-x
\]
\end{eulerformula}
\begin{eulercomment}
- aspect(1.2);\\
Perintah aspect() digunakan untuk mengatur rasio aspek dari jendela
grafik. Hal ini berarti perintah aspect(1.2); akan menghasilkan plot
dengan perbandingan rasio panjang dan lebar 2:1.\\
- figure(3,4);\\
Perintah ini digunakan untuk membuat jendela grafik dengan ukuran 3x4.\\
Jadi, akan ada total 12 jendela grafik yang akan ditampilkan dalam
tata letak 3 baris dan 4 kolom.\\
- figure(1); plot2d("x\textasciicircum{}3-x",grid=0); ...\\
Adalah perintah untuk beralih ke jendela grafik pertama dan menggambar
plot dari fungsi x\textasciicircum{}3 - x tanpa grid, frame, atau sumbu.\\
- figure(2); plot2d("x\textasciicircum{}3-x",grid=1); ...\\
Adalah perintah untuk beralih ke jendela grafik kedua dan menggambar
plot dari fungsi x\textasciicircum{}3 - x dengan grid hanya pada sumbu x dan y.\\
- figure(3); plot2d("x\textasciicircum{}3-x",grid=2); ...\\
Adalah perintah untuk beralih ke jendela grafik ketiga dan menggambar
plot dari fungsi x\textasciicircum{}3 - x dengan tampilan default, termasuk tanda-tanda
default pada sumbu.\\
- figure(4); plot2d("x\textasciicircum{}3-x",grid=3); ...\\
Adalah perintah untuk beralih ke jendela grafik keempat dan menggambar
plot dari fungsi x\textasciicircum{}3 - x dengan grid pada sumbu x dan y, serta
label-label sumbu yang ada di dalam jendela.\\
- figure(5); plot2d("x\textasciicircum{}3-x",grid=4); ...\\
Adalah perintah untuk beralih ke jendela grafik kelima dan menggambar
plot dari fungsi x\textasciicircum{}3 - x tanpa tanda-tanda sumbu, hanya label-label
yang ada.\\
- figure(6); plot2d("x\textasciicircum{}3-x",grid=5); ...\\
Adalah perintah untuk beralih ke jendela grafik keenam dan menggambar
plot dari fungsi x\textasciicircum{}3 - x dengan tampilan default, tetapi tanpa margin
di sekitar plot.\\
- figure(7); plot2d("x\textasciicircum{}3-x",grid=6); ...\\
Adalah perintah untuk beralih ke jendela grafik ketujuh dan menggambar
plot dari fungsi x\textasciicircum{}3 - x hanya dengan sumbu-sumbu (tanpa grid atau
label).\\
- figure(8); plot2d("x\textasciicircum{}3-x",grid=7); ...\\
Adalah perintah untuk beralih ke jendela grafik kedelapan dan
menggambar plot dari fungsi x\textasciicircum{}3 - x hanya dengan sumbu-sumbu dan
tanda-tanda pada sumbu.\\
- figure(9); plot2d("x\textasciicircum{}3-x",grid=8); ...\\
Adalah perintah untuk beralih ke jendela grafik kesembilan dan
menggambar plot dari fungsi x\textasciicircum{}3 - x hanya dengan sumbu-sumbu dan
tanda-tanda pada sumbu, dengan tanda-tanda yang lebih halus pada
sumbu.\\
- figure(10); plot2d("x\textasciicircum{}3-x",grid=9); ...\\
Adalah perintah untuk beralih ke jendela grafik kesepuluh dan
menggambar plot dari fungsi x\textasciicircum{}3 - x dengan tanda-tanda default kecil
di dalam jendela.\\
- figure(11); plot2d("x\textasciicircum{}3-x",grid=10); ...\\
Adalah perintah untuk beralih ke jendela grafik kesebelas dan
menggambar plot dari fungsi x\textasciicircum{}3 - x hanya dengan sumbu-sumbu, tanpa
tanda-tanda.\\
- figure(0);\\
Adalah perintah untuk beralih kembali ke jendela grafik utama atau
jendela grafik dengan nomor 0 setelah semua perintah dalam urutan
selesai dieksekusi.

Dari ketiga contoh di atas, dapat kita katakan bahwa untuk menggambar
beberapa kurva sekaligus itu dapat dilakukan dengan satu baris
perintah ataupun dengan cara mendefinisikannya 1 per 1.

Terlihat beberapa jenis grid memiliki tampilan yang mirip atau sama,
seperti 1 dan 2, 2 dan 5, 4 dan 9, 7 dan 8, untuk dapat membedakannya
secara lebih jelas, ubah grid dari contoh di bawah ini.
\end{eulercomment}
\begin{eulerprompt}
>reset;
>aspect(1.3);
>figure(1,3);...
>figure (1); plot2d("x^2*exp(-x)",0,10);...
>figure (2); plot2d("2*exp(x)",-5,5);...
>figure (3); plot2d("exp(x^2)",-2,2);...
>figure (0):
\end{eulerprompt}
\eulerimg{20}{images/Vikram Zaky Ardianto_22305144028_Plot 2d-031.png}
\begin{eulercomment}
Contoh lain:
\end{eulercomment}
\begin{eulerprompt}
>reset;
>aspect(3/4);
>figure(2,1);...
>for a=1:2; figure(a); plot2d("2*x*log(x^2)",0,3,grid=a); end;...
>figure(0):
\end{eulerprompt}
\eulerimg{34}{images/Vikram Zaky Ardianto_22305144028_Plot 2d-032.png}
\eulerheading{Menggambar Beberapa Kurva pada bidang }
\begin{eulercomment}
* koordinat yang sama 

Plot lebih dari satu fungsi (multiple function) ke dalam satu jendela
dapat dilakukan dengan berbagai cara. Salah satu caranya adalah
menggunakan \textgreater{}add untuk beberapa panggilan ke plot2d secara
keseluruhan, kecuali panggilan pertama.

Berikut contohnya:\\
menggambar kurva\\
\end{eulercomment}
\begin{eulerformula}
\[
 f(x)=cos(x)
\]
\end{eulerformula}
\begin{eulerformula}
\[
f(x)= x^2
\]
\end{eulerformula}
\begin{eulerprompt}
>aspect(); plot2d("cos(x)",r=3); plot2d("x^2",style=".",>add):
\end{eulerprompt}
\eulerimg{27}{images/Vikram Zaky Ardianto_22305144028_Plot 2d-035.png}
\begin{eulerformula}
\[
f(x)=cos(x)-1
\]
\end{eulerformula}
\begin{eulerformula}
\[
f(x)= sin(x)-1
\]
\end{eulerformula}
\begin{eulerprompt}
>aspect(2); plot2d("cos(x)-1",-1,6); plot2d("sin(x)-1",style="--",>add):
\end{eulerprompt}
\eulerimg{13}{images/Vikram Zaky Ardianto_22305144028_Plot 2d-036.png}
\begin{eulercomment}
Selain menggunakan \textgreater{}add kita juga bisa menambahkannya secara langsung

Berikut contohnya:\\
Menggambar kurva\\
\end{eulercomment}
\begin{eulerformula}
\[
f(x)= 2x+1
\]
\end{eulerformula}
\begin{eulerformula}
\[
f(x)= -2x+1
\]
\end{eulerformula}
\begin{eulerprompt}
>plot2d(["2x+1","x"],0,8):
\end{eulerprompt}
\eulerimg{13}{images/Vikram Zaky Ardianto_22305144028_Plot 2d-037.png}
\begin{eulerformula}
\[
f(x)=sin(2x)
\]
\end{eulerformula}
\begin{eulerformula}
\[
f(x)=cos(3x)
\]
\end{eulerformula}
\begin{eulerprompt}
>aspect(1.5); plot2d(["sin(2x)","cos(3x)"],0,8):
\end{eulerprompt}
\eulerimg{17}{images/Vikram Zaky Ardianto_22305144028_Plot 2d-038.png}
\begin{eulercomment}
Kegunaan \textgreater{}add yang lain juga bisa untuk menambahkan titik pada kurva.

Berikut contohnya:\\
Menambahkan sebuah titik di\\
\end{eulercomment}
\begin{eulerformula}
\[
f(x)= x+4
\]
\end{eulerformula}
\begin{eulerprompt}
>aspect(); plot2d("x+4",-2,5,); plot2d(2,6,>points,>add):
\end{eulerprompt}
\eulerimg{27}{images/Vikram Zaky Ardianto_22305144028_Plot 2d-039.png}
\begin{eulercomment}
Kita juga bisa mencari titik perpotongan dengan cara berikut:

\end{eulercomment}
\begin{eulerformula}
\[
sin(x)=2x
\]
\end{eulerformula}
\begin{eulerprompt}
>plot2d(["sin(x)","2x"],r=2,cx=1,cy=1, ...
>  color=[black,blue],style=["-","."], ...
>  grid=1);
>x0=solve("sin(x)-2x",1);  ...
>  plot2d(x0,x0,>points,>add);  ...
>  label("sin(x) = 2x",x0,x0,pos="cl",offset=20):
\end{eulerprompt}
\eulerimg{27}{images/Vikram Zaky Ardianto_22305144028_Plot 2d-040.png}
\begin{eulerprompt}
>function f(x,a) := x^2+a*x-x/a; ...
>plot2d("f",-10,10;1,title="a=1"):
\end{eulerprompt}
\eulerimg{27}{images/Vikram Zaky Ardianto_22305144028_Plot 2d-041.png}
\begin{eulerprompt}
> plot2d(\{\{"f",1\}\},-10,10); ...
>for a=1:10; plot2d(\{\{"f",a\}\},>add); end:
\end{eulerprompt}
\eulerimg{27}{images/Vikram Zaky Ardianto_22305144028_Plot 2d-042.png}
\begin{eulerprompt}
>function f(x,a) := x^2*exp(-x^2/a); ...
>plot2d("f",-10,10;5,thickness=2,title="a=5"):
\end{eulerprompt}
\eulerimg{27}{images/Vikram Zaky Ardianto_22305144028_Plot 2d-043.png}
\begin{eulerprompt}
>plot2d(\{\{"f",1\}\},-8,8); ...
>for a=2:5; plot2d(\{\{"f",a\}\},>add,thickness=2); end:
\end{eulerprompt}
\eulerimg{27}{images/Vikram Zaky Ardianto_22305144028_Plot 2d-044.png}
\begin{eulerprompt}
>aspect(2.1); &plot2d(1/x,[x,-1,1]):
\end{eulerprompt}
\eulerimg{27}{images/Vikram Zaky Ardianto_22305144028_Plot 2d-045.png}
\begin{eulerprompt}
>x=linspace(-1,1,50);...
>plot2d("1/x"):
\end{eulerprompt}
\eulerimg{12}{images/Vikram Zaky Ardianto_22305144028_Plot 2d-046.png}
\eulerheading{Menuliskan Label koordinat,label kurva,}
\begin{eulercomment}
* dan keterangan kurva(legend)

Dalam EMT, untuk menambahkan judul dapat dilakukan dengan title="..."\\
untuk menambahkan sumbu x dan sumbu y dapat dilakukan dengan x1="...",
y1="..."\\
sebagai contoh:
\end{eulercomment}
\begin{eulerprompt}
>plot2d("x^2-4*x"):
\end{eulerprompt}
\eulerimg{12}{images/Vikram Zaky Ardianto_22305144028_Plot 2d-047.png}
\begin{eulercomment}
untuk menambahkan judul dapat dilakukan dengan title="..."\\
untuk menambahkan sumbu x dan sumbu y dapat dilakukan dengan x1="...",
y1="..."
\end{eulercomment}
\begin{eulerprompt}
>plot2d("x^2-4*x",title="FUNGSI y=x^2-4*x",yl="Sumbu y",xl="Sumbu x"):
\end{eulerprompt}
\eulerimg{12}{images/Vikram Zaky Ardianto_22305144028_Plot 2d-048.png}
\begin{eulercomment}
Selain itu juga dapat dengan cara lain seperti contoh berikut:
\end{eulercomment}
\begin{eulerprompt}
>expr := "x^3-x"; ...
>  plot2d(expr,title="y="+expr,xl="Sumbu x",yl="Sumbu y"); ...
>  label("(1,0)",1,0);  label("Max",E,expr(E),pos="lc"): 
\end{eulerprompt}
\eulerimg{12}{images/Vikram Zaky Ardianto_22305144028_Plot 2d-049.png}
\eulerheading{Mengatur ukuran gambar,format(style),dan warna kurva}
\begin{eulercomment}
Untuk mengubah ukuran, dapat dilakukan dengan menggunakan
aspect="...", semakin besar nilai aspect, maka ukuran kurva akan
semakin kecil, begitupun sebaliknya

untuk mengganti style, dapat dipilih dengan berbagai pilihan\\
style="...", dapat dipilih dari, misal : "-","\_',"-.",".-.","-.-".

untuk warna dapat dipilih sebagai salah satu warna default\\
color="...", warna default= red,green,blue,yellow, dll

sebagai contoh:
\end{eulercomment}
\begin{eulerprompt}
>aspect(1); plot2d("exp(x^2-3)"):
\end{eulerprompt}
\eulerimg{27}{images/Vikram Zaky Ardianto_22305144028_Plot 2d-050.png}
\begin{eulercomment}
ukuran kurva dapat diganti dengan mengganti nilai aspect="...",
semakin besar nilai aspect, maka ukuran kurva akan semakin kecil Untuk
mengganti warna dapat ditambahkan dengan color="...", sedangkan untuk
mengganti format(style) dapat dilakukan dengan menambahkan style="..."
\end{eulercomment}
\begin{eulerprompt}
>aspect(2); plot2d("exp(x^2-3)", color=red, style="--"):
\end{eulerprompt}
\eulerimg{13}{images/Vikram Zaky Ardianto_22305144028_Plot 2d-051.png}
\begin{eulercomment}
Berikut adalah tampilan warna EMT yang telah ditentukan
\end{eulercomment}
\begin{eulerprompt}
>aspect (1) ; columnsplot (ones(1,16),lab=0:15,grid=0, color=0:15) :
\end{eulerprompt}
\eulerimg{27}{images/Vikram Zaky Ardianto_22305144028_Plot 2d-052.png}
\begin{eulercomment}
selain menggunakan warna default, untuk mengubah warna dapat juga
dengan menggunakan kode warna di atas\\
sebagai contoh:
\end{eulercomment}
\begin{eulerprompt}
>aspect(1); plot2d("exp(x^3+2*x)",r=3, color=1, style="--"):
\end{eulerprompt}
\eulerimg{27}{images/Vikram Zaky Ardianto_22305144028_Plot 2d-053.png}
\begin{euleroutput}
  
\end{euleroutput}
\eulerheading{Membuat Gambar Kurva yang Bersifat Interaktif}
\begin{eulercomment}
Saat memplot fungsi atau ekspresi, parameter \textgreater{}user memungkinkan
pengguna untuk memperbesar dan menggeser plot dengan tombol kursor
atau mouse. Pengguna dapat

- perbesar dengan + atau -\\
- pindahkan plot dengan tombol kursor\\
- pilih jendela plot dengan mouse\\
- atur ulang tampilan dengan spasi\\
- keluar dengan kembali

Tombol spasi akan mengatur ulang plot ke jendela plot asli.

Saat memplot data, flag \textgreater{}user hanya akan menunggu penekanan tombol.
\end{eulercomment}
\begin{eulerprompt}
>plot2d(\{\{"x^3-a*x",a=1\}\},>user,title="Press any key!"); ...
>insimg;  
\end{eulerprompt}
\eulerimg{27}{images/Vikram Zaky Ardianto_22305144028_Plot 2d-054.png}
\begin{eulerprompt}
>plot2d("exp(x)*sin(x)",user=true, ...
>  title="+/- or cursor keys (return to exit)"):
\end{eulerprompt}
\eulerimg{27}{images/Vikram Zaky Ardianto_22305144028_Plot 2d-055.png}
\begin{eulercomment}
Berikut ini menunjukkan cara interaksi pengguna tingkat lanjut (lihat
tutorial tentang pemrograman untuk detailnya).

Fungsi bawaan mousedrag() menunggu event mouse atau keyboard. Ini
melaporkan mouse ke bawah, mouse dipindahkan atau mouse ke atas, dan
penekanan tombol. Fungsi dragpoints() memanfaatkan ini, dan
memungkinkan pengguna menyeret titik mana pun dalam plot.

Kita membutuhkan fungsi plot terlebih dahulu. Sebagai contoh, kita
interpolasi dalam 5 titik dengan polinomial. Fungsi harus diplot ke
area plot tetap.
\end{eulercomment}
\begin{eulerprompt}
>function plotf(xp,yp,select) ...
\end{eulerprompt}
\begin{eulerudf}
    d=interp(xp,yp);
    plot2d("interpval(xp,d,x)";d,xp,r=2);
    plot2d(xp,yp,>points,>add);
    if select>0 then
      plot2d(xp[select],yp[select],color=red,>points,>add);
    endif;
    title("Drag one point, or press space or return!");
  endfunction
\end{eulerudf}
\begin{eulercomment}
Perhatikan parameter titik koma di plot2d (d dan xp), yang diteruskan
ke evaluasi fungsi interp(). Tanpa ini, kita harus menulis fungsi
plotinterp() terlebih dahulu, mengakses nilai secara global.

Sekarang kita menghasilkan beberapa nilai acak, dan membiarkan
pengguna menyeret poin.
\end{eulercomment}
\begin{eulerprompt}
>t=-1:0.5:1; dragpoints("plotf",t,random(size(t))-0.5):
\end{eulerprompt}
\eulerimg{27}{images/Vikram Zaky Ardianto_22305144028_Plot 2d-056.png}
\begin{eulercomment}
Ada juga fungsi, yang memplot fungsi lain tergantung pada vektor
parameter, dan memungkinkan pengguna menyesuaikan parameter ini.

Pertama kita membutuhkan fungsi plot.
\end{eulercomment}
\begin{eulerprompt}
>function plotf([a,b]) := plot2d("exp(a*x)*cos(2pi*b*x)",0,2pi;a,b);
\end{eulerprompt}
\begin{eulercomment}
Kemudian kita membutuhkan nama untuk parameter, nilai awal dan matriks
rentang nx2, opsional baris judul.\\
Ada slider interaktif, yang dapat mengatur nilai oleh pengguna. Fungsi
dragvalues() menyediakan ini.
\end{eulercomment}
\begin{eulerprompt}
>dragvalues("plotf",["a","b"],[-1,2],[[-2,2];[1,10]], ...
>  heading="Drag these values:",hcolor=black):
\end{eulerprompt}
\eulerimg{27}{images/Vikram Zaky Ardianto_22305144028_Plot 2d-057.png}
\begin{eulercomment}
Dimungkinkan untuk membatasi nilai yang diseret ke bilangan bulat.
Sebagai contoh, kita menulis fungsi plot, yang memplot polinomial
Taylor derajat n ke fungsi kosinus.
\end{eulercomment}
\begin{eulerprompt}
>function plotf(n) ...
\end{eulerprompt}
\begin{eulerudf}
  plot2d("cos(x)",0,2pi,>square,grid=6);
  plot2d(&"taylor(cos(x),x,0,@n)",color=blue,>add);
  textbox("Taylor polynomial of degree "+n,0.1,0.02,style="t",>left);
  endfunction
\end{eulerudf}
\begin{eulercomment}
Sekarang kami mengizinkan derajat n bervariasi dari 0 hingga 20 dalam
20 pemberhentian. Hasil dragvalues() digunakan untuk memplot sketsa
dengan n ini, dan untuk memasukkan plot ke dalam buku catatan.
\end{eulercomment}
\begin{eulerprompt}
>nd=dragvalues("plotf","degree",2,[0,20],20,y=0.8, ...
>   heading="Drag the value:"); ...
>plotf(nd):
\end{eulerprompt}
\eulerimg{27}{images/Vikram Zaky Ardianto_22305144028_Plot 2d-058.png}
\begin{eulercomment}
Berikut ini adalah demonstrasi sederhana dari fungsi tersebut.
Pengguna dapat menggambar di atas jendela plot, meninggalkan jejak
poin.
\end{eulercomment}
\begin{eulerprompt}
>function dragtest ...
\end{eulerprompt}
\begin{eulerudf}
    plot2d(none,r=1,title="Drag with the mouse, or press any key!");
    start=0;
    repeat
      \{flag,m,time\}=mousedrag();
      if flag==0 then return; endif;
      if flag==2 then
        hold on; mark(m[1],m[2]); hold off;
      endif;
    end
  endfunction
\end{eulerudf}
\begin{eulerprompt}
>dragtest // lihat hasilnya dan cobalah lakukan!
\end{eulerprompt}
\eulerheading{Menggambar Sekumpulan Kurva dengan Satu Perintah}
\begin{eulercomment}
Secara default, EMT menghitung tick sumbu otomatis dan menambahkan
label ke setiap tick. Ini dapat diubah dengan parameter grid. Gaya
default sumbu dan label dapat dimodifikasi. Selain itu, label dan
judul dapat ditambahkan secara manual. Untuk mengatur ulang ke gaya
default, gunakan reset().
\end{eulercomment}
\begin{eulerprompt}
>aspect();
>figure(3,4); ...
> figure(1); plot2d("x^3-x",grid=0); ... // no grid, frame or axis
> figure(2); plot2d("x^3-x",grid=1); ... // x-y-axis
> figure(3); plot2d("x^3-x",grid=2); ... // default ticks
> figure(4); plot2d("x^3-x",grid=3); ... // x-y- axis with labels inside
> figure(5); plot2d("x^3-x",grid=4); ... // no ticks, only labels
> figure(6); plot2d("x^3-x",grid=5); ... // default, but no margin
> figure(7); plot2d("x^3-x",grid=6); ... // axes only
> figure(8); plot2d("x^3-x",grid=7); ... // axes only, ticks at axis
> figure(9); plot2d("x^3-x",grid=8); ... // axes only, finer ticks at axis
> figure(10); plot2d("x^3-x",grid=9); ... // default, small ticks inside
> figure(11); plot2d("x^3-x",grid=10); ...// no ticks, axes only
> figure(0):
\end{eulerprompt}
\eulerimg{27}{images/Vikram Zaky Ardianto_22305144028_Plot 2d-059.png}
\begin{eulercomment}
Parameter \textless{}frame mematikan frame, dan framecolor=blue mengatur frame
ke warna biru.

Jika Anda ingin centang sendiri, Anda dapat menggunakan style=0, dan
menambahkan semuanya nanti.
\end{eulercomment}
\begin{eulerprompt}
>aspect(1.5); 
>plot2d("x^3-x",grid=0); // plot
>frame; xgrid([-1,0,1]); ygrid(0): // add frame and grid
\end{eulerprompt}
\eulerimg{17}{images/Vikram Zaky Ardianto_22305144028_Plot 2d-060.png}
\begin{eulercomment}
Untuk judul plot dan label sumbu, lihat contoh berikut.
\end{eulercomment}
\begin{eulerprompt}
>plot2d("exp(x)",-1,1);
>textcolor(black); // set the text color to black
>title(latex("y=e^x")); // title above the plot
>xlabel(latex("x")); // "x" for x-axis
>ylabel(latex("y"),>vertical); // vertical "y" for y-axis
>label(latex("(0,1)"),0,1,color=blue): // label a point
\end{eulerprompt}
\eulerimg{17}{images/Vikram Zaky Ardianto_22305144028_Plot 2d-061.png}
\begin{eulercomment}
Sumbu dapat digambar secara terpisah dengan xaxis() dan yaxis().
\end{eulercomment}
\begin{eulerprompt}
>plot2d("x^3-x",<grid,<frame);
>xaxis(0,xx=-2:1,style="->"); yaxis(0,yy=-5:5,style="->"):
\end{eulerprompt}
\eulerimg{17}{images/Vikram Zaky Ardianto_22305144028_Plot 2d-062.png}
\begin{eulercomment}
Teks pada plot dapat diatur dengan label(). Dalam contoh berikut, "lc"
berarti tengah bawah. Ini mengatur posisi label relatif terhadap
koordinat plot.
\end{eulercomment}
\begin{eulerprompt}
>function f(x) &= x^3-x
\end{eulerprompt}
\begin{euleroutput}
  
                                   3
                                  x  - x
  
\end{euleroutput}
\begin{eulerprompt}
>plot2d(f,-1,1,>square);
>x0=fmin(f,0,1); // compute point of minimum
>label("Rel. Min.",x0,f(x0),pos="lc"): // add a label there
\end{eulerprompt}
\eulerimg{17}{images/Vikram Zaky Ardianto_22305144028_Plot 2d-063.png}
\begin{eulercomment}
Ada juga kotak teks.
\end{eulercomment}
\begin{eulerprompt}
>plot2d(&f(x),-1,1,-2,2); // function
>plot2d(&diff(f(x),x),>add,style="--",color=red); // derivative
>labelbox(["f","f'"],["-","--"],[black,red]): // label box
\end{eulerprompt}
\eulerimg{17}{images/Vikram Zaky Ardianto_22305144028_Plot 2d-064.png}
\begin{eulerprompt}
>plot2d(["exp(x)","1+x"],color=[black,blue],style=["-","-.-"]):
\end{eulerprompt}
\eulerimg{17}{images/Vikram Zaky Ardianto_22305144028_Plot 2d-065.png}
\begin{eulerprompt}
>gridstyle("->",color=gray,textcolor=gray,framecolor=gray);  ...
> plot2d("x^3-x",grid=1);   ...
> settitle("y=x^3-x",color=black); ...
> label("x",2,0,pos="bc",color=gray);  ...
> label("y",0,6,pos="cl",color=gray); ...
> reset():
\end{eulerprompt}
\eulerimg{27}{images/Vikram Zaky Ardianto_22305144028_Plot 2d-066.png}
\begin{eulercomment}
Untuk kontrol lebih, sumbu x dan sumbu y dapat dilakukan secara
manual.

Perintah fullwindow() memperluas jendela plot karena kita tidak lagi
membutuhkan tempat untuk label di luar jendela plot. Gunakan
shrinkwindow() atau reset() untuk mengatur ulang ke default.
\end{eulercomment}
\begin{eulerprompt}
>fullwindow; ...
> gridstyle(color=darkgray,textcolor=darkgray); ...
> plot2d(["2^x","1","2^(-x)"],a=-2,b=2,c=0,d=4,<grid,color=4:6,<frame); ...
> xaxis(0,-2:1,style="->"); xaxis(0,2,"x",<axis); ...
> yaxis(0,4,"y",style="->"); ...
> yaxis(-2,1:4,>left); ...
> yaxis(2,2^(-2:2),style=".",<left); ...
> labelbox(["2^x","1","2^-x"],colors=4:6,x=0.8,y=0.2); ...
> reset:
\end{eulerprompt}
\eulerimg{27}{images/Vikram Zaky Ardianto_22305144028_Plot 2d-067.png}
\begin{eulercomment}
Berikut adalah contoh lain, di mana string Unicode digunakan dan sumbu
di luar area plot.
\end{eulercomment}
\begin{eulerprompt}
>aspect(1.5); 
>plot2d(["sin(x)","cos(x)"],0,2pi,color=[red,green],<grid,<frame); ...
> xaxis(-1.1,(0:2)*pi,xt=["0",u"&pi;",u"2&pi;"],style="-",>ticks,>zero);  ...
> xgrid((0:0.5:2)*pi,<ticks); ...
> yaxis(-0.1*pi,-1:0.2:1,style="-",>zero,>grid); ...
> labelbox(["sin","cos"],colors=[red,green],x=0.5,y=0.2,>left); ...
> xlabel(u"&phi;"); ylabel(u"f(&phi;)"):
\end{eulerprompt}
\eulerimg{17}{images/Vikram Zaky Ardianto_22305144028_Plot 2d-068.png}
\eulerheading{Merencanakan Data 2D}
\begin{eulercomment}
Jika x dan y adalah vektor data, data ini akan digunakan sebagai
koordinat x dan y dari suatu kurva. Dalam hal ini, a, b, c, dan d,
atau radius r dapat ditentukan, atau jendela plot akan menyesuaikan
secara otomatis dengan data. Atau, \textgreater{}persegi dapat diatur untuk menjaga
rasio aspek persegi.

Memplot ekspresi hanyalah singkatan untuk plot data. Untuk plot data,
Anda memerlukan satu atau beberapa baris nilai x, dan satu atau
beberapa baris nilai y. Dari rentang dan nilai-x, fungsi plot2d akan
menghitung data yang akan diplot, secara default dengan evaluasi
fungsi yang adaptif. Untuk plot titik gunakan "\textgreater{}points", untuk garis
campuran dan titik gunakan "\textgreater{}addpoints".

Tapi Anda bisa memasukkan data secara langsung.

- Gunakan vektor baris untuk x dan y untuk satu fungsi.\\
- Matriks untuk x dan y diplot baris demi baris.

Berikut adalah contoh dengan satu baris untuk x dan y.
\end{eulercomment}
\begin{eulerprompt}
>x=-10:0.1:10; y=exp(-x^2)*x; plot2d(x,y):
\end{eulerprompt}
\eulerimg{17}{images/Vikram Zaky Ardianto_22305144028_Plot 2d-069.png}
\begin{eulercomment}
Data juga dapat diplot sebagai titik. Gunakan poin=true untuk ini.
Plotnya bekerja seperti poligon, tetapi hanya menggambar
sudut-sudutnya.

- style="...": Pilih dari "[]", "\textless{}\textgreater{}", "o", ".", "..", "+", "*", "[]#",
"\textless{} \textgreater{}#", "o#", "..#", "#", "\textbar{}".

Untuk memplot set poin gunakan \textgreater{}points. Jika warna adalah vektor
warna, setiap titik\\
mendapat warna yang berbeda. Untuk matriks koordinat dan vektor kolom,
warna berlaku untuk baris matriks.\\
Parameter \textgreater{}addpoints menambahkan titik ke segmen garis untuk plot
data.
\end{eulercomment}
\begin{eulerprompt}
>xdata=[1,1.5,2.5,3,4]; ydata=[3,3.1,2.8,2.9,2.7]; // data
>plot2d(xdata,ydata,a=0.5,b=4.5,c=2.5,d=3.5,style="."); // lines
>plot2d(xdata,ydata,>points,>add,style="o"): // add points
\end{eulerprompt}
\eulerimg{17}{images/Vikram Zaky Ardianto_22305144028_Plot 2d-070.png}
\begin{eulerprompt}
>p=polyfit(xdata,ydata,1); // get regression line
>plot2d("polyval(p,x)",>add,color=red): // add plot of line
\end{eulerprompt}
\eulerimg{17}{images/Vikram Zaky Ardianto_22305144028_Plot 2d-071.png}
\eulerheading{Menggambar Daerah Yang Dibatasi Kurva}
\begin{eulercomment}
Plot data benar-benar poligon. Kita juga dapat memplot kurva atau
kurva terisi.

- terisi=benar mengisi plot.\\
- style="...": Pilih dari "#", "/", "\textbackslash{}", "\textbackslash{}/".\\
- fillcolor: Lihat di atas untuk warna yang tersedia.

Warna isian ditentukan oleh argumen "fillcolor", dan pada \textless{}outline
opsional mencegah menggambar batas untuk semua gaya kecuali yang
default.
\end{eulercomment}
\begin{eulerprompt}
>t=linspace(0,2pi,1000); // parameter for curve
>x=sin(t)*exp(t/pi); y=cos(t)*exp(t/pi); // x(t) and y(t)
>figure(1,2); aspect(16/9)
>figure(1); plot2d(x,y,r=10); // plot curve
>figure(2); plot2d(x,y,r=10,>filled,style="/",fillcolor=red); // fill curve
>figure(0):
\end{eulerprompt}
\eulerimg{14}{images/Vikram Zaky Ardianto_22305144028_Plot 2d-072.png}
\begin{eulercomment}
Dalam contoh berikut kami memplot elips terisi dan dua segi enam
terisi menggunakan kurva tertutup dengan 6 titik dengan gaya isian
berbeda.
\end{eulercomment}
\begin{eulerprompt}
>x=linspace(0,2pi,1000); plot2d(sin(x),cos(x)*0.5,r=1,>filled,style="/"):
\end{eulerprompt}
\eulerimg{14}{images/Vikram Zaky Ardianto_22305144028_Plot 2d-073.png}
\begin{eulerprompt}
>t=linspace(0,2pi,6); ...
>plot2d(cos(t),sin(t),>filled,style="/",fillcolor=red,r=1.2):
\end{eulerprompt}
\eulerimg{14}{images/Vikram Zaky Ardianto_22305144028_Plot 2d-074.png}
\begin{eulerprompt}
>t=linspace(0,2pi,6); plot2d(cos(t),sin(t),>filled,style="#"):
\end{eulerprompt}
\eulerimg{14}{images/Vikram Zaky Ardianto_22305144028_Plot 2d-075.png}
\begin{eulercomment}
Contoh lainnya adalah segi tujuh, yang kita buat dengan 7 titik pada
lingkaran satuan.
\end{eulercomment}
\begin{eulerprompt}
>t=linspace(0,2pi,7);  ...
> plot2d(cos(t),sin(t),r=1,>filled,style="/",fillcolor=red):
\end{eulerprompt}
\eulerimg{14}{images/Vikram Zaky Ardianto_22305144028_Plot 2d-076.png}
\begin{eulercomment}
Berikut ini adalah himpunan nilai maksimal dari empat kondisi linier
yang kurang dari atau sama dengan 3. Ini adalah A[k].v\textless{}=3 untuk semua
baris A. Untuk mendapatkan sudut yang bagus, kita menggunakan n yang
relatif besar.
\end{eulercomment}
\begin{eulerprompt}
>A=[2,1;1,2;-1,0;0,-1];
>function f(x,y) := max([x,y].A');
>plot2d("f",r=4,level=[0;3],color=green,n=111):
\end{eulerprompt}
\eulerimg{14}{images/Vikram Zaky Ardianto_22305144028_Plot 2d-077.png}
\begin{eulercomment}
Poin utama dari bahasa matriks adalah memungkinkan untuk menghasilkan
tabel fungsi dengan mudah.
\end{eulercomment}
\begin{eulerprompt}
>t=linspace(0,2pi,1000); x=cos(3*t); y=sin(4*t);
\end{eulerprompt}
\begin{eulercomment}
Kami sekarang memiliki vektor x dan y nilai. plot2d() dapat memplot
nilai-nilai ini\\
sebagai kurva yang menghubungkan titik-titik. Plotnya bisa diisi. Pada
kasus ini\\
ini menghasilkan hasil yang bagus karena aturan lilitan, yang
digunakan untuk\\
isi.
\end{eulercomment}
\begin{eulerprompt}
>plot2d(x,y,<grid,<frame,>filled):
\end{eulerprompt}
\eulerimg{14}{images/Vikram Zaky Ardianto_22305144028_Plot 2d-078.png}
\begin{eulercomment}
Sebuah vektor interval diplot terhadap nilai x sebagai daerah terisi\\
antara nilai interval bawah dan atas.

Hal ini dapat berguna untuk memplot kesalahan perhitungan. Tapi itu
bisa\\
juga digunakan untuk memplot kesalahan statistik.
\end{eulercomment}
\begin{eulerprompt}
>t=0:0.1:1; ...
> plot2d(t,interval(t-random(size(t)),t+random(size(t))),style="|");  ...
> plot2d(t,t,add=true):
\end{eulerprompt}
\eulerimg{14}{images/Vikram Zaky Ardianto_22305144028_Plot 2d-079.png}
\begin{eulercomment}
Jika x adalah vektor yang diurutkan, dan y adalah vektor interval,
maka plot2d akan memplot rentang interval yang terisi dalam bidang.
Gaya isian sama dengan gaya poligon.
\end{eulercomment}
\begin{eulerprompt}
>t=-1:0.01:1; x=~t-0.01,t+0.01~; y=x^3-x;
>plot2d(t,y):
\end{eulerprompt}
\eulerimg{14}{images/Vikram Zaky Ardianto_22305144028_Plot 2d-080.png}
\begin{eulercomment}
Dimungkinkan untuk mengisi wilayah nilai untuk fungsi tertentu. Untuk\\
ini, level harus berupa matriks 2xn. Baris pertama adalah batas bawah\\
dan baris kedua berisi batas atas.
\end{eulercomment}
\begin{eulerprompt}
>expr := "2*x^2+x*y+3*y^4+y"; // define an expression f(x,y)
>plot2d(expr,level=[0;1],style="-",color=blue): // 0 <= f(x,y) <= 1
\end{eulerprompt}
\eulerimg{14}{images/Vikram Zaky Ardianto_22305144028_Plot 2d-081.png}
\begin{eulercomment}
Kami juga dapat mengisi rentang nilai seperti

\end{eulercomment}
\begin{eulerformula}
\[
-1 \le (x^2+y^2)^2-x^2+y^2 \le 0.
\]
\end{eulerformula}
\begin{eulercomment}
\end{eulercomment}
\begin{eulerprompt}
>plot2d("(x^2+y^2)^2-x^2+y^2",r=1.2,level=[-1;0],style="/"):
\end{eulerprompt}
\eulerimg{14}{images/Vikram Zaky Ardianto_22305144028_Plot 2d-082.png}
\begin{eulerprompt}
>plot2d("cos(x)","sin(x)^3",xmin=0,xmax=2pi,>filled,style="/"):
\end{eulerprompt}
\eulerimg{14}{images/Vikram Zaky Ardianto_22305144028_Plot 2d-083.png}
\eulerheading{Grafik Fungsi Parametrik}
\begin{eulercomment}
Nilai-x tidak perlu diurutkan. (x,y) hanya menggambarkan kurva. Jika x
diurutkan, kurva tersebut merupakan grafik fungsi.

Dalam contoh berikut, kami memplot spiral

\end{eulercomment}
\begin{eulerformula}
\[
\gamma(t) = t \cdot (\cos(2\pi t),\sin(2\pi t))
\]
\end{eulerformula}
\begin{eulercomment}
Kita perlu menggunakan banyak titik untuk tampilan yang halus atau
fungsi adaptive() untuk mengevaluasi ekspresi (lihat fungsi adaptive()
untuk lebih jelasnya).
\end{eulercomment}
\begin{eulerprompt}
>t=linspace(0,1,1000); ...
>plot2d(t*cos(2*pi*t),t*sin(2*pi*t),r=1):
\end{eulerprompt}
\eulerimg{14}{images/Vikram Zaky Ardianto_22305144028_Plot 2d-084.png}
\begin{eulercomment}
Atau, dimungkinkan untuk menggunakan dua ekspresi untuk kurva. Berikut
ini plot kurva yang sama seperti di atas.
\end{eulercomment}
\begin{eulerprompt}
>plot2d("x*cos(2*pi*x)","x*sin(2*pi*x)",xmin=0,xmax=1,r=1):
\end{eulerprompt}
\eulerimg{14}{images/Vikram Zaky Ardianto_22305144028_Plot 2d-085.png}
\begin{eulerprompt}
>t=linspace(0,1,1000); r=exp(-t); x=r*cos(2pi*t); y=r*sin(2pi*t);
>plot2d(x,y,r=1):
\end{eulerprompt}
\eulerimg{14}{images/Vikram Zaky Ardianto_22305144028_Plot 2d-086.png}
\begin{eulercomment}
Dalam contoh berikutnya, kami memplot kurva

\end{eulercomment}
\begin{eulerformula}
\[
\gamma(t) = (r(t) \cos(t), r(t) \sin(t))
\]
\end{eulerformula}
\begin{eulercomment}
dengan

\end{eulercomment}
\begin{eulerformula}
\[
r(t) = 1 + \dfrac{\sin(3t)}{2}.
\]
\end{eulerformula}
\begin{eulerprompt}
>t=linspace(0,2pi,1000); r=1+sin(3*t)/2; x=r*cos(t); y=r*sin(t); ...
>plot2d(x,y,>filled,fillcolor=red,style="/",r=1.5):
\end{eulerprompt}
\eulerimg{14}{images/Vikram Zaky Ardianto_22305144028_Plot 2d-087.png}
\eulerheading{Menggambar Grafik Bilangan Kompleks}
\begin{eulercomment}
Array bilangan kompleks juga dapat diplot. Kemudian titik-titik grid
akan terhubung. Jika sejumlah garis kisi ditentukan (atau vektor garis
kisi 1x2) dalam argumen cgrid, hanya garis kisi tersebut yang
terlihat.

Matriks bilangan kompleks akan secara otomatis diplot sebagai kisi di
bidang kompleks.

Dalam contoh berikut, kami memplot gambar lingkaran satuan di bawah
fungsi eksponensial. Parameter cgrid menyembunyikan beberapa kurva
grid.
\end{eulercomment}
\begin{eulerprompt}
>aspect(); r=linspace(0,1,50); a=linspace(0,2pi,80)'; z=r*exp(I*a);...
>plot2d(z,a=-1.25,b=1.25,c=-1.25,d=1.25,cgrid=10):
\end{eulerprompt}
\eulerimg{27}{images/Vikram Zaky Ardianto_22305144028_Plot 2d-088.png}
\begin{eulerprompt}
>aspect(1.25); r=linspace(0,1,50); a=linspace(0,2pi,200)'; z=r*exp(I*a);
>plot2d(exp(z),cgrid=[40,10]):
\end{eulerprompt}
\eulerimg{21}{images/Vikram Zaky Ardianto_22305144028_Plot 2d-089.png}
\begin{eulerprompt}
>r=linspace(0,1,10); a=linspace(0,2pi,40)'; z=r*exp(I*a);
>plot2d(exp(z),>points,>add):
\end{eulerprompt}
\eulerimg{21}{images/Vikram Zaky Ardianto_22305144028_Plot 2d-090.png}
\begin{eulercomment}
Sebuah vektor bilangan kompleks secara otomatis diplot sebagai kurva
pada bidang kompleks dengan bagian real dan bagian imajiner.

Dalam contoh, kami memplot lingkaran satuan dengan

\end{eulercomment}
\begin{eulerformula}
\[
\gamma(t) = e^{it}
\]
\end{eulerformula}
\begin{eulerprompt}
>t=linspace(0,2pi,1000); ...
>plot2d(exp(I*t)+exp(4*I*t),r=2):
\end{eulerprompt}
\eulerimg{21}{images/Vikram Zaky Ardianto_22305144028_Plot 2d-091.png}
\eulerheading{Plot Statistik}
\begin{eulercomment}
Ada banyak fungsi yang dikhususkan pada plot statistik. Salah satu
plot yang sering digunakan adalah plot kolom.

Jumlah kumulatif dari nilai terdistribusi 0-1-normal menghasilkan
jalan acak.
\end{eulercomment}
\begin{eulerprompt}
>plot2d(cumsum(randnormal(1,1000))):
\end{eulerprompt}
\eulerimg{21}{images/Vikram Zaky Ardianto_22305144028_Plot 2d-092.png}
\begin{eulercomment}
Menggunakan dua baris menunjukkan jalan dalam dua dimensi.
\end{eulercomment}
\begin{eulerprompt}
>X=cumsum(randnormal(2,1000)); plot2d(X[1],X[2]):
\end{eulerprompt}
\eulerimg{21}{images/Vikram Zaky Ardianto_22305144028_Plot 2d-093.png}
\begin{eulerprompt}
>columnsplot(cumsum(random(10)),style="/",color=blue):
\end{eulerprompt}
\eulerimg{21}{images/Vikram Zaky Ardianto_22305144028_Plot 2d-094.png}
\begin{eulercomment}
Itu juga dapat menampilkan string sebagai label.
\end{eulercomment}
\begin{eulerprompt}
>months=["Jan","Feb","Mar","Apr","May","Jun", ...
>  "Jul","Aug","Sep","Oct","Nov","Dec"];
>values=[10,12,12,18,22,28,30,26,22,18,12,8];
>columnsplot(values,lab=months,color=red,style="-");
>title("Temperature"):
\end{eulerprompt}
\eulerimg{21}{images/Vikram Zaky Ardianto_22305144028_Plot 2d-095.png}
\begin{eulerprompt}
>k=0:10;
>plot2d(k,bin(10,k),>bar):
\end{eulerprompt}
\eulerimg{21}{images/Vikram Zaky Ardianto_22305144028_Plot 2d-096.png}
\begin{eulerprompt}
>plot2d(k,bin(10,k)); plot2d(k,bin(10,k),>points,>add):
\end{eulerprompt}
\eulerimg{21}{images/Vikram Zaky Ardianto_22305144028_Plot 2d-097.png}
\begin{eulerprompt}
>plot2d(normal(1000),normal(1000),>points,grid=6,style=".."):
\end{eulerprompt}
\eulerimg{21}{images/Vikram Zaky Ardianto_22305144028_Plot 2d-098.png}
\begin{eulerprompt}
>plot2d(normal(1,1000),>distribution,style="O"):
\end{eulerprompt}
\eulerimg{21}{images/Vikram Zaky Ardianto_22305144028_Plot 2d-099.png}
\begin{eulerprompt}
>plot2d("qnormal",0,5;2.5,0.5,>filled):
\end{eulerprompt}
\eulerimg{21}{images/Vikram Zaky Ardianto_22305144028_Plot 2d-100.png}
\begin{eulercomment}
Untuk memplot distribusi statistik eksperimental, Anda dapat
menggunakan distribution=n dengan plot2d.
\end{eulercomment}
\begin{eulerprompt}
>w=randexponential(1,1000); // exponential distribution
>plot2d(w,>distribution): // or distribution=n with n intervals
\end{eulerprompt}
\eulerimg{21}{images/Vikram Zaky Ardianto_22305144028_Plot 2d-101.png}
\begin{eulercomment}
Atau Anda dapat menghitung distribusi dari data dan memplot hasilnya
dengan \textgreater{}bar di plot3d, atau dengan plot kolom.
\end{eulercomment}
\begin{eulerprompt}
>w=normal(1000); // 0-1-normal distribution
>\{x,y\}=histo(w,10,v=[-6,-4,-2,-1,0,1,2,4,6]); // interval bounds v
>plot2d(x,y,>bar):
\end{eulerprompt}
\eulerimg{21}{images/Vikram Zaky Ardianto_22305144028_Plot 2d-102.png}
\begin{eulercomment}
Fungsi statplot() menyetel gaya dengan string sederhana.
\end{eulercomment}
\begin{eulerprompt}
>statplot(1:10,cumsum(random(10)),"b"):
\end{eulerprompt}
\eulerimg{21}{images/Vikram Zaky Ardianto_22305144028_Plot 2d-103.png}
\begin{eulerprompt}
>n=10; i=0:n; ...
>plot2d(i,bin(n,i)/2^n,a=0,b=10,c=0,d=0.3); ...
>plot2d(i,bin(n,i)/2^n,points=true,style="ow",add=true,color=blue):
\end{eulerprompt}
\eulerimg{21}{images/Vikram Zaky Ardianto_22305144028_Plot 2d-104.png}
\begin{eulercomment}
Selain itu, data dapat diplot sebagai batang. Dalam hal ini, x harus
diurutkan dan satu elemen lebih panjang dari y. Bilah akan memanjang
dari x[i] ke x[i+1] dengan nilai y[i]. Jika x memiliki ukuran yang
sama dengan y, maka akan diperpanjang satu elemen dengan spasi
terakhir.

Gaya isian dapat digunakan seperti di atas.
\end{eulercomment}
\begin{eulerprompt}
>n=10; k=bin(n,0:n); ...
>plot2d(-0.5:n+0.5,k,bar=true,fillcolor=lightgray):
\end{eulerprompt}
\eulerimg{21}{images/Vikram Zaky Ardianto_22305144028_Plot 2d-105.png}
\begin{eulercomment}
Data untuk plot batang (bar=1) dan histogram (histogram=1) dapat
dinyatakan secara eksplisit dalam xv dan yv, atau dapat dihitung dari
distribusi empiris dalam xv dengan \textgreater{}distribution (atau
distribution=n). Histogram nilai xv akan dihitung secara otomatis
dengan \textgreater{}histogram. Jika \textgreater{}even ditentukan, nilai xv akan dihitung dalam
interval bilangan bulat.
\end{eulercomment}
\begin{eulerprompt}
>plot2d(normal(10000),distribution=50):
\end{eulerprompt}
\eulerimg{21}{images/Vikram Zaky Ardianto_22305144028_Plot 2d-106.png}
\begin{eulerprompt}
>k=0:10; m=bin(10,k); x=(0:11)-0.5; plot2d(x,m,>bar):
\end{eulerprompt}
\eulerimg{21}{images/Vikram Zaky Ardianto_22305144028_Plot 2d-107.png}
\begin{eulerprompt}
>columnsplot(m,k):
\end{eulerprompt}
\eulerimg{21}{images/Vikram Zaky Ardianto_22305144028_Plot 2d-108.png}
\begin{eulerprompt}
>plot2d(random(600)*6,histogram=6):
\end{eulerprompt}
\eulerimg{21}{images/Vikram Zaky Ardianto_22305144028_Plot 2d-109.png}
\begin{eulercomment}
Untuk distribusi, ada parameter distribution=n, yang menghitung nilai
secara otomatis dan mencetak distribusi relatif dengan n sub-interval.
\end{eulercomment}
\begin{eulerprompt}
>plot2d(normal(1,1000),distribution=10,style="\(\backslash\)/"):
\end{eulerprompt}
\eulerimg{21}{images/Vikram Zaky Ardianto_22305144028_Plot 2d-110.png}
\begin{eulercomment}
Dengan parameter even=true, ini akan menggunakan interval integer.
\end{eulercomment}
\begin{eulerprompt}
>plot2d(intrandom(1,1000,10),distribution=10,even=true):
\end{eulerprompt}
\eulerimg{21}{images/Vikram Zaky Ardianto_22305144028_Plot 2d-111.png}
\begin{eulercomment}
Perhatikan bahwa ada banyak plot statistik, yang mungkin berguna.
Silahkan lihat tutorial tentang statistik.
\end{eulercomment}
\begin{eulerprompt}
>columnsplot(getmultiplicities(1:6,intrandom(1,6000,6))):
\end{eulerprompt}
\eulerimg{21}{images/Vikram Zaky Ardianto_22305144028_Plot 2d-112.png}
\begin{eulerprompt}
>plot2d(normal(1,1000),>distribution); ...
>  plot2d("qnormal(x)",color=red,thickness=2,>add):
\end{eulerprompt}
\eulerimg{21}{images/Vikram Zaky Ardianto_22305144028_Plot 2d-113.png}
\begin{eulercomment}
Ada juga banyak plot khusus untuk statistik. Boxplot menunjukkan
kuartil dari distribusi ini dan banyak outlier. Menurut definisi,
outlier dalam boxplot adalah data yang melebihi 1,5 kali kisaran 50\%
tengah plot.
\end{eulercomment}
\begin{eulerprompt}
>M=normal(5,1000); boxplot(quartiles(M)):
\end{eulerprompt}
\eulerimg{21}{images/Vikram Zaky Ardianto_22305144028_Plot 2d-114.png}
\eulerheading{Fungsi Implisit}
\begin{eulercomment}
Plot implisit menunjukkan garis level yang menyelesaikan f(x,y)=level,
di mana "level" dapat berupa nilai tunggal atau vektor nilai. Jika
level="auto", akan ada garis level nc, yang akan menyebar antara
fungsi minimum dan maksimum secara merata. Warna yang lebih gelap atau
lebih terang dapat ditambahkan dengan \textgreater{}hue untuk menunjukkan nilai
fungsi. Untuk fungsi implisit, xv harus berupa fungsi atau ekspresi
dari parameter x dan y, atau, sebagai alternatif, xv dapat berupa
matriks nilai.

Euler dapat menandai garis level

\end{eulercomment}
\begin{eulerformula}
\[
f(x,y) = c
\]
\end{eulerformula}
\begin{eulercomment}
dari fungsi apapun.

Untuk menggambar himpunan f(x,y)=c untuk satu atau lebih konstanta c,
Anda dapat menggunakan plot2d() dengan plot implisitnya di dalam
bidang. Parameter untuk c adalah level=c, di mana c dapat berupa
vektor garis level. Selain itu, skema warna dapat digambar di latar
belakang untuk menunjukkan nilai fungsi untuk setiap titik dalam plot.
Parameter "n" menentukan kehalusan plot.
\end{eulercomment}
\begin{eulerprompt}
>aspect(1.5); 
>plot2d("x^2+y^2-x*y-x",r=1.5,level=0,contourcolor=red):
\end{eulerprompt}
\eulerimg{17}{images/Vikram Zaky Ardianto_22305144028_Plot 2d-115.png}
\begin{eulerprompt}
>expr := "2*x^2+x*y+3*y^4+y"; // define an expression f(x,y)
>plot2d(expr,level=0): // Solutions of f(x,y)=0
\end{eulerprompt}
\eulerimg{17}{images/Vikram Zaky Ardianto_22305144028_Plot 2d-116.png}
\begin{eulerprompt}
>plot2d(expr,level=0:0.5:20,>hue,contourcolor=white,n=200): // nice
\end{eulerprompt}
\eulerimg{17}{images/Vikram Zaky Ardianto_22305144028_Plot 2d-117.png}
\begin{eulerprompt}
>plot2d(expr,level=0:0.5:20,>hue,>spectral,n=200,grid=4): // nicer
\end{eulerprompt}
\eulerimg{17}{images/Vikram Zaky Ardianto_22305144028_Plot 2d-118.png}
\begin{eulercomment}
Ini berfungsi untuk plot data juga. Tetapi Anda harus menentukan
rentangnya\\
untuk label sumbu.
\end{eulercomment}
\begin{eulerprompt}
>x=-2:0.05:1; y=x'; z=expr(x,y);
>plot2d(z,level=0,a=-1,b=2,c=-2,d=1,>hue):
\end{eulerprompt}
\eulerimg{17}{images/Vikram Zaky Ardianto_22305144028_Plot 2d-119.png}
\begin{eulerprompt}
>plot2d("x^3-y^2",>contour,>hue,>spectral):
\end{eulerprompt}
\eulerimg{17}{images/Vikram Zaky Ardianto_22305144028_Plot 2d-120.png}
\begin{eulerprompt}
>plot2d("x^3-y^2",level=0,contourwidth=3,>add,contourcolor=red):
\end{eulerprompt}
\eulerimg{17}{images/Vikram Zaky Ardianto_22305144028_Plot 2d-121.png}
\begin{eulerprompt}
>z=z+normal(size(z))*0.2;
>plot2d(z,level=0.5,a=-1,b=2,c=-2,d=1):
\end{eulerprompt}
\eulerimg{17}{images/Vikram Zaky Ardianto_22305144028_Plot 2d-122.png}
\begin{eulerprompt}
>plot2d(expr,level=[0:0.2:5;0.05:0.2:5.05],color=lightgray):
\end{eulerprompt}
\eulerimg{17}{images/Vikram Zaky Ardianto_22305144028_Plot 2d-123.png}
\begin{eulerprompt}
>plot2d("x^2+y^3+x*y",level=1,r=4,n=100):
\end{eulerprompt}
\eulerimg{17}{images/Vikram Zaky Ardianto_22305144028_Plot 2d-124.png}
\begin{eulerprompt}
>plot2d("x^2+2*y^2-x*y",level=0:0.1:10,n=100,contourcolor=white,>hue):
\end{eulerprompt}
\eulerimg{17}{images/Vikram Zaky Ardianto_22305144028_Plot 2d-125.png}
\begin{eulercomment}
Juga dimungkinkan untuk mengisi set

\end{eulercomment}
\begin{eulerformula}
\[
a \le f(x,y) \le b
\]
\end{eulerformula}
\begin{eulercomment}
dengan rentang tingkat.

Dimungkinkan untuk mengisi wilayah nilai untuk fungsi tertentu. Untuk
ini, level harus berupa matriks 2xn. Baris pertama adalah batas bawah
dan baris kedua berisi batas atas.
\end{eulercomment}
\begin{eulerprompt}
>plot2d(expr,level=[0;1],style="-",color=blue): // 0 <= f(x,y) <= 1
\end{eulerprompt}
\eulerimg{17}{images/Vikram Zaky Ardianto_22305144028_Plot 2d-126.png}
\begin{eulercomment}
Plot implisit juga dapat menunjukkan rentang level. Kemudian level
harus berupa matriks 2xn dari interval level, di mana baris pertama
berisi awal dan baris kedua adalah akhir dari setiap interval. Atau,
vektor baris sederhana dapat digunakan untuk level, dan parameter dl
memperluas nilai level ke interval.
\end{eulercomment}
\begin{eulerprompt}
>plot2d("x^4+y^4",r=1.5,level=[0;1],color=blue,style="/"):
\end{eulerprompt}
\eulerimg{17}{images/Vikram Zaky Ardianto_22305144028_Plot 2d-127.png}
\begin{eulerprompt}
>plot2d("x^2+y^3+x*y",level=[0,2,4;1,3,5],style="/",r=2,n=100):
\end{eulerprompt}
\eulerimg{17}{images/Vikram Zaky Ardianto_22305144028_Plot 2d-128.png}
\begin{eulerprompt}
>plot2d("x^2+y^3+x*y",level=-10:20,r=2,style="-",dl=0.1,n=100):
\end{eulerprompt}
\eulerimg{17}{images/Vikram Zaky Ardianto_22305144028_Plot 2d-129.png}
\begin{eulerprompt}
>plot2d("sin(x)*cos(y)",r=pi,>hue,>levels,n=100):
\end{eulerprompt}
\eulerimg{17}{images/Vikram Zaky Ardianto_22305144028_Plot 2d-130.png}
\begin{eulercomment}
Dimungkinkan juga untuk menandai suatu wilayah

\end{eulercomment}
\begin{eulerformula}
\[
a \le f(x,y) \le b.
\]
\end{eulerformula}
\begin{eulercomment}
Ini dilakukan dengan menambahkan level dengan dua baris.
\end{eulercomment}
\begin{eulerprompt}
>plot2d("(x^2+y^2-1)^3-x^2*y^3",r=1.3, ...
>  style="#",color=red,<outline, ...
>  level=[-2;0],n=100):
\end{eulerprompt}
\eulerimg{17}{images/Vikram Zaky Ardianto_22305144028_Plot 2d-131.png}
\begin{eulercomment}
Dimungkinkan untuk menentukan level tertentu. Misalnya, kita dapat
memplot solusi persamaan seperti

\end{eulercomment}
\begin{eulerformula}
\[
x^3-xy+x^2y^2=6
\]
\end{eulerformula}
\begin{eulerprompt}
>plot2d("x^3-x*y+x^2*y^2",r=6,level=1,n=100):
\end{eulerprompt}
\eulerimg{17}{images/Vikram Zaky Ardianto_22305144028_Plot 2d-132.png}
\begin{eulerprompt}
>function starplot1 (v, style="/", color=green, lab=none) ...
\end{eulerprompt}
\begin{eulerudf}
    if !holding() then clg; endif;
    w=window(); window(0,0,1024,1024);
    h=holding(1);
    r=max(abs(v))*1.2;
    setplot(-r,r,-r,r);
    n=cols(v); t=linspace(0,2pi,n);
    v=v|v[1]; c=v*cos(t); s=v*sin(t);
    cl=barcolor(color); st=barstyle(style);
    loop 1 to n
      polygon([0,c[#],c[#+1]],[0,s[#],s[#+1]],1);
      if lab!=none then
        rlab=v[#]+r*0.1;
        \{col,row\}=toscreen(cos(t[#])*rlab,sin(t[#])*rlab);
        ctext(""+lab[#],col,row-textheight()/2);
      endif;
    end;
    barcolor(cl); barstyle(st);
    holding(h);
    window(w);
  endfunction
\end{eulerudf}
\begin{eulercomment}
Tidak ada kotak atau sumbu kutu di sini. Selain itu, kami menggunakan
jendela penuh untuk plot.

Kami memanggil reset sebelum kami menguji plot ini untuk mengembalikan
default grafis. Ini tidak perlu, jika Anda yakin plot Anda berhasil.
\end{eulercomment}
\begin{eulerprompt}
>reset; starplot1(normal(1,10)+5,color=red,lab=1:10):
\end{eulerprompt}
\eulerimg{27}{images/Vikram Zaky Ardianto_22305144028_Plot 2d-133.png}
\begin{eulercomment}
Terkadang, Anda mungkin ingin merencanakan sesuatu yang tidak dapat
dilakukan plot2d, tetapi hampir.

Dalam fungsi berikut, kami melakukan plot impuls logaritmik. plot2d
dapat melakukan plot logaritmik, tetapi tidak untuk batang impuls.
\end{eulercomment}
\begin{eulerprompt}
>function logimpulseplot1 (x,y) ...
\end{eulerprompt}
\begin{eulerudf}
    \{x0,y0\}=makeimpulse(x,log(y)/log(10));
    plot2d(x0,y0,>bar,grid=0);
    h=holding(1);
    frame();
    xgrid(ticks(x));
    p=plot();
    for i=-10 to 10;
      if i<=p[4] and i>=p[3] then
         ygrid(i,yt="10^"+i);
      endif;
    end;
    holding(h);
  endfunction
\end{eulerudf}
\begin{eulercomment}
Mari kita uji dengan nilai yang terdistribusi secara eksponensial.
\end{eulercomment}
\begin{eulerprompt}
>aspect(1.5); x=1:10; y=-log(random(size(x)))*200; ...
>logimpulseplot1(x,y):
\end{eulerprompt}
\eulerimg{17}{images/Vikram Zaky Ardianto_22305144028_Plot 2d-134.png}
\begin{eulercomment}
Mari kita menganimasikan kurva 2D menggunakan plot langsung. Perintah
plot(x,y) hanya memplot kurva ke jendela plot. setplot(a,b,c,d)
mengatur jendela ini.

Fungsi wait(0) memaksa plot untuk muncul di jendela grafik. Jika
tidak, menggambar ulang terjadi dalam interval waktu yang jarang.
\end{eulercomment}
\begin{eulerprompt}
>function animliss (n,m) ...
\end{eulerprompt}
\begin{eulerudf}
  t=linspace(0,2pi,500);
  f=0;
  c=framecolor(0);
  l=linewidth(2);
  setplot(-1,1,-1,1);
  repeat
    clg;
    plot(sin(n*t),cos(m*t+f));
    wait(0);
    if testkey() then break; endif;
    f=f+0.02;
  end;
  framecolor(c);
  linewidth(l);
  endfunction
\end{eulerudf}
\begin{eulercomment}
Tekan sembarang tombol untuk menghentikan animasi ini.
\end{eulercomment}
\begin{eulerprompt}
>animliss(2,3); // lihat hasilnya, jika sudah puas, tekan ENTER
\end{eulerprompt}
\eulerheading{Plot Logaritmik}
\begin{eulercomment}
EMT menggunakan parameter "logplot" untuk skala logaritmik.\\
Plot logaritma dapat diplot baik menggunakan skala logaritma dalam y
dengan logplot=1, atau menggunakan skala logaritma dalam x dan y
dengan logplot=2, atau dalam x dengan logplot=3.

\end{eulercomment}
\begin{eulerttcomment}
 - logplot=1: y-logaritma
 - logplot=2: x-y-logaritma
 - logplot=3: x-logaritma
\end{eulerttcomment}
\begin{eulerprompt}
>plot2d("exp(x^3-x)*x^2",1,5,logplot=1):
\end{eulerprompt}
\eulerimg{17}{images/Vikram Zaky Ardianto_22305144028_Plot 2d-135.png}
\begin{eulerprompt}
>plot2d("exp(x+sin(x))",0,100,logplot=1):
\end{eulerprompt}
\eulerimg{17}{images/Vikram Zaky Ardianto_22305144028_Plot 2d-136.png}
\begin{eulerprompt}
>plot2d("exp(x+sin(x))",10,100,logplot=2):
\end{eulerprompt}
\eulerimg{17}{images/Vikram Zaky Ardianto_22305144028_Plot 2d-137.png}
\begin{eulerprompt}
>plot2d("gamma(x)",1,10,logplot=1):
\end{eulerprompt}
\eulerimg{17}{images/Vikram Zaky Ardianto_22305144028_Plot 2d-138.png}
\begin{eulerprompt}
>plot2d("log(x*(2+sin(x/100)))",10,1000,logplot=3):
\end{eulerprompt}
\eulerimg{17}{images/Vikram Zaky Ardianto_22305144028_Plot 2d-139.png}
\begin{eulercomment}
Ini juga berfungsi dengan plot data.
\end{eulercomment}
\begin{eulerprompt}
>x=10^(1:20); y=x^2-x;
>plot2d(x,y,logplot=2):
\end{eulerprompt}
\eulerimg{17}{images/Vikram Zaky Ardianto_22305144028_Plot 2d-140.png}
\end{eulernotebook}
\end{document}


\newpage
\chapter{KB Pekan 5: Menggunakan EMT untuk mengambar grafik 3 dimensi (3D)}
\documentclass{article}

\usepackage{eumat}

\begin{document}
\begin{eulernotebook}
\eulerheading{Penggunaan Software EMT untuk Plot 3D}
\begin{eulercomment}
Vikram Zaky Ardianto\\
22305144028\\
MATEMATIKA E 2022

\begin{eulercomment}
\eulerheading{1. Menggambar Grafik Fungsi Dua Variabel}
\begin{eulercomment}
* dalam Bentuk Ekspresi Langsung 
Fungsi Dua Variabel didefinisikan sebagai sebuah fungsi bernilai real
dari dua variabel real, yakni fungsi f yang memadankan setiap pasangan\\
terurut (x,y) pada suatu himpunan D dari bidang dengan bilangan real\\
tunggal f (x,y).

Di dalam program numerik EMT, ekspresi adalah string. Jika ditandai
sebagai simbolis, mereka akan mencetak melalui Maxima, jika tidak
melalui EMT. Ekspresi dalam string digunakan untuk membuat plot dan
banyak fungsi numerik. Untuk ini, variabel dalam ekspresi harus "x"
dan "y".

Untuk grafik suatu fungsi, gunakan

- ekspresi sederhana dalam x dan y,\\
- nama fungsi dari dua variabel\\
- atau matriks data.

\end{eulercomment}
\eulersubheading{Grafik Fungsi Linear}
\begin{eulercomment}
Fungsi linear dua variabel biasanya dinyatakan dalam bentuk\\
\end{eulercomment}
\begin{eulerformula}
\[
f(x,y)=ax+by+c
\]
\end{eulerformula}
\begin{eulerprompt}
>plot3d("x^3+2*y^2"):
\end{eulerprompt}
\eulerimg{17}{images/Vikram Zaky Ardianto_22305144028_Plot 3D-002.png}
\begin{eulerprompt}
>plot3d("x^2+3*y^2"):
\end{eulerprompt}
\eulerimg{17}{images/Vikram Zaky Ardianto_22305144028_Plot 3D-003.png}
\eulersubheading{Grafik Fungsi Kuadrat}
\begin{eulercomment}
Fungsi kuadrat dua variabel biasanya dinyatakan dalam bentuk\\
\end{eulercomment}
\begin{eulerformula}
\[
f(x,y)=ax^2+by^2+cxy+dx+ey+f
\]
\end{eulerformula}
\begin{eulerprompt}
>plot3d("2*x^2*y+2*y^2"):
\end{eulerprompt}
\eulerimg{17}{images/Vikram Zaky Ardianto_22305144028_Plot 3D-004.png}
\begin{eulerprompt}
>plot3d("2*x^2+2*y^2+-x+y-4"):
\end{eulerprompt}
\eulerimg{17}{images/Vikram Zaky Ardianto_22305144028_Plot 3D-005.png}
\eulersubheading{Grafik Fungsi Akar Kuadrat}
\begin{eulerprompt}
>plot3d("sqrt(2*x^2+3*y^2)"):
\end{eulerprompt}
\eulerimg{17}{images/Vikram Zaky Ardianto_22305144028_Plot 3D-006.png}
\begin{eulerprompt}
>plot3d("sqrt(x^3+4*y^2)"):
\end{eulerprompt}
\eulerimg{17}{images/Vikram Zaky Ardianto_22305144028_Plot 3D-007.png}
\eulersubheading{Grafik Fungsi Trigonometri}
\begin{eulerprompt}
>plot3d("2*cos(x)*sin(y)"):
\end{eulerprompt}
\eulerimg{17}{images/Vikram Zaky Ardianto_22305144028_Plot 3D-008.png}
\begin{eulerprompt}
>aspect(2); plot3d("2*x^2+sin(y)",-6,6,0,6*pi):
\end{eulerprompt}
\eulerimg{13}{images/Vikram Zaky Ardianto_22305144028_Plot 3D-009.png}
\begin{eulercomment}
1. aspect(1.5) mengatur aspek rasio pada grafik 3D.\\
2. plot3d("2*x\textasciicircum{}2+sin(y)",-6,6,0,6*pi) adalah fungsi matematika yang
digunakan untuk membuat grafik 3D.\\
3. -5,5 mengatur rentang sumbu x yang akan ditampilkan pada grafik.\\
4. 0,6*pi mengatur rentang sumbu y yang akan ditampilkan pada grafik.

\end{eulercomment}
\eulersubheading{Grafik Fungsi Eksponensial}
\begin{eulercomment}
Fungsi eksponensial dua variabel bisa dinyatakan\\
\end{eulercomment}
\begin{eulerformula}
\[
f(x,y)=a.b^{xy}
\]
\end{eulerformula}
\begin{eulerprompt}
>plot3d("10*2^(2*x*y)"):
\end{eulerprompt}
\eulerimg{13}{images/Vikram Zaky Ardianto_22305144028_Plot 3D-011.png}
\begin{eulerprompt}
>plot3d("-3^(x*y)"):
\end{eulerprompt}
\eulerimg{13}{images/Vikram Zaky Ardianto_22305144028_Plot 3D-012.png}
\eulersubheading{Grafik Fungsi Logaritma}
\begin{eulercomment}
Fungsi logaritma dua variabel bisa dinyatakan sebagai\\
\end{eulercomment}
\begin{eulerformula}
\[
f(x,y)=log_b(xy)
\]
\end{eulerformula}
\begin{eulerprompt}
>plot3d("log(2*x*y)"):
\end{eulerprompt}
\eulerimg{13}{images/Vikram Zaky Ardianto_22305144028_Plot 3D-013.png}
\begin{eulerprompt}
>plot3d("log(3*x*y)"):
\end{eulerprompt}
\eulerimg{13}{images/Vikram Zaky Ardianto_22305144028_Plot 3D-014.png}
\begin{eulerprompt}
> 
\end{eulerprompt}
\eulerheading{2. Menggambar Grafik Fungsi Dua Variabel yang}
\begin{eulercomment}
* Rumusnya Disimpan dalam Variabel Ekspresi
Fungsi ini dapat memplot plot 3D dengan grafik fungsi dua\\
variabel, permukaan berparameter, kurva ruang, awan titik,\\
penyelesaian persamaan tiga variabel. Semua plot 3D bisa\\
ditampilkan sebagai anaglyph.

fungsi plot3d (x, y, z, xmin, xmax, ymin, ymax, n, a

Parameter\\
x : ekspresi dalam x dan y\\
x,y,z : matriks koordinat suatu permukaan\\
x,y,z : ekspresi dalam x dan y untuk permukaan parametrik\\
x,y,z : ekspresi dalam x untuk memplot kurva ruang

xmin,xmax,ymin,ymax :\\
\end{eulercomment}
\begin{eulerttcomment}
  x,y batas ekspresi
\end{eulerttcomment}
\begin{eulercomment}

contoh:

ekspresi dalam string
\end{eulercomment}
\begin{eulerprompt}
>expr := "x^2+sin(y)"
\end{eulerprompt}
\begin{euleroutput}
  x^2+sin(y)
\end{euleroutput}
\begin{eulercomment}
plot ekspresi
\end{eulercomment}
\begin{eulerprompt}
>plot3d(expr,-5,5,0,6*pi):
\end{eulerprompt}
\eulerimg{17}{images/Vikram Zaky Ardianto_22305144028_Plot 3D-015.png}
\begin{eulercomment}
1. x\textasciicircum{}2+sin(y) adalah fungsi matematika yang digunakan untuk membuat
grafik 3D.\\
2. -5,5 mengatur rentang sumbu x yang akan ditampilkan pada grafik.\\
3. 0,6*pi mengatur rentang sumbu y yang akan ditampilkan pada grafik.
\end{eulercomment}
\begin{eulerprompt}
>expr := "4*x^3*y" 
\end{eulerprompt}
\begin{euleroutput}
  4*x^3*y
\end{euleroutput}
\begin{eulerprompt}
>aspect(2); plot3d(expr): 
\end{eulerprompt}
\eulerimg{13}{images/Vikram Zaky Ardianto_22305144028_Plot 3D-016.png}
\begin{eulercomment}
1. aspect(2) mengatur aspek rasio pada grafik 3D.\\
2. plot3d(expr) adalah fungsi matematika yang digunakan untuk membuat
grafik 3D.
\end{eulercomment}
\begin{eulerprompt}
>expr := "cos(x)*sin(y)"
\end{eulerprompt}
\begin{euleroutput}
  cos(x)*sin(y)
\end{euleroutput}
\begin{eulerprompt}
>plot3d(expr): 
\end{eulerprompt}
\eulerimg{17}{images/Vikram Zaky Ardianto_22305144028_Plot 3D-017.png}
\begin{eulerprompt}
>expr := "y^2-x^2"
\end{eulerprompt}
\begin{euleroutput}
  y^2-x^2
\end{euleroutput}
\begin{eulerprompt}
>aspect(1.5); plot3d(expr,-5,5,-5,5):
\end{eulerprompt}
\eulerimg{17}{images/Vikram Zaky Ardianto_22305144028_Plot 3D-018.png}
\begin{eulercomment}
1. aspect(1.5) mengatur aspek rasio pada grafik 3D.\\
2. plot3d(expr,-5,5,-5,5) adalah fungsi matematika yang digunakan
untuk membuat grafik 3D.\\
3. -5,5 mengatur rentang sumbu x yang akan ditampilkan pada grafik.\\
4. -5,5 mengatur rentang sumbu y yang akan ditampilkan pada grafik.

\end{eulercomment}
\eulersubheading{Fungsi umum untuk plot 3D.}
\begin{eulercomment}
Fungsi plot3d (x, y, z, xmin, xmax, ymin, ymax, n, a,  ..,\\
c, d, r, scale, fscale, frame, angle, height, zoom, distance, ..)

Rentang plot untuk fungsi dapat ditentukan dengan\\
- a,b: rentang x\\
- c,d: rentang y\\
- r : persegi simetris di sekitar (0,0).\\
- n : jumlah subinterval untuk plot.

Ada beberapa parameter untuk menskalakan fungsi atau mengubah tampilan
grafik.\\
- fscale: menskalakan ke nilai fungsi (defaultnya adalah \textless{}fscale).\\
- scale: angka atau vektor 1x2 untuk menskalakan ke arah x dan y.\\
- frame: jenis bingkai (default 1).

Tampilan dapat diubah dengan berbagai cara.\\
- distance: jarak pandang ke plot.\\
- zoom: nilai zoom.\\
- angle: sudut terhadap sumbu y negatif dalam radian.\\
- height: ketinggian pandangan dalam radian.

Nilai default dapat diperiksa atau diubah dengan fungsi view(). Ini
mengembalikan parameter dalam urutan di atas.
\end{eulercomment}
\begin{eulerprompt}
>view 
\end{eulerprompt}
\begin{euleroutput}
  [5,  2.6,  2,  0.4]
\end{euleroutput}
\begin{eulercomment}
Jarak yang lebih dekat membutuhkan lebih sedikit zoom. Efeknya lebih
seperti lensa sudut lebar.

contoh soal:
\end{eulercomment}
\begin{eulerprompt}
>plot3d("exp(-(x^2+y^2)/5)",r=10,n=80,fscale=4,scale=1.2,frame=3,>user):
\end{eulerprompt}
\eulerimg{17}{images/Vikram Zaky Ardianto_22305144028_Plot 3D-019.png}
\begin{eulercomment}
1. exp(-(x\textasciicircum{}2+y\textasciicircum{}2)/5) adalah fungsi matematika yang digunakan untuk
membuat grafik 3D.\\
2. r=10 mengatur jarak maksimum dari pusat grafik ke tepi grafik.\\
3. n=80 mengatur jumlah titik yang digunakan untuk membuat grafik.\\
4. fscale=4 mengatur faktor skala untuk warna.\\
5. scale=1.2 mengatur faktor skala untuk ukuran grafik.\\
6. frame=3 mengatur jenis bingkai yang digunakan untuk grafik.
\end{eulercomment}
\begin{eulercomment}
Pada contoh berikut, sudut=0 dan tinggi=0 dilihat dari sumbu y
negatif. Label sumbu untuk y disembunyikan dalam kasus ini.
\end{eulercomment}
\begin{eulerprompt}
>plot3d("x^2+y",distance=3,zoom=1,angle=pi/2,height=0):
\end{eulerprompt}
\eulerimg{17}{images/Vikram Zaky Ardianto_22305144028_Plot 3D-020.png}
\begin{eulercomment}
1. x\textasciicircum{}2+y adalah fungsi matematika yang digunakan untuk membuat grafik
3D.\\
2. distance=3 mengatur jarak pandang dari grafik.\\
3. zoom=1 mengatur faktor perbesaran grafik.\\
4. angle=pi/2 mengatur sudut pandang grafik dalam radian.\\
5. height=0 mengatur ketinggian pandangan dari grafik.

Plot selalu terlihat berada di tengah kubus plot. Anda dapat
memindahkan bagian tengah dengan parameter tengah.
\end{eulercomment}
\begin{eulerprompt}
>plot3d("x^4+y^2",a=0,b=1,c=-1,d=1,angle=-20°,height=20°, ...
>  center=[0.4,0,0],zoom=5):
\end{eulerprompt}
\eulerimg{17}{images/Vikram Zaky Ardianto_22305144028_Plot 3D-021.png}
\begin{eulercomment}
1. x\textasciicircum{}4+y\textasciicircum{}2 adalah fungsi matematika yang digunakan untuk membuat
grafik 3D.\\
2. a=0,b=1,c=-1,d=1 mengatur rentang sumbu x dan y yang akan
ditampilkan pada grafik.\\
3. angle=-20° mengatur sudut pandang grafik dalam derajat.\\
4. height=20° mengatur ketinggian pandangan dari grafik dalam derajat.\\
5. center=[0.4,0,0] mengatur pusat pandangan dari grafik.\\
6. zoom=5 mengatur faktor perbesaran grafik.

Plotnya diskalakan agar sesuai dengan unit kubus untuk dilihat. Jadi
tidak perlu mengubah jarak atau zoom tergantung ukuran plot. Namun
labelnya mengacu pada ukuran sebenarnya.

Jika Anda mematikannya dengan scale=false, Anda harus berhati-hati
agar plot tetap masuk ke dalam jendela plotting, dengan mengubah jarak
pandang atau zoom, dan memindahkan bagian tengah.

\end{eulercomment}
\begin{eulerprompt}
>plot3d("5*exp(-x^2-y^2)",r=2,<fscale,<scale,distance=13,height=50°, ...
>  center=[0,0,-2],frame=3):
\end{eulerprompt}
\eulerimg{17}{images/Vikram Zaky Ardianto_22305144028_Plot 3D-022.png}
\begin{eulercomment}
1. 5*exp(-x\textasciicircum{}2-y\textasciicircum{}2) adalah fungsi matematika yang digunakan untuk
membuat grafik 3D.\\
2. r=2 mengatur jarak maksimum dari pusat grafik ke tepi grafik.\\
3. \textless{}fscale mengatur faktor skala untuk warna.\\
4. \textless{}scale mengatur faktor skala untuk ukuran grafik.\\
5. distance=13 mengatur jarak pandang dari grafik.\\
6. height=50° mengatur ketinggian pandangan dari grafik dalam derajat.\\
7. center=[0,0,-2] mengatur pusat pandangan dari grafik.\\
8. frame=3 mengatur jenis bingkai yang digunakan untuk grafik.

Plot kutub juga tersedia. Parameter polar=true menggambar plot kutub.
Fungsi tersebut harus tetap merupakan fungsi dari x dan y. Parameter
"fscale" menskalakan fungsi dengan skalanya sendiri. Kalau tidak,
fungsinya akan diskalakan agar sesuai dengan kubus.
\end{eulercomment}
\begin{eulerprompt}
>plot3d("1/(x^2+y^2+1)",r=5,>polar, ...
>fscale=2,>hue,n=100,zoom=4,>contour,color=blue):
\end{eulerprompt}
\eulerimg{17}{images/Vikram Zaky Ardianto_22305144028_Plot 3D-023.png}
\begin{eulercomment}
1. 1/(x\textasciicircum{}2+y\textasciicircum{}2+1) adalah fungsi matematika yang digunakan untuk membuat
grafik 3D.\\
2. r=5 mengatur jarak maksimum dari pusat grafik ke tepi grafik.\\
3. polar mengatur tampilan grafik dalam koordinat polar.\\
4. fscale=2 mengatur faktor skala untuk warna.\\
5. hue mengatur skala warna yang digunakan pada grafik.\\
6. n=100 mengatur jumlah titik yang digunakan untuk membuat grafik.\\
7. zoom=4 mengatur faktor perbesaran grafik.\\
8. contour mengatur tampilan garis kontur pada grafik.\\
9. color=blue mengatur warna garis kontur pada grafik.
\end{eulercomment}
\begin{eulerprompt}
>function f(r) := exp(-r/2)*cos(r); ...
>plot3d("f(x^2+y^2)",>polar,scale=[1,1,0.4],r=pi,frame=3,zoom=4):
\end{eulerprompt}
\eulerimg{17}{images/Vikram Zaky Ardianto_22305144028_Plot 3D-024.png}
\begin{eulercomment}
1. function f(r) := exp(-r/2)*cos(r) adalah fungsi matematika yang
didefinisikan sebagai f(r) = e\textasciicircum{}(-r/2) * cos(r).\\
2. plot3d("f(x\textasciicircum{}2+y\textasciicircum{}2)",polar,scale=[1,1,0.4],r=pi,frame=3,zoom=4)
adalah perintah untuk membuat grafik 3D dari fungsi f(x\textasciicircum{}2+y\textasciicircum{}2).\\
3. polar mengatur tampilan grafik dalam koordinat polar.\\
4. scale=[1,1,0.4] mengatur faktor skala untuk ukuran grafik.\\
5. r=pi mengatur jarak maksimum dari pusat grafik ke tepi grafik.\\
6. frame=3 mengatur jenis bingkai yang digunakan untuk grafik.\\
7. zoom=4 mengatur faktor perbesaran grafik.

Parameter memutar memutar fungsi di x di sekitar sumbu x.

- rotate=1: Menggunakan sumbu x\\
- rotate=2: Menggunakan sumbu z
\end{eulercomment}
\begin{eulerprompt}
>plot3d("x^2+1",a=-1,b=1,rotate=true,grid=5):
\end{eulerprompt}
\eulerimg{17}{images/Vikram Zaky Ardianto_22305144028_Plot 3D-025.png}
\begin{eulercomment}
1. x\textasciicircum{}2+1 adalah fungsi matematika yang digunakan untuk membuat grafik
3D.\\
2. a=-1,b=1 mengatur rentang sumbu x yang akan ditampilkan pada
grafik.\\
3. rotate=true mengatur grafik agar dapat diputar secara interaktif.\\
4. grid=5 mengatur jumlah garis koordinat yang ditampilkan pada
grafik.
\end{eulercomment}
\begin{eulerprompt}
>plot3d("x^2+1",a=-1,b=1,rotate=2,grid=5):
\end{eulerprompt}
\eulerimg{17}{images/Vikram Zaky Ardianto_22305144028_Plot 3D-026.png}
\begin{eulercomment}
1. x\textasciicircum{}2+1 adalah fungsi matematika yang digunakan untuk membuat grafik
3D.\\
2. a=-1,b=1 mengatur rentang sumbu x yang akan ditampilkan pada
grafik.\\
3. rotate=2 mengatur grafik agar dapat diputar secara interaktif
dengan menggunakan mouse.\\
4. grid=5 mengatur jumlah garis koordinat yang ditampilkan pada
grafik.
\end{eulercomment}
\begin{eulerprompt}
>plot3d("sqrt(25-x^2)",a=0,b=5,rotate=1):
\end{eulerprompt}
\eulerimg{17}{images/Vikram Zaky Ardianto_22305144028_Plot 3D-027.png}
\begin{eulercomment}
1. sqrt(25-x\textasciicircum{}2) adalah fungsi matematika yang digunakan untuk membuat
grafik 3D.\\
2. a=0,b=5 mengatur rentang sumbu x yang akan ditampilkan pada grafik.\\
3. rotate=1 mengatur grafik agar dapat diputar secara interaktif.
\end{eulercomment}
\begin{eulerprompt}
>plot3d("x*sin(x)",a=0,b=6pi,rotate=2):
\end{eulerprompt}
\eulerimg{17}{images/Vikram Zaky Ardianto_22305144028_Plot 3D-028.png}
\begin{eulercomment}
1. x*sin(x) adalah fungsi matematika yang digunakan untuk membuat
grafik 3D.\\
2. a=0,b=6pi mengatur rentang sumbu x yang akan ditampilkan pada
grafik.\\
3. rotate=2 mengatur grafik agar dapat diputar secara interaktif.

Berikut adalah plot dengan tiga fungsi.
\end{eulercomment}
\begin{eulerprompt}
>plot3d("x","x^2+y^2","y",r=2,zoom=3.5,frame=3):
\end{eulerprompt}
\eulerimg{17}{images/Vikram Zaky Ardianto_22305144028_Plot 3D-029.png}
\begin{eulercomment}
1. x adalah fungsi matematika yang digunakan untuk menentukan nilai
sumbu x pada grafik.\\
2. x\textasciicircum{}2+y\textasciicircum{}2 adalah fungsi matematika yang digunakan untuk menentukan
nilai sumbu z pada grafik.\\
3. y adalah fungsi matematika yang digunakan untuk menentukan nilai
sumbu y pada grafik.\\
4. r=2 mengatur jarak maksimum dari pusat grafik ke tepi grafik.\\
5. zoom=3.5 mengatur faktor perbesaran grafik.\\
6. frame=3 mengatur jenis bingkai yang digunakan untuk grafik.
\end{eulercomment}
\eulerheading{3. Menggambar Grafik Fungsi Dua Variabel yang }
\begin{eulercomment}
* Fungsinya Didefinisikan sebagai Fungsi Numerik

\end{eulercomment}
\eulersubheading{Fungsi Dua Variabel}
\begin{eulercomment}
Fungsi dua variabel adalah sebuah fungsi yang bernilai real dari dua
variabel real. Fungsi ini memadankan setiap pasangan terurut (x,y)
pada suatu himpunan D dari bidang dengan bilangan real tunggal f(x,y).
Dalam matematika, fungsi dua variabel atau lebih digunakan untuk
menggambarkan hubungan antara dua atau lebih variabel.

\end{eulercomment}
\eulersubheading{Fungsi Numerik}
\begin{eulercomment}
Fungsi numerik adalah suatu fungsi matematika yang menghasilkan nilai
numerik sebagai output-nya. Fungsi ini dapat dinyatakan dalam bentuk
persamaan matematika atau algoritma komputasi.

Contoh:

Fungsi\\
\end{eulercomment}
\begin{eulerformula}
\[
f(x,y) = 5x+y
\]
\end{eulerformula}
\begin{eulercomment}
Misal input nilai x=2 dan y=3, maka akan dihasilkan nilai z yaitu

\end{eulercomment}
\begin{eulerformula}
\[
z = f(x,y) = 5(2)+3 = 10+3 = 13
\]
\end{eulerformula}
\begin{eulercomment}
\end{eulercomment}
\eulersubheading{Gambar Grafik Fungsi}
\begin{eulercomment}
Fungsi satu baris numerik didefinisikan oleh ":=".

Langkah-langkah untuk memvisualisasikan grafik fungsi dua variabel
yang fungsi nya didefinisikan sebagai fungsi numerik dalam plot3d:

1. Buat fungsi numerik yang akan digunakan untuk memvisualisasikan
data.\\
function f(x,y):=ax+by\\
dimana a dan b adalah konstanta

2. Gunakan fungsi plot3d() untuk membuat grafik tiga dimensi dari
fungsi numerik.\\
plot3d("f"):

\end{eulercomment}
\eulersubheading{Contoh}
\begin{eulercomment}
Fungsi matematika f(x,y) dapat digambarkan dalam bentuk grafik tiga
dimensi menggunakan perintah plot3d. Berikut adalah contoh penggunaan
perintah plot3d untuk menggambarkan fungsi tersebut:

1. Fungsi Linear Dua Variabel

\end{eulercomment}
\begin{eulerformula}
\[
f(x,y)=5x+3y+1
\]
\end{eulerformula}
\begin{eulerprompt}
>function f(x,y):= 5*x+3*y+1
>plot3d("f"):
\end{eulerprompt}
\eulerimg{17}{images/Vikram Zaky Ardianto_22305144028_Plot 3D-030.png}
\begin{eulercomment}
- Fungsi f(x,y) didefinisikan sebagai 5x+3y+1.\\
- Perintah "plot3d("f")" digunakan untuk memplot grafik 3D dari fungsi
f(x,y) menggunakan fungsi plot3d di EMT.\\
- Grafik yang dihasilkan akan menampilkan fungsi dalam tiga dimensi,
dengan sumbu x dan y mewakili variabel masukan dan sumbu z mewakili
nilai keluaran fungsi. Grafik akan menunjukkan bentuk fungsi dan
perubahannya seiring dengan perubahan variabel masukan.

\end{eulercomment}
\eulersubheading{}
\begin{eulercomment}
2. Fungsi Kuadrat Dua Variabel

\end{eulercomment}
\begin{eulerformula}
\[
f(x,y)=x^2+3y^2+21
\]
\end{eulerformula}
\begin{eulerprompt}
>function f(x,y):= x^2+(3*y)^2+27
>plot3d("f"):
\end{eulerprompt}
\eulerimg{17}{images/Vikram Zaky Ardianto_22305144028_Plot 3D-031.png}
\begin{eulercomment}
- Perintah "function f(x,y):= x\textasciicircum{}2+(3*y)\textasciicircum{}2+27" berarti mendefinisikan
fungsi matematika f(x,y) sebagai x pangkat 2 ditambah 3 kali y pangkat
2 ditambah 27.\\
- Perintah "plot3d("f")" berarti membuat grafik tiga dimensi dari
fungsi f(x,y) yang telah didefinisikan sebelumnya.

\end{eulercomment}
\eulersubheading{}
\begin{eulercomment}
3. Fungsi Logaritma Dua Variabel

\end{eulercomment}
\begin{eulerformula}
\[
f(x,y)= \log(2xy)
\]
\end{eulerformula}
\begin{eulerprompt}
>function f(x,y):= log((2*x)*y)
>plot3d("f"):
\end{eulerprompt}
\eulerimg{17}{images/Vikram Zaky Ardianto_22305144028_Plot 3D-032.png}
\begin{eulercomment}
- Input yang diberikan adalah fungsi matematika dua variabel, f(x,y),
yang didefinisikan sebagai logaritma hasil kali 2x dan y.\\
- Perintah "plot3d("f")" digunakan untuk memplot grafik fungsi f(x,y)
dalam ruang tiga dimensi.

\end{eulercomment}
\eulersubheading{}
\begin{eulercomment}
4. Fungsi Eksponen Dua Variabel

\end{eulercomment}
\begin{eulerformula}
\[
f(x,y)=x^{5y+10}
\]
\end{eulerformula}
\begin{eulerprompt}
>function f(x,y):= x^(5*y+10)
>plot3d("f"):
\end{eulerprompt}
\eulerimg{17}{images/Vikram Zaky Ardianto_22305144028_Plot 3D-033.png}
\begin{eulercomment}
- Perintah `fungsi f(x,y):= x\textasciicircum{}(5y+10)` adalah fungsi matematika dua
variabel `x` dan `y` dan dengan rumus x\textasciicircum{}(5y+10)\\
- Perintah `plot3d("f")` digunakan untuk memplot fungsi dalam ruang
tiga dimensi. Plot yang dihasilkan akan menampilkan nilai fungsi
sebagai permukaan pada bidang x-y, dengan tinggi permukaan mewakili
nilai fungsi pada titik tersebut.

\end{eulercomment}
\eulersubheading{}
\begin{eulercomment}
5. Fungsi Trigonometri Dua Variabel

\end{eulercomment}
\begin{eulerformula}
\[
f(x,y)=cos(x)+sin(y)
\]
\end{eulerformula}
\begin{eulerprompt}
>function f(x,y):= cos(x)*sin(y)
>plot3d("f"):
\end{eulerprompt}
\eulerimg{17}{images/Vikram Zaky Ardianto_22305144028_Plot 3D-034.png}
\begin{eulercomment}
- Perintah "function f(x,y):= cos(x)*sin(y)" adalah perintah untuk
mendefinisikan fungsi matematika f(x,y) yang menghasilkan nilai
cosinus dari x dikalikan dengan sinus dari y.\\
- Perintah "plot3d("f")" adalah perintah untuk membuat grafik tiga
dimensi dari fungsi f(x,y) yang telah didefinisikan sebelumnya.



\begin{eulercomment}
\eulerheading{4. Menggambar Grafik Fungsi Dua Variabel yang}
\begin{eulercomment}
* Fungsinya Didefinisikan sebagai Fungsi Simbolik

\end{eulercomment}
\eulersubheading{Fungsi Simbolik}
\begin{eulercomment}
Fungsi simbolik adalah fungsi yang dinyatakan dengan menggunakan
simbol-simbol matematika, seperti huruf dan operasi matematika,
daripada menggunakan angka konkret atau ekspresi numerik. Fungsi
simbolik sering digunakan untuk menggambarkan hubungan antara
variabel-variabel matematika dalam bentuk yang lebih umum dan abstrak.

Contoh fungsi simbolik yang umum adalah:

\end{eulercomment}
\begin{eulerformula}
\[
g(x,y) = 2x + y
\]
\end{eulerformula}
\begin{eulercomment}
Dalam contoh di atas, g(x) adalah fungsi simbolik yang mengaitkan
setiap nilai x dengan hasil dari ekspresi matematika 2x + 3. Fungsi
ini dapat digunakan untuk menghitung nilai fungsi untuk berbagai nilai
x.

\end{eulercomment}
\eulersubheading{Perbedaan Fungsi Numerik dan Fungsi Simbolik}
\begin{eulercomment}
1. Fungsi Numerik\\
Fungsi numerik dinyatakan dalam bentuk yang lebih konkret menggunakan
angka-angka nyata.

Contoh fungsi numerik adalah

\end{eulercomment}
\begin{eulerformula}
\[
g(x,y) = 2x + y + 3
\]
\end{eulerformula}
\begin{eulercomment}
dimana kita memberikan nilai numerik kepada "x dan y"\\
misalnya, x = 5 dan y = 2, maka hasilnya adalah angka konkret yaitu
g(5,2) = 15

2. Fungsi Simbolik\\
Fungsi simbolik dinyatakan menggunakan simbol-simbol matematika
seperti huruf (variabel) dan operasi matematika.

Contoh fungsi simbolik adalah

\end{eulercomment}
\begin{eulerformula}
\[
g(x,y) = 2x + y + 3
\]
\end{eulerformula}
\begin{eulercomment}
"g" adalah simbol fungsi\\
"x,y" adalah variabel,\\
2x + 3 adalah ekspresi matematika yang menggambarkan hubungan antara
"x,y" dan hasil fungsi.

\end{eulercomment}
\eulersubheading{Gambar Grafik Fungsi}
\begin{eulercomment}
Fungsi satu baris simbolik didefinisikan oleh "\&=".

Langkah-langkah untuk memvisualisasikan grafik fungsi dua variabel
yang fungsi nya didefinisikan sebagai fungsi simbolik dalam plot3d:

1. Buat fungsi simbolik yang akan digunakan untuk memvisualisasikan
data.\\
function g(x,y)\&= ax+by;\\
dimana a dan b adalah konstanta

2. Gunakan fungsi plot3d() untuk membuat grafik tiga dimensi dari
fungsi simbolik\\
plot3d("g"):

3. Menentukan rentang variabel\\
misal\\
plot3d("g(x,y)",-10,10,-5,5):\\
dengan batasan x dari -10 hingga 10 dan batasan y dari -5 hingga 5

\end{eulercomment}
\eulersubheading{Contoh}
\begin{eulercomment}
1. Fungsi Linear Dua Variabel

\end{eulercomment}
\begin{eulerformula}
\[
g(x,y)=x-2y+6
\]
\end{eulerformula}
\begin{eulercomment}
\end{eulercomment}
\begin{eulerprompt}
>function g(x,y)&= x-2*y+6;
>plot3d("g(x,y)"):
\end{eulerprompt}
\eulerimg{17}{images/Vikram Zaky Ardianto_22305144028_Plot 3D-035.png}
\begin{eulercomment}
- Fungsi g(x,y) adalah fungsi matematika yang mengambil dua variabel,
x dan y, dan menghasilkan sebuah nilai berdasarkan rumus x - 2y + 6.\\
- Perintah "plot3d" digunakan untuk menghasilkan grafik tiga dimensi
dari fungsi tersebut.
\end{eulercomment}
\begin{eulerprompt}
>plot3d("g(x,y)",-1,2,0,2*pi):
\end{eulerprompt}
\eulerimg{17}{images/Vikram Zaky Ardianto_22305144028_Plot 3D-036.png}
\begin{eulercomment}
- Perintah "plot3d("g(x,y)",-1,2,0,2*pi)" adalah perintah untuk
menggambar grafik fungsi tiga dimensi "g(x,y)" pada rentang x dari -1
hingga 2 dan rentang y dari 0 hingga 2pi.

\end{eulercomment}
\eulersubheading{}
\begin{eulercomment}
2. Fungsi Kuadrat Dua Variabel

\end{eulercomment}
\begin{eulerformula}
\[
g(x,y)=x^2+y^2+5
\]
\end{eulerformula}
\begin{eulerprompt}
>function g(x,y)&= x^2+y^2+5;
>plot3d("g(x,y)"):
\end{eulerprompt}
\eulerimg{17}{images/Vikram Zaky Ardianto_22305144028_Plot 3D-037.png}
\begin{eulercomment}
- Fungsi g(x,y) adalah fungsi matematika yang mengambil dua variabel,
x dan y, dan menghasilkan sebuah nilai berdasarkan rumus x\textasciicircum{}2+y\textasciicircum{}2+5\\
- Perintah "plot3d" digunakan untuk menghasilkan grafik tiga dimensi
dari fungsi tersebut.
\end{eulercomment}
\begin{eulerprompt}
>plot3d("g(x,y)",-10,10,-1,5):
\end{eulerprompt}
\eulerimg{17}{images/Vikram Zaky Ardianto_22305144028_Plot 3D-038.png}
\begin{eulercomment}
- Perintah "plot3d("g(x,y)",-10,10,-1,5)" adalah perintah untuk
menggambar grafik fungsi tiga dimensi g(x,y) pada rentang x dari -10
hingga 10 dan rentang y dari -1 hingga 5

\end{eulercomment}
\eulersubheading{}
\begin{eulercomment}
3. Fungsi Logaritma Dua Variabel

\end{eulercomment}
\begin{eulerformula}
\[
g(x,y) = \log(xy5)
\]
\end{eulerformula}
\begin{eulerprompt}
>function g(x,y)&= log(x*y*5);
>plot3d("g(x,y)"):
\end{eulerprompt}
\eulerimg{17}{images/Vikram Zaky Ardianto_22305144028_Plot 3D-039.png}
\begin{eulercomment}
- Fungsi g(x,y) adalah fungsi matematika yang mengambil dua variabel,
x dan y, dan menghasilkan sebuah nilai berdasarkan rumus logaritma x
dikalikan y dikalikan 5\\
- Perintah "plot3d" digunakan untuk menghasilkan grafik tiga dimensi
dari fungsi tersebut.
\end{eulercomment}
\begin{eulerprompt}
>plot3d("g(x,y)",-1,1,-5,5):
\end{eulerprompt}
\eulerimg{17}{images/Vikram Zaky Ardianto_22305144028_Plot 3D-040.png}
\begin{eulercomment}
- Perintah "plot3d("g(x,y)",-1,1,-5,5)" adalah perintah untuk
menggambar grafik fungsi tiga dimensi g(x,y) pada rentang x dari -1
hingga 1 dan rentang y dari -5 hingga 5
\end{eulercomment}
\begin{eulerprompt}
>plot3d("g(x,y)",-1,1,-5,5,zoom=4.5):
\end{eulerprompt}
\eulerimg{17}{images/Vikram Zaky Ardianto_22305144028_Plot 3D-041.png}
\begin{eulercomment}
- plot3d: perintah untuk membuat grafik 3D.\\
- "g(x,y)": fungsi matematika yang akan digunakan untuk membuat
grafik.\\
- -1,1: rentang nilai variabel x yang akan digunakan dalam grafik.\\
- -5,5: rentang nilai variabel y yang akan digunakan dalam grafik.\\
- zoom=4.5: perintah untuk memperbesar tampilan grafik dengan faktor
4.5.

\end{eulercomment}
\eulersubheading{}
\begin{eulercomment}
4. Fungsi Eksponen Dua Variabel

\end{eulercomment}
\begin{eulerformula}
\[
g(x,y) = 2^{xy5}
\]
\end{eulerformula}
\begin{eulerprompt}
>function g(x,y)&= 2^(x*y*5);
>plot3d("g(x,y)"):
\end{eulerprompt}
\eulerimg{17}{images/Vikram Zaky Ardianto_22305144028_Plot 3D-042.png}
\begin{eulercomment}
- Fungsi g(x,y) adalah fungsi matematika yang mengambil dua variabel,
x dan y, dan menghasilkan sebuah nilai berdasarkan rumus 2\textasciicircum{}(xy5)\\
- Perintah "plot3d" digunakan untuk menghasilkan grafik tiga dimensi
dari fungsi tersebut.
\end{eulercomment}
\begin{eulerprompt}
>plot3d("g(x,y)",-1,5,-1,3,frame=3,zoom=3):
\end{eulerprompt}
\eulerimg{17}{images/Vikram Zaky Ardianto_22305144028_Plot 3D-043.png}
\begin{eulercomment}
- Peintah plot3d("g(x,y)",-1,5,-1,3,frame=3,zoom=3) adalah perintah
untuk membuat plot tiga dimensi dari fungsi `g(x,y)` dengan batas `x`
dari `-1` hingga `5` dan batas `y` dari `-1` hingga `3`.

- plot3d: perintah untuk membuat plot tiga dimensi.\\
- "g(x,y)"`: fungsi yang akan diplot.\\
- (-1,5): batas `x` dari `-1` hingga `5`.\\
- (-1,3): batas `y` dari `-1` hingga `3`.\\
- frame=3: menampilkan frame nomor 3.\\
- zoom=3: memperbesar tampilan plot sebanyak 3 kali.

\end{eulercomment}
\eulersubheading{}
\begin{eulercomment}
5. Fungsi Trigonometri Dua Variabel

\end{eulercomment}
\begin{eulerformula}
\[
g(x,y)=tan(x) - cot(y)
\]
\end{eulerformula}
\begin{eulerprompt}
>function g(x,y)&= tan(x)-cos(y);
>plot3d("g(x,y)"):
\end{eulerprompt}
\eulerimg{17}{images/Vikram Zaky Ardianto_22305144028_Plot 3D-044.png}
\begin{eulercomment}
- Fungsi g(x,y) adalah fungsi matematika yang mengambil dua variabel,
x dan y, dan menghasilkan sebuah nilai berdasarkan rumus tan(x)-cos(y)\\
- Perintah "plot3d" digunakan untuk menghasilkan grafik tiga dimensi
dari fungsi tersebut.
\end{eulercomment}
\begin{eulerprompt}
>plot3d("g(x,y)",-1,3,0,2*pi,frame=1,zoom=3.5):
\end{eulerprompt}
\eulerimg{17}{images/Vikram Zaky Ardianto_22305144028_Plot 3D-045.png}
\begin{eulercomment}
- Peintah plot3d("g(x,y)",-1,3,0,2*pi,frame=5,zoom=3) adalah perintah
untuk membuat plot tiga dimensi dari fungsi `g(x,y)` dengan batas `x`
dari `-1` hingga `3` dan batas `y` dari `0` hingga `2pi`.

- plot3d: perintah untuk membuat plot tiga dimensi.\\
- "g(x,y)"`: fungsi yang akan diplot.\\
- (-1,3): batas `x` dari `-1` hingga `3`.\\
- (0,2pi): batas `y` dari `0` hingga `2pi`.\\
- frame=1: menampilkan frame nomor 1.\\
- zoom=3.5: memperbesar tampilan plot sebanyak 3.5 kali.

\end{eulercomment}
\eulersubheading{}
\begin{eulercomment}
6. Fungsi Akar Kuadrat

\end{eulercomment}
\begin{eulerformula}
\[
P(x,y)= \sqrt{10x^2+2y^2}
\]
\end{eulerformula}
\begin{eulerprompt}
>function P(x,y) &= sqrt(10*x^2+2*y^2);
>plot3d("P(x,y)"):
\end{eulerprompt}
\eulerimg{17}{images/Vikram Zaky Ardianto_22305144028_Plot 3D-046.png}
\begin{eulercomment}
- Fungsi P(x,y) adalah fungsi matematika yang mengambil dua variabel,
x dan y, dan menghasilkan sebuah nilai berdasarkan rumus akar kuadrat
dari 10x\textasciicircum{}2+2y\textasciicircum{}2\\
- Perintah "plot3d" digunakan untuk menghasilkan grafik tiga dimensi
dari fungsi tersebut.
\end{eulercomment}
\begin{eulerprompt}
>plot3d("P(x,y)",-2,2,0,3*pi,frame=5,zoom=2,scale=1):
\end{eulerprompt}
\eulerimg{17}{images/Vikram Zaky Ardianto_22305144028_Plot 3D-047.png}
\begin{eulercomment}
- P(x,y): Merupakan fungsi yang akan digambarkan dalam grafik tiga
dimensi.\\
- (-2,2): Merupakan rentang nilai dari sumbu x yang akan digunakan
dalam grafik.\\
- (0,3pi): Merupakan rentang nilai dari sumbu y yang akan digunakan
dalam grafik. Nilai pi dikalikan dengan 3 agar rentang nilai y
mencakup tiga putaran lingkaran penuh.\\
- frame=5: Menentukan nomor bingkai (frame) yang akan digunakan dalam
animasi grafik.\\
- zoom=2: Menentukan faktor pembesaran grafik. Dengan memperbesar
tampilan, kita dapat melihat detail yang lebih kecil pada plot.\\
- scale=1: Menentukan skala grafik. Dengan mengatur skala, kita dapat
mengubah jarak antara titik-titik pada sumbu tersebut.
\end{eulercomment}
\eulerheading{5. Menggambar Data $x$, $y$, $z$}
\begin{eulercomment}
* pada ruang Tiga Dimensi (3D)

Definisi

\end{eulercomment}
\begin{eulerttcomment}
  Menggambar data pada ruang tiga dimensi (3D) adalah proses
\end{eulerttcomment}
\begin{eulercomment}
visualisasi data yang mengubah informasi dalam tiga dimensi, yaitu
panjang, lebar, dan tinggi, menjadi representasi visual yang dapat
dipahami dan dianalisis.

Tujuan:

\end{eulercomment}
\begin{eulerttcomment}
  Tujuan dari menggambar data 3D adalah untuk membantu pemahaman dan
\end{eulerttcomment}
\begin{eulercomment}
interpretasi data yang lebih baik, terutama ketika data tersebut
memiliki komponen yang tidak dapat direpresentasikan dengan baik dalam
dua dimensi.

Sama seperti plot2d, plot3d menerima data. Untuk objek 3D, Anda perlu
menyediakan matriks nilai x-, y- dan z, atau tiga fungsi atau ekspresi
fx(x,y), fy(x,y), fz(x,y).

\end{eulercomment}
\begin{eulerformula}
\[
\gamma(t,s) = (x(t,s),y(t,s),z(t,s))
\]
\end{eulerformula}
\begin{eulercomment}
Karena x,y,z adalah matriks, kita asumsikan bahwa (t,s) melalui sebuah
kotak persegi. Hasilnya, Anda dapat memplot gambar persegi panjang di
ruang angkasa.

Kita dapat menggunakan bahasa matriks Euler untuk menghasilkan
koordinat secara efektif.

Dalam contoh berikut, kami menggunakan vektor nilai t dan vektor kolom
nilai s untuk membuat parameter permukaan bola. Dalam gambar kita
dapat menandai daerah, dalam kasus kita daerah kutub.

\end{eulercomment}
\eulersubheading{Contoh 1}
\begin{eulerprompt}
>t=-1:0.1:1; s=(-1:0.1:1)'; plot3d(t,s,t*s,grid=10):
\end{eulerprompt}
\eulerimg{17}{images/Vikram Zaky Ardianto_22305144028_Plot 3D-049.png}
\begin{eulercomment}
Baris pertama kode "t=-1:0.1:1" membuat vektor baris t yang berisi
nilai dari -1 hingga 1 dengan interval 0.1. Baris kedua
"s=(-1:0.1:1)'" membuat vektor kolom s yang berisi nilai dari -1
hingga 1 dengan interval 0.1. Operator transpose ' digunakan untuk
mengubah vektor baris t menjadi vektor kolom.\\
Baris ketiga "plot3d(t,s,ts,grid=10)" membuat plot tiga dimensi dari
fungsi f(x,y) = xy pada domain [-1,1] x [-1,1]. Plot dibuat
menggunakan fungsi plot3d, yang mengambil tiga argumen: koordinat x,
y, dan z dari titik-titik yang akan diplot. Dalam hal ini, koordinat x
diberikan oleh vektor t, koordinat y diberikan oleh vektor s, dan
koordinat z diberikan oleh hasil perkalian t dan s, yaitu ts.
Parameter grid diatur menjadi 10, yang menunjukkan jumlah garis grid
yang akan ditampilkan pada plot.

\end{eulercomment}
\eulersubheading{Contoh 2}
\begin{eulercomment}
Tentu saja, titik cloud juga dimungkinkan. Untuk memplot data titik
dalam ruang, kita membutuhkan tiga vektor untuk koordinat titik-titik
tersebut.

Gayanya sama seperti di plot2d dengan points=true;
\end{eulercomment}
\begin{eulerprompt}
>n=500;...
>plot3d(normal(1,n),normal(1,n),normal(1,n),points=true,style="."):
\end{eulerprompt}
\eulerimg{17}{images/Vikram Zaky Ardianto_22305144028_Plot 3D-050.png}
\begin{eulercomment}
Kode "n=500;
plot3d(normal(1,n),normal(1,n),normal(1,n),points=true,style=".")"
digunakan untuk membuat plot tiga dimensi dari tiga vektor normal yang
dihasilkan secara acak dengan menggunakan fungsi "normal()" pada Euler
Math Toolbox (EMT). Parameter "n=500" menunjukkan bahwa setiap vektor
normal memiliki 500 elemen. Parameter "points=true" digunakan untuk
menampilkan titik-titik pada plot, sedangkan parameter "style='.'"
digunakan untuk mengatur gaya titik pada plot menjadi titik bulat.

\end{eulercomment}
\eulersubheading{Contoh 3}
\begin{eulercomment}
\end{eulercomment}
\begin{eulerttcomment}
 Dengan lebih banyak usaha, kami dapat menghasilkan banyak permukaan.
\end{eulerttcomment}
\begin{eulercomment}
\end{eulercomment}
\begin{eulerttcomment}
 Dalam contoh berikut, kita membuat tampilan bayangan dari bola yang
\end{eulerttcomment}
\begin{eulercomment}
terdistorsi. Koordinat biasa untuk bola adalah

\end{eulercomment}
\begin{eulerformula}
\[
\gamma(t,s) = (\cos(t)\cos(s),\sin(t)\sin(s),\cos(s))
\]
\end{eulerformula}
\begin{eulercomment}
dengan

\end{eulercomment}
\begin{eulerformula}
\[
0 \le t \le 2\pi, \quad \frac{-\pi}{2} \le s \le \frac{\pi}{2}.
\]
\end{eulerformula}
\begin{eulercomment}
Kami mendistorsi ini dengan sebuah faktor

\end{eulercomment}
\begin{eulerformula}
\[
d(t,s) = \frac{\cos(4t)+\cos(8s)}{4}
\]
\end{eulerformula}
\begin{eulerprompt}
>t=linspace(0,2pi,320); s=linspace(-pi/2,pi/2,160)';...
>d=1+0.2*(cos(4*t)+cos(8*s));...
>plot3d(cos(t)*cos(s)*d,sin(t)*cos(s)*d,sin(s)*d,hue=1,...
>light=[1,0,1],frame=0,zoom=5):
\end{eulerprompt}
\eulerimg{17}{images/Vikram Zaky Ardianto_22305144028_Plot 3D-051.png}
\begin{eulercomment}
Kode ini terdiri dari beberapa baris. Baris pertama
"t=linspace(0,2pi,320)" membuat vektor t yang berisi 320 nilai yang
sama terdistribusi secara merata antara 0 dan 2p. Baris kedua
"s=linspace(-pi/2,pi/2,160)'" membuat vektor s yang berisi 160 nilai
yang sama terdistribusi secara merata antara -p/2 dan p/2. Operator
transpose ' digunakan untuk mengubah vektor baris s menjadi vektor
kolom.

Baris ketiga "d=1+0.2*(cos(4t)+cos(8s))" membuat vektor d yang berisi
nilai dari 1 + 0.2 * (cos(4t) + cos(8s)). Baris keempat
"plot3d(cos(t)*cos(s)*d,sin(t)*cos(s)*d,sin(s)*d,hue=1,light=[1,0,1],frame=0,zoom=5)"
membuat plot tiga dimensi dari fungsi f(x,y) = 2x\textasciicircum{}2 + y\textasciicircum{}3. Plot dibuat
menggunakan fungsi plot3d, yang mengambil empat argumen: koordinat x,
y, dan z dari titik-titik yang akan diplot, serta beberapa parameter
lainnya. Dalam hal ini, koordinat x diberikan oleh ekspresi
cos(t)*cos(s)*d, koordinat y diberikan oleh ekspresi sin(t)*cos(s)*d,
dan koordinat z diberikan oleh ekspresi sin(s)*d. Parameter "hue=1"
digunakan untuk mengatur warna pada plot berdasarkan nilai fungsinya.
Parameter "light=[1,0,1]" digunakan untuk mengatur pencahayaan pada
plot. Parameter "frame=0" digunakan untuk menghilangkan frame pada
plot. Parameter "zoom=5" digunakan untuk mengatur level zoom pada
plot.
\end{eulercomment}
\eulerheading{6. Grafik Tiga Dimensi yang}
\begin{eulercomment}
* Bersifat Interaktif dan animasi grafik 3D
\end{eulercomment}
\begin{eulercomment}
Membuat gambar grafik tiga dimensi (3D) yang bersifat interaktif dan
animasi grafik 3D adalah proses menciptakan visualisasi tiga dimensi
yang memungkinkan pengguna berinteraksi dengan objek-objek 3D.
Interaktivitas dalam gambar 3D memungkinkan pengguna untuk melakukan
tindakan seperti mengubah sudut pandang, memindahkan objek, atau
berinteraksi dengan elemen-elemen dalam adegan 3D. Animasi grafik 3D
dapat mencakup pergerakan, tetapi juga dapat berarti perubahan dalam
tampilan atau atribut objek tanpa pergerakan fisik yang mencolok.

CONTOH GAMBAR
\end{eulercomment}
\begin{eulerprompt}
>function testplot () := plot3d("x^2+y^3"); ...
>rotate("testplot"); testplot(): 
\end{eulerprompt}
\eulerimg{17}{images/Vikram Zaky Ardianto_22305144028_Plot 3D-052.png}
\begin{eulerprompt}
>function testplot () := plot3d("x^2+y",distance=3,zoom=1,angle=pi/2,height=0); ...
>rotate("testplot"); testplot(): 
\end{eulerprompt}
\eulerimg{17}{images/Vikram Zaky Ardianto_22305144028_Plot 3D-053.png}
\begin{eulercomment}
Hilangkan command angle untuk bisa merotasikan grafik,dan height = 0
untuk membuat posisi sejajar dengan mata jadi tidak mempengaruhi
pergerakan hanya berbeda sudut pandang saja
\end{eulercomment}
\begin{eulerprompt}
>plot3d("exp(-x^2+y^2)",>user, ...
>  title="Turn with the vector keys (press return to finish)"):
\end{eulerprompt}
\eulerimg{17}{images/Vikram Zaky Ardianto_22305144028_Plot 3D-054.png}
\begin{eulerprompt}
>plot3d("exp(x^2+y^2)",>user, ...
>title="Coba gerakan)")
\end{eulerprompt}
\begin{eulercomment}
Interaksi pengguna dimungkinkan dengan parameter. Pengguna dapat
menekan tombol berikut.\\
1. kiri, kanan, atas, bawah: memutar sudut pandang\\
2. +,-: memperbesar atau memperkecil\\
3. a: menghasilkan anaglyph (lihat di bawah)\\
4. l: beralih memutar sumber cahaya (lihat di bawah)\\
5. spasi: disetel ulang ke default\\
6. enter: akhiri interaksi
\end{eulercomment}
\begin{eulerprompt}
>plot3d("exp(-(x^2+y^2)/5)",r=10,n=80,fscale=4,scale=1.2,frame=3,>user):
\end{eulerprompt}
\eulerimg{17}{images/Vikram Zaky Ardianto_22305144028_Plot 3D-055.png}
\begin{eulercomment}
Parameter "r=10" menunjukkan jari-jari bola yang digunakan untuk
membuat plot tiga dimensi. Dalam hal ini, jari-jari bola yang
digunakan adalah 10.\\
Parameter "n=80" menunjukkan jumlah titik yang digunakan untuk membuat
plot. Semakin besar nilai n, semakin banyak titik yang digunakan untuk
membuat plot, sehingga plot akan menjadi lebih halus dan akurat.\\
Parameter "fscale=4" menunjukkan faktor skala pada sumbu z. Dalam hal
ini, faktor skala pada sumbu z adalah 4.\\
Parameter "scale=1.2" menunjukkan faktor skala pada plot. Semakin
besar nilai scale, semakin besar ukuran plot yang dihasilkan.\\
Parameter "frame=3" menunjukkan jenis frame yang digunakan pada plot.
Dalam hal ini, jenis frame yang digunakan adalah frame kotak dengan
sumbu x, y, dan z yang ditampilkan.
\end{eulercomment}
\begin{eulerprompt}
>plot3d("x^2+y",distance=3,zoom=1,angle=pi/2,height=0):
\end{eulerprompt}
\eulerimg{17}{images/Vikram Zaky Ardianto_22305144028_Plot 3D-056.png}
\begin{eulercomment}
Tampilan dapat diubah dengan berbagai cara.

- distance: jarak pandang ke plot.\\
- zoom: nilai zoom.\\
- angle: sudut terhadap sumbu y negatif dalam radian.\\
- height: ketinggian tampilan dalam radian.
\end{eulercomment}
\begin{eulerprompt}
>plot3d("x^4+y^2",a=0,b=1,c=-1,d=1, angle=-20?, height=20?, ...
>  center=[0.4,0,0], zoom=5):
\end{eulerprompt}
\begin{euleroutput}
  Closing bracket missing in function call!
  Error in:
  plot3d("x^4+y^2",a=0,b=1,c=-1,d=1, angle=-20?, height=20?,   c ...
                                              ^
\end{euleroutput}
\begin{eulercomment}
Plot selalu terlihat berada di tengah kubus plot. Anda dapat
memindahkan bagian tengah dengan parameter center.

Parameter center digunakan untuk memindahkan pusat plot ke lokasi
tertentu dalam ruang. Dalam hal ini, pusat plot diatur ke titik (0.4,
0, 0) dalam ruang tiga dimensi. Parameter center berguna ketika kita
ingin mengubah sudut pandang plot atau ketika kita ingin menyelaraskan
plot dengan objek lain dalam scene. Dengan menentukan pusat plot, kita
dapat mengontrol posisi kamera dan arah tampilan plot.

Ada beberapa parameter untuk menskalakan fungsi atau mengubah tampilan
grafik.

fscale: menskalakan ke nilai fungsi (defaultnya adalah \textless{}fscale).\\
scale: angka atau vektor 1x2 untuk diskalakan ke arah x dan y.\\
frame: jenis bingkai (default 1).
\end{eulercomment}
\begin{eulerprompt}
>function testplot () := plot3d("5*exp(-x^2-y^2)",r=2,<fscale,<scale,distance=13,height=50?, ...
>center=[0,0,-2],frame=3); ...
>rotate("testplot"); testplot():
\end{eulerprompt}
\begin{euleroutput}
  Closing bracket missing in function call!
  testplot:
      useglobal; return plot3d("5*exp(-x^2-y^2)",r=2,<fscale,<scale ...
  Try "trace errors" to inspect local variables after errors.
  rotate:
      f$(args());
\end{euleroutput}
\begin{eulerprompt}
>plot3d("x^2+1",a=-1,b=1,rotate=true,grid=5):
\end{eulerprompt}
\eulerimg{17}{images/Vikram Zaky Ardianto_22305144028_Plot 3D-057.png}
\begin{eulercomment}
Penjelasan:\\
Secara umum, parameter "a" dan "b" digunakan untuk menentukan rentang
nilai variabel independen dalam suatu fungsi. Dalam kasus ini, "a=-1"
dan "b=1" menunjukkan bahwa fungsi tersebut akan diplot pada interval
[-1, 1]. Parameter "rotate=true" menunjukkan bahwa grafik akan diputar
untuk memberikan tampilan bentuk tiga dimensi yang lebih baik.
Parameter "grid=5" menunjukkan bahwa grid dengan jarak 5 unit akan
ditampilkan pada grafik.

Parameter memutar memutar fungsi dalam x di sekitar sumbu x.

- rotate=1: Menggunakan sumbu x\\
- rotate=2: Menggunakan sumbu z
\end{eulercomment}
\begin{eulerprompt}
>plot3d("x^2+1",a=-1,b=1,rotate=2,grid=5):
\end{eulerprompt}
\eulerimg{17}{images/Vikram Zaky Ardianto_22305144028_Plot 3D-058.png}
\begin{eulerprompt}
>function testplot () := plot3d("sqrt(25-x^2)",a=0,b=5,rotate=1); ...
>rotate("testplot"); testplot():
\end{eulerprompt}
\eulerimg{17}{images/Vikram Zaky Ardianto_22305144028_Plot 3D-059.png}
\begin{eulerprompt}
>function testplot () := plot3d("x^4+y^2",a=0,b=1,c=-1,d=1,height=20?, ...
>center=[0.4,0,0],zoom=5); ...
>rotate("testplot"); testplot():
\end{eulerprompt}
\begin{euleroutput}
  Closing bracket missing in function call!
  testplot:
      useglobal; return plot3d("x^4+y^2",a=0,b=1,c=-1,d=1,height=20 ...
  Try "trace errors" to inspect local variables after errors.
  rotate:
      f$(args());
\end{euleroutput}
\begin{eulerprompt}
>function testplot () := plot3d("1/(x^2+y^2+1)",r=5,>polar, ...
>fscale=2,>hue,n=100,zoom=4,>contour,color=red); ...
>rotate("testplot"); testplot():
\end{eulerprompt}
\eulerimg{17}{images/Vikram Zaky Ardianto_22305144028_Plot 3D-060.png}
\begin{eulercomment}
Parameter "r=5" menunjukkan jari-jari bola yang digunakan untuk
membuat plot tiga dimensi. Dalam hal ini, jari-jari bola yang
digunakan adalah 5.\\
Parameter "\textgreater{}polar" menunjukkan bahwa plot yang dibuat adalah plot
polar tiga dimensi. Plot polar adalah plot yang dibuat dengan
menggunakan koordinat polar, yaitu koordinat yang terdiri dari jarak
dan sudut.\\
Parameter "fscale=2" menunjukkan faktor skala pada sumbu z. Dalam hal
ini, faktor skala pada sumbu z adalah 2.\\
Parameter "\textgreater{}hue" menunjukkan bahwa warna pada plot akan diatur
berdasarkan nilai fungsinya. Semakin tinggi nilai fungsinya, semakin
terang warnanya.\\
Parameter "n=100" menunjukkan jumlah titik yang digunakan untuk
membuat plot. Semakin besar nilai n, semakin banyak titik yang
digunakan untuk membuat plot, sehingga plot akan menjadi lebih halus
dan akurat.\\
Parameter "zoom=4" menunjukkan level zoom pada plot.\\
Parameter "\textgreater{}contour" menunjukkan bahwa garis kontur akan ditampilkan
pada plot.\\
Parameter "color=blue" menunjukkan warna garis kontur pada plot. Dalam
hal ini, warna yang digunakan adalah biru.

Untuk plotnya, Euler menambahkan garis grid. Sebaliknya dimungkinkan
untuk menggunakan garis level dan satu warna atau warna spektral.
Euler dapat menggambar ketinggian fungsi pada sebuah plot dengan
bayangan. Di semua plot 3D, Euler dapat menghasilkan anaglyph
merah/cyan.

-hue: Mengaktifkan bayangan cahaya, bukan kabel.\\
-contour: Membuat plot garis kontur otomatis pada plot.\\
-level=... (atau level): Vektor nilai garis kontur.
\end{eulercomment}
\begin{eulerprompt}
>function testplot () := plot3d("x^2-y^2",0,5,0,5,level=-1:0.1:1,color=blue); ...
>rotate("testplot"); testplot():
\end{eulerprompt}
\eulerimg{17}{images/Vikram Zaky Ardianto_22305144028_Plot 3D-061.png}
\begin{eulercomment}
Parameter "level=-1:0.1:1" menunjukkan rentang nilai fungsinya yang
akan ditampilkan pada plot. Dalam hal ini, rentang nilai fungsinya
adalah dari -1 hingga 1 dengan interval 0.1.
\end{eulercomment}
\begin{eulerprompt}
>function testplot () := plot3d("x^2+y^4",>cp,cpcolor=green,cpdelta=0.2); ...
>rotate("testplot"); testplot():
\end{eulerprompt}
\eulerimg{17}{images/Vikram Zaky Ardianto_22305144028_Plot 3D-062.png}
\begin{eulercomment}
Parameter "\textgreater{}cp" menunjukkan bahwa titik kontrol akan ditambahkan pada
plot. Titik kontrol digunakan untuk menentukan bentuk dan posisi plot
tiga dimensi.\\
Parameter "cpcolor=green" menunjukkan warna titik kontrol yang akan
digunakan. Dalam hal ini, warna yang digunakan adalah hijau.\\
Parameter "cpdelta=0.2" menunjukkan jarak antara titik kontrol.
Semakin kecil nilai cpdelta, semakin banyak titik kontrol yang akan
ditambahkan pada plot.
\end{eulercomment}
\begin{eulerprompt}
>plot3d("-x^2-y^2", ...
>hue=true,light=[0,1,1],amb=0,user=true, ...
> title="Press l and cursor keys (return to exit)"):
\end{eulerprompt}
\eulerimg{17}{images/Vikram Zaky Ardianto_22305144028_Plot 3D-063.png}
\begin{eulercomment}
parameter "hue=true" menunjukkan bahwa warna pada plot akan diatur
berdasarkan nilai fungsinya. Semakin tinggi nilai fungsinya, semakin
terang warnanya.\\
Parameter "light=light=[0,1,1] menunjukkan intensitas cahaya pada
plot. Nilai light=[0,1,1] menunjukkan bahwa cahaya datang dari arah
positif y dan z.\\
Parameter "amb=0" menunjukkan intensitas cahaya ambient pada plot.
Nilai 0 menunjukkan bahwa tidak ada cahaya ambient yang digunakan.
\end{eulercomment}
\begin{eulerprompt}
>function testplot () := plot3d("-x^2-y^2",color=rgb(0.2,0.2,0),hue=true,frame=false, ...
> zoom=3,contourcolor=red,level=-2:0.1:1,dl=0.01); ...
>rotate("testplot"); testplot():
\end{eulerprompt}
\eulerimg{17}{images/Vikram Zaky Ardianto_22305144028_Plot 3D-064.png}
\begin{eulercomment}
Parameter "frame=false" digunakan untuk menghilangkan frame pada plot
tiga dimensi. Parameter "color=rgb(0.2,0.2,0)" menunjukkan warna dasar
plot. Dalam hal ini, warna yang digunakan adalah hitam dengan nilai
RGB (0.2, 0.2, 0). Parameter "dl=0.01" menunjukkan jarak antara
titik-titik pada plot. Semakin kecil nilai dl, semakin banyak titik
yang digunakan untuk membuat plot, sehingga plot akan menjadi lebih
halus dan akurat. Namun, semakin kecil nilai dl, semakin lama waktu
yang dibutuhkan untuk membuat plot.
\end{eulercomment}
\begin{eulerprompt}
>function testplot () := plot3d("x^2+y^3",>contour,>spectral); ...
>rotate("testplot"); testplot():
\end{eulerprompt}
\eulerimg{17}{images/Vikram Zaky Ardianto_22305144028_Plot 3D-065.png}
\begin{eulerprompt}
>function testplot () := plot3d("x^2+y^3", >transparent, grid=10, wirecolor=red); ...
>rotate("testplot"); testplot():
\end{eulerprompt}
\eulerimg{17}{images/Vikram Zaky Ardianto_22305144028_Plot 3D-066.png}
\eulerheading{7. Fungsi Parametrik 3D}
\begin{eulercomment}
Fungsi parametrik merupakan jenis fungsi matematika yang menggambarkan
hubungan antara dua atau lebih variabel, dimana masing-masing
koordinat (x, y, z...) dinyatakan sebagai fungsi lain dari beberapa
parameter. Fungsi parametrik dapat digunakan untuk menggambar kurva,
lintasan, atau hubungan antara berbagai variabel yang bergantung pada
parameter-parameter tertentu.

Sebagai contoh :
\end{eulercomment}
\begin{eulerprompt}
>plot3d("cos(x)*cos(y)^3","sin(x)*cos(y)^3","sin(y)", a=0,b=2*pi,c=pi/2,d=-pi/2,...
>>hue,color=blue,light=[0,1,3],<frame,...
>n=90,grid=[20,50],wirecolor=black,zoom=5):
\end{eulerprompt}
\eulerimg{17}{images/Vikram Zaky Ardianto_22305144028_Plot 3D-067.png}
\begin{eulerprompt}
>plot3d("cos(x)*cos(y)","sin(x)*cos(y)","cos(x)", a=0,b=2*pi,c=pi/2,d=-pi/2,...
>>hue,color=blue,light=[0,1,3],<frame,...
>n=90,grid=[20,50],wirecolor=black,zoom=5):
\end{eulerprompt}
\eulerimg{17}{images/Vikram Zaky Ardianto_22305144028_Plot 3D-068.png}
\begin{eulerprompt}
>plot3d("cos(x)^3*sin(y)","sin(x)^2*sin(y)","cos(x)^2", a=0,b=2*pi,c=pi/2,d=-pi/2,...
>>hue,color=blue,light=[0,1,5],<frame,...
>n=90,grid=[20,50],wirecolor=black,zoom=5):
\end{eulerprompt}
\eulerimg{17}{images/Vikram Zaky Ardianto_22305144028_Plot 3D-069.png}
\begin{eulerprompt}
>plot3d("cos(x)^2*cos(y)","sin(x)^2*cos(y)","cos(x)^2", a=0,b=2*pi,c=pi/2,d=-pi/2,...
>>hue,color=blue,light=[0,1,5],<frame,...
>n=90,grid=[10,50],wirecolor=black,zoom=5):
\end{eulerprompt}
\eulerimg{17}{images/Vikram Zaky Ardianto_22305144028_Plot 3D-070.png}
\begin{eulerprompt}
>plot3d("cos(x)*cos(y)","sin(x)*cos(y)","sin(y)", a=0,b=2*pi,c=pi/2,d=-pi/2,...
>>hue,color=blue,light=[0,1,3],<frame,...
>n=90,grid=[20,50],wirecolor=black,zoom=5):
\end{eulerprompt}
\eulerimg{17}{images/Vikram Zaky Ardianto_22305144028_Plot 3D-071.png}
\eulerheading{8. Menggambar Fungsi Implisit Implisit}
\begin{eulercomment}
Fungsi implisit (implicit function) adalah fungsi yang memuat lebih
dari satu variabel, berjenis variabel bebas dan variabel terikat yang
berada dalam satu ruas sehingga tidak bisa dipisahkan pada ruas yang
berbeda.

\end{eulercomment}
\begin{eulerformula}
\[
F(x,y,z)=0
\]
\end{eulerformula}
\begin{eulercomment}
(1 persamaan dan 3 variabel), terdiri dari 2 variabel bebas dan 1
terikat

Misalnya,\\
\end{eulercomment}
\begin{eulerformula}
\[
F(x, y, z) = x^2 + y^2 + z^2 = 1
\]
\end{eulerformula}
\begin{eulercomment}
adalah persamaan implisit yang menggambarkan bola dengan jari-jari 1
dan pusat di (0,0,0).

\end{eulercomment}
\begin{eulerprompt}
>zplot3d("x^2+y^3+z*y-1", r=5, implicit=3):
\end{eulerprompt}
\begin{euleroutput}
  Function zplot3d not found.
  Try list ... to find functions!
  Error in:
  zplot3d("x^2+y^3+z*y-1", r=5, implicit=3): ...
                                           ^
\end{euleroutput}
\begin{eulerprompt}
>c=1; d=1;
>plot3d("((x^2+y^2-c^2)^2+(z^2-1)^2)*((y^2+z^2-c^2)^2+(x^2-1)^2)*((z^2+x^2-c^2)^2+(y^2-1)^2)-d", r=2, <frame,>implicit,>user):
\end{eulerprompt}
\eulerimg{17}{images/Vikram Zaky Ardianto_22305144028_Plot 3D-074.png}
\begin{eulerprompt}
>plot3d("x^2+y^2+4*x*z+z^3",>implicit, r=2, zoom=2.5):
\end{eulerprompt}
\eulerimg{17}{images/Vikram Zaky Ardianto_22305144028_Plot 3D-075.png}
\begin{eulercomment}
Selain plot kontur yang sudah di jelaskan sebelumnya, pada EMT juga
ada plot umplisit dalam tiga dimensi. Euler menghasilkan potongan
melalui objek. Fitur plot3d termasuk plot implisit. Plot-plot ini
menunjukkan himpunan nol dari sebuah fungsi dalam tiga variabel.

Solusi dari\\
\end{eulercomment}
\begin{eulerformula}
\[
f(x,y,z) = 0
\]
\end{eulerformula}
\begin{eulercomment}
dapat divisualisasikan dalam potongan yang sejajar dengan bidang x-y,
bidang x-z, dan bidang y-z.

- implisit = 1: potong sejajar dengan bidang-y-z\\
- implicit = 2: memotong sejajar dengan bidang x-z\\
- implicit=4: memotong sejajar dengan bidang x-y

Ambil contoh dari persamaan latex pada fungsi implisit tadi dan
tambahkan nilai-nilai ini, sehingga kita dapat memplot persamaan ini\\
\end{eulercomment}
\begin{eulerformula}
\[
M = {(x,y,z) :{ x^2+y^3+zy=1}}
\]
\end{eulerformula}
\begin{eulerprompt}
>plot3d("x^2+y^3+z*y", r=1, implicit=2):
\end{eulerprompt}
\eulerimg{17}{images/Vikram Zaky Ardianto_22305144028_Plot 3D-076.png}
\begin{eulercomment}
Contoh fungsi implisit yang lainnya
\end{eulercomment}
\begin{eulerprompt}
>plot3d("x^3+y^3+z*y-1",r=7,implicit=4):
\end{eulerprompt}
\eulerimg{17}{images/Vikram Zaky Ardianto_22305144028_Plot 3D-077.png}
\begin{eulerprompt}
>plot3d("2*x^2 + 3*y^2 + z^2 - 25",r=8,implicit=2):
\end{eulerprompt}
\eulerimg{17}{images/Vikram Zaky Ardianto_22305144028_Plot 3D-078.png}
\begin{eulerprompt}
>plot3d("x^5 + 5*y^3 + 3*z^2 - 5*x - 7*y - 5*z + 10",r=5,implicit=2):
\end{eulerprompt}
\eulerimg{17}{images/Vikram Zaky Ardianto_22305144028_Plot 3D-079.png}
\begin{eulerprompt}
>plot3d("x^3+y^5+5*x*z+z^3",>implicit,r=3,zoom=2):
\end{eulerprompt}
\eulerimg{17}{images/Vikram Zaky Ardianto_22305144028_Plot 3D-080.png}
\begin{eulerprompt}
>plot3d("x^2+y^2+4*x*z+z^3-5",>implicit,r=2,zoom=2.5):
\end{eulerprompt}
\eulerimg{17}{images/Vikram Zaky Ardianto_22305144028_Plot 3D-081.png}
\eulerheading{9. Menggambar Titik pada Ruang Tiga Dimensi}
\begin{eulercomment}
Untuk menggambar titik pada ruang tiga dimensi kita memerlukan tiga
vektor untuk koordinat titik serta menambahkan parameter points=true.

\end{eulercomment}
\begin{eulerprompt}
>n=510; ...
>plot3d(normal(1,n),normal(1,n),normal(1,n),points=true,style="."):
\end{eulerprompt}
\eulerimg{13}{images/Vikram Zaky Ardianto_22305144028_Plot 3D-082.png}
\begin{eulercomment}
\textgreater{}n=510; ...\\
\textgreater{}plot3d(normal(1,n),normal(1,n),normal(1,n),points=true,style="."):

- plot3d() untuk menjalankan perintah membuat plot 3D.\\
- normal(1,n) sebagai titik koordinat yang akan diplot pada sumbu
x,y,z dengan nilai angka dstribusi normal yang dicetak secara random
sebanyak n sehingga membentuk matriks 1xn atau 1x500.\\
- points=true sebagai parameter yang memerintahkan plot3d akan
menampilkan points(titik-titik).

\end{eulercomment}
\begin{eulerprompt}
>n=30; ...
>plot3d(normal(1,n),normal(1,n),normal(1,n),points=true,style="*"):
\end{eulerprompt}
\eulerimg{13}{images/Vikram Zaky Ardianto_22305144028_Plot 3D-083.png}
\begin{eulercomment}
1. n = 30; Ini adalah pernyataan untuk menginisialisasi variabel n
dengan nilai 30. Variabel ini kemungkinan akan digunakan untuk
menentukan jumlah titik yang akan digunakan dalam plot.

2. normal(1, n): Fungsi normal digunakan untuk menghasilkan
nilai-nilai acak yang terdistribusi secara normal (disebut juga
Gaussian) dengan rata-rata 1 dan deviasi standar 1. Ini berarti Anda
akan mendapatkan n nilai acak yang terdistribusi secara normal dengan
rata-rata 1 dan deviasi standar 1. Anda melakukan ini untuk
mendapatkan koordinat x, y, dan z untuk plot 3D.

3. plot3d(...): Ini adalah fungsi yang digunakan untuk membuat plot 3D
dengan parameter-parameter berikut:

\end{eulercomment}
\begin{eulerttcomment}
 -normal(1, n): Ini adalah data koordinat x, y, dan z yang telah
\end{eulerttcomment}
\begin{eulercomment}
dihasilkan sebelumnya.\\
\end{eulercomment}
\begin{eulerttcomment}
 -points = true: Ini mengatur agar titik-titik data ditampilkan dalam
\end{eulerttcomment}
\begin{eulercomment}
plot.\\
\end{eulercomment}
\begin{eulerttcomment}
 -style = "*": Ini mengatur gaya plot menjadi tanda bintang (*).
\end{eulerttcomment}
\begin{eulerprompt}
>x=[1,2,3,4]; y=[4,5,6,1]; z=[6,1,2,3];
>plot3d(x,y,z,points=true,style="."):
\end{eulerprompt}
\eulerimg{13}{images/Vikram Zaky Ardianto_22305144028_Plot 3D-084.png}
\begin{eulercomment}
\textgreater{}x=[1,2,3,4]; y=[4,5,6,1]; z=[6,1,2,3];\\
\textgreater{}plot3d(x,y,z,points=true,style="."):

- plot3d() untuk menjalankan perintah membuat plot 3D.\\
- Titik koordinat yang akan diplot pada sumbu x,y,z telah
didefinisikan oleh vektor baris x,y,z sebelumnya.\\
- points=true sebagai parameter yang memerintahkan plot3d akan
menampilkan points(titik-titik).

\end{eulercomment}
\begin{eulerprompt}
>x=[1,1,0,-4,4]; y=[2,-11,7,1,9]; z=[0,8,4,7,7];
>plot3d(x,y,z,points=true,zoom=3,style="/"):
\end{eulerprompt}
\eulerimg{13}{images/Vikram Zaky Ardianto_22305144028_Plot 3D-085.png}
\begin{eulercomment}
1.x, y, dan z: Ini adalah tiga vektor yang berisi koordinat
titik-titik dalam tiga dimensi. Dalam contoh ini, x berisi [1, 1, 0,
-4, 4], y berisi [2, -11, 7, 1, 9], dan z berisi [0, 8, 4, 7, 7].
Setiap elemen dalam vektor-vektor ini mewakili koordinat satu titik
dalam ruang 3D.

2.plot3d(...): Ini adalah fungsi yang digunakan untuk membuat plot 3D
dengan parameter-parameter berikut:

-.x, y, dan z: Ini adalah data koordinat yang akan digunakan untuk
membuat plot.\\
-points = true: Ini mengatur agar titik-titik data ditampilkan dalam
plot. Dengan points = true, Anda akan melihat titik-titik yang
mewakili koordinat data.\\
-zoom = 3: Ini mengatur tingkat zoom plot. Dengan zoom = 3, plot akan
diperbesar sebanyak tiga kali dari ukuran defaultnya.\\
-style = "/": Ini mengatur gaya plot menjadi tanda garis miring ("/").
\end{eulercomment}
\begin{eulerprompt}
>a=random(1,5); b=linspace(10,18,4); c=normal(1,5); ...
>plot3d(a,b,c,scale=[5,1,3],points=true,style="'"):
\end{eulerprompt}
\eulerimg{13}{images/Vikram Zaky Ardianto_22305144028_Plot 3D-086.png}
\begin{eulercomment}
\textgreater{}a=random(1,5); b=linspace(10,18,4); c=normal(1,5); ...\\
\end{eulercomment}
\begin{eulerttcomment}
 plot3d(a,b,c,scale=[5,1,3],points=true,style="'"):
\end{eulerttcomment}
\begin{eulercomment}

- plot3d() untuk menjalankan perintah membuat plot 3D.\\
- Titik koordinat yang akan diplot pada sumbu x,y,z didefinisikan oleh
x=a=random(1,5) yaitu bilangan acak dari 0-1 sebanyak 5 bilangan,
y=b=linspace(10,18,4) yaitu bilangan dari 10 hingga 18 dengan selisih
yang sama sebanyak 5 bilangan, z=c=normal(1,5) yaitu bilangan acak
distribusi normal sebanyak 5 bilangan.\\
- points=true sebagai parameter yang memerintahkan plot3d akan
menampilkan points(titik-titik).
\end{eulercomment}
\begin{eulercomment}

\begin{eulercomment}
\eulerheading{10. Mengatur Tampilan, Warna dan }
\begin{eulercomment}
* Angle Gambar Permukaan 3D

Dalam plot3d terdapat banyak function terkait tampilan gambar 3D, di
antaranya:

sliced:\\
\end{eulercomment}
\begin{eulerttcomment}
  Memplot versi irisan (0=tidak, 1=arah-x, 2=arah-y).
\end{eulerttcomment}
\begin{eulercomment}
hue :\\
\end{eulercomment}
\begin{eulerttcomment}
  Menghitung bayangan menggunakan sumber cahaya.
\end{eulerttcomment}
\begin{eulercomment}
light, amb, max :\\
\end{eulercomment}
\begin{eulerttcomment}
  Mengontrol pengaturan bayangan titik cahaya, ambient dan maksimum.
\end{eulerttcomment}
\begin{eulercomment}
contour  :\\
\end{eulercomment}
\begin{eulerttcomment}
  Menampilkan garis level tebal (dengan level otomatis).
\end{eulerttcomment}
\begin{eulercomment}
spectral:\\
\end{eulercomment}
\begin{eulerttcomment}
  Gunakan warna spektral alih-alih rona monokrom. Terdapat
  skema spektral dari spektral = 1 (> spektral) hingga spektral = 9.
  >Default >spectral adalah rona warna dan ini setara dengan
  color=-2 hingga color=-10.
\end{eulerttcomment}
\begin{eulercomment}
xhue, yhue, zhue:\\
\end{eulercomment}
\begin{eulerttcomment}
  Gunakan koordinat ini sebagai pengganti sumber cahaya.
\end{eulerttcomment}
\begin{eulercomment}
hues :\\
\end{eulercomment}
\begin{eulerttcomment}
  Matriks nilai rona dari 0 sampai 1 untuk bayangan untuk plot x-y-z.
  Matriks harus memiliki ukuran yang kompatibel dengan x, y, z.
\end{eulerttcomment}
\begin{eulercomment}
contourcolor :\\
\end{eulercomment}
\begin{eulerttcomment}
  Warna garis kontur.
\end{eulerttcomment}
\begin{eulercomment}
contourwidth :\\
\end{eulercomment}
\begin{eulerttcomment}
  Lebar garis kontur.
\end{eulerttcomment}
\begin{eulercomment}
fillcolor :\\
\end{eulercomment}
\begin{eulerttcomment}
  Warna isian untuk permukaan 3d tanpa rona.
\end{eulerttcomment}
\begin{eulercomment}
user :\\
\end{eulercomment}
\begin{eulerttcomment}
  Pengguna dapat memutar plot dengan keyboard kiri, kanan, atas,
  bawah. +,- memperbesar plot. Spasi mengatur ulang plot. Return
  mengakhiri interaksi pengguna. Tombol a menghasilkan plot anaglyph.
  Tombol l mengalihkan pergerakan sumber cahaya untuk plot rona.
  Tombol c menggerakkan plot ke atas, bawah, kiri, atau kanan.
\end{eulerttcomment}
\begin{eulercomment}
rotate :\\
\end{eulercomment}
\begin{eulerttcomment}
  Memutar plot sebuah fungsi dalam satu ekspresi dalam x.
\end{eulerttcomment}
\begin{eulercomment}
anaglyph :\\
\end{eulercomment}
\begin{eulerttcomment}
  Menghasilkan plot 3d anaglyph (>anaglyph). Plot ini membutuhkan
  kacamata merah untuk dapat dilihat dengan baik.
\end{eulerttcomment}
\begin{eulercomment}
viewangle :\\
\end{eulercomment}
\begin{eulerttcomment}
  Sudut pandang default, diputar di sekitar z-
\end{eulerttcomment}
\begin{eulercomment}
zoom :\\
\end{eulercomment}
\begin{eulerttcomment}
  Pembesaran tampilan. Standarnya adalah sekitar 2,6.
\end{eulerttcomment}
\begin{eulercomment}
view :\\
\end{eulercomment}
\begin{eulerttcomment}
  Tampilan lengkap, vektor 1x4 yang berisi jarak, zoom, sudut pandang,
  tinggi pandang.
\end{eulerttcomment}
\begin{eulercomment}
center :\\
\end{eulercomment}
\begin{eulerttcomment}
  Vektor ini memindahkan pusat plot. Hal ini diperlukan jika plot
  tidak dipusatkan di (0,0,0) secara otomatis.
\end{eulerttcomment}
\begin{eulercomment}
style :\\
\end{eulercomment}
\begin{eulerttcomment}
  Gaya plot.
\end{eulerttcomment}
\begin{eulercomment}
color :\\
\end{eulercomment}
\begin{eulerttcomment}
  Warna untuk objek dan permukaan yang diarsir
\end{eulerttcomment}
\begin{eulercomment}
wirecolor :\\
\end{eulercomment}
\begin{eulerttcomment}
  Warna untuk plot kawat
\end{eulerttcomment}
\begin{eulercomment}
cp :\\
\end{eulercomment}
\begin{eulerttcomment}
  Menggambar bidang kontur di bawah plot (>cp).
\end{eulerttcomment}
\begin{eulercomment}
cpcolor :\\
\end{eulercomment}
\begin{eulerttcomment}
  Warna untuk bidang kontur.
\end{eulerttcomment}
\begin{eulerprompt}
>plot3d("x*y",r=4,title="z=x*y",zoom=5):
\end{eulerprompt}
\eulerimg{13}{images/Vikram Zaky Ardianto_22305144028_Plot 3D-087.png}
\begin{eulerprompt}
>plot3d("x*y^3",>user,r=1,>anaglyph,title="Press cursor keys or return!"):
\end{eulerprompt}
\eulerimg{13}{images/Vikram Zaky Ardianto_22305144028_Plot 3D-088.png}
\begin{eulerprompt}
>plot3d("x^2*y^3",r=0.9,zlabel="x^2*y^3",>user,zoom=3, ...
>fillcolor=[2,6],>cp,cpcolor=blue):
\end{eulerprompt}
\eulerimg{13}{images/Vikram Zaky Ardianto_22305144028_Plot 3D-089.png}
\begin{eulerprompt}
>plot3d("x^2+y^3",angle=80°,>contour,spectral=2):
\end{eulerprompt}
\eulerimg{13}{images/Vikram Zaky Ardianto_22305144028_Plot 3D-090.png}
\begin{eulerprompt}
>plot3d("x^y-y^x",a=0,b=4,c=0,d=4,angle=90°,>contour, ...
>  contourwidth=4,contourcolor=red):
\end{eulerprompt}
\eulerimg{13}{images/Vikram Zaky Ardianto_22305144028_Plot 3D-091.png}
\begin{eulerprompt}
>plot3d("x^2+3y^2",>wire,>anaglyph,title="Use Red/Cyan Glasses!"):
\end{eulerprompt}
\eulerimg{13}{images/Vikram Zaky Ardianto_22305144028_Plot 3D-092.png}
\begin{eulerprompt}
>plot3d("x^3+10y^2",0,2,0,10,scale=[5,1,2],zoom=3,grid=10,>transparent):
\end{eulerprompt}
\eulerimg{13}{images/Vikram Zaky Ardianto_22305144028_Plot 3D-093.png}
\begin{eulerprompt}
>x=-1:0.05:1; y=x'; plot3d(x,x*y^2,y,>user,>hue,angle=20°):
\end{eulerprompt}
\eulerimg{13}{images/Vikram Zaky Ardianto_22305144028_Plot 3D-094.png}
\begin{eulerprompt}
>X=normal(3,50); plot3d(X[1],X[2],X[3],>points,style="/",zoom=3,>user):
\end{eulerprompt}
\eulerimg{13}{images/Vikram Zaky Ardianto_22305144028_Plot 3D-095.png}
\eulerheading{11. Plot Kontur}
\begin{eulercomment}
Untuk plot, Euler menambahkan garis grid. Sebagai gantinya
dimungkinkan untuk menggunakan garis level dan rona satu warna atau
rona berwarna spektral. Euler dapat menggambar tinggi fungsi pada plot
dengan bayangan. Di semua plot 3D, Euler dapat menghasilkan anaglyph
merah/sian.

- \textgreater{}hue: Menyalakan bayangan cahaya alih-alih kabel.\\
- \textgreater{}contour: Memplot garis kontur otomatis pada plot.\\
- level=... (atau levels): Sebuah vektor nilai untuk garis kontur.

Standarnya adalah level="auto", yang menghitung beberapa garis level
secara otomatis. Seperti yang Anda lihat di plot, level sebenarnya
adalah rentang level.

Gaya default dapat diubah. Untuk plot kontur berikut, kami menggunakan
grid yang lebih halus untuk 100x100 poin, skala fungsi dan plot, dan
menggunakan sudut pandang yang berbeda.
\end{eulercomment}
\begin{eulerprompt}
>plot3d("exp(-x^2-y^3)",r=2,n=100,level="thin", ...
\end{eulerprompt}
\begin{eulercomment}
1."exp(-x\textasciicircum{}2-y\textasciicircum{}3)": Ini adalah fungsi matematika yang akan digunakan
untuk membuat plot 3D. Fungsi ini adalah fungsi Gaussian dua dimensi
yang bergantung pada variabel x dan y.

2.r = 2: Ini adalah parameter r yang mengatur jangkauan atau rentang
dari plot 3D. Dalam hal ini, r diatur menjadi 2, yang mungkin mengacu
pada jangkauan plot dalam koordinat x dan y.

3.n = 100: Ini adalah parameter n yang mengatur jumlah titik sampel
dalam plot. Dalam hal ini, ada 100 titik sampel yang akan digunakan
untuk menggambarkan plot.

4.level = "thin": Ini adalah parameter level yang mengatur ketebalan
atau tipe garis yang digunakan dalam plot. "Thin" mungkin mengacu pada
penggunaan garis tipis dalam plot.

5.contour: Ini adalah parameter yang menambahkan garis kontur ke plot.
Ini memungkinkan Anda melihat kontur atau garis isovalue dalam plot
yang menggambarkan tingkat nilai fungsi.

6.spectral: Ini adalah parameter yang mengatur skema warna plot
menjadi skema warna spektral.

7.fscale = 1: Ini adalah parameter fscale yang mengatur skala faktor
untuk plot.

8.scale = 1.1: Ini adalah parameter scale yang mengatur faktor skala
untuk plot.

9.angle = 45°: Ini adalah parameter angle yang mengatur sudut tampilan
plot. Dalam hal ini, plot akan dilihat dari sudut 45 derajat.

10.height = 20°: Ini adalah parameter height yang mengatur tinggi plot
dalam derajat.
\end{eulercomment}
\begin{eulerprompt}
>>contour,>spectral,fscale=1,scale=1.1,angle=45°,height=20°):
\end{eulerprompt}
\eulerimg{13}{images/Vikram Zaky Ardianto_22305144028_Plot 3D-096.png}
\begin{eulerprompt}
>plot3d("exp(x*y)",angle=100°,>contour,color=green):
\end{eulerprompt}
\eulerimg{13}{images/Vikram Zaky Ardianto_22305144028_Plot 3D-097.png}
\begin{eulercomment}
1. "exp(x*y)": Ini adalah fungsi matematika yang akan digunakan untuk
membuat plot 3D. Fungsi ini adalah eksponensial dari hasil perkalian
antara x dan y. Dalam konteks ini, x dan y adalah variabel-variabel
dalam plot.

2. angle = 100°: Ini adalah parameter angle yang mengatur sudut
tampilan plot. Dalam hal ini, plot akan dilihat dari sudut 100
derajat.

3. contour: Ini adalah parameter yang menambahkan garis kontur ke
plot. Ini memungkinkan Anda melihat kontur atau garis isovalue dalam
plot yang menggambarkan tingkat nilai fungsi.

4. color = green: Ini adalah parameter color yang mengatur warna plot.
Dalam hal ini, plot akan menggunakan warna hijau.
\end{eulercomment}
\begin{eulerprompt}
>plot3d("x^2+y^2",>spectral,>contour,n=100):
\end{eulerprompt}
\eulerimg{13}{images/Vikram Zaky Ardianto_22305144028_Plot 3D-098.png}
\begin{eulercomment}
1. "x\textasciicircum{}2 + y\textasciicircum{}2": Ini adalah fungsi matematika yang akan digunakan untuk
membuat plot 3D. Fungsi ini adalah fungsi kuadrat dari variabel x dan
y. Dalam konteks ini, x dan y adalah variabel-variabel dalam plot.

2. spectral: Ini adalah parameter yang mengatur skema warna plot
menjadi skema warna spektral. Dengan menggunakan skema warna spektral,
berbagai nilai dalam plot akan diberikan warna yang berbeda, yang
memudahkan untuk memahami perubahan nilai dalam fungsi.

3. contour: Ini adalah parameter yang menambahkan garis kontur ke
plot. Ini memungkinkan Anda melihat kontur atau garis isovalue dalam
plot yang menggambarkan tingkat nilai fungsi.

4. n = 100: Ini adalah parameter n yang mengatur jumlah titik sampel
dalam plot. Dalam hal ini, ada 100 titik sampel yang akan digunakan
untuk menggambarkan plot. Semakin banyak titik sampel, semakin halus
plotnya.
\end{eulercomment}
\begin{eulerprompt}
>plot3d("x^2-y^2",0,1,0,1,angle=220°,level=-1:0.2:1,color=redgreen):
\end{eulerprompt}
\eulerimg{13}{images/Vikram Zaky Ardianto_22305144028_Plot 3D-099.png}
\begin{eulercomment}
1. "x\textasciicircum{}2 - y\textasciicircum{}2": Ini adalah fungsi matematika yang akan digunakan untuk
membuat plot 3D. Fungsi ini adalah perbedaan antara kuadrat variabel x
dan kuadrat variabel y. Dalam konteks ini, x dan y adalah
variabel-variabel dalam plot.

2. 0, 1, 0, 1: Ini adalah parameter yang mengatur batasan tampilan
plot. Angka-angka ini mewakili batas minimum dan maksimum untuk x dan
y. Dalam hal ini, plot akan dibatasi pada rentang x dan y antara 0 dan
1.

3. angle = 220°: Ini adalah parameter angle yang mengatur sudut
tampilan plot. Dalam hal ini, plot akan dilihat dari sudut 220
derajat.

4. level = -1:0.2:1: Ini adalah parameter level yang mengatur tingkat
nilai fungsi yang akan ditampilkan dalam plot. Rentang ini (-1 hingga
1) akan dibagi menjadi beberapa tingkat, dengan selang 0.2 antara
masing-masing tingkat. Ini akan menghasilkan garis kontur pada tingkat
nilai fungsi tertentu.

5. color = redgreen: Ini adalah parameter color yang mengatur skema
warna plot. Warna yang digunakan adalah kombinasi warna merah dan
hijau.
\end{eulercomment}
\begin{eulerprompt}
>plot3d("x^2+y^3",level=[-0.1,0.9;0,1], ...
>  >spectral,angle=30°,grid=10,contourcolor=gray):
\end{eulerprompt}
\eulerimg{13}{images/Vikram Zaky Ardianto_22305144028_Plot 3D-100.png}
\begin{eulercomment}
1."x\textasciicircum{}2 + y\textasciicircum{}3": Ini adalah fungsi matematika yang akan digunakan untuk
membuat plot 3D. Fungsi ini adalah hasil penjumlahan dari kuadrat
variabel x dan kubik variabel y. Dalam konteks ini, x dan y adalah
variabel-variabel dalam plot.

2.level = [-0.1, 0.9; 0, 1]: Ini adalah parameter level yang mengatur
tingkat nilai fungsi yang akan ditampilkan dalam plot. Parameter ini
didefinisikan sebagai matriks dua baris dengan dua kolom. Setiap baris
berisi batasan tingkat nilai fungsi yang akan ditampilkan dalam plot.
Misalnya, [-0.1, 0.9] menunjukkan bahwa tingkat nilai akan ditampilkan
dari -0.1 hingga 0.9, dan [0, 1] menunjukkan bahwa tingkat nilai kedua
akan ditampilkan dari 0 hingga 1.

3.spectral: Ini adalah parameter yang mengatur skema warna plot
menjadi skema warna spektral.

4.angle = 30°: Ini adalah parameter angle yang mengatur sudut tampilan
plot. Dalam hal ini, plot akan dilihat dari sudut 30 derajat.

5.grid = 10: Ini adalah parameter grid yang mengatur jumlah garis kisi
dalam plot. Dalam hal ini, akan ada 10 garis kisi dalam plot.

6.contourcolor = gray: Ini adalah parameter contourcolor yang mengatur
warna garis kontur dalam plot menjadi abu-abu (gray).

Dalam contoh berikut, kami memplot himpunan, di mana

\end{eulercomment}
\begin{eulerformula}
\[
f(x,y) = x^y-y^x = 0
\]
\end{eulerformula}
\begin{eulercomment}
Kami menggunakan satu garis tipis untuk garis level.
\end{eulercomment}
\begin{eulerprompt}
>plot3d("x^y-y^x",level=0,a=0,b=6,c=0,d=6,contourcolor=red,n=100):
\end{eulerprompt}
\eulerimg{13}{images/Vikram Zaky Ardianto_22305144028_Plot 3D-101.png}
\eulerheading{12. Menggambar Grafik Tiga Dimensi}
\begin{eulercomment}
* alam modus anaglif
\end{eulercomment}
\begin{eulerprompt}
>X=cumsum(normal(3,100)); ...
> plot3d(X[1],X[2],X[3],>anaglyph,>wire):
\end{eulerprompt}
\eulerimg{13}{images/Vikram Zaky Ardianto_22305144028_Plot 3D-102.png}
\begin{eulercomment}
1.X = cumsum(normal(3, 100));: Ini adalah urutan perintah yang
melakukan beberapa operasi berurutan.

a)normal(3, 100): Ini adalah panggilan fungsi normal yang digunakan
untuk menghasilkan 100 bilangan acak dengan distribusi normal
(Gaussian) dengan rata-rata 3 dan deviasi standar 1. Hasilnya adalah
vektor tiga dimensi yang berisi 100 titik acak dalam ruang tiga
dimensi.\\
b)cumsum(...): Ini adalah panggilan fungsi cumsum yang digunakan untuk
menghitung kumulatif dari vektor 3D yang dihasilkan sebelumnya. Dengan
kata lain, ini akan menghasilkan vektor yang merupakan akumulasi
(penjumlahan berulang) dari vektor 3D tersebut. Hasilnya adalah vektor
tiga dimensi yang menggambarkan perjalanan dalam ruang 3D berdasarkan
perubahan titik acak.\\
2.plot3d(X[1], X[2], X[3], anaglyph, wire);: Ini adalah perintah untuk
membuat plot 3D dari data yang telah dihasilkan sebelumnya.

a).X[1], X[2], dan X[3] adalah komponen vektor tiga dimensi X yang
akan digunakan sebagai koordinat dalam plot 3D. X[1] digunakan sebagai
koordinat sumbu x, X[2] digunakan sebagai koordinat sumbu y, dan X[3]
digunakan sebagai koordinat sumbu z.\\
b)anaglyph: Ini adalah parameter yang mengatur plot menggunakan efek
anaglif. Anaglif adalah teknik yang digunakan untuk menghasilkan efek
tiga dimensi (3D) dengan menggunakan dua gambar yang sedikit berbeda
untuk mata kiri dan kanan, dan penonton memerlukan kacamata anaglif
khusus untuk melihat efek 3D.\\
c)wire: Ini adalah parameter yang mengatur plot sebagai plot tali
(wireframe), yang berarti hanya garis-garis yang menghubungkan
titik-titik yang akan ditampilkan dalam plot.\\
4.X[1], X[2], dan X[3]: Ini adalah komponen dari vektor X. X[1]
digunakan sebagai koordinat sumbu x, X[2] digunakan sebagai koordinat
sumbu y, dan X[3] digunakan sebagai koordinat sumbu z. Dengan
menggunakan komponen vektor ini sebagai koordinat, Anda membuat plot
3D yang merepresentasikan perubahan dalam tiga dimensi berdasarkan
data dalam vektor X.

5.anaglyph: Ini adalah parameter yang mengatur plot menggunakan efek
anaglif. Efek anaglif adalah teknik yang digunakan untuk menghasilkan
efek tiga dimensi (3D) dengan menggunakan dua gambar yang sedikit
berbeda untuk mata kiri dan kanan. Penonton memerlukan kacamata
anaglif khusus dengan lensa berwarna berbeda untuk mata kiri dan kanan
untuk melihat efek 3D ini. Penggunaan anaglyph dalam kode ini
menunjukkan bahwa plot akan dibuat dengan efek anaglif.

6.wire: Ini adalah parameter yang mengatur plot sebagai plot tali
(wireframe). Wireframe adalah gaya plot di mana hanya garis-garis yang
menghubungkan titik-titik yang akan ditampilkan dalam plot. Dengan
pengaturan wire, Anda akan melihat plot dalam bentuk rangkaian
garis-garis yang menggambarkan bentuk objek dalam tampilan 3D.
\end{eulercomment}
\begin{eulerprompt}
>plot3d("x^2+y^3",>anaglyph,>contour,angle=30°):
\end{eulerprompt}
\eulerimg{13}{images/Vikram Zaky Ardianto_22305144028_Plot 3D-103.png}
\begin{eulercomment}
1. "x\textasciicircum{}2 + y\textasciicircum{}3": Ini adalah fungsi matematika yang akan digunakan untuk
membuat plot 3D. Fungsi ini adalah hasil penjumlahan dari kuadrat
variabel x dan kubik variabel y. Dalam konteks ini, x dan y adalah
variabel-variabel dalam plot.

2. anaglyph: Ini adalah parameter yang mengatur plot menggunakan efek
anaglif. Anaglif adalah teknik yang digunakan untuk menghasilkan efek
tiga dimensi (3D) dengan menggunakan dua gambar yang sedikit berbeda
untuk mata kiri dan kanan. Penonton memerlukan kacamata anaglif khusus
dengan lensa berwarna berbeda untuk mata kiri dan kanan untuk melihat
efek 3D ini. Dengan pengaturan ini, plot akan dibuat dengan efek 3D
anaglif.

3. contour: Ini adalah parameter yang menambahkan garis kontur ke
plot. Ini memungkinkan Anda melihat kontur atau garis isovalue dalam
plot yang menggambarkan tingkat nilai fungsi.

4. angle = 30°: Ini adalah parameter angle yang mengatur sudut
tampilan plot. Dalam hal ini, plot akan dilihat dari sudut 30 derajat.
\end{eulercomment}
\begin{eulerprompt}
>u=linspace(-1,1,10); v=linspace(0,2*pi,50)'; ...
>X=(3+u*cos(v/2))*cos(v); Y=(3+u*cos(v/2))*sin(v); Z=u*sin(v/2); ...
>plot3d(X,Y,Z,>anaglyph,<frame,>wire,scale=2.3):
\end{eulerprompt}
\eulerimg{13}{images/Vikram Zaky Ardianto_22305144028_Plot 3D-104.png}
\begin{eulercomment}
1. u = linspace(-1, 1, 10);: Ini adalah perintah untuk membuat vektor
u yang berisi 10 nilai yang dihasilkan secara merata dalam rentang -1
hingga 1. Vektor ini akan digunakan dalam perhitungan selanjutnya.

2.v = linspace(0, 2 * pi, 50)';: Ini adalah perintah untuk membuat
vektor v yang berisi 50 nilai yang dihasilkan secara merata dalam
rentang 0 hingga 2p (dua kali nilai p). Vektor ini juga akan digunakan
dalam perhitungan selanjutnya.

3.X, Y, dan Z: Ini adalah perintah-perintah yang digunakan untuk
menghasilkan data koordinat dalam tiga dimensi. Data ini dihasilkan
berdasarkan persamaan yang menggunakan nilai u dan v. Data ini akan
digunakan untuk membuat plot 3D.

4.plot3d(X, Y, Z, anaglyph, \textless{}frame, wire, scale = 2.3);: Ini adalah
perintah untuk membuat plot 3D berdasarkan data X, Y, dan Z yang telah
dihasilkan sebelumnya. Parameter-parameter yang digunakan dalam
perintah ini adalah:

a)anaglyph: Ini adalah parameter yang mengatur plot menggunakan efek
anaglif, yang memberikan efek tiga dimensi (3D) saat melihat plot
dengan kacamata anaglif.\\
b)\textless{}frame: Ini adalah parameter yang mengatur agar frame (kerangka)
plot ditampilkan. Ini adalah bingkai atau batasan dari plot.\\
c)wire: Ini adalah parameter yang mengatur plot dalam bentuk rangkaian
garis-garis (wireframe).\\
d)scale = 2.3: Ini adalah parameter yang mengatur faktor skala plot
sebesar 2.3. Ini akan memperbesar plot.
\end{eulercomment}
\begin{eulerprompt}
>u:=linspace(-pi,pi,160); v:=linspace(-pi,pi,400)';  ...
>x:=(4*(1+.25*sin(3*v))+cos(u))*cos(2*v); ...
>y:=(4*(1+.25*sin(3*v))+cos(u))*sin(2*v); ...
> z=sin(u)+2*cos(3*v); ...
>plot3d(x,y,z,frame=0,scale=1.5,hue=1,light=[1,0,-1],zoom=2.8,>anaglyph):
\end{eulerprompt}
\eulerimg{13}{images/Vikram Zaky Ardianto_22305144028_Plot 3D-105.png}
\eulerheading{13. Plot Statistik batang 3d}
\begin{eulercomment}
Plot bar juga dimungkinkan. Untuk ini, kita harus menyediakan

- x: vektor baris dengan n+1 elemen\\
- y: vektor kolom dengan n+1 elemen\\
- z: matriks nilai nxn.

z bisa lebih besar, tetapi hanya nilai nxn yang akan digunakan.

Dalam contoh, pertama-tama kita menghitung nilainya. Kemudian kita
sesuaikan x dan y, sehingga vektor berpusat pada nilai yang digunakan.
\end{eulercomment}
\begin{eulerprompt}
>x=-1:0.1:1; y=x'; z=x^2+y^2; ...
>xa=(x|1.1)-0.05; ya=(y_1.1)-0.05; ...
>plot3d(xa,ya,z,bar=true):
\end{eulerprompt}
\eulerimg{13}{images/Vikram Zaky Ardianto_22305144028_Plot 3D-106.png}
\begin{eulercomment}
Dimungkinkan untuk membagi plot permukaan menjadi dua atau lebih
bagian.
\end{eulercomment}
\begin{eulerprompt}
>x=-1:0.1:1; y=x'; z=x+y; d=zeros(size(x)); ...
>plot3d(x,y,z,disconnect=2:2:20):
\end{eulerprompt}
\eulerimg{13}{images/Vikram Zaky Ardianto_22305144028_Plot 3D-107.png}
\begin{eulercomment}
Jika memuat atau menghasilkan matriks data M dari file dan perlu
memplotnya dalam 3D, Anda dapat menskalakan matriks ke [-1,1] dengan
scale(M), atau menskalakan matriks dengan \textgreater{}zscale. Ini dapat
dikombinasikan dengan faktor penskalaan individu yang diterapkan
sebagai tambahan.
\end{eulercomment}
\begin{eulerprompt}
>i=1:20; j=i'; ...
>plot3d(i*j^2+100*normal(20,20),>zscale,scale=[1,1,1.5],angle=-40°,zoom=1.8):
\end{eulerprompt}
\eulerimg{13}{images/Vikram Zaky Ardianto_22305144028_Plot 3D-108.png}
\begin{eulerprompt}
>Z=intrandom(5,100,6); v=zeros(5,6); ...
>loop 1 to 5; v[#]=getmultiplicities(1:6,Z[#]); end; ...
>columnsplot3d(v',scols=1:5,ccols=[1:5]):
\end{eulerprompt}
\eulerimg{13}{images/Vikram Zaky Ardianto_22305144028_Plot 3D-109.png}
\eulerheading{14. Permukaan Benda Putar}
\begin{eulerprompt}
>plot2d("(x^2+y^2-1)^3-x^2*y^3",r=1.3, ...
>style="#",color=red,<outline, ...
>level=[-2;0],n=100):
\end{eulerprompt}
\eulerimg{13}{images/Vikram Zaky Ardianto_22305144028_Plot 3D-110.png}
\begin{eulerprompt}
>ekspresi &= (x^2+y^2-1)^3-x^2*y^3; $ekspresi
\end{eulerprompt}
\begin{eulerformula}
\[
\left(y^2+x^2-1\right)^3-x^2\,y^3
\]
\end{eulerformula}
\begin{eulercomment}
Kami ingin memutar kurva jantung di sekitar sumbu y. Berikut adalah
ungkapan, yang mendefinisikan hati:

\end{eulercomment}
\begin{eulerformula}
\[
f(x,y)=(x^2+y^2-1)^3-x^2.y^3.
\]
\end{eulerformula}
\begin{eulercomment}
Selanjutnya kita atur

\end{eulercomment}
\begin{eulerformula}
\[
x=r.cos(a),\quad y=r.sin(a).
\]
\end{eulerformula}
\begin{eulerprompt}
>function fr(r,a) &= ekspresi with [x=r*cos(a),y=r*sin(a)] | trigreduce; $fr(r,a)
\end{eulerprompt}
\begin{eulerformula}
\[
\left(r^2-1\right)^3+\frac{\left(\sin \left(5\,a\right)-\sin \left(  3\,a\right)-2\,\sin a\right)\,r^5}{16}
\]
\end{eulerformula}
\begin{eulercomment}
Hal ini memungkinkan untuk mendefinisikan fungsi numerik, yang
memecahkan r, jika a diberikan. Dengan fungsi itu kita dapat memplot
jantung yang diputar sebagai permukaan parametrik.
\end{eulercomment}
\begin{eulerprompt}
>function map f(a) := bisect("fr",0,2;a); ...
>t=linspace(-pi/2,pi/2,100); r=f(t);  ...
>s=linspace(pi,2pi,100)'; ...
>plot3d(r*cos(t)*sin(s),r*cos(t)*cos(s),r*sin(t), ...
>>hue,<frame,color=red,zoom=4,amb=0,max=0.7,grid=12,height=50°):
\end{eulerprompt}
\eulerimg{13}{images/Vikram Zaky Ardianto_22305144028_Plot 3D-113.png}
\begin{eulercomment}
Berikut ini adalah plot 3D dari gambar di atas yang diputar di sekitar
sumbu z. Kami mendefinisikan fungsi, yang menggambarkan objek.
\end{eulercomment}
\begin{eulerprompt}
>function f(x,y,z) ...
\end{eulerprompt}
\begin{eulerudf}
  r=x^2+y^2; ...
  return (r+z^2-1)^3-r*z^3;
  endfunction
\end{eulerudf}
\begin{eulerprompt}
>plot3d("f(x,y,z)", ...
>xmin=0,xmax=1.2,ymin=-1.2,ymax=1.2,zmin=-1.2,zmax=1.4, ...
>implicit=1,angle=-30°,zoom=2.5,n=[10,60,60],>anaglyph):
\end{eulerprompt}
\eulerimg{13}{images/Vikram Zaky Ardianto_22305144028_Plot 3D-114.png}
\eulerheading{15. Menggambar Povray}
\begin{eulercomment}
Dengan bantuan file Euler povray.e, Euler dapat menghasilkan file
Povray. Hasilnya sangat bagus untuk dilihat.

Anda perlu menginstal Povray (32bit atau 64bit) dari
http://www.povray.org/, dan meletakkan sub-direktori "bin" dari Povray ke jalur lingkungan, atau mengatur variabel "defaultpovray" dengan path lengkap yang menunjuk ke "pvengine.exe".

Antarmuka Povray dari Euler menghasilkan file Povray di direktori home
pengguna, dan memanggil Povray untuk mengurai file-file ini. Nama file
default adalah current.pov, dan direktori default adalah eulerhome(),
biasanya c:\textbackslash{}Users\textbackslash{}Username\textbackslash{}Euler. Povray menghasilkan file PNG, yang
dapat dimuat oleh Euler ke dalam buku catatan. Untuk membersihkan
file-file ini, gunakan povclear().

Fungsi pov3d memiliki semangat yang sama dengan plot3d. Ini dapat
menghasilkan grafik fungsi f(x,y), atau permukaan dengan koordinat
X,Y,Z dalam matriks, termasuk garis level opsional. Fungsi ini memulai
raytracer secara otomatis, dan memuat adegan ke dalam notebook Euler.

Selain pov3d(), ada banyak fungsi yang menghasilkan objek Povray.
Fungsi-fungsi ini mengembalikan string, yang berisi kode Povray untuk
objek. Untuk menggunakan fungsi ini, mulai file Povray dengan
povstart(). Kemudian gunakan writeln(...) untuk menulis objek ke file
adegan. Terakhir, akhiri file dengan povend(). Secara default,
raytracer akan dimulai, dan PNG akan dimasukkan ke dalam notebook
Euler.

Fungsi objek memiliki parameter yang disebut "look", yang membutuhkan
string dengan kode Povray untuk tekstur dan hasil akhir objek. Fungsi
povlook() dapat digunakan untuk menghasilkan string ini. Ini memiliki
parameter untuk warna, transparansi, Phong Shading dll.

Perhatikan bahwa alam semesta Povray memiliki sistem koordinat lain.
Antarmuka ini menerjemahkan semua koordinat ke sistem Povray. Jadi
Anda dapat terus berpikir dalam sistem koordinat Euler dengan z
menunjuk vertikal ke atas, a nd x,y,z sumbu dalam arti tangan kanan.\\
Anda perlu memuat file povray.
\end{eulercomment}
\begin{eulerprompt}
>load povray;
\end{eulerprompt}
\begin{eulercomment}
Pastikan, direktori bin Povray ada di jalur. Jika tidak, edit variabel
berikut sehingga berisi path ke povray yang dapat dieksekusi.
\end{eulercomment}
\begin{eulerprompt}
>defaultpovray="D:\(\backslash\)poray\(\backslash\)bin\(\backslash\)pvengine.exe"
\end{eulerprompt}
\begin{euleroutput}
  D:\(\backslash\)poray\(\backslash\)bin\(\backslash\)pvengine.exe
\end{euleroutput}
\begin{eulercomment}
Untuk kesan pertama, kami memplot fungsi sederhana. Perintah berikut
menghasilkan file povray di direktori pengguna Anda, dan menjalankan
Povray untuk ray tracing file ini.

Jika Anda memulai perintah berikut, GUI Povray akan terbuka,
menjalankan file, dan menutup secara otomatis. Karena alasan keamanan,
Anda akan ditanya, apakah Anda ingin mengizinkan file exe untuk
dijalankan. Anda dapat menekan batal untuk menghentikan pertanyaan
lebih lanjut. Anda mungkin harus menekan OK di jendela Povray untuk
mengakui dialog awal Povray.
\end{eulercomment}
\begin{eulerprompt}
>pov3d("x^2+y^2",zoom=3);
\end{eulerprompt}
\begin{euleroutput}
  
\end{euleroutput}
\begin{euleroutput}
  
\end{euleroutput}
\begin{eulercomment}
1. u := linspace(-pi, pi, 160);: Ini adalah perintah untuk membuat
vektor u yang berisi 160 nilai yang dihasilkan secara merata dalam
rentang dari -p hingga p.

2.v := linspace(-pi, pi, 400)';: Ini adalah perintah untuk membuat
vektor v yang berisi 400 nilai yang dihasilkan secara merata dalam
rentang dari -p hingga p. Vektor ini diubah menjadi matriks kolom
dengan penambahan tanda apostrof (') di belakangnya.

3.x, y, dan z: Ini adalah perintah-perintah yang digunakan untuk
menghasilkan data koordinat dalam tiga dimensi. Data ini dihasilkan
berdasarkan persamaan yang menggunakan nilai u dan v. Data ini akan
digunakan untuk membuat plot 3D.

4.plot3d(x, y, z, frame = 0, scale = 1.5, hue = 1, light = [1, 0, -1],
zoom = 2.8, anaglyph);: Ini adalah perintah untuk membuat plot 3D
berdasarkan data x, y, dan z yang telah dihasilkan sebelumnya.
Parameter-parameter yang digunakan dalam perintah ini adalah:

a)frame = 0: Ini adalah parameter yang mengatur agar frame (kerangka)
plot tidak ditampilkan.\\
b)scale = 1.5: Ini adalah parameter yang mengatur faktor skala plot
sebesar 1.5. Ini akan memperbesar plot.\\
c)hue = 1: Ini adalah parameter yang mengatur warna plot dengan skala
warna tunggal.\\
d)light = [1, 0, -1]: Ini adalah parameter yang mengatur pencahayaan
plot dengan arah cahaya yang ditentukan oleh vektor [1, 0, -1].\\
e)zoom = 2.8: Ini adalah parameter yang mengatur tingkat zoom plot
sebesar 2.8. Ini akan memperbesar plot.\\
f)anaglyph: Ini adalah parameter yang mengatur plot menggunakan efek
anaglif, yang memberikan efek tiga dimensi (3D) saat dilihat dengan
kacamata anaglif.
\end{eulercomment}
\begin{eulerprompt}
> 
\end{eulerprompt}
\begin{euleroutput}
  
\end{euleroutput}
\end{eulernotebook}
\end{document}


\newpage
\chapter{KB Pekan 6-7: Menggunakan EMT untuk kalkulus}
\documentclass{article}

\usepackage{eumat}

\begin{document}
\begin{eulernotebook}
\begin{eulercomment}
Vikram Zaky Ardianto\\
Matematika E\\
22305144028\\
\begin{eulercomment}
\eulerheading{Menggunnakan EMT untuk Kalkulus}
\begin{eulercomment}
Materi Kalkulus mencakup di antaranya:

- Fungsi (fungsi aljabar, trigonometri, eksponensial, logaritma,
komposisi fungsi)\\
- Limit Fungsi,\\
- Turunan Fungsi,\\
- Integral Tak Tentu,\\
- Integral Tentu dan Aplikasinya,\\
- Barisan dan Deret (kekonvergenan barisan dan deret).

EMT (bersama Maxima) dapat digunakan untuk melakukan semua perhitungan
di dalam kalkulus, baik secara numerik maupun analitik (eksak).

\end{eulercomment}
\eulersubheading{Mendefinisikan Fungsi}
\begin{eulercomment}
Terdapat beberapa cara mendefinisikan fungsi pada EMT, yakni:

- Menggunakan format nama\_fungsi := rumus fungsi (untuk fungsi
numerik),\\
- Menggunakan format nama\_fungsi \&= rumus fungsi (untuk fungsi
simbolik, namun dapat dihitung secara numerik),\\
- Menggunakan format nama\_fungsi \&\&= rumus fungsi (untuk fungsi
simbolik murni, tidak dapat dihitung langsung),\\
- Fungsi sebagai program EMT.

Setiap format harus diawali dengan perintah function (bukan sebagai
ekspresi).

Berikut adalah adalah beberapa contoh cara mendefinisikan fungsi.
\end{eulercomment}
\begin{eulerprompt}
>function f(x) := 2*x^2+exp(sin(x)) // fungsi numerik
>f(0), f(1), f(pi)
\end{eulerprompt}
\begin{euleroutput}
  1
  4.31977682472
  20.7392088022
\end{euleroutput}
\begin{eulerprompt}
>function g(x) := sqrt(x^2-3*x)/(x+1)
>g(3)
\end{eulerprompt}
\begin{euleroutput}
  0
\end{euleroutput}
\begin{eulerprompt}
>g(0)
\end{eulerprompt}
\begin{euleroutput}
  0
\end{euleroutput}
\begin{eulerprompt}
>f(g(5)) // komposisi fungsi
\end{eulerprompt}
\begin{euleroutput}
  2.20920171961
\end{euleroutput}
\begin{eulerprompt}
>g(f(5))
\end{eulerprompt}
\begin{euleroutput}
  0.950898070639
\end{euleroutput}
\begin{eulerprompt}
>f(0:10) // nilai-nilai f(1), f(2), ..., f(10)
\end{eulerprompt}
\begin{euleroutput}
  [1,  4.31978,  10.4826,  19.1516,  32.4692,  50.3833,  72.7562,
  99.929,  130.69,  163.51,  200.58]
\end{euleroutput}
\begin{eulerprompt}
>fmap(0:10) // sama dengan f(0:10), berlaku untuk semua fungsi
\end{eulerprompt}
\begin{euleroutput}
  [1,  4.31978,  10.4826,  19.1516,  32.4692,  50.3833,  72.7562,
  99.929,  130.69,  163.51,  200.58]
\end{euleroutput}
\begin{eulercomment}
Misalkan kita akan mendefinisikan fungsi

\end{eulercomment}
\begin{eulerformula}
\[
f(x) = \begin{cases} x^3 & x>0 \\ x^2 & x\le 0. \end{cases}
\]
\end{eulerformula}
\begin{eulercomment}
Fungsi tersebut tidak dapat didefinisikan sebagai fungsi numerik
secara "inline" menggunakan format :=, melainkan didefinisikan sebagai
program. Perhatikan, kata "map" digunakan agar fungsi dapat menerima
vektor sebagai input, dan hasilnya berupa vektor. Jika tanpa kata
"map" fungsinya hanya dapat menerima input satu nilai.
\end{eulercomment}
\begin{eulerprompt}
>function map f(x) ...
\end{eulerprompt}
\begin{eulerudf}
    if x>0 then return x^3
    else return x^2
    endif;
  endfunction
\end{eulerudf}
\begin{eulerprompt}
>f(1)
\end{eulerprompt}
\begin{euleroutput}
  1
\end{euleroutput}
\begin{eulerprompt}
>f(-2)
\end{eulerprompt}
\begin{euleroutput}
  4
\end{euleroutput}
\begin{eulerprompt}
>f(-5:5)
\end{eulerprompt}
\begin{euleroutput}
  [25,  16,  9,  4,  1,  0,  1,  8,  27,  64,  125]
\end{euleroutput}
\begin{eulerprompt}
>aspect(1.5); plot2d("f(x)",-5,5):
\end{eulerprompt}
\eulerimg{17}{images/Vikram Zaky Ardianto_22305144028_Kalkulus-001.png}
\begin{eulerprompt}
>function f(x) &= 2*E^x // fungsi simbolik
\end{eulerprompt}
\begin{euleroutput}
  
                                      x
                                   2 E
  
\end{euleroutput}
\begin{eulerprompt}
>function g(x) &= 3*x+1
\end{eulerprompt}
\begin{euleroutput}
  
                                 3 x + 1
  
\end{euleroutput}
\begin{eulerprompt}
>function h(x) &= f(g(x)) // komposisi fungsi
\end{eulerprompt}
\begin{euleroutput}
  
                                   3 x + 1
                                2 E
  
\end{euleroutput}
\eulerheading{Latihan}
\begin{eulercomment}
Bukalah buku Kalkulus. Cari dan pilih beberapa (paling sedikit 5
fungsi berbeda tipe/bentuk/jenis) fungsi dari buku tersebut, kemudian
definisikan di EMT pada baris-baris perintah berikut (jika perlu
tambahkan lagi). Untuk setiap fungsi, hitung beberapa nilainya, baik
untuk satu nilai maupun vektor. Gambar grafik tersebut.

Juga, carilah fungsi beberapa (dua) variabel. Lakukan hal sama seperti
di atas.

Jawab:\\
\end{eulercomment}
\begin{eulerformula}
\[
\text{A). FUNGSI 1 VARIABEL}
\]
\end{eulerformula}
\begin{eulercomment}
1. Fungsi 1
\end{eulercomment}
\begin{eulerprompt}
>function k(x) := x*(x^4-9)^3
>k(3), k(5), k(7)
\end{eulerprompt}
\begin{euleroutput}
  1119744
  1168724480
  95803542016
\end{euleroutput}
\begin{eulerprompt}
>kmap(-3:3)
\end{eulerprompt}
\begin{euleroutput}
  [-1.11974e+06,  -686,  512,  0,  -512,  686,  1.11974e+06]
\end{euleroutput}
\begin{eulerprompt}
>plot2d("k(x)"):
\end{eulerprompt}
\eulerimg{17}{images/Vikram Zaky Ardianto_22305144028_Kalkulus-002.png}
\begin{eulercomment}
2. Fungsi 2
\end{eulercomment}
\begin{eulerprompt}
>function y(x) := (x)^3/(4-x^4) 
>y(2), y(-2), y(1)
\end{eulerprompt}
\begin{euleroutput}
  -0.666666666667
  0.666666666667
  0.333333333333
\end{euleroutput}
\begin{eulerprompt}
>ymap(-5:-5)
\end{eulerprompt}
\begin{euleroutput}
  0.201288244767
\end{euleroutput}
\begin{eulerprompt}
>plot2d("y(x)"):
\end{eulerprompt}
\eulerimg{17}{images/Vikram Zaky Ardianto_22305144028_Kalkulus-003.png}
\begin{eulercomment}
3. Fungsi 3
\end{eulercomment}
\begin{eulerprompt}
>function k(x) := 8*x/(2*x+11)+2
>k(2), k(-1), k(-3), k(4)
\end{eulerprompt}
\begin{euleroutput}
  3.06666666667
  1.11111111111
  -2.8
  3.68421052632
\end{euleroutput}
\begin{eulerprompt}
>kmap(2:5)
\end{eulerprompt}
\begin{euleroutput}
  [3.06667,  3.41176,  3.68421,  3.90476]
\end{euleroutput}
\begin{eulerprompt}
>plot2d("k(x)"):
\end{eulerprompt}
\eulerimg{17}{images/Vikram Zaky Ardianto_22305144028_Kalkulus-004.png}
\begin{eulercomment}
4. Fungsi 4
\end{eulercomment}
\begin{eulerprompt}
>function j(x) := 8*x^9/(x^7-3)
>j(5), j(4), j(3)
\end{eulerprompt}
\begin{euleroutput}
  200.007680295
  128.023441792
  72.0989010989
\end{euleroutput}
\begin{eulerprompt}
>jmap(5:8)
\end{eulerprompt}
\begin{euleroutput}
  [200.008,  288.003,  392.001,  512.001]
\end{euleroutput}
\begin{eulerprompt}
>plot2d("j(x)",-3,3,-600,600):
\end{eulerprompt}
\eulerimg{17}{images/Vikram Zaky Ardianto_22305144028_Kalkulus-005.png}
\begin{eulercomment}
5. Fungsi 5
\end{eulercomment}
\begin{eulerprompt}
>function l(x) := (-cos(x))*sin(8*x)
>l(pi), l(0), l(pi/3)
\end{eulerprompt}
\begin{euleroutput}
  0
  0
  -0.433012701892
\end{euleroutput}
\begin{eulerprompt}
>lmap(0:3pi)
\end{eulerprompt}
\begin{euleroutput}
  [0,  -0.534553,  -0.11981,  -0.896516,  0.360437,  -0.21136,  0.737655,
  0.393198,  0.133864,  0.231266]
\end{euleroutput}
\begin{eulerprompt}
>plot2d("j(x)"):
\end{eulerprompt}
\eulerimg{17}{images/Vikram Zaky Ardianto_22305144028_Kalkulus-006.png}
\begin{eulercomment}
6. Fungsi 6
\end{eulercomment}
\begin{eulerprompt}
>function z(x) := x*sqrt(9x+12)
>z(11), z(9), z(8)
\end{eulerprompt}
\begin{euleroutput}
  115.892191281
  86.7928568489
  73.3212111193
\end{euleroutput}
\begin{eulerprompt}
>zmap(3:12)
\end{eulerprompt}
\begin{euleroutput}
  [18.735,  27.7128,  37.7492,  48.7442,  60.6218,  73.3212,  86.7929,
  100.995,  115.892,  131.453]
\end{euleroutput}
\begin{eulerprompt}
>plot2d("z(x)"):
\end{eulerprompt}
\eulerimg{17}{images/Vikram Zaky Ardianto_22305144028_Kalkulus-007.png}
\begin{eulercomment}
\end{eulercomment}
\begin{eulerformula}
\[
\text{B). FUNGSI 2 VARIABEL}
\]
\end{eulerformula}
\begin{eulercomment}
1. Fungsi 1
\end{eulercomment}
\begin{eulerprompt}
>function a(x,y) ...
\end{eulerprompt}
\begin{eulerudf}
  return x^2+y^2-24
  endfunction
\end{eulerudf}
\begin{eulerprompt}
>a(2,1), a(5,4), a(2,4)
\end{eulerprompt}
\begin{euleroutput}
  -19
  17
  -4
\end{euleroutput}
\begin{eulerprompt}
>amap(-2:2,3:3)
\end{eulerprompt}
\begin{euleroutput}
  [-11,  -14,  -15,  -14,  -11]
\end{euleroutput}
\begin{eulerprompt}
>aspect=1.5; plot3d("a(x,y)",a=-100,b=100,c=-80,d=80,angle=35°,height=30°,r=pi,n=100):
\end{eulerprompt}
\eulerimg{17}{images/Vikram Zaky Ardianto_22305144028_Kalkulus-008.png}
\begin{eulercomment}
2. Fungsi 2
\end{eulercomment}
\begin{eulerprompt}
>function q(x,y) ...
\end{eulerprompt}
\begin{eulerudf}
  return y^2/(x^2/3)
  endfunction
\end{eulerudf}
\begin{eulerprompt}
>q(4,2), q(2,3), q(4,3)
\end{eulerprompt}
\begin{euleroutput}
  0.75
  6.75
  1.6875
\end{euleroutput}
\begin{eulerprompt}
>qmap(2:2,-2:2)
\end{eulerprompt}
\begin{euleroutput}
  [3,  0.75,  0,  0.75,  3]
\end{euleroutput}
\begin{eulerprompt}
>aspect=1.5; plot3d("q(x,y)",a=-100,b=100,c=-80,d=80,angle=35°,height=30°,r=pi,n=100):
\end{eulerprompt}
\eulerimg{17}{images/Vikram Zaky Ardianto_22305144028_Kalkulus-009.png}
\eulerheading{Menghitung Limit}
\begin{eulercomment}
Perhitungan limit pada EMT dapat dilakukan dengan menggunakan fungsi
Maxima, yakni "limit". Fungsi "limit" dapat digunakan untuk menghitung
limit fungsi dalam bentuk ekspresi maupun fungsi yang sudah
didefinisikan sebelumnya. Nilai limit dapat dihitung pada sebarang
nilai atau pada tak hingga (-inf, minf, dan inf). Limit kiri dan limit
kanan juga dapat dihitung, dengan cara memberi opsi "plus" atau
"minus". Hasil limit dapat berupa nilai, "und' (tak definisi), "ind"
(tak tentu namun terbatas), "infinity" (kompleks tak hingga).

Perhatikan beberapa contoh berikut. Perhatikan cara menampilkan
perhitungan secara lengkap, tidak hanya menampilkan hasilnya saja.
\end{eulercomment}
\begin{eulerprompt}
>$showev('limit(1/(2*x-1),x,0))
\end{eulerprompt}
\begin{eulerformula}
\[
\lim_{x\rightarrow 0}{\frac{1}{2\,x-1}}=-1
\]
\end{eulerformula}
\begin{eulerprompt}
>$showev('limit((x^2-3*x-10)/(x-5),x,5))
\end{eulerprompt}
\begin{eulerformula}
\[
\lim_{x\rightarrow 5}{\frac{x^2-3\,x-10}{x-5}}=7
\]
\end{eulerformula}
\begin{eulerprompt}
>$showev('limit(sin(x)/x,x,0))
\end{eulerprompt}
\begin{eulerformula}
\[
\lim_{x\rightarrow 0}{\frac{\sin x}{x}}=1
\]
\end{eulerformula}
\begin{eulerprompt}
>plot2d("sin(x)/x",-pi,pi):
\end{eulerprompt}
\eulerimg{17}{images/Vikram Zaky Ardianto_22305144028_Kalkulus-013.png}
\begin{eulerprompt}
>$showev('limit(sin(x^3)/x,x,0))
\end{eulerprompt}
\begin{eulerformula}
\[
\lim_{x\rightarrow 0}{\frac{\sin x^3}{x}}=0
\]
\end{eulerformula}
\begin{eulerprompt}
>$showev('limit(log(x), x, minf))
\end{eulerprompt}
\begin{eulerformula}
\[
\lim_{x\rightarrow  -\infty }{\log x}={\it infinity}
\]
\end{eulerformula}
\begin{eulerprompt}
>$showev('limit((-2)^x,x, inf))
\end{eulerprompt}
\begin{eulerformula}
\[
\lim_{x\rightarrow \infty }{\left(-2\right)^{x}}={\it infinity}
\]
\end{eulerformula}
\begin{eulerprompt}
>$showev('limit(t-sqrt(2-t),t,2,minus))
\end{eulerprompt}
\begin{eulerformula}
\[
\lim_{t\uparrow 2}{t-\sqrt{2-t}}=2
\]
\end{eulerformula}
\begin{eulerprompt}
>$showev('limit(t-sqrt(2-t),t,5,plus)) // Perhatikan hasilnya
\end{eulerprompt}
\begin{eulerformula}
\[
\lim_{t\downarrow 5}{t-\sqrt{2-t}}=5-\sqrt{3}\,i
\]
\end{eulerformula}
\begin{eulerprompt}
>plot2d("x-sqrt(2-x)",-2,5):
\end{eulerprompt}
\eulerimg{17}{images/Vikram Zaky Ardianto_22305144028_Kalkulus-019.png}
\begin{eulerprompt}
>$showev('limit((x^2-9)/(2*x^2-5*x-3),x,3))
\end{eulerprompt}
\begin{eulerformula}
\[
\lim_{x\rightarrow 3}{\frac{x^2-9}{2\,x^2-5\,x-3}}=\frac{6}{7}
\]
\end{eulerformula}
\begin{eulerprompt}
>$showev('limit((1-cos(x))/x,x,0))
\end{eulerprompt}
\begin{eulerformula}
\[
\lim_{x\rightarrow 0}{\frac{1-\cos x}{x}}=0
\]
\end{eulerformula}
\begin{eulerprompt}
>$showev('limit((x^2+abs(x))/(x^2-abs(x)),x,0))
\end{eulerprompt}
\begin{eulerformula}
\[
\lim_{x\rightarrow 0}{\frac{\left| x\right| +x^2}{x^2-\left| x  \right| }}=-1
\]
\end{eulerformula}
\begin{eulerprompt}
>$showev('limit((1+1/x)^x,x,inf))
\end{eulerprompt}
\begin{eulerformula}
\[
\lim_{x\rightarrow \infty }{\left(\frac{1}{x}+1\right)^{x}}=e
\]
\end{eulerformula}
\begin{eulerprompt}
>$showev('limit((1+k/x)^x,x,inf))
\end{eulerprompt}
\begin{eulerformula}
\[
\lim_{x\rightarrow \infty }{\left(\frac{k}{x}+1\right)^{x}}=e^{k}
\]
\end{eulerformula}
\begin{eulerprompt}
>$showev('limit((1+x)^(1/x),x,0))
\end{eulerprompt}
\begin{eulerformula}
\[
\lim_{x\rightarrow 0}{\left(x+1\right)^{\frac{1}{x}}}=e
\]
\end{eulerformula}
\begin{eulerprompt}
>$showev('limit((x/(x+k))^x,x,inf))
\end{eulerprompt}
\begin{eulerformula}
\[
\lim_{x\rightarrow \infty }{\left(\frac{x}{x+k}\right)^{x}}=e^ {- k   }
\]
\end{eulerformula}
\begin{eulerprompt}
>$showev('limit(sin(1/x),x,0))
\end{eulerprompt}
\begin{eulerformula}
\[
\lim_{x\rightarrow 0}{\sin \left(\frac{1}{x}\right)}={\it ind}
\]
\end{eulerformula}
\begin{eulerprompt}
>$showev('limit(sin(1/x),x,inf))
\end{eulerprompt}
\begin{eulerformula}
\[
\lim_{x\rightarrow \infty }{\sin \left(\frac{1}{x}\right)}=0
\]
\end{eulerformula}
\begin{eulerprompt}
>plot2d("sin(1/x)",-5,5):
\end{eulerprompt}
\eulerimg{17}{images/Vikram Zaky Ardianto_22305144028_Kalkulus-029.png}
\eulerheading{Latihan}
\begin{eulercomment}
Bukalah buku Kalkulus. Cari dan pilih beberapa (paling sedikit 5
fungsi berbeda tipe/bentuk/jenis) fungsi dari buku tersebut, kemudian
definisikan di EMT pada baris-baris perintah berikut (jika perlu
tambahkan lagi). Untuk setiap fungsi, hitung nilai limit fungsi
tersebut di beberapa nilai dan di tak hingga. Gambar grafik fungsi
tersebut untuk mengkonfirmasi nilai-nilai limit tersebut.

Jawab:\\
1. Fungsi 1\\
\end{eulercomment}
\begin{eulerformula}
\[
\text{$f(x)=\frac{3x-6}{x+2}$}
\]
\end{eulerformula}
\begin{eulerprompt}
>$showev('limit((3*x-6)/(x+2),x,2))
\end{eulerprompt}
\begin{eulerformula}
\[
\lim_{x\rightarrow 2}{\frac{3\,x-6}{x+2}}=0
\]
\end{eulerformula}
\begin{eulerprompt}
>plot2d("(3*x-6)/(x+2)",-2,3.5,-1,5):
\end{eulerprompt}
\eulerimg{17}{images/Vikram Zaky Ardianto_22305144028_Kalkulus-031.png}
\begin{eulercomment}
2. Fungsi 2\\
\end{eulercomment}
\begin{eulerformula}
\[
\text{$f(x)=\frac{cos 2x}{sin x - cos x}$}
\]
\end{eulerformula}
\begin{eulerprompt}
>$showev('limit(cos(2*x)/(sin(x) - cos (x)),x,0))
\end{eulerprompt}
\begin{eulerformula}
\[
\lim_{x\rightarrow 0}{\frac{\cos \left(2\,x\right)}{\sin x-\cos x}}=  -1
\]
\end{eulerformula}
\begin{eulerprompt}
>plot2d("cos(2*x)/(sin(x) - cos (x))",-1,1):
\end{eulerprompt}
\eulerimg{17}{images/Vikram Zaky Ardianto_22305144028_Kalkulus-033.png}
\begin{eulercomment}
3. Fungsi 3\\
\end{eulercomment}
\begin{eulerformula}
\[
\text{$f(x)=\frac{2x^2-2x+5}{3x^2+x-6}$}
\]
\end{eulerformula}
\begin{eulerprompt}
>$showev('limit(((2*x^2-2*x+5)/(3*x^2+x-6)),x,3))
\end{eulerprompt}
\begin{eulerformula}
\[
\lim_{x\rightarrow 3}{\frac{2\,x^2-2\,x+5}{3\,x^2+x-6}}=\frac{17}{  24}
\]
\end{eulerformula}
\begin{eulerprompt}
>plot2d("(2*x^2-2*x+5)/(3*x^2+x-6)",-2,10,-10,5):
\end{eulerprompt}
\eulerimg{17}{images/Vikram Zaky Ardianto_22305144028_Kalkulus-035.png}
\begin{eulercomment}
4. Fungsi 4\\
\end{eulercomment}
\begin{eulerformula}
\[
\text{$f(x)=4x^2-3$}
\]
\end{eulerformula}
\begin{eulerprompt}
>$showev('limit((4*x^2-3),x,0))
\end{eulerprompt}
\begin{eulerformula}
\[
\lim_{x\rightarrow 0}{4\,x^2-3}=-3
\]
\end{eulerformula}
\begin{eulerprompt}
>plot2d("(4*x^2-3)"):
\end{eulerprompt}
\eulerimg{17}{images/Vikram Zaky Ardianto_22305144028_Kalkulus-037.png}
\begin{eulercomment}
5. Fungsi 5\\
\end{eulercomment}
\begin{eulerformula}
\[
\text{$f(x)=x^{x^{x}}$}
\]
\end{eulerformula}
\begin{eulerprompt}
>$showev('limit((x^(x^(x))),x,0,plus))
\end{eulerprompt}
\begin{eulerformula}
\[
\lim_{x\downarrow 0}{x^{x^{x}}}=0
\]
\end{eulerformula}
\begin{eulerprompt}
>plot2d("(x^(x^(x)))",-3,3,-1,7):
\end{eulerprompt}
\eulerimg{17}{images/Vikram Zaky Ardianto_22305144028_Kalkulus-039.png}
\begin{eulercomment}
6. Fungsi 6\\
\end{eulercomment}
\begin{eulerformula}
\[
\text{$f(x)=\frac{3xtanx}{1-cos4x}$}
\]
\end{eulerformula}
\begin{eulerprompt}
>$showev('limit((3*x*tan(x))/(1-cos(4*x)),x,0))
\end{eulerprompt}
\begin{eulerformula}
\[
3\,\left(\lim_{x\rightarrow 0}{\frac{x\,\tan x}{1-\cos \left(4\,x  \right)}}\right)=\frac{3}{8}
\]
\end{eulerformula}
\begin{eulerprompt}
>plot2d("(3*x*tan(x))/(1-cos(4*x))",-pi/2,2pi,0,2pi):
\end{eulerprompt}
\eulerimg{17}{images/Vikram Zaky Ardianto_22305144028_Kalkulus-041.png}
\eulerheading{Turunan Fungsi}
\begin{eulercomment}
Definisi turunan:

\end{eulercomment}
\begin{eulerformula}
\[
f'(x) = \lim_{h\to 0} \frac{f(x+h)-f(x)}{h}
\]
\end{eulerformula}
\begin{eulercomment}
Berikut adalah contoh-contoh menentukan turunan fungsi dengan
menggunakan definisi turunan (limit).
\end{eulercomment}
\begin{eulerprompt}
>$showev('limit(((x+h)^n-x^n)/h,h,0)) // turunan x^n
\end{eulerprompt}
\begin{eulerformula}
\[
\lim_{h\rightarrow 0}{\frac{\left(x+h\right)^{n}-x^{n}}{h}}=n\,x^{n  -1}
\]
\end{eulerformula}
\begin{eulercomment}
Mengapa hasilnya seperti itu? Tuliskan atau tunjukkan bahwa hasil
limit tersebut benar, sehingga benar turunan fungsinya benar.  Tulis
penjelasan Anda di komentar ini.

Sebagai petunjuk, ekspansikan (x+h)\textasciicircum{}n dengan menggunakan teorema
binomial.\\
Jawab:\\
\end{eulercomment}
\begin{eulerformula}
\[
\text{Akan ditunjukkan bahwa \: $f'(x)=\lim_{h\to 0} \frac{(x+h)^n-x^n}{h}=nx^{n-1}$}
\]
\end{eulerformula}
\begin{eulercomment}
\end{eulercomment}
\begin{eulerformula}
\[
\text{Pertama, ekspansikan $(x+h)^n$, yakni: }
\]
\end{eulerformula}
\begin{eulercomment}
\end{eulercomment}
\begin{eulerformula}
\[
\text{$(x+h)^n=\sum_{k=0}^{n} \binom{n}{k}x^{n-k}h^k$}
\]
\end{eulerformula}
\begin{eulercomment}
\end{eulercomment}
\begin{eulerformula}
\[
\text{$\Leftrightarrow \: (x+h)^n=\binom{n}{0}x^{n}+\binom{n}{1}x^{n-1}h+\binom{n}{2}x^{n-2}h^2+ ...+\binom{n}{n}h^n$}
\]
\end{eulerformula}
\begin{eulercomment}
\end{eulercomment}
\begin{eulerformula}
\[
\text{$\Leftrightarrow \: (x+h)^n=x^{n}+nx^{n-1}h+\binom{n}{2}x^{n-2}h^2+\binom{n}{3}x^{n-3}h^3+ ...+h^n$}
\]
\end{eulerformula}
\begin{eulercomment}
\end{eulercomment}
\begin{eulerformula}
\[
\text{Sehingga, $f'(x)$ menjadi:\: $f'(x)=\lim_{h\to 0} \frac{(x+h)^n-x^n}{h}$}
\]
\end{eulerformula}
\begin{eulercomment}
\end{eulercomment}
\begin{eulerformula}
\[
\text{$\Leftrightarrow f'(x)=\lim_{h\to 0} \frac{x^{n}+nx^{n-1}h+\binom{n}{2}x^{n-2}h^2+\binom{n}{3}x^{n-3}h^3+ ...+h^n-x^n}{h}$}
\]
\end{eulerformula}
\begin{eulercomment}
\end{eulercomment}
\begin{eulerformula}
\[
\text{$\Leftrightarrow f'(x)=\lim_{h\to 0} nx^{n-1}+\binom{n}{2}x^{n-2}h+\binom{n}{3}x^{n-3}h^2+ ...+h^{n-1}$}
\]
\end{eulerformula}
\begin{eulercomment}
\end{eulercomment}
\begin{eulerformula}
\[
\text{$\Leftrightarrow f'(x)=nx^{n-1}$. Terbukti.}
\]
\end{eulerformula}
\begin{eulerprompt}
>$showev('limit((sin(x+h)-sin(x))/h,h,0)) // turunan sin(x)
\end{eulerprompt}
\begin{eulerformula}
\[
\lim_{h\rightarrow 0}{\frac{\sin \left(x+h\right)-\sin x}{h}}=\cos   x
\]
\end{eulerformula}
\begin{eulercomment}
Mengapa hasilnya seperti itu? Tuliskan atau tunjukkan bahwa hasil
limit tersebut\\
benar, sehingga benar turunan fungsinya benar.  Tulis penjelasan Anda
di komentar ini.

Sebagai petunjuk, ekspansikan sin(x+h) dengan menggunakan rumus jumlah
dua sudut.\\
Jawab:\\
\end{eulercomment}
\begin{eulerformula}
\[
\text{Akan ditunjukkan bahwa\: $\lim_{h\to 0} \frac{\sin(x+h)-\sin x}{h}=\cos x$}
\]
\end{eulerformula}
\begin{eulercomment}
\end{eulercomment}
\begin{eulerformula}
\[
\text{Diketahui bahwa:}
\]
\end{eulerformula}
\begin{eulercomment}
\end{eulercomment}
\begin{eulerformula}
\[
\text{$1).\: \sin(x+h)=\sin x\cos h+\cos x\sin h$}
\]
\end{eulerformula}
\begin{eulerformula}
\[
\text{$2).\: \lim_{h\to 0} \frac{1-\cos h}{h}=0$}
\]
\end{eulerformula}
\begin{eulerformula}
\[
\text{$3).\: \lim_{h\to 0} \frac{\sin h}{h}=1$}
\]
\end{eulerformula}
\begin{eulercomment}
\end{eulercomment}
\begin{eulerformula}
\[
\text{$\lim_{h\to 0} \frac{\sin(x+h)-\sin x}{h}$}
\]
\end{eulerformula}
\begin{eulercomment}
\end{eulercomment}
\begin{eulerformula}
\[
\text{$=\lim_{h\to 0} \frac{\sin x\cos h+\cos x\sin h-\sin x}{h}$}
\]
\end{eulerformula}
\begin{eulercomment}
\end{eulercomment}
\begin{eulerformula}
\[
\text{$=\lim_{h\to 0} \left[-\sin x\cdot\frac{1-\cos h}{h}+\cos x\cdot\frac{\sin h}{h}\right]$}
\]
\end{eulerformula}
\begin{eulercomment}
\end{eulercomment}
\begin{eulerformula}
\[
\text{$=(-\sin x)\left[\lim_{h\to 0} \frac{1-\cos h}{h}+(\cos x)\lim_{h\to 0} \frac{\sin h}{h}\right]$}
\]
\end{eulerformula}
\begin{eulercomment}
\end{eulercomment}
\begin{eulerformula}
\[
\text{$=(-\sin x)(0)+(\cos x)(1)=\cos x$. Terbukti.}
\]
\end{eulerformula}
\begin{eulerprompt}
>$showev('limit((log(x+h)-log(x))/h,h,0)) // turunan log(x)
\end{eulerprompt}
\begin{eulerformula}
\[
\lim_{h\rightarrow 0}{\frac{\log \left(x+h\right)-\log x}{h}}=  \frac{1}{x}
\]
\end{eulerformula}
\begin{eulercomment}
Mengapa hasilnya seperti itu? Tuliskan atau tunjukkan bahwa hasil
limit tersebut\\
benar, sehingga benar turunan fungsinya benar.  Tulis penjelasan Anda
di komentar ini.

Sebagai petunjuk, gunakan sifat-sifat logaritma dan hasil limit pada
bagian sebelumnya di atas.\\
Jawab:\\
Bukti:\\
\end{eulercomment}
\begin{eulerformula}
\[
\text{Ambil $f(x)=^a\log x$.}
\]
\end{eulerformula}
\begin{eulercomment}
\end{eulercomment}
\begin{eulerformula}
\[
\text{$\lim_{h\to 0} \frac{^a\log (x+h)-^a\log x}{h}$}
\]
\end{eulerformula}
\begin{eulercomment}
\end{eulercomment}
\begin{eulerformula}
\[
\text{$=\lim _{h\to 0} \frac{^a\log \frac{(x+h)}{x}}{h}$}
\]
\end{eulerformula}
\begin{eulercomment}
\end{eulercomment}
\begin{eulerformula}
\[
\text{$=\lim_{h\to 0} \frac{^a\log (1+\frac{h}{x})}{h}$}
\]
\end{eulerformula}
\begin{eulercomment}
\end{eulercomment}
\begin{eulerformula}
\[
\text{$=\lim_{h\to 0} \frac{^a\log (1+\frac{h}{x})}{\frac{h}{x}x}$}
\]
\end{eulerformula}
\begin{eulercomment}
\end{eulercomment}
\begin{eulerformula}
\[
\text{$=\lim_{h\to 0} \frac{\frac{x}{h}\cdot ^a\log (1+\frac{h}{x})}{x}$}
\]
\end{eulerformula}
\begin{eulercomment}
\end{eulercomment}
\begin{eulerformula}
\[
\text{$=\lim_{h\to 0} \frac{^a\log (1+\frac{h}{x})^\frac{x}{h}}{x}$}
\]
\end{eulerformula}
\begin{eulercomment}
\end{eulercomment}
\begin{eulerformula}
\[
\text{$=\frac{\lim_{h\to 0}\: ^a\log (1+\frac{h}{x})^\frac{x}{h}}{\lim _{h\to 0}\: x}$}
\]
\end{eulerformula}
\begin{eulercomment}
\end{eulercomment}
\begin{eulerformula}
\[
\text{$=\frac{1}{x\cdot ^e\log a}$}
\]
\end{eulerformula}
\begin{eulercomment}
\end{eulercomment}
\begin{eulerformula}
\[
\text{$=\frac{1}{x\cdot \ln a}$}
\]
\end{eulerformula}
\begin{eulercomment}
\end{eulercomment}
\begin{eulerformula}
\[
\text{Menggunakan hasil di atas, maka:}
\]
\end{eulerformula}
\begin{eulercomment}
\end{eulercomment}
\begin{eulerformula}
\[
\text{$\frac{d\: \ln x}{dx}=\frac{d\: ^e\log x}{dx}=\frac{1}{x\cdot \ln e}=\frac{1}{x}.$ Terbukti.}
\]
\end{eulerformula}
\begin{eulerprompt}
>$showev('limit((1/(x+h)-1/x)/h,h,0)) // turunan 1/x
\end{eulerprompt}
\begin{eulerformula}
\[
\lim_{h\rightarrow 0}{\frac{\frac{1}{x+h}-\frac{1}{x}}{h}}=-\frac{1  }{x^2}
\]
\end{eulerformula}
\begin{eulerprompt}
>$showev('limit((E^(x+h)-E^x)/h,h,0)) // turunan f(x)=e^x
\end{eulerprompt}
\begin{euleroutput}
  Answering "Is x an integer?" with "integer"
  Answering "Is x an integer?" with "integer"
  Answering "Is x an integer?" with "integer"
  Answering "Is x an integer?" with "integer"
  Answering "Is x an integer?" with "integer"
  Maxima is asking
  Acceptable answers are: yes, y, Y, no, n, N, unknown, uk
  Is x an integer?
  
  Use assume!
  Error in:
   $showev('limit((E^(x+h)-E^x)/h,h,0)) // turunan f(x)=e^x ...
                                       ^
\end{euleroutput}
\begin{eulercomment}
Maxima bermasalah dengan limit:

\end{eulercomment}
\begin{eulerformula}
\[
\lim_{h\to 0}\frac{e^{x+h}-e^x}{h}.
\]
\end{eulerformula}
\begin{eulercomment}
Oleh karena itu diperlukan trik khusus agar hasilnya benar.
\end{eulercomment}
\begin{eulerprompt}
>$showev('limit((E^h-1)/h,h,0))
\end{eulerprompt}
\begin{eulerformula}
\[
\lim_{h\rightarrow 0}{\frac{e^{h}-1}{h}}=1
\]
\end{eulerformula}
\begin{eulerprompt}
>$factor(E^(x+h)-E^x)
\end{eulerprompt}
\begin{eulerformula}
\[
\left(e^{h}-1\right)\,e^{x}
\]
\end{eulerformula}
\begin{eulerprompt}
>$showev('limit(factor((E^(x+h)-E^x)/h),h,0)) // turunan f(x)=e^x
\end{eulerprompt}
\begin{eulerformula}
\[
\left(\lim_{h\rightarrow 0}{\frac{e^{h}-1}{h}}\right)\,e^{x}=e^{x}
\]
\end{eulerformula}
\begin{eulerprompt}
>function f(x) &= x^x
\end{eulerprompt}
\begin{euleroutput}
  
                                     x
                                    x
  
\end{euleroutput}
\begin{eulerprompt}
>$showev('limit((f(x+h)-f(x))/h,h,0)) // turunan f(x)=x^x
\end{eulerprompt}
\begin{eulerformula}
\[
\lim_{h\rightarrow 0}{\frac{\left(x+h\right)^{x+h}-x^{x}}{h}}=  {\it infinity}
\]
\end{eulerformula}
\begin{eulercomment}
Di sini Maxima juga bermasalah terkait limit:

\end{eulercomment}
\begin{eulerformula}
\[
lim_{h\to 0} \frac{(x+h)^{x+h}-x^x}{h}.
\]
\end{eulerformula}
\begin{eulercomment}
Dalam hal ini diperlukan asumsi nilai x.
\end{eulercomment}
\begin{eulerprompt}
>&assume(x>0); $showev('limit((f(x+h)-f(x))/h,h,0)) // turunan f(x)=x^x
\end{eulerprompt}
\begin{eulerformula}
\[
\lim_{h\rightarrow 0}{\frac{\left(x+h\right)^{x+h}-x^{x}}{h}}=x^{x}  \,\left(\log x+1\right)
\]
\end{eulerformula}
\begin{eulerprompt}
>&forget(x>0) // jangan lupa, lupakan asumsi untuk kembali ke semula
\end{eulerprompt}
\begin{euleroutput}
  
                                 [x > 0]
  
\end{euleroutput}
\begin{eulerprompt}
>&forget(x<0)
\end{eulerprompt}
\begin{euleroutput}
  
                                 [x < 0]
  
\end{euleroutput}
\begin{eulerprompt}
>&facts()
\end{eulerprompt}
\begin{euleroutput}
  
                                    []
  
\end{euleroutput}
\begin{eulerprompt}
>$showev('limit((asin(x+h)-asin(x))/h,h,0)) // turunan arcsin(x)
\end{eulerprompt}
\begin{eulerformula}
\[
\lim_{h\rightarrow 0}{\frac{\arcsin \left(x+h\right)-\arcsin x}{h}}=  \frac{1}{\sqrt{1-x^2}}
\]
\end{eulerformula}
\begin{eulerprompt}
>$showev('limit((tan(x+h)-tan(x))/h,h,0)) // turunan tan(x)
\end{eulerprompt}
\begin{eulerformula}
\[
\lim_{h\rightarrow 0}{\frac{\tan \left(x+h\right)-\tan x}{h}}=  \frac{1}{\cos ^2x}
\]
\end{eulerformula}
\begin{eulerprompt}
>function f(x) &= sinh(x) // definisikan f(x)=sinh(x)
\end{eulerprompt}
\begin{euleroutput}
  
                                 sinh(x)
  
\end{euleroutput}
\begin{eulerprompt}
>function df(x) &= limit((f(x+h)-f(x))/h,h,0); $df(x) // df(x) = f'(x)
\end{eulerprompt}
\begin{eulerformula}
\[
\frac{e^ {- x }\,\left(e^{2\,x}+1\right)}{2}
\]
\end{eulerformula}
\begin{eulercomment}
Hasilnya adalah cosh(x), karena

\end{eulercomment}
\begin{eulerformula}
\[
\frac{e^x+e^{-x}}{2}=\cosh(x).
\]
\end{eulerformula}
\begin{eulerprompt}
>plot2d(["f(x)","df(x)"],-pi,pi,color=[blue,red]):
\end{eulerprompt}
\eulerimg{17}{images/Vikram Zaky Ardianto_22305144028_Kalkulus-054.png}
\eulerheading{Latihan}
\begin{eulercomment}
Bukalah buku Kalkulus. Cari dan pilih beberapa (paling sedikit 5
fungsi berbeda tipe/bentuk/jenis) fungsi dari buku tersebut, kemudian
definisikan di EMT pada baris-baris perintah berikut (jika perlu
tambahkan lagi). Untuk setiap fungsi, tentukan turunannya dengan
menggunakan definisi turunan (limit), seperti contoh-contoh tersebut.
Gambar grafik fungsi asli dan fungsi turunannya pada sumbu koordinat
yang sama.

Jawab:\\
1. Fungsi 1
\end{eulercomment}
\begin{eulerprompt}
>function f(x) := x^2
>$showev('limit((((x+h)^2-x^2)/h),h,0)) // turunan x^2
\end{eulerprompt}
\begin{eulerformula}
\[
\lim_{h\rightarrow 0}{\frac{\left(x+h\right)^2-x^2}{h}}=2\,x
\]
\end{eulerformula}
\begin{eulerprompt}
>function df(x) &= limit((((x+h)^2-x^2)/h),h,0);  $df(x)// df(x) = f'(x)
\end{eulerprompt}
\begin{eulerformula}
\[
2\,x
\]
\end{eulerformula}
\begin{eulerprompt}
>plot2d(["f(x)","df(x)"],-pi,pi,color=[blue,red]), label("f(x)",2,0.6), label("df(x)",2,0.17):
\end{eulerprompt}
\eulerimg{17}{images/Vikram Zaky Ardianto_22305144028_Kalkulus-057.png}
\begin{eulercomment}
2. Fungsi 2
\end{eulercomment}
\begin{eulerprompt}
>function f(x) := sin(x)*cos(x)
>$showev('limit(((sin(x+h)*cos(x+h))-sin(x)*cos(x))/h,h,0)) // turunan sin(x)*cos(x)
\end{eulerprompt}
\begin{eulerformula}
\[
\lim_{h\rightarrow 0}{\frac{\cos \left(x+h\right)\,\sin \left(x+h  \right)-\cos x\,\sin x}{h}}=\cos ^2x-\sin ^2x
\]
\end{eulerformula}
\begin{eulerprompt}
>function df(x) &= limit(((sin(x+h)*cos(x+h))-sin(x)*cos(x))/h,h,0);  $df(x)// df(x) = f'(x)
\end{eulerprompt}
\begin{eulerformula}
\[
\cos ^2x-\sin ^2x
\]
\end{eulerformula}
\begin{eulerprompt}
>plot2d(["f(x)","df(x)"],-pi,pi,color=[blue,red]), label("f(x)",1,0), label("df(x)",2.3,1.2):
\end{eulerprompt}
\eulerimg{17}{images/Vikram Zaky Ardianto_22305144028_Kalkulus-060.png}
\begin{eulercomment}
3. Fungsi 3
\end{eulercomment}
\begin{eulerprompt}
>function f(x) := sqrt(x)*4
>$showev('limit((sqrt(x+h)*4-sqrt(x)*4)/h,h,0)) // turunan sqrt(x)*4
\end{eulerprompt}
\begin{eulerformula}
\[
\lim_{h\rightarrow 0}{\frac{4\,\sqrt{x+h}-4\,\sqrt{x}}{h}}=\frac{2  }{\sqrt{x}}
\]
\end{eulerformula}
\begin{eulerprompt}
>function df(x) &= limit((sqrt(x+h)*4-sqrt(x)*4)/h,h,0);  $df(x)// df(x) = f'(x)
\end{eulerprompt}
\begin{eulerformula}
\[
\frac{2}{\sqrt{x}}
\]
\end{eulerformula}
\begin{eulerprompt}
>plot2d(["f(x)","df(x)"],-pi,pi,color=[blue,red]), label("f(x)",-2,11), label("df(x)",-2,-10):
\end{eulerprompt}
\eulerimg{17}{images/Vikram Zaky Ardianto_22305144028_Kalkulus-063.png}
\begin{eulercomment}
4. Fungsi 4
\end{eulercomment}
\begin{eulerprompt}
>function f(x) := cos(1/x)
>$showev('limit((cos(1/(x+h))-cos(1/x))/h,h,0)) // turunan cos(1/x)
\end{eulerprompt}
\begin{eulerformula}
\[
\lim_{h\rightarrow 0}{\frac{\cos \left(\frac{1}{x+h}\right)-\cos   \left(\frac{1}{x}\right)}{h}}=\frac{\sin \left(\frac{1}{x}\right)}{x  ^2}
\]
\end{eulerformula}
\begin{eulerprompt}
>function df(x) &= limit((cos(1/(x+h))-cos(1/x))/h,h,0);  $df(x)// df(x) = f'(x)
\end{eulerprompt}
\begin{eulerformula}
\[
\frac{\sin \left(\frac{1}{x}\right)}{x^2}
\]
\end{eulerformula}
\begin{eulerprompt}
>plot2d(["f(x)","df(x)"],-pi,pi,color=[blue,red]), label("f(x)",2,0.4), label("df(x)",1,-0.5):
\end{eulerprompt}
\eulerimg{17}{images/Vikram Zaky Ardianto_22305144028_Kalkulus-066.png}
\begin{eulercomment}
5. Fungsi 5
\end{eulercomment}
\begin{eulerprompt}
>function f(x) := (log(x))^5
>$showev('limit(((log(x+h))^5-(log(x))^5)/h,h,0)) // turunan (log(x))^5
\end{eulerprompt}
\begin{eulerformula}
\[
\lim_{h\rightarrow 0}{\frac{\log ^5\left(x+h\right)-\log ^5x}{h}}=  \frac{5\,\log ^4x}{x}
\]
\end{eulerformula}
\begin{eulerprompt}
>function df(x) &= limit(((log(x+h))^5-(log(x))^5)/h,h,0);  $df(x)// df(x) = f'(x)
\end{eulerprompt}
\begin{eulerformula}
\[
\frac{5\,\log ^4x}{x}
\]
\end{eulerformula}
\begin{eulerprompt}
>plot2d(["f(x)","df(x)"],-50,100,-10,50,color=[blue,red]), label("f(x)",25,35), label("df(x)",50,1):
\end{eulerprompt}
\eulerimg{17}{images/Vikram Zaky Ardianto_22305144028_Kalkulus-069.png}
\begin{eulercomment}
6. Fungsi 6
\end{eulercomment}
\begin{eulerprompt}
>function f(x) := sqrt(tan(x))
>$showev('limit((sqrt(tan(x+h))-sqrt(tan(x)))/h,h,0)) // turunan exp(x)*cos(x)
\end{eulerprompt}
\begin{eulerformula}
\[
\lim_{h\rightarrow 0}{\frac{\sqrt{\tan \left(x+h\right)}-\sqrt{  \tan x}}{h}}=\frac{1}{2\,\cos ^2x\,\sqrt{\tan x}}
\]
\end{eulerformula}
\begin{eulerprompt}
>function df(x) &= limit((sqrt(tan(x+h))-sqrt(tan(x)))/h,h,0);  $df(x)// df(x) = f'(x)
\end{eulerprompt}
\begin{eulerformula}
\[
\frac{1}{2\,\cos ^2x\,\sqrt{\tan x}}
\]
\end{eulerformula}
\begin{eulerprompt}
>plot2d(["f(x)","df(x)"],-10,10,-10,10,color=[blue,red]), label("f(x)",4.5,0), label("df(x)",5.5,5):
\end{eulerprompt}
\eulerimg{17}{images/Vikram Zaky Ardianto_22305144028_Kalkulus-072.png}
\eulerheading{Integral}
\begin{eulercomment}
EMT dapat digunakan untuk menghitung integral, baik integral tak tentu
maupun integral tentu. Untuk integral tak tentu (simbolik) sudah tentu
EMT menggunakan Maxima, sedangkan untuk perhitungan integral tentu EMT
sudah menyediakan beberapa fungsi yang mengimplementasikan algoritma
kuadratur (perhitungan integral tentu menggunakan metode numerik).

Pada notebook ini akan ditunjukkan perhitungan integral tentu dengan
menggunakan Teorema Dasar Kalkulus:

\end{eulercomment}
\begin{eulerformula}
\[
\int_a^b f(x)\ dx = F(b)-F(a), \quad \text{ dengan  } F'(x) = f(x).
\]
\end{eulerformula}
\begin{eulercomment}
Fungsi untuk menentukan integral adalah integrate. Fungsi ini dapat
digunakan untuk menentukan, baik integral tentu maupun tak tentu (jika
fungsinya memiliki antiderivatif). Untuk perhitungan integral tentu
fungsi integrate menggunakan metode numerik (kecuali fungsinya tidak
integrabel, kita tidak akan menggunakan metode ini).
\end{eulercomment}
\begin{eulerprompt}
>$showev('integrate(x^n,x))
\end{eulerprompt}
\begin{euleroutput}
  Answering "Is n equal to -1?" with "no"
\end{euleroutput}
\begin{eulerformula}
\[
\int {x^{n}}{\;dx}=\frac{x^{n+1}}{n+1}
\]
\end{eulerformula}
\begin{eulerprompt}
>$showev('integrate(1/(1+x),x))
\end{eulerprompt}
\begin{eulerformula}
\[
\int {\frac{1}{x+1}}{\;dx}=\log \left(x+1\right)
\]
\end{eulerformula}
\begin{eulerprompt}
>$showev('integrate(1/(1+x^2),x))
\end{eulerprompt}
\begin{eulerformula}
\[
\int {\frac{1}{x^2+1}}{\;dx}=\arctan x
\]
\end{eulerformula}
\begin{eulerprompt}
>$showev('integrate(1/sqrt(1-x^2),x))
\end{eulerprompt}
\begin{eulerformula}
\[
\int {\frac{1}{\sqrt{1-x^2}}}{\;dx}=\arcsin x
\]
\end{eulerformula}
\begin{eulerprompt}
>$showev('integrate(sin(x),x,0,pi))
\end{eulerprompt}
\begin{eulerformula}
\[
\int_{0}^{\pi}{\sin x\;dx}=2
\]
\end{eulerformula}
\begin{eulerprompt}
>$showev('integrate(sin(x),x,a,b))
\end{eulerprompt}
\begin{eulerformula}
\[
\int_{a}^{b}{\sin x\;dx}=\cos a-\cos b
\]
\end{eulerformula}
\begin{eulerprompt}
>$showev('integrate(x^n,x,a,b))
\end{eulerprompt}
\begin{euleroutput}
  Answering "Is n positive, negative or zero?" with "positive"
\end{euleroutput}
\begin{eulerformula}
\[
\int_{a}^{b}{x^{n}\;dx}=\frac{b^{n+1}}{n+1}-\frac{a^{n+1}}{n+1}
\]
\end{eulerformula}
\begin{eulerprompt}
>$showev('integrate(x^2*sqrt(2*x+1),x))
\end{eulerprompt}
\begin{eulerformula}
\[
\int {x^2\,\sqrt{2\,x+1}}{\;dx}=\frac{\left(2\,x+1\right)^{\frac{7  }{2}}}{28}-\frac{\left(2\,x+1\right)^{\frac{5}{2}}}{10}+\frac{\left(  2\,x+1\right)^{\frac{3}{2}}}{12}
\]
\end{eulerformula}
\begin{eulerprompt}
>$showev('integrate(x^2*sqrt(2*x+1),x,0,2))
\end{eulerprompt}
\begin{eulerformula}
\[
\int_{0}^{2}{x^2\,\sqrt{2\,x+1}\;dx}=\frac{2\,5^{\frac{5}{2}}}{21}-  \frac{2}{105}
\]
\end{eulerformula}
\begin{eulerprompt}
>$ratsimp(%)
\end{eulerprompt}
\begin{eulerformula}
\[
\int_{0}^{2}{x^2\,\sqrt{2\,x+1}\;dx}=\frac{2\,5^{\frac{7}{2}}-2}{  105}
\]
\end{eulerformula}
\begin{eulerprompt}
>$showev('integrate((sin(sqrt(x)+a)*E^sqrt(x))/sqrt(x),x,0,pi^2))
\end{eulerprompt}
\begin{eulerformula}
\[
\int_{0}^{\pi^2}{\frac{\sin \left(\sqrt{x}+a\right)\,e^{\sqrt{x}}}{  \sqrt{x}}\;dx}=\left(-e^{\pi}-1\right)\,\sin a+\left(e^{\pi}+1  \right)\,\cos a
\]
\end{eulerformula}
\begin{eulerprompt}
>$factor(%)
\end{eulerprompt}
\begin{eulerformula}
\[
\int_{0}^{\pi^2}{\frac{\sin \left(\sqrt{x}+a\right)\,e^{\sqrt{x}}}{  \sqrt{x}}\;dx}=\left(-e^{\pi}-1\right)\,\left(\sin a-\cos a\right)
\]
\end{eulerformula}
\begin{eulerprompt}
>function map f(x) &= E^(-x^2); $f(x)
\end{eulerprompt}
\begin{eulerformula}
\[
e^ {- x^2 }
\]
\end{eulerformula}
\begin{eulerprompt}
>$showev('integrate(f(x),x))
\end{eulerprompt}
\begin{eulerformula}
\[
\int {e^ {- x^2 }}{\;dx}=\frac{\sqrt{\pi}\,\mathrm{erf}\left(x  \right)}{2}
\]
\end{eulerformula}
\begin{eulercomment}
Fungsi f tidak memiliki antiturunan, integralnya masih memuat integral
lain.

\end{eulercomment}
\begin{eulerformula}
\[
erf(x) = \int \frac{e^{-x^2}}{\sqrt{\pi}} \ dx.
\]
\end{eulerformula}
\begin{eulercomment}
Kita tidak dapat menggunakan teorema Dasar kalkulus untuk menghitung
integral tentu fungsi tersebut jika semua batasnya berhingga. Dalam
hal ini dapat digunakan metode numerik (rumus kuadratur).

Misalkan kita akan menghitung:

maxima: 'integrate(f(x),x,0,pi)
\end{eulercomment}
\begin{eulerprompt}
>x=0:0.1:pi-0.1; plot2d(x,f(x+0.1),>bar); plot2d("f(x)",0,pi,>add):
\end{eulerprompt}
\eulerimg{17}{images/Vikram Zaky Ardianto_22305144028_Kalkulus-087.png}
\begin{eulercomment}
Integral tentu

maxima: 'integrate(f(x),x,0,pi)

dapat dihampiri dengan jumlah luas persegi-persegi panjang di bawah
kurva y=f(x) tersebut. Langkah-langkahnya adalah sebagai berikut.
\end{eulercomment}
\begin{eulerprompt}
>t &= makelist(a,a,0,pi-0.1,0.1); // t sebagai list untuk menyimpan nilai-nilai x
>fx &= makelist(f(t[i]+0.1),i,1,length(t)); // simpan nilai-nilai f(x)
>// jangan menggunakan x sebagai list, kecuali Anda pakar Maxima!
\end{eulerprompt}
\begin{eulercomment}
Hasilnya adalah:

maxima: 'integrate(f(x),x,0,pi) = 0.1*sum(fx[i],i,1,length(fx))

Jumlah tersebut diperoleh dari hasil kali lebar sub-subinterval (=0.1)
dan jumlah nilai-nilai f(x) untuk x = 0.1, 0.2, 0.3, ..., 3.2.
\end{eulercomment}
\begin{eulerprompt}
>0.1*sum(f(x+0.1)) // cek langsung dengan perhitungan numerik EMT
\end{eulerprompt}
\begin{euleroutput}
  0.836219610253
\end{euleroutput}
\begin{eulercomment}
Untuk mendapatkan nilai integral tentu yang mendekati nilai sebenarnya, lebar
sub-intervalnya dapat diperkecil lagi, sehingga daerah di bawah kurva tertutup
semuanya, misalnya dapat digunakan lebar subinterval 0.001. (Silakan dicoba!)

Meskipun Maxima tidak dapat menghitung integral tentu fungsi tersebut untuk
batas-batas yang berhingga, namun integral tersebut dapat dihitung secara eksak jika
batas-batasnya tak hingga. Ini adalah salah satu keajaiban di dalam matematika, yang
terbatas tidak dapat dihitung secara eksak, namun yang tak hingga malah dapat
dihitung secara eksak.
\end{eulercomment}
\begin{eulerprompt}
>$showev('integrate(f(x),x,0,inf))
\end{eulerprompt}
\begin{eulerformula}
\[
\int_{0}^{\infty }{e^ {- x^2 }\;dx}=\frac{\sqrt{\pi}}{2}
\]
\end{eulerformula}
\begin{eulercomment}
Berikut adalah contoh lain fungsi yang tidak memiliki antiderivatif, sehingga
integral tentunya hanya dapat dihitung dengan metode numerik.
\end{eulercomment}
\begin{eulerprompt}
>function f(x) &= x^x; $f(x)
\end{eulerprompt}
\begin{eulerformula}
\[
x^{x}
\]
\end{eulerformula}
\begin{eulerprompt}
>$showev('integrate(f(x),x,0,1))
\end{eulerprompt}
\begin{eulerformula}
\[
\int_{0}^{1}{x^{x}\;dx}=\int_{0}^{1}{x^{x}\;dx}
\]
\end{eulerformula}
\begin{eulerprompt}
>x=0:0.1:1-0.01; plot2d(x,f(x+0.01),>bar); plot2d("f(x)",0,1,>add):
\end{eulerprompt}
\eulerimg{17}{images/Vikram Zaky Ardianto_22305144028_Kalkulus-091.png}
\begin{eulercomment}
Maxima gagal menghitung integral tentu tersebut secara langsung menggunakan perintah
integrate. Berikut kita lakukan seperti contoh sebelumnya untuk mendapat hasil atau
pendekatan nilai integral tentu tersebut.
\end{eulercomment}
\begin{eulerprompt}
>t &= makelist(a,a,0,1-0.01,0.01);
>fx &= makelist(f(t[i]+0.01),i,1,length(t));
\end{eulerprompt}
\eulerheading{Latihan}
\begin{eulercomment}
- Bukalah buku Kalkulus.\\
- Cari dan pilih beberapa (paling sedikit 5 fungsi berbeda
tipe/bentuk/jenis) fungsi dari buku tersebut, kemudian definisikan di
EMT pada baris-baris perintah berikut (jika perlu tambahkan lagi).\\
- Untuk setiap fungsi, tentukan anti turunannya (jika ada), hitunglah
integral tentu dengan batas-batas yang menarik (Anda tentukan
sendiri), seperti contoh-contoh tersebut.\\
- Lakukan hal yang sama untuk fungsi-fungsi yang tidak dapat
diintegralkan (cari sedikitnya 3 fungsi).\\
- Gambar grafik fungsi dan daerah integrasinya pada sumbu koordinat
yang sama.\\
- Gunakan integral tentu untuk mencari luas daerah yang dibatasi oleh
dua kurva yang berpotongan di dua titik. (Cari dan gambar kedua kurva
dan arsir (warnai) daerah yang dibatasi oleh keduanya.)\\
- Gunakan integral tentu untuk menghitung volume benda putar kurva y=
f(x) yang diputar mengelilingi sumbu x dari x=a sampai x=b, yakni

\end{eulercomment}
\begin{eulerformula}
\[
V = \int_a^b \pi (f(x))^2\ dx.
\]
\end{eulerformula}
\begin{eulercomment}
(Pilih fungsinya dan gambar kurva dan benda putar yang dihasilkan.
Anda dapat mencari contoh-contoh bagaimana cara menggambar benda hasil
perputaran suatu kurva.)\\
- Gunakan integral tentu untuk menghitung panjang kurva y=f(x) dari
x=a sampai x=b dengan menggunakan rumus:

\end{eulercomment}
\begin{eulerformula}
\[
S = \int_a^b \sqrt{1+(f'(x))^2} \ dx.
\]
\end{eulerformula}
\begin{eulercomment}
(Pilih fungsi dan gambar kurvanya.)

Jawab:\\
1. Fungsi 1
\end{eulercomment}
\begin{eulerprompt}
>function f(x) &= 5*x^2; $f(x)
\end{eulerprompt}
\begin{eulerformula}
\[
5\,x^2
\]
\end{eulerformula}
\begin{eulerprompt}
>$showev('integrate(f(x),x))
\end{eulerprompt}
\begin{eulerformula}
\[
5\,\int {x^2}{\;dx}=\frac{5\,x^3}{3}
\]
\end{eulerformula}
\begin{eulerprompt}
>$showev('integrate(f(x),x,2,3))
\end{eulerprompt}
\begin{eulerformula}
\[
5\,\int_{2}^{3}{x^2\;dx}=\frac{95}{3}
\]
\end{eulerformula}
\begin{eulerprompt}
>x=0.01:0.03:4; plot2d(x,f(x+0.01),>bar); plot2d("f(x)",2,3,>add):
\end{eulerprompt}
\eulerimg{17}{images/Vikram Zaky Ardianto_22305144028_Kalkulus-095.png}
\begin{eulercomment}
2. Fungsi 2
\end{eulercomment}
\begin{eulerprompt}
>function f(x) &= cos(2*x+5); $f(x)
\end{eulerprompt}
\begin{eulerformula}
\[
\cos \left(2\,x+5\right)
\]
\end{eulerformula}
\begin{eulerprompt}
>$showev('integrate(f(x),x))
\end{eulerprompt}
\begin{eulerformula}
\[
\int {\cos \left(2\,x+5\right)}{\;dx}=\frac{\sin \left(2\,x+5  \right)}{2}
\]
\end{eulerformula}
\begin{eulerprompt}
>$showev('integrate(f(x),x,pi,2*pi))
\end{eulerprompt}
\begin{eulerformula}
\[
\int_{\pi}^{2\,\pi}{\cos \left(2\,x+5\right)\;dx}=0
\]
\end{eulerformula}
\begin{eulerprompt}
>x=0:0.05:pi-0.1; plot2d(x,f(x+0.03),>bar); plot2d("f(x)",pi,2*pi,>add):
\end{eulerprompt}
\eulerimg{17}{images/Vikram Zaky Ardianto_22305144028_Kalkulus-099.png}
\begin{eulercomment}
3. Fungsi 3
\end{eulercomment}
\begin{eulerprompt}
>function f(x) &= (sin(x))*(cos((x)))^2; $f(x)
\end{eulerprompt}
\begin{eulerformula}
\[
\cos ^2x\,\sin x
\]
\end{eulerformula}
\begin{eulerprompt}
>$showev('integrate(f(x),x))
\end{eulerprompt}
\begin{eulerformula}
\[
\int {\cos ^2x\,\sin x}{\;dx}=-\frac{\cos ^3x}{3}
\]
\end{eulerformula}
\begin{eulerprompt}
>$showev('integrate(f(x),x,0,pi))
\end{eulerprompt}
\begin{eulerformula}
\[
\int_{0}^{\pi}{\cos ^2x\,\sin x\;dx}=\frac{2}{3}
\]
\end{eulerformula}
\begin{eulerprompt}
>x=-pi:0.04:pi; plot2d(x,f(x+0.01),>bar); plot2d("f(x)",0,pi,>add):
\end{eulerprompt}
\eulerimg{17}{images/Vikram Zaky Ardianto_22305144028_Kalkulus-103.png}
\begin{eulercomment}
4. Fungsi 4
\end{eulercomment}
\begin{eulerprompt}
>function f(x) &= (x^2*(2-x^3)^(1/2)); $f(x)
\end{eulerprompt}
\begin{eulerformula}
\[
x^2\,\sqrt{2-x^3}
\]
\end{eulerformula}
\begin{eulerprompt}
>$showev('integrate(f(x),x))
\end{eulerprompt}
\begin{eulerformula}
\[
\int {x^2\,\sqrt{2-x^3}}{\;dx}=-\frac{2\,\left(2-x^3\right)^{\frac{  3}{2}}}{9}
\]
\end{eulerformula}
\begin{eulerprompt}
>$showev('integrate(f(x),x,0,1))
\end{eulerprompt}
\begin{eulerformula}
\[
\int_{0}^{1}{x^2\,\sqrt{2-x^3}\;dx}=\frac{2^{\frac{5}{2}}}{9}-  \frac{2}{9}
\]
\end{eulerformula}
\begin{eulerprompt}
>x=-1:0.04:1; plot2d(x,f(x+0.01),>bar); plot2d("f(x)",0,1,>add):
\end{eulerprompt}
\eulerimg{17}{images/Vikram Zaky Ardianto_22305144028_Kalkulus-107.png}
\begin{eulercomment}
5. Fungsi 5
\end{eulercomment}
\begin{eulerprompt}
>function f(x) &= sqrt(24-x^2); $f(x)
\end{eulerprompt}
\begin{eulerformula}
\[
\sqrt{24-x^2}
\]
\end{eulerformula}
\begin{eulerprompt}
>$showev('integrate(f(x),x))
\end{eulerprompt}
\begin{eulerformula}
\[
\int {\sqrt{24-x^2}}{\;dx}=12\,\arcsin \left(\frac{x}{2\,\sqrt{6}}  \right)+\frac{x\,\sqrt{24-x^2}}{2}
\]
\end{eulerformula}
\begin{eulerprompt}
>$showev('integrate(f(x),x,1,2))
\end{eulerprompt}
\begin{eulerformula}
\[
\int_{1}^{2}{\sqrt{24-x^2}\;dx}=12\,\arcsin \left(\frac{1}{\sqrt{6}  }\right)-\frac{24\,\arcsin \left(\frac{1}{2\,\sqrt{6}}\right)+\sqrt{  23}}{2}+2\,\sqrt{5}
\]
\end{eulerformula}
\begin{eulerprompt}
>x=-2:0.04:1; plot2d(x,f(x+0.01),>bar); plot2d("f(x)",1,2,>add):
\end{eulerprompt}
\eulerimg{17}{images/Vikram Zaky Ardianto_22305144028_Kalkulus-111.png}
\begin{eulercomment}
6. Fungsi 6
\end{eulercomment}
\begin{eulerprompt}
>t &= makelist(a,a,0,1-0.01,0.01);
>fx &= makelist(f(t[i]+0.01),i,1,length(t));
>function f(x) &= x^2+50; $f(x)
\end{eulerprompt}
\begin{eulerformula}
\[
x^2+50
\]
\end{eulerformula}
\begin{eulerprompt}
>x=0:0.1:pi-0.01; plot2d(x,f(x+0.01),>bar); plot2d("f(x)",0,pi,>add):
\end{eulerprompt}
\eulerimg{17}{images/Vikram Zaky Ardianto_22305144028_Kalkulus-113.png}
\begin{eulerprompt}
>0.01*sum(f(x+0.01))
\end{eulerprompt}
\begin{euleroutput}
  17.051552
\end{euleroutput}
\begin{eulercomment}
7. Fungsi 7
\end{eulercomment}
\begin{eulerprompt}
>t &= makelist(a,a,0,1-0.01,0.01);
>fx &= makelist(f(t[i]+0.01),i,1,length(t));
>function f(x) &= cos(x)/x; $f(x)
\end{eulerprompt}
\begin{eulerformula}
\[
\frac{\cos x}{x}
\]
\end{eulerformula}
\begin{eulerprompt}
>x=-pi:0.07:pi-0.01; plot2d(x,f(x+0.01),>bar); plot2d("f(x)",0,pi,>add):
\end{eulerprompt}
\eulerimg{17}{images/Vikram Zaky Ardianto_22305144028_Kalkulus-115.png}
\begin{eulerprompt}
>0.01*sum(f(x+0.01))
\end{eulerprompt}
\begin{euleroutput}
  0.415163991256
\end{euleroutput}
\begin{eulercomment}
8. Fungsi 8
\end{eulercomment}
\begin{eulerprompt}
>t &= makelist(a,a,0,1-0.01,0.01);
>fx &= makelist(f(t[i]+0.01),i,1,length(t));
>function f(x) &= sqrt(x^2-1); $f(x)
\end{eulerprompt}
\begin{eulerformula}
\[
\sqrt{x^2-1}
\]
\end{eulerformula}
\begin{eulerprompt}
>x=3:0.04:pi-0.01; plot2d(x,f(x+0.01),>bar); plot2d("f(x)",0,2,>add):
\end{eulerprompt}
\eulerimg{17}{images/Vikram Zaky Ardianto_22305144028_Kalkulus-117.png}
\begin{eulerprompt}
>0.01*sum(f(x+0.01))
\end{eulerprompt}
\begin{euleroutput}
  0.11610107668
\end{euleroutput}
\eulersubheading{Luas daerah dibatasi 2 kurva}
\begin{eulercomment}
1). Fungsi 1
\end{eulercomment}
\begin{eulerprompt}
>function f(x) &= x^3; $f(x)
\end{eulerprompt}
\begin{eulerformula}
\[
x^3
\]
\end{eulerformula}
\begin{eulerprompt}
>function g(x) &= x; $g(x)
\end{eulerprompt}
\begin{eulerformula}
\[
x
\]
\end{eulerformula}
\begin{eulerprompt}
>plot2d(["x^4","x^3"],-2,2,-1,2):
\end{eulerprompt}
\eulerimg{17}{images/Vikram Zaky Ardianto_22305144028_Kalkulus-120.png}
\begin{eulerprompt}
>function h(x) &= f(x)-g(x); $h(x)
\end{eulerprompt}
\begin{eulerformula}
\[
x^3-x
\]
\end{eulerformula}
\begin{eulerprompt}
>$showev('integrate(h(x),x))
\end{eulerprompt}
\begin{eulerformula}
\[
\int {x^3-x}{\;dx}=\frac{x^4}{4}-\frac{x^2}{2}
\]
\end{eulerformula}
\begin{eulerprompt}
>$&solve(f(x)=g(x))
\end{eulerprompt}
\begin{eulerformula}
\[
\left[ x=-1 , x=1 , x=0 \right] 
\]
\end{eulerformula}
\begin{eulerprompt}
>$showev('integrate(h(x),x,0,1)) // menghitung luas daerah yang dibatasi 2 kurva
\end{eulerprompt}
\begin{eulerformula}
\[
\int_{0}^{1}{x^3-x\;dx}=-\frac{1}{4}
\]
\end{eulerformula}
\begin{eulercomment}
\end{eulercomment}
\begin{eulerformula}
\[
\text{Arsiran daerah yang dibatasi kurva $f(x)$ dan $g(x)$ sebagai berikut:}
\]
\end{eulerformula}
\begin{eulerprompt}
>x=-1:0.01:1; plot2d(x,f(x),>bar,>filled,style="-",fillcolor=orange,>grid); plot2d(x,g(x),>bar,>add,>filled,style="-",fillcolor=white); label("f(x)",0,2.1); label("g(x)",0.5,0.3):
\end{eulerprompt}
\eulerimg{17}{images/Vikram Zaky Ardianto_22305144028_Kalkulus-125.png}
\begin{eulercomment}
2). Fungsi 2
\end{eulercomment}
\begin{eulerprompt}
>function f(x) &= x^3+1; $f(x)
\end{eulerprompt}
\begin{eulerformula}
\[
x^3+1
\]
\end{eulerformula}
\begin{eulerprompt}
>function g(x) &= x^2; $g(x)
\end{eulerprompt}
\begin{eulerformula}
\[
x^2
\]
\end{eulerformula}
\begin{eulerprompt}
>plot2d(["-x^2+2","x^2"],-2,2,-1,2):
\end{eulerprompt}
\eulerimg{17}{images/Vikram Zaky Ardianto_22305144028_Kalkulus-128.png}
\begin{eulerprompt}
>function h(x) &= f(x)-g(x); $h(x)
\end{eulerprompt}
\begin{eulerformula}
\[
x^3-x^2+1
\]
\end{eulerformula}
\begin{eulerprompt}
>$&solve(f(x)=g(x))
\end{eulerprompt}
\begin{eulerformula}
\[
\left[ x=\frac{\frac{\sqrt{3}\,i}{2}-\frac{1}{2}}{9\,\left(\frac{  \sqrt{23}}{2\,3^{\frac{3}{2}}}-\frac{25}{54}\right)^{\frac{1}{3}}}+  \left(\frac{\sqrt{23}}{2\,3^{\frac{3}{2}}}-\frac{25}{54}\right)^{  \frac{1}{3}}\,\left(-\frac{\sqrt{3}\,i}{2}-\frac{1}{2}\right)+\frac{  1}{3} , x=\left(\frac{\sqrt{23}}{2\,3^{\frac{3}{2}}}-\frac{25}{54}  \right)^{\frac{1}{3}}\,\left(\frac{\sqrt{3}\,i}{2}-\frac{1}{2}  \right)+\frac{-\frac{\sqrt{3}\,i}{2}-\frac{1}{2}}{9\,\left(\frac{  \sqrt{23}}{2\,3^{\frac{3}{2}}}-\frac{25}{54}\right)^{\frac{1}{3}}}+  \frac{1}{3} , x=\left(\frac{\sqrt{23}}{2\,3^{\frac{3}{2}}}-\frac{25  }{54}\right)^{\frac{1}{3}}+\frac{1}{9\,\left(\frac{\sqrt{23}}{2\,3^{  \frac{3}{2}}}-\frac{25}{54}\right)^{\frac{1}{3}}}+\frac{1}{3}   \right] 
\]
\end{eulerformula}
\begin{eulerprompt}
>$showev('integrate(h(x),x,-1,1)) // menghitung luas daerah yang dibatasi 2 kurva
\end{eulerprompt}
\begin{eulerformula}
\[
\int_{-1}^{1}{x^3-x^2+1\;dx}=\frac{4}{3}
\]
\end{eulerformula}
\begin{eulercomment}
\end{eulercomment}
\begin{eulerformula}
\[
\text{Arsiran daerah yang dibatasi kurva $f(x)$ dan $g(x)$ sebagai berikut:}
\]
\end{eulerformula}
\begin{eulerprompt}
>x=-1:0.01:1; plot2d(x,f(x),>bar,>filled,style="-",fillcolor=orange,>grid); plot2d(x,g(x),>bar,>add,>filled,style="-",fillcolor=white); label("f(x)",0,2.1); label("g(x)",0.5,0.3):
\end{eulerprompt}
\eulerimg{17}{images/Vikram Zaky Ardianto_22305144028_Kalkulus-132.png}
\eulersubheading{Volume benda putar}
\begin{eulercomment}
Menghitung volume hasil perputaran kurva\\
\end{eulercomment}
\begin{eulerformula}
\[
m(x)=x^3+1
\]
\end{eulerformula}
\begin{eulercomment}
dari x=-1 sampai x=0. Diputar terhadap sumbu-x.\\
Jawab:
\end{eulercomment}
\begin{eulerprompt}
>function m(x) &= x^4+3; $m(x)
\end{eulerprompt}
\begin{eulerformula}
\[
x^4+3
\]
\end{eulerformula}
\begin{eulerprompt}
>$showev('integrate(pi*(m(x))^2,x,-1,0)) // Menghitung volume hasil perputaran m(x)
\end{eulerprompt}
\begin{eulerformula}
\[
\pi\,\int_{-1}^{0}{\left(x^4+3\right)^2\;dx}=\frac{464\,\pi}{45}
\]
\end{eulerformula}
\begin{eulercomment}
Daerah di bawah kurva yang akan dirotasi terhadap sumbu x sebagai
berikut:
\end{eulercomment}
\begin{eulerprompt}
>plot2d("m(x)",-1,0,-1,2,grid=7,>filled, style="/\(\backslash\)"): 
\end{eulerprompt}
\eulerimg{17}{images/Vikram Zaky Ardianto_22305144028_Kalkulus-135.png}
\begin{eulercomment}
Hasil perputaran m(x) terhadap sumbu x sebagai berikut:
\end{eulercomment}
\begin{eulerprompt}
>plot3d("m(x)",-1,0,-1,1,>rotate,angle=6.3,>hue,>contour,color=redgreen,height=11):
\end{eulerprompt}
\eulerimg{17}{images/Vikram Zaky Ardianto_22305144028_Kalkulus-136.png}
\begin{eulercomment}
\end{eulercomment}
\eulersubheading{Menghitung panjang kurva}
\begin{eulercomment}
Menghitung panjang kurva\\
\end{eulercomment}
\begin{eulerformula}
\[
\text{$y=x^2-x+1$}
\]
\end{eulerformula}
\begin{eulercomment}
dari x=1 sampai x=3.
\end{eulercomment}
\begin{eulerprompt}
>function d(x) &= x^2-x+1; $d(x)
\end{eulerprompt}
\begin{eulerformula}
\[
x^2-x+1
\]
\end{eulerformula}
\begin{eulerprompt}
>plot2d("d(x)",-5,6): // gambar kurva d(x)
\end{eulerprompt}
\eulerimg{17}{images/Vikram Zaky Ardianto_22305144028_Kalkulus-138.png}
\begin{eulerprompt}
>$showev('limit((d(x+h)-d(x))/h,h,0))
\end{eulerprompt}
\begin{eulerformula}
\[
\lim_{h\rightarrow 0}{\frac{\left(x+h\right)^2-x^2-h}{h}}=2\,x-1
\]
\end{eulerformula}
\begin{eulerprompt}
>function dd(x) &= limit((d(x+h)-d(x))/h,h,0); $dd(x)
\end{eulerprompt}
\begin{eulerformula}
\[
2\,x-1
\]
\end{eulerformula}
\begin{eulerprompt}
>function q(x) &= ((dd(x))^2); $q(x)
\end{eulerprompt}
\begin{eulerformula}
\[
\left(2\,x-1\right)^2
\]
\end{eulerformula}
\begin{eulerprompt}
>$showev('integrate(sqrt(1+q(x)),x,1,3)) // menghitung panjang kurva
\end{eulerprompt}
\begin{eulerformula}
\[
\int_{1}^{3}{\sqrt{\left(2\,x-1\right)^2+1}\;dx}=\frac{  {\rm asinh}\; 5+5\,\sqrt{26}}{4}-\frac{{\rm asinh}\; 1+\sqrt{2}}{4}
\]
\end{eulerformula}
\begin{eulercomment}
Jadi, panjang kurva\\
\end{eulercomment}
\begin{eulerformula}
\[
\text{$y=x^2-x+1$}
\]
\end{eulerformula}
\begin{eulercomment}
dari x=0 sampai x=4 adalah\\
\end{eulercomment}
\begin{eulerformula}
\[
\text{$S=\frac{asinh 5+5sqrt(26)}{4}-\frac{asinh(1)+sqrt(2)}{4}$}.
\]
\end{eulerformula}
\begin{eulercomment}
\begin{eulercomment}
\eulerheading{Barisan dan Deret}
\begin{eulercomment}
(Catatan: bagian ini belum lengkap. Anda dapat membaca contoh-contoh
pengguanaan EMT dan Maxima untuk menghitung limit barisan, rumus
jumlah parsial suatu deret, jumlah tak hingga suatu deret konvergen,
dan sebagainya. Anda dapat mengeksplor contoh-contoh di EMT atau
perbagai panduan penggunaan Maxima di software Maxima atau dari
Internet.)

Barisan dapat didefinisikan dengan beberapa cara di dalam EMT, di
antaranya:

- dengan cara yang sama seperti mendefinisikan vektor dengan
elemen-elemen beraturan (menggunakan titik dua ":");\\
- menggunakan perintah "sequence" dan rumus barisan (suku ke -n);\\
- menggunakan perintah "iterate" atau "niterate";\\
- menggunakan fungsi Maxima "create\_list" atau "makelist" untuk
menghasilkan barisan simbolik;\\
- menggunakan fungsi biasa yang inputnya vektor atau barisan;\\
- menggunakan fungsi rekursif.

EMT menyediakan beberapa perintah (fungsi) terkait barisan, yakni:

- sum: menghitung jumlah semua elemen suatu barisan\\
- cumsum: jumlah kumulatif suatu barisan\\
- differences: selisih antar elemen-elemen berturutan

EMT juga dapat digunakan untuk menghitung jumlah deret berhingga
maupun deret tak hingga, dengan menggunakan perintah (fungsi) "sum".
Perhitungan dapat dilakukan secara numerik maupun simbolik dan eksak.

Berikut adalah beberapa contoh perhitungan barisan dan deret
menggunakan EMT.
\end{eulercomment}
\begin{eulerprompt}
>1:10 // barisan sederhana
\end{eulerprompt}
\begin{euleroutput}
  [1,  2,  3,  4,  5,  6,  7,  8,  9,  10]
\end{euleroutput}
\begin{eulerprompt}
>1:2:30
\end{eulerprompt}
\begin{euleroutput}
  [1,  3,  5,  7,  9,  11,  13,  15,  17,  19,  21,  23,  25,  27,  29]
\end{euleroutput}
\begin{eulerprompt}
>sum(1:2:30), sum(1/(1:2:30))
\end{eulerprompt}
\begin{euleroutput}
  225
  2.33587263431
\end{euleroutput}
\begin{eulerprompt}
>$'sum(k, k, 1, n) = factor(ev(sum(k, k, 1, n),simpsum=true)) // simpsum:menghitung deret secara simbolik
\end{eulerprompt}
\begin{eulerformula}
\[
\sum_{k=1}^{n}{k}=\frac{n\,\left(n+1\right)}{2}
\]
\end{eulerformula}
\begin{eulerprompt}
>$'sum(1/(3^k+k), k, 0, inf) = factor(ev(sum(1/(3^k+k), k, 0, inf),simpsum=true))
\end{eulerprompt}
\begin{eulerformula}
\[
\sum_{k=0}^{\infty }{\frac{1}{3^{k}+k}}=\sum_{k=0}^{\infty }{\frac{  1}{3^{k}+k}}
\]
\end{eulerformula}
\begin{eulercomment}
Di sini masih gagal, hasilnya tidak dihitung.
\end{eulercomment}
\begin{eulerprompt}
>$'sum(1/x^2, x, 1, inf)= ev(sum(1/x^2, x, 1, inf),simpsum=true) // ev: menghitung nilai ekspresi
\end{eulerprompt}
\begin{eulerformula}
\[
\sum_{x=1}^{\infty }{\frac{1}{x^2}}=\frac{\pi^2}{6}
\]
\end{eulerformula}
\begin{eulerprompt}
>$'sum((-1)^(k-1)/k, k, 1, inf) = factor(ev(sum((-1)^(x-1)/x, x, 1, inf),simpsum=true))
\end{eulerprompt}
\begin{eulerformula}
\[
\sum_{k=1}^{\infty }{\frac{\left(-1\right)^{k-1}}{k}}=-\sum_{x=1}^{  \infty }{\frac{\left(-1\right)^{x}}{x}}
\]
\end{eulerformula}
\begin{eulercomment}
Di sini masih gagal, hasilnya tidak dihitung.
\end{eulercomment}
\begin{eulerprompt}
>$'sum((-1)^k/(2*k-1), k, 1, inf) = factor(ev(sum((-1)^k/(2*k-1), k, 1, inf),simpsum=true))
\end{eulerprompt}
\begin{eulerformula}
\[
\sum_{k=1}^{\infty }{\frac{\left(-1\right)^{k}}{2\,k-1}}=\sum_{k=1  }^{\infty }{\frac{\left(-1\right)^{k}}{2\,k-1}}
\]
\end{eulerformula}
\begin{eulerprompt}
>$ev(sum(1/n!, n, 0, inf),simpsum=true)
\end{eulerprompt}
\begin{eulerformula}
\[
\sum_{n=0}^{\infty }{\frac{1}{n!}}
\]
\end{eulerformula}
\begin{eulercomment}
Di sini masih gagal, hasilnya tidak dihitung, harusnya hasilnya e.
\end{eulercomment}
\begin{eulerprompt}
>&assume(abs(x)<1); $'sum(a*x^k, k, 0, inf)=ev(sum(a*x^k, k, 0, inf),simpsum=true), &forget(abs(x)<1);
\end{eulerprompt}
\begin{eulerformula}
\[
a\,\sum_{k=0}^{\infty }{x^{k}}=\frac{a}{1-x}
\]
\end{eulerformula}
\begin{eulercomment}
Deret geometri tak hingga, dengan asumsi rasional antara -1 dan 1.
\end{eulercomment}
\eulerheading{Deret Taylor}
\begin{eulercomment}
Deret Taylor suatu fungsi f yang diferensiabel sampai tak hingga di
sekitar x=a adalah:

\end{eulercomment}
\begin{eulerformula}
\[
f(x) = \sum_{k=0}^\infty \frac{(x-a)^k f^{(k)}(a)}{k!}.
\]
\end{eulerformula}
\begin{eulerprompt}
>$'e^x =taylor(exp(x),x,0,10) // deret Taylor e^x di sekitar x=0, sampai suku ke-11
\end{eulerprompt}
\begin{eulerformula}
\[
e^{x}=\frac{x^{10}}{3628800}+\frac{x^9}{362880}+\frac{x^8}{40320}+  \frac{x^7}{5040}+\frac{x^6}{720}+\frac{x^5}{120}+\frac{x^4}{24}+  \frac{x^3}{6}+\frac{x^2}{2}+x+1
\]
\end{eulerformula}
\begin{eulerprompt}
>$'log(x)=taylor(log(x),x,1,10)// deret log(x) di sekitar x=1
\end{eulerprompt}
\begin{eulerformula}
\[
\log x=x-\frac{\left(x-1\right)^{10}}{10}+\frac{\left(x-1\right)^9  }{9}-\frac{\left(x-1\right)^8}{8}+\frac{\left(x-1\right)^7}{7}-  \frac{\left(x-1\right)^6}{6}+\frac{\left(x-1\right)^5}{5}-\frac{  \left(x-1\right)^4}{4}+\frac{\left(x-1\right)^3}{3}-\frac{\left(x-1  \right)^2}{2}-1
\]
\end{eulerformula}
\end{eulernotebook}
\end{document}


\newpage
\chapter{KB Pekan 8: Menggunakan EMT untuk Geometri}
\documentclass{article}

\usepackage{eumat}

\begin{document}
\begin{eulernotebook}
\begin{eulercomment}
Vikram Zaky Ardianto\\
22305144028\\
Matematika E

\end{eulercomment}
\eulersubheading{Visualisasi dan Perhitungan Geometri dengan EMT}
\begin{eulercomment}
Euler menyediakan beberapa fungsi untuk melakukan visualisasi dan
perhitungan geometri, baik secara numerik maupun analitik (seperti
biasanya tentunya, menggunakan Maxima). Fungsi-fungsi untuk
visualisasi dan perhitungan geometeri tersebut disimpan di dalam file
program "geometry.e", sehingga file tersebut harus dipanggil sebelum
menggunakan fungsi-fungsi atau perintah-perintah untuk geometri.
\end{eulercomment}
\begin{eulerprompt}
>load geometry
\end{eulerprompt}
\begin{euleroutput}
  Numerical and symbolic geometry.
\end{euleroutput}
\eulersubheading{Fungsi-fungsi Geometri}
\begin{eulercomment}
Fungsi-fungsi untuk Menggambar Objek Geometri:

\end{eulercomment}
\begin{eulerttcomment}
  defaultd:=textheight()*1.5: nilai asli untuk parameter d
  setPlotrange(x1,x2,y1,y2): menentukan rentang x dan y pada bidang
\end{eulerttcomment}
\begin{eulercomment}
koordinat\\
\end{eulercomment}
\begin{eulerttcomment}
  setPlotRange(r): pusat bidang koordinat (0,0) dan batas-batas
\end{eulerttcomment}
\begin{eulercomment}
sumbu-x dan y adalah -r sd r\\
\end{eulercomment}
\begin{eulerttcomment}
  plotPoint (P, "P"): menggambar titik P dan diberi label "P"
  plotSegment (A,B, "AB", d): menggambar ruas garis AB, diberi label
\end{eulerttcomment}
\begin{eulercomment}
"AB" sejauh d\\
\end{eulercomment}
\begin{eulerttcomment}
  plotLine (g, "g", d): menggambar garis g diberi label "g" sejauh d
  plotCircle (c,"c",v,d): Menggambar lingkaran c dan diberi label "c"
  plotLabel (label, P, V, d): menuliskan label pada posisi P
\end{eulerttcomment}
\begin{eulercomment}

Fungsi-fungsi Geometri Analitik (numerik maupun simbolik):

\end{eulercomment}
\begin{eulerttcomment}
  turn(v, phi): memutar vektor v sejauh phi
  turnLeft(v):   memutar vektor v ke kiri
  turnRight(v):  memutar vektor v ke kanan
  normalize(v): normal vektor v
  crossProduct(v, w): hasil kali silang vektorv dan w.
  lineThrough(A, B): garis melalui A dan B, hasilnya [a,b,c] sdh.
\end{eulerttcomment}
\begin{eulercomment}
ax+by=c.\\
\end{eulercomment}
\begin{eulerttcomment}
  lineWithDirection(A,v): garis melalui A searah vektor v
  getLineDirection(g): vektor arah (gradien) garis g
  getNormal(g): vektor normal (tegak lurus) garis g
  getPointOnLine(g):  titik pada garis g
  perpendicular(A, g):  garis melalui A tegak lurus garis g
  parallel (A, g):  garis melalui A sejajar garis g
  lineIntersection(g, h):  titik potong garis g dan h
  projectToLine(A, g):   proyeksi titik A pada garis g
  distance(A, B):  jarak titik A dan B
  distanceSquared(A, B):  kuadrat jarak A dan B
  quadrance(A, B): kuadrat jarak A dan B
  areaTriangle(A, B, C):  luas segitiga ABC
  computeAngle(A, B, C):   besar sudut <ABC
  angleBisector(A, B, C): garis bagi sudut <ABC
  circleWithCenter (A, r): lingkaran dengan pusat A dan jari-jari r
  getCircleCenter(c):  pusat lingkaran c
  getCircleRadius(c):  jari-jari lingkaran c
  circleThrough(A,B,C):  lingkaran melalui A, B, C
  middlePerpendicular(A, B): titik tengah AB
  lineCircleIntersections(g, c): titik potong garis g dan lingkran c
  circleCircleIntersections (c1, c2):  titik potong lingkaran c1 dan
\end{eulerttcomment}
\begin{eulercomment}
c2\\
\end{eulercomment}
\begin{eulerttcomment}
  planeThrough(A, B, C):  bidang melalui titik A, B, C
\end{eulerttcomment}
\begin{eulercomment}

Fungsi-fungsi Khusus Untuk Geometri Simbolik:

\end{eulercomment}
\begin{eulerttcomment}
  getLineEquation (g,x,y): persamaan garis g dinyatakan dalam x dan y
  getHesseForm (g,x,y,A): bentuk Hesse garis g dinyatakan dalam x dan
\end{eulerttcomment}
\begin{eulercomment}
y dengan titik A pada\\
\end{eulercomment}
\begin{eulerttcomment}
  sisi positif (kanan/atas) garis
  quad(A,B): kuadrat jarak AB
  spread(a,b,c): Spread segitiga dengan panjang sisi-sisi a,b,c, yakni
\end{eulerttcomment}
\begin{eulercomment}
sin(alpha)\textasciicircum{}2 dengan\\
\end{eulercomment}
\begin{eulerttcomment}
  alpha sudut yang menghadap sisi a.
  crosslaw(a,b,c,sa): persamaan 3 quads dan 1 spread pada segitiga
\end{eulerttcomment}
\begin{eulercomment}
dengan panjang sisi a, b, c.\\
\end{eulercomment}
\begin{eulerttcomment}
  triplespread(sa,sb,sc): persamaan 3 spread sa,sb,sc yang memebntuk
\end{eulerttcomment}
\begin{eulercomment}
suatu segitiga\\
\end{eulercomment}
\begin{eulerttcomment}
  doublespread(sa): Spread sudut rangkap Spread 2*phi, dengan
\end{eulerttcomment}
\begin{eulercomment}
sa=sin(phi)\textasciicircum{}2 spread a.

\end{eulercomment}
\eulersubheading{Contoh 1: Luas, Lingkaran Luar, Lingkaran Dalam Segitiga}
\begin{eulercomment}
Untuk menggambar objek-objek geometri, langkah pertama adalah
menentukan rentang sumbu-sumbu koordinat. Semua objek geometri akan
digambar pada satu bidang koordinat, sampai didefinisikan bidang
koordinat yang baru.
\end{eulercomment}
\begin{eulerprompt}
>setPlotRange(-0.5,2.5,-0.5,2.5); // mendefinisikan bidang koordinat baru 
\end{eulerprompt}
\begin{eulercomment}
Sekarang tetapkan tiga poin dan plot mereka.
\end{eulercomment}
\begin{eulerprompt}
>A=[1,0]; plotPoint(A,"A"); // definisi dan gambar tiga titik
>B=[0,1]; plotPoint(B,"B");
>C=[2,2]; plotPoint(C,"C");
\end{eulerprompt}
\begin{eulercomment}
Kemudian tiga segmen.
\end{eulercomment}
\begin{eulerprompt}
>plotSegment(A,B,"c"); // c=AB
>plotSegment(B,C,"a"); // a=BC
>plotSegment(A,C,"b"); // b=AC
\end{eulerprompt}
\begin{eulercomment}
Fungsi geometri meliputi fungsi untuk membuat garis dan lingkaran.
Format garis adalah [a,b,c], yang mewakili garis dengan persamaan
ax+by=c.
\end{eulercomment}
\begin{eulerprompt}
>lineThrough(B,C) // garis yang melalui B dan C
\end{eulerprompt}
\begin{euleroutput}
  [-1,  2,  2]
\end{euleroutput}
\begin{eulercomment}
Hitunglah garis tegak lurus yang melalui A pada BC.
\end{eulercomment}
\begin{eulerprompt}
>h=perpendicular(A,lineThrough(B,C)); // garis h tegak lurus BC melalui A
\end{eulerprompt}
\begin{eulercomment}
Dan persimpangannya dengan BC.
\end{eulercomment}
\begin{eulerprompt}
>D=lineIntersection(h,lineThrough(B,C)); // D adalah titik potong h dan BC
\end{eulerprompt}
\begin{eulercomment}
Plot itu.
\end{eulercomment}
\begin{eulerprompt}
>plotPoint(D,value=1); // koordinat D ditampilkan
>aspect(1); plotSegment(A,D): // tampilkan semua gambar hasil plot...()
\end{eulerprompt}
\eulerimg{27}{images/Vikram Zaky Ardianto_22305144028_Geometri-001.png}
\begin{eulercomment}
Hitung luas ABC:

\end{eulercomment}
\begin{eulerformula}
\[
L_{\triangle ABC}= \frac{1}{2}AD.BC.
\]
\end{eulerformula}
\begin{eulerprompt}
>norm(A-D)*norm(B-C)/2 // AD=norm(A-D), BC=norm(B-C)
\end{eulerprompt}
\begin{euleroutput}
  1.5
\end{euleroutput}
\begin{eulercomment}
Bandingkan dengan rumus determinan.
\end{eulercomment}
\begin{eulerprompt}
>areaTriangle(A,B,C) // hitung luas segitiga langusng dengan fungsi
\end{eulerprompt}
\begin{euleroutput}
  1.5
\end{euleroutput}
\begin{eulercomment}
Cara lain menghitung luas segitigas ABC:
\end{eulercomment}
\begin{eulerprompt}
>distance(A,D)*distance(B,C)/2
\end{eulerprompt}
\begin{euleroutput}
  1.5
\end{euleroutput}
\begin{eulercomment}
Sudut di C
\end{eulercomment}
\begin{eulerprompt}
>degprint(computeAngle(B,C,A))
\end{eulerprompt}
\begin{euleroutput}
  36°52'11.63''
\end{euleroutput}
\begin{eulercomment}
Sekarang lingkaran luar segitiga.
\end{eulercomment}
\begin{eulerprompt}
>c=circleThrough(A,B,C); // lingkaran luar segitiga ABC
>R=getCircleRadius(c); // jari2 lingkaran luar 
>O=getCircleCenter(c); // titik pusat lingkaran c 
>plotPoint(O,"O"); // gambar titik "O"
>plotCircle(c,"Lingkaran luar segitiga ABC"):
\end{eulerprompt}
\eulerimg{27}{images/Vikram Zaky Ardianto_22305144028_Geometri-002.png}
\begin{eulercomment}
Tampilkan koordinat titik pusat dan jari-jari lingkaran luar.
\end{eulercomment}
\begin{eulerprompt}
>O, R
\end{eulerprompt}
\begin{euleroutput}
  [1.16667,  1.16667]
  1.17851130198
\end{euleroutput}
\begin{eulercomment}
Sekarang akan digambar lingkaran dalam segitiga ABC. Titik pusat lingkaran dalam adalah
titik potong garis-garis bagi sudut.
\end{eulercomment}
\begin{eulerprompt}
>l=angleBisector(A,C,B); // garis bagi <ACB
>g=angleBisector(C,A,B); // garis bagi <CAB
>P=lineIntersection(l,g) // titik potong kedua garis bagi sudut
\end{eulerprompt}
\begin{euleroutput}
  [0.86038,  0.86038]
\end{euleroutput}
\begin{eulercomment}
Tambahkan semuanya ke plot.
\end{eulercomment}
\begin{eulerprompt}
>color(5); plotLine(l); plotLine(g); color(1); // gambar kedua garis bagi sudut
>plotPoint(P,"P"); // gambar titik potongnya
>r=norm(P-projectToLine(P,lineThrough(A,B))) // jari-jari lingkaran dalam
\end{eulerprompt}
\begin{euleroutput}
  0.509653732104
\end{euleroutput}
\begin{eulerprompt}
>plotCircle(circleWithCenter(P,r),"Lingkaran dalam segitiga ABC"): // gambar lingkaran dalam
\end{eulerprompt}
\eulerimg{27}{images/Vikram Zaky Ardianto_22305144028_Geometri-003.png}
\eulersubheading{Latihan}
\begin{eulercomment}
1. Tentukan ketiga titik singgung lingkaran dalam dengan sisi-sisi
segitiga ABC.
\end{eulercomment}
\begin{eulerprompt}
>setPlotRange(-2.5,4.5,-2.5,4.5);
>A=[-2,1]; plotPoint(A,"A");
>B=[1,-2]; plotPoint(B,"B");
>C=[4,4]; plotPoint(C,"C");
\end{eulerprompt}
\begin{eulercomment}
2. Gambar segitiga dengan titik-titik sudut ketiga titik singgung
tersebut.
\end{eulercomment}
\begin{eulerprompt}
>plotSegment(A,B,"c")
>plotSegment(B,C,"a")
>plotSegment(A,C,"b")
>aspect(1):
\end{eulerprompt}
\eulerimg{27}{images/Vikram Zaky Ardianto_22305144028_Geometri-004.png}
\begin{eulercomment}
3. Tunjukkan bahwa garis bagi sudut yang ke tiga juga melalui titik
pusat lingkaran dalam.
\end{eulercomment}
\begin{eulerprompt}
>l=angleBisector(A,C,B);
>g=angleBisector(C,A,B);
>P=lineIntersection(l,g)
\end{eulerprompt}
\begin{euleroutput}
  [0.581139,  0.581139]
\end{euleroutput}
\begin{eulerprompt}
>color(5); plotLine(l); plotLine(g); color(1);
>plotPoint(P,"P");
>r=norm(P-projectToLine(P,lineThrough(A,B)))
\end{eulerprompt}
\begin{euleroutput}
  1.52896119631
\end{euleroutput}
\begin{eulerprompt}
>plotCircle(circleWithCenter(P,r),"Lingkaran dalam segitiga ABC"):
\end{eulerprompt}
\eulerimg{27}{images/Vikram Zaky Ardianto_22305144028_Geometri-005.png}
\begin{eulercomment}
Jadi, terbukti bahwa garis bagi sudut yang ketiga juga melalui titik
pusat lingkaran dalam.

4. Gambar jari-jari lingkaran dalam.
\end{eulercomment}
\begin{eulerprompt}
>r=norm(P-projectToLine(P,lineThrough(A,B)))
\end{eulerprompt}
\begin{euleroutput}
  1.52896119631
\end{euleroutput}
\begin{eulerprompt}
>plotCircle(circleWithCenter(P,r),"Lingkaran dalam segitiga ABC"):
\end{eulerprompt}
\eulerimg{27}{images/Vikram Zaky Ardianto_22305144028_Geometri-006.png}
\eulersubheading{Contoh 2: Geometri Simbolik}
\begin{eulercomment}
Kita dapat menghitung geometri eksak dan simbolik menggunakan Maxima.

File geometri.e menyediakan fungsi yang sama (dan lebih banyak lagi)
di Maxima. Namun, kita dapat menggunakan perhitungan simbolis
sekarang.
\end{eulercomment}
\begin{eulerprompt}
>A &= [1,0]; B &= [0,1]; C &= [2,2]; // menentukan tiga titik A, B, C
\end{eulerprompt}
\begin{eulercomment}
Fungsi untuk garis dan lingkaran bekerja seperti fungsi Euler, tetapi
memberikan perhitungan simbolis.
\end{eulercomment}
\begin{eulerprompt}
>c &= lineThrough(B,C) // c=BC
\end{eulerprompt}
\begin{euleroutput}
  
                               [- 1, 2, 2]
  
\end{euleroutput}
\begin{eulercomment}
Kita bisa mendapatkan persamaan garis dengan mudah.
\end{eulercomment}
\begin{eulerprompt}
>$getLineEquation(c,x,y), $solve(%,y) | expand // persamaan garis c
\end{eulerprompt}
\begin{eulerformula}
\[
\left[ y=\frac{x}{2}+1 \right] 
\]
\end{eulerformula}
\eulerimg{1}{images/Vikram Zaky Ardianto_22305144028_Geometri-008-large.png}
\begin{eulerprompt}
>$getLineEquation(lineThrough(A,[x1,y1]),x,y) // persamaan garis melalui A dan (x1, y1)
\end{eulerprompt}
\begin{eulerformula}
\[
\left({\it x_1}-1\right)\,y-x\,{\it y_1}=-{\it y_1}
\]
\end{eulerformula}
\begin{eulerprompt}
>h &= perpendicular(A,lineThrough(B,C)) // h melalui A tegak lurus BC
\end{eulerprompt}
\begin{euleroutput}
  
                                [2, 1, 2]
  
\end{euleroutput}
\begin{eulerprompt}
>Q &= lineIntersection(c,h) // Q titik potong garis c=BC dan h
\end{eulerprompt}
\begin{euleroutput}
  
                                   2  6
                                  [-, -]
                                   5  5
  
\end{euleroutput}
\begin{eulerprompt}
>$projectToLine(A,lineThrough(B,C)) // proyeksi A pada BC
\end{eulerprompt}
\begin{eulerformula}
\[
\left[ \frac{2}{5} , \frac{6}{5} \right] 
\]
\end{eulerformula}
\begin{eulerprompt}
>$distance(A,Q) // jarak AQ
\end{eulerprompt}
\begin{eulerformula}
\[
\frac{3}{\sqrt{5}}
\]
\end{eulerformula}
\begin{eulerprompt}
>cc &= circleThrough(A,B,C); $cc // (titik pusat dan jari-jari) lingkaran melalui A, B, C
\end{eulerprompt}
\begin{eulerformula}
\[
\left[ \frac{7}{6} , \frac{7}{6} , \frac{5}{3\,\sqrt{2}} \right] 
\]
\end{eulerformula}
\begin{eulerprompt}
>r&=getCircleRadius(cc); $r , $float(r) // tampilkan nilai jari-jari
\end{eulerprompt}
\begin{eulerformula}
\[
1.178511301977579
\]
\end{eulerformula}
\eulerimg{0}{images/Vikram Zaky Ardianto_22305144028_Geometri-014-large.png}
\begin{eulerprompt}
>$computeAngle(A,C,B) // nilai <ACB
\end{eulerprompt}
\begin{eulerformula}
\[
\arccos \left(\frac{4}{5}\right)
\]
\end{eulerformula}
\begin{eulerprompt}
>$solve(getLineEquation(angleBisector(A,C,B),x,y),y)[1] // persamaan garis bagi <ACB
\end{eulerprompt}
\begin{eulerformula}
\[
y=x
\]
\end{eulerformula}
\begin{eulerprompt}
>P &= lineIntersection(angleBisector(A,C,B),angleBisector(C,B,A)); $P // titik potong 2 garis bagi sudut
\end{eulerprompt}
\begin{eulerformula}
\[
\left[ \frac{\sqrt{2}\,\sqrt{5}+2}{6} , \frac{\sqrt{2}\,\sqrt{5}+2  }{6} \right] 
\]
\end{eulerformula}
\begin{eulerprompt}
>P() // hasilnya sama dengan perhitungan sebelumnya
\end{eulerprompt}
\begin{euleroutput}
  [0.86038,  0.86038]
\end{euleroutput}
\eulersubheading{Garis dan Lingkaran yang Berpotongan}
\begin{eulercomment}
Tentu saja, kita juga dapat memotong garis dengan lingkaran, dan
lingkaran dengan lingkaran.
\end{eulercomment}
\begin{eulerprompt}
>A &:= [1,0]; c=circleWithCenter(A,4);
>B &:= [1,2]; C &:= [2,1]; l=lineThrough(B,C);
>setPlotRange(5); plotCircle(c); plotLine(l);
\end{eulerprompt}
\begin{eulercomment}
Perpotongan garis dengan lingkaran menghasilkan dua titik dan jumlah
titik potong.
\end{eulercomment}
\begin{eulerprompt}
>\{P1,P2,f\}=lineCircleIntersections(l,c);
>P1, P2,
\end{eulerprompt}
\begin{euleroutput}
  [4.64575,  -1.64575]
  [-0.645751,  3.64575]
\end{euleroutput}
\begin{eulerprompt}
>plotPoint(P1); plotPoint(P2):
\end{eulerprompt}
\eulerimg{27}{images/Vikram Zaky Ardianto_22305144028_Geometri-018.png}
\begin{eulercomment}
Begitu pula di Maxima.
\end{eulercomment}
\begin{eulerprompt}
>c &= circleWithCenter(A,4) // lingkaran dengan pusat A jari-jari 4
\end{eulerprompt}
\begin{euleroutput}
  
                                [1, 0, 4]
  
\end{euleroutput}
\begin{eulerprompt}
>l &= lineThrough(B,C) // garis l melalui B dan C
\end{eulerprompt}
\begin{euleroutput}
  
                                [1, 1, 3]
  
\end{euleroutput}
\begin{eulerprompt}
>$lineCircleIntersections(l,c) | radcan, // titik potong lingkaran c dan garis l
\end{eulerprompt}
\begin{eulerformula}
\[
\left[ \left[ \sqrt{7}+2 , 1-\sqrt{7} \right]  , \left[ 2-\sqrt{7}   , \sqrt{7}+1 \right]  \right] 
\]
\end{eulerformula}
\begin{eulercomment}
Akan ditunjukkan bahwa sudut-sudut yang menghadap bsuusr yang sama adalah sama besar.
\end{eulercomment}
\begin{eulerprompt}
>C=A+normalize([-2,-3])*4; plotPoint(C); plotSegment(P1,C); plotSegment(P2,C);
>degprint(computeAngle(P1,C,P2))
\end{eulerprompt}
\begin{euleroutput}
  69°17'42.68''
\end{euleroutput}
\begin{eulerprompt}
>C=A+normalize([-4,-3])*4; plotPoint(C); plotSegment(P1,C); plotSegment(P2,C);
>degprint(computeAngle(P1,C,P2))
\end{eulerprompt}
\begin{euleroutput}
  69°17'42.68''
\end{euleroutput}
\begin{eulerprompt}
>insimg;
\end{eulerprompt}
\eulerimg{27}{images/Vikram Zaky Ardianto_22305144028_Geometri-020.png}
\eulersubheading{Garis Sumbu}
\begin{eulercomment}
Berikut adalah langkah-langkah menggambar garis sumbu ruas garis AB:

1. Gambar lingkaran dengan pusat A melalui B.\\
2. Gambar lingkaran dengan pusat B melalui A.\\
3. Tarik garis melallui kedua titik potong kedua lingkaran tersebut. Garis ini merupakan
garis sumbu (melalui titik tengah dan tegak lurus) AB.
\end{eulercomment}
\begin{eulerprompt}
>A=[2,2]; B=[-1,-2];
>c1=circleWithCenter(A,distance(A,B));
>c2=circleWithCenter(B,distance(A,B));
>\{P1,P2,f\}=circleCircleIntersections(c1,c2);
>l=lineThrough(P1,P2);
>setPlotRange(5); plotCircle(c1); plotCircle(c2);
>plotPoint(A); plotPoint(B); plotSegment(A,B); plotLine(l):
\end{eulerprompt}
\eulerimg{27}{images/Vikram Zaky Ardianto_22305144028_Geometri-021.png}
\begin{eulercomment}
Selanjutnya, kami melakukan hal yang sama di Maxima dengan koordinat
umum.
\end{eulercomment}
\begin{eulerprompt}
>A &= [a1,a2]; B &= [b1,b2];
>c1 &= circleWithCenter(A,distance(A,B));
>c2 &= circleWithCenter(B,distance(A,B));
>P &= circleCircleIntersections(c1,c2); P1 &= P[1]; P2 &= P[2];
\end{eulerprompt}
\begin{eulercomment}
Persamaan untuk persimpangan cukup terlibat. Tetapi kita dapat
menyederhanakannya, jika kita memecahkan y.
\end{eulercomment}
\begin{eulerprompt}
>g &= getLineEquation(lineThrough(P1,P2),x,y);
>$solve(g,y)
\end{eulerprompt}
\begin{eulerformula}
\[
\left[ y=\frac{-\left(2\,{\it b_1}-2\,{\it a_1}\right)\,x+{\it b_2}  ^2+{\it b_1}^2-{\it a_2}^2-{\it a_1}^2}{2\,{\it b_2}-2\,{\it a_2}}   \right] 
\]
\end{eulerformula}
\begin{eulercomment}
Ini memang sama dengan tegak lurus tengah, yang dihitung dengan cara
yang sama sekali berbeda.
\end{eulercomment}
\begin{eulerprompt}
>$solve(getLineEquation(middlePerpendicular(A,B),x,y),y)
\end{eulerprompt}
\begin{eulerformula}
\[
\left[ y=\frac{-\left(2\,{\it b_1}-2\,{\it a_1}\right)\,x+{\it b_2}  ^2+{\it b_1}^2-{\it a_2}^2-{\it a_1}^2}{2\,{\it b_2}-2\,{\it a_2}}   \right] 
\]
\end{eulerformula}
\begin{eulerprompt}
>h &=getLineEquation(lineThrough(A,B),x,y);
>$solve(h,y)
\end{eulerprompt}
\begin{eulerformula}
\[
\left[ y=\frac{\left({\it b_2}-{\it a_2}\right)\,x-{\it a_1}\,  {\it b_2}+{\it a_2}\,{\it b_1}}{{\it b_1}-{\it a_1}} \right] 
\]
\end{eulerformula}
\begin{eulercomment}
Perhatikan hasil kali gradien garis g dan h adalah:

\end{eulercomment}
\begin{eulerformula}
\[
\frac{-(b_1-a_1)}{(b_2-a_2)}\times \frac{(b_2-a_2)}{(b_1-a_1)} = -1.
\]
\end{eulerformula}
\begin{eulercomment}
Artinya kedua garis tegak lurus.

\end{eulercomment}
\eulersubheading{Contoh 3: Rumus Heron}
\begin{eulercomment}
Rumus Heron menyatakan bahwa luas segitiga dengan panjang sisi-sisi a,
b dan c adalah:

\end{eulercomment}
\begin{eulerformula}
\[
L = \sqrt{s(s-a)(s-b)(s-c)}\quad \text{ dengan } s=(a+b+c)/2,
\]
\end{eulerformula}
\begin{eulercomment}
Untuk membuktikan hal ini kita misalkan C(0,0), B(a,0) dan A(x,y),
b=AC, c=AB. Luas segitiga ABC adalah

\end{eulercomment}
\begin{eulerformula}
\[
L_{\triangle ABC}=\frac{1}{2}a\times y.
\]
\end{eulerformula}
\begin{eulercomment}
Nilai y didapat dengan menyelesaikan sistem persamaan:

\end{eulercomment}
\begin{eulerformula}
\[
x^2+y^2=b^2, \quad (x-a)^2+y^2=c^2.
\]
\end{eulerformula}
\begin{eulerprompt}
>sol &= solve([x^2+y^2=b^2,(x-a)^2+y^2=c^2],[x,y])
\end{eulerprompt}
\begin{euleroutput}
  
                                    []
  
\end{euleroutput}
\begin{eulerprompt}
>setPlotRange(-1,10,-1,8); plotPoint([0,0], "C(0,0)"); plotPoint([5.5,0], "B(a,0)");  ...
>plotPoint([7.5,6], "A(x,y)");
>plotSegment([0,0],[5.5,0], "a",25); plotSegment([5.5,0],[7.5,6],"c",15);  ...
>plotSegment([0,0],[7.5,6],"b",25); 
>plotSegment([7.5,6],[7.5,0],"t=y",25):
\end{eulerprompt}
\eulerimg{27}{images/Vikram Zaky Ardianto_22305144028_Geometri-025.png}
\begin{eulerprompt}
>sol &= solve([x^2+y^2=b^2,(x-a)^2+y^2=c^2],[x,y])
\end{eulerprompt}
\begin{euleroutput}
  
                                    []
  
\end{euleroutput}
\begin{eulercomment}
Ekstrak solusi y.
\end{eulercomment}
\begin{eulerprompt}
>ysol &= y with sol[2][2]; $ysol
\end{eulerprompt}
\begin{euleroutput}
  Maxima said:
  part: invalid index of list or matrix.
   -- an error. To debug this try: debugmode(true);
  
  Error in:
  ysol &= y with sol[2][2]; $ysol ...
                          ^
\end{euleroutput}
\begin{eulercomment}
Kami mendapatkan rumus Heron.
\end{eulercomment}
\begin{eulerprompt}
>function H(a,b,c) &= sqrt(factor((ysol*a/2)^2)); $'H(a,b,c)=H(a,b,c)
\end{eulerprompt}
\begin{eulerformula}
\[
H\left(a , b , \left[ 1 , 0 , 4 \right] \right)=\frac{\left| a  \right| \,\left| {\it ysol}\right| }{2}
\]
\end{eulerformula}
\begin{eulerprompt}
>$'Luas=H(3,4,5) // luas segitiga dengan panjang sisi-sisi 3, 4, 5
\end{eulerprompt}
\begin{eulerformula}
\[
{\it Luas}=\frac{3\,\left| {\it ysol}\right| }{2}
\]
\end{eulerformula}
\begin{eulercomment}
Tentu saja, setiap segitiga persegi panjang adalah kasus yang
terkenal.
\end{eulercomment}
\begin{eulerprompt}
>H(3,4,5) //luas segitiga siku-siku dengan panjang sisi 3, 4, 5
\end{eulerprompt}
\begin{euleroutput}
  Variable or function ysol not found.
  Try "trace errors" to inspect local variables after errors.
  H:
      useglobal; return abs(a)*abs(ysol)/2 
  Error in:
  H(3,4,5) //luas segitiga siku-siku dengan panjang sisi 3, 4, 5 ...
          ^
\end{euleroutput}
\begin{eulercomment}
Dan juga jelas, bahwa ini adalah segitiga dengan luas maksimal dan dua
sisi 3 dan 4.
\end{eulercomment}
\begin{eulerprompt}
>aspect (1.5); plot2d(&H(3,4,x),1,7): // Kurva luas segitiga sengan panjang sisi 3, 4, x (1<= x <=7)
\end{eulerprompt}
\begin{euleroutput}
  Variable or function ysol not found.
  Error in expression: 3*abs(ysol)/2
   %ploteval:
      y0=f$(x[1],args());
  adaptiveevalone:
      s=%ploteval(g$,t;args());
  Try "trace errors" to inspect local variables after errors.
  plot2d:
      dw/n,dw/n^2,dw/n,auto;args());
\end{euleroutput}
\begin{eulercomment}
Kasus umum juga berfungsi.
\end{eulercomment}
\begin{eulerprompt}
>$solve(diff(H(a,b,c)^2,c)=0,c)
\end{eulerprompt}
\begin{euleroutput}
  Maxima said:
  diff: second argument must be a variable; found [1,0,4]
   -- an error. To debug this try: debugmode(true);
  
  Error in:
   $solve(diff(H(a,b,c)^2,c)=0,c) ...
                                ^
\end{euleroutput}
\begin{eulercomment}
Sekarang mari kita cari himpunan semua titik di mana b+c=d untuk
beberapa konstanta d. Diketahui bahwa ini adalah elips.
\end{eulercomment}
\begin{eulerprompt}
>s1 &= subst(d-c,b,sol[2]); $s1
\end{eulerprompt}
\begin{euleroutput}
  Maxima said:
  part: invalid index of list or matrix.
   -- an error. To debug this try: debugmode(true);
  
  Error in:
  s1 &= subst(d-c,b,sol[2]); $s1 ...
                           ^
\end{euleroutput}
\begin{eulercomment}
Dan buat fungsi ini.
\end{eulercomment}
\begin{eulerprompt}
>function fx(a,c,d) &= rhs(s1[1]); $fx(a,c,d), function fy(a,c,d) &= rhs(s1[2]); $fy(a,c,d)
\end{eulerprompt}
\begin{eulerformula}
\[
0
\]
\end{eulerformula}
\eulerimg{0}{images/Vikram Zaky Ardianto_22305144028_Geometri-029-large.png}
\begin{eulercomment}
Sekarang kita bisa menggambar setnya. Sisi b bervariasi dari 1 hingga
4. Diketahui bahwa kita mendapatkan elips.
\end{eulercomment}
\begin{eulerprompt}
>aspect(1); plot2d(&fx(3,x,5),&fy(3,x,5),xmin=1,xmax=4,square=1):
\end{eulerprompt}
\eulerimg{27}{images/Vikram Zaky Ardianto_22305144028_Geometri-030.png}
\begin{eulercomment}
Kita dapat memeriksa persamaan umum untuk elips ini, yaitu.

\end{eulercomment}
\begin{eulerformula}
\[
\frac{(x-x_m)^2}{u^2}+\frac{(y-y_m)}{v^2}=1,
\]
\end{eulerformula}
\begin{eulercomment}
di mana (xm,ym) adalah pusat, dan u dan v adalah setengah sumbu.
\end{eulercomment}
\begin{eulerprompt}
>$ratsimp((fx(a,c,d)-a/2)^2/u^2+fy(a,c,d)^2/v^2 with [u=d/2,v=sqrt(d^2-a^2)/2])
\end{eulerprompt}
\begin{eulerformula}
\[
\frac{a^2}{d^2}
\]
\end{eulerformula}
\begin{eulercomment}
Kita lihat bahwa tinggi dan luas segitiga adalah maksimal untuk x=0.
Jadi luas segitiga dengan a+b+c=d maksimal jika segitiga sama sisi.
Kami ingin menurunkan ini secara analitis.
\end{eulercomment}
\begin{eulerprompt}
>eqns &= [diff(H(a,b,d-(a+b))^2,a)=0,diff(H(a,b,d-(a+b))^2,b)=0]; $eqns
\end{eulerprompt}
\begin{eulerformula}
\[
\left[ \frac{a\,{\it ysol}^2}{2}=0 , 0=0 \right] 
\]
\end{eulerformula}
\begin{eulercomment}
Kami mendapatkan beberapa minima, yang termasuk dalam segitiga dengan
satu sisi 0, dan solusinya a=b=c=d/3.
\end{eulercomment}
\begin{eulerprompt}
>$solve(eqns,[a,b])
\end{eulerprompt}
\begin{eulerformula}
\[
\left[ \left[ a=0 , b={\it \%r_1} \right]  \right] 
\]
\end{eulerformula}
\begin{eulercomment}
Ada juga metode Lagrange, memaksimalkan H(a,b,c)\textasciicircum{}2 terhadap a+b+d=d.
\end{eulercomment}
\begin{eulerprompt}
>&solve([diff(H(a,b,c)^2,a)=la,diff(H(a,b,c)^2,b)=la, ...
>   diff(H(a,b,c)^2,c)=la,a+b+c=d],[a,b,c,la])
\end{eulerprompt}
\begin{euleroutput}
  Maxima said:
  diff: second argument must be a variable; found [1,0,4]
   -- an error. To debug this try: debugmode(true);
  
  Error in:
  ... la,    diff(H(a,b,c)^2,c)=la,a+b+c=d],[a,b,c,la]) ...
                                                       ^
\end{euleroutput}
\begin{eulercomment}
Kita bisa membuat plot situasinya
\end{eulercomment}
\begin{eulercomment}
Pertama-tama atur poin di Maxima.
\end{eulercomment}
\begin{eulerprompt}
>A &= at([x,y],sol[2]); $A
\end{eulerprompt}
\begin{euleroutput}
  Maxima said:
  part: invalid index of list or matrix.
   -- an error. To debug this try: debugmode(true);
  
  Error in:
  A &= at([x,y],sol[2]); $A ...
                       ^
\end{euleroutput}
\begin{eulerprompt}
>B &= [0,0]; $B, C &= [a,0]; $C
\end{eulerprompt}
\begin{eulerformula}
\[
\left[ a , 0 \right] 
\]
\end{eulerformula}
\eulerimg{0}{images/Vikram Zaky Ardianto_22305144028_Geometri-035-large.png}
\begin{eulercomment}
Kemudian atur rentang plot, dan plot titik-titiknya.
\end{eulercomment}
\begin{eulerprompt}
>setPlotRange(0,5,-2,3); ...
>a=4; b=3; c=2; ...
>plotPoint(mxmeval("B"),"B"); plotPoint(mxmeval("C"),"C"); ...
>plotPoint(mxmeval("A"),"A"):
\end{eulerprompt}
\begin{euleroutput}
  Variable a1 not found!
  Use global variables or parameters for string evaluation.
  Error in Evaluate, superfluous characters found.
  Try "trace errors" to inspect local variables after errors.
  mxmeval:
      return evaluate(mxm(s));
  Error in:
  ... otPoint(mxmeval("C"),"C"); plotPoint(mxmeval("A"),"A"): ...
                                                       ^
\end{euleroutput}
\begin{eulercomment}
Plot segmen.
\end{eulercomment}
\begin{eulerprompt}
>plotSegment(mxmeval("A"),mxmeval("C")); ...
>plotSegment(mxmeval("B"),mxmeval("C")); ...
>plotSegment(mxmeval("B"),mxmeval("A")):
\end{eulerprompt}
\begin{euleroutput}
  Variable a1 not found!
  Use global variables or parameters for string evaluation.
  Error in Evaluate, superfluous characters found.
  Try "trace errors" to inspect local variables after errors.
  mxmeval:
      return evaluate(mxm(s));
  Error in:
  plotSegment(mxmeval("A"),mxmeval("C")); plotSegment(mxmeval("B ...
                          ^
\end{euleroutput}
\begin{eulercomment}
Hitung tegak lurus tengah di Maxima.
\end{eulercomment}
\begin{eulerprompt}
>h &= middlePerpendicular(A,B); g &= middlePerpendicular(B,C);
\end{eulerprompt}
\begin{eulercomment}
Dan pusat lingkaran.
\end{eulercomment}
\begin{eulerprompt}
>U &= lineIntersection(h,g);
\end{eulerprompt}
\begin{eulercomment}
Kami mendapatkan rumus untuk jari-jari lingkaran.
\end{eulercomment}
\begin{eulerprompt}
>&assume(a>0,b>0,c>0); $distance(U,B) | radcan
\end{eulerprompt}
\begin{eulerformula}
\[
\frac{\sqrt{{\it a_2}^2+{\it a_1}^2}\,\sqrt{{\it a_2}^2+{\it a_1}^2  -2\,a\,{\it a_1}+a^2}}{2\,\left| {\it a_2}\right| }
\]
\end{eulerformula}
\begin{eulercomment}
Mari kita tambahkan ini ke plot.
\end{eulercomment}
\begin{eulerprompt}
>plotPoint(U()); ...
>plotCircle(circleWithCenter(mxmeval("U"),mxmeval("distance(U,C)"))):
\end{eulerprompt}
\begin{euleroutput}
  Variable a2 not found!
  Use global variables or parameters for string evaluation.
  Error in ^
  Error in expression: [a/2,(a2^2+a1^2-a*a1)/(2*a2)]
  Error in:
  plotPoint(U()); plotCircle(circleWithCenter(mxmeval("U"),mxmev ...
               ^
\end{euleroutput}
\begin{eulercomment}
Menggunakan geometri, kami memperoleh rumus sederhana

\end{eulercomment}
\begin{eulerformula}
\[
\frac{a}{\sin(\alpha)}=2r
\]
\end{eulerformula}
\begin{eulercomment}
untuk radiusnya. Kami dapat memeriksa, apakah ini benar dengan Maxima.
Maxima akan memfaktorkan ini hanya jika kita kuadratkan.
\end{eulercomment}
\begin{eulerprompt}
>$c^2/sin(computeAngle(A,B,C))^2  | factor
\end{eulerprompt}
\begin{eulerformula}
\[
\left[ \frac{{\it a_2}^2+{\it a_1}^2}{{\it a_2}^2} , 0 , \frac{16\,  \left({\it a_2}^2+{\it a_1}^2\right)}{{\it a_2}^2} \right] 
\]
\end{eulerformula}
\eulersubheading{Contoh 4: Garis Euler dan Parabola}
\begin{eulercomment}
Garis Euler adalah garis yang ditentukan dari sembarang segitiga yang
tidak sama sisi. Ini adalah garis tengah segitiga, dan melewati
beberapa titik penting yang ditentukan dari segitiga, termasuk
orthocenter, circumcenter, centroid, titik Exeter dan pusat lingkaran
sembilan titik segitiga.

Untuk demonstrasi, kami menghitung dan memplot garis Euler dalam
sebuah segitiga.

Pertama, kita mendefinisikan sudut-sudut segitiga di Euler. Kami
menggunakan definisi, yang terlihat dalam ekspresi simbolis.
\end{eulercomment}
\begin{eulerprompt}
>A::=[-1,-1]; B::=[2,0]; C::=[1,2];
\end{eulerprompt}
\begin{eulercomment}
Untuk memplot objek geometris, kami menyiapkan area plot, dan
menambahkan titik ke sana. Semua plot objek geometris ditambahkan ke
plot saat ini.
\end{eulercomment}
\begin{eulerprompt}
>setPlotRange(3); plotPoint(A,"A"); plotPoint(B,"B"); plotPoint(C,"C");
\end{eulerprompt}
\begin{eulercomment}
Kita juga bisa menambahkan sisi segitiga.
\end{eulercomment}
\begin{eulerprompt}
>plotSegment(A,B,""); plotSegment(B,C,""); plotSegment(C,A,""):
\end{eulerprompt}
\eulerimg{27}{images/Vikram Zaky Ardianto_22305144028_Geometri-038.png}
\begin{eulercomment}
Berikut adalah luas segitiga, menggunakan rumus determinan. Tentu
saja, kita harus mengambil nilai absolut dari hasil ini.
\end{eulercomment}
\begin{eulerprompt}
>$areaTriangle(A,B,C)
\end{eulerprompt}
\begin{eulerformula}
\[
-\frac{7}{2}
\]
\end{eulerformula}
\begin{eulercomment}
Kita dapat menghitung koefisien sisi c.
\end{eulercomment}
\begin{eulerprompt}
>c &= lineThrough(A,B)
\end{eulerprompt}
\begin{euleroutput}
  
                              [- 1, 3, - 2]
  
\end{euleroutput}
\begin{eulercomment}
Dan juga dapatkan rumus untuk baris ini.
\end{eulercomment}
\begin{eulerprompt}
>$getLineEquation(c,x,y)
\end{eulerprompt}
\begin{eulerformula}
\[
3\,y-x=-2
\]
\end{eulerformula}
\begin{eulercomment}
Untuk bentuk Hesse, kita perlu menentukan sebuah titik, sehingga titik
tersebut berada di sisi positif dari bentuk Hesse. Memasukkan titik
menghasilkan jarak positif ke garis.
\end{eulercomment}
\begin{eulerprompt}
>$getHesseForm(c,x,y,C), $at(%,[x=C[1],y=C[2]])
\end{eulerprompt}
\begin{eulerformula}
\[
\frac{7}{\sqrt{10}}
\]
\end{eulerformula}
\eulerimg{1}{images/Vikram Zaky Ardianto_22305144028_Geometri-042-large.png}
\begin{eulercomment}
Sekarang kita hitung lingkaran luar ABC.
\end{eulercomment}
\begin{eulerprompt}
>LL &= circleThrough(A,B,C); $getCircleEquation(LL,x,y)
\end{eulerprompt}
\begin{eulerformula}
\[
\left(y-\frac{5}{14}\right)^2+\left(x-\frac{3}{14}\right)^2=\frac{  325}{98}
\]
\end{eulerformula}
\begin{eulerprompt}
>O &= getCircleCenter(LL); $O
\end{eulerprompt}
\begin{eulerformula}
\[
\left[ \frac{3}{14} , \frac{5}{14} \right] 
\]
\end{eulerformula}
\begin{eulercomment}
Gambarkan lingkaran dan pusatnya. Cu dan U adalah simbolis. Kami
mengevaluasi ekspresi ini untuk Euler.
\end{eulercomment}
\begin{eulerprompt}
>plotCircle(LL()); plotPoint(O(),"O"):
\end{eulerprompt}
\eulerimg{27}{images/Vikram Zaky Ardianto_22305144028_Geometri-045.png}
\begin{eulercomment}
Kita dapat menghitung perpotongan ketinggian di ABC (orthocenter)
secara numerik dengan perintah berikut.
\end{eulercomment}
\begin{eulerprompt}
>H &= lineIntersection(perpendicular(A,lineThrough(C,B)),...
>  perpendicular(B,lineThrough(A,C))); $H
\end{eulerprompt}
\begin{eulerformula}
\[
\left[ \frac{11}{7} , \frac{2}{7} \right] 
\]
\end{eulerformula}
\begin{eulercomment}
Sekarang kita dapat menghitung garis Euler dari segitiga.
\end{eulercomment}
\begin{eulerprompt}
>el &= lineThrough(H,O); $getLineEquation(el,x,y)
\end{eulerprompt}
\begin{eulerformula}
\[
-\frac{19\,y}{14}-\frac{x}{14}=-\frac{1}{2}
\]
\end{eulerformula}
\begin{eulercomment}
Tambahkan ke plot kami.
\end{eulercomment}
\begin{eulerprompt}
>plotPoint(H(),"H"); plotLine(el(),"Garis Euler"):
\end{eulerprompt}
\eulerimg{27}{images/Vikram Zaky Ardianto_22305144028_Geometri-048.png}
\begin{eulercomment}
Pusat gravitasi harus berada di garis ini.
\end{eulercomment}
\begin{eulerprompt}
>M &= (A+B+C)/3; $getLineEquation(el,x,y) with [x=M[1],y=M[2]]
\end{eulerprompt}
\begin{eulerformula}
\[
-\frac{1}{2}=-\frac{1}{2}
\]
\end{eulerformula}
\begin{eulerprompt}
>plotPoint(M(),"M"): // titik berat
\end{eulerprompt}
\eulerimg{27}{images/Vikram Zaky Ardianto_22305144028_Geometri-050.png}
\begin{eulercomment}
Teorinya memberitahu kita MH=2*MO. Kita perlu menyederhanakan dengan
radcan untuk mencapai ini.
\end{eulercomment}
\begin{eulerprompt}
>$distance(M,H)/distance(M,O)|radcan
\end{eulerprompt}
\begin{eulerformula}
\[
2
\]
\end{eulerformula}
\begin{eulercomment}
Fungsi termasuk fungsi untuk sudut juga.
\end{eulercomment}
\begin{eulerprompt}
>$computeAngle(A,C,B), degprint(%())
\end{eulerprompt}
\begin{eulerformula}
\[
\arccos \left(\frac{4}{\sqrt{5}\,\sqrt{13}}\right)
\]
\end{eulerformula}
\begin{euleroutput}
  60°15'18.43''
\end{euleroutput}
\begin{eulercomment}
Persamaan untuk pusat incircle tidak terlalu bagus.
\end{eulercomment}
\begin{eulerprompt}
>Q &= lineIntersection(angleBisector(A,C,B),angleBisector(C,B,A))|radcan; $Q
\end{eulerprompt}
\begin{eulerformula}
\[
\left[ \frac{\left(2^{\frac{3}{2}}+1\right)\,\sqrt{5}\,\sqrt{13}-15  \,\sqrt{2}+3}{14} , \frac{\left(\sqrt{2}-3\right)\,\sqrt{5}\,\sqrt{  13}+5\,2^{\frac{3}{2}}+5}{14} \right] 
\]
\end{eulerformula}
\begin{eulercomment}
Mari kita hitung juga ekspresi untuk jari-jari lingkaran yang
tertulis.
\end{eulercomment}
\begin{eulerprompt}
>r &= distance(Q,projectToLine(Q,lineThrough(A,B)))|ratsimp; $r
\end{eulerprompt}
\begin{eulerformula}
\[
\frac{\sqrt{\left(-41\,\sqrt{2}-31\right)\,\sqrt{5}\,\sqrt{13}+115  \,\sqrt{2}+614}}{7\,\sqrt{2}}
\]
\end{eulerformula}
\begin{eulerprompt}
>LD &=  circleWithCenter(Q,r); // Lingkaran dalam
\end{eulerprompt}
\begin{eulercomment}
Mari kita tambahkan ini ke plot.
\end{eulercomment}
\begin{eulerprompt}
>color(5); plotCircle(LD()):
\end{eulerprompt}
\eulerimg{27}{images/Vikram Zaky Ardianto_22305144028_Geometri-055.png}
\eulersubheading{Parabola}
\begin{eulercomment}
Selanjutnya akan dicari persamaan tempat kedudukan titik-titik yang berjarak sama ke titik C
dan ke garis AB.
\end{eulercomment}
\begin{eulerprompt}
>p &= getHesseForm(lineThrough(A,B),x,y,C)-distance([x,y],C); $p='0
\end{eulerprompt}
\begin{eulerformula}
\[
\frac{3\,y-x+2}{\sqrt{10}}-\sqrt{\left(2-y\right)^2+\left(1-x  \right)^2}=0
\]
\end{eulerformula}
\begin{eulercomment}
Persamaan tersebut dapat digambar menjadi satu dengan gambar sebelumnya.
\end{eulercomment}
\begin{eulerprompt}
>plot2d(p,level=0,add=1,contourcolor=6):
\end{eulerprompt}
\eulerimg{27}{images/Vikram Zaky Ardianto_22305144028_Geometri-057.png}
\begin{eulercomment}
Ini seharusnya menjadi beberapa fungsi, tetapi pemecah default Maxima
hanya dapat menemukan solusinya, jika kita kuadratkan persamaannya.
Akibatnya, kami mendapatkan solusi palsu.
\end{eulercomment}
\begin{eulerprompt}
>akar &= solve(getHesseForm(lineThrough(A,B),x,y,C)^2-distance([x,y],C)^2,y)
\end{eulerprompt}
\begin{euleroutput}
  
          [y = - 3 x - sqrt(70) sqrt(9 - 2 x) + 26, 
                                y = - 3 x + sqrt(70) sqrt(9 - 2 x) + 26]
  
\end{euleroutput}
\begin{eulercomment}
Solusi pertama adalah

maxima: akar[1]

Menambahkan solusi pertama ke plot menunjukkan, bahwa itu memang jalan
yang kita cari. Teorinya memberi tahu kita bahwa itu adalah parabola
yang diputar.
\end{eulercomment}
\begin{eulerprompt}
>plot2d(&rhs(akar[1]),add=1):
\end{eulerprompt}
\eulerimg{27}{images/Vikram Zaky Ardianto_22305144028_Geometri-058.png}
\begin{eulerprompt}
>function g(x) &= rhs(akar[1]); $'g(x)= g(x)// fungsi yang mendefinisikan kurva di atas
\end{eulerprompt}
\begin{eulerformula}
\[
g\left(x\right)=-3\,x-\sqrt{70}\,\sqrt{9-2\,x}+26
\]
\end{eulerformula}
\begin{eulerprompt}
>T &=[-1, g(-1)]; // ambil sebarang titik pada kurva tersebut
>dTC &= distance(T,C); $fullratsimp(dTC), $float(%) // jarak T ke C
\end{eulerprompt}
\begin{eulerformula}
\[
2.135605779339061
\]
\end{eulerformula}
\eulerimg{0}{images/Vikram Zaky Ardianto_22305144028_Geometri-061-large.png}
\begin{eulerprompt}
>U &= projectToLine(T,lineThrough(A,B)); $U // proyeksi T pada garis AB 
\end{eulerprompt}
\begin{eulerformula}
\[
\left[ \frac{80-3\,\sqrt{11}\,\sqrt{70}}{10} , \frac{20-\sqrt{11}\,  \sqrt{70}}{10} \right] 
\]
\end{eulerformula}
\begin{eulerprompt}
>dU2AB &= distance(T,U); $fullratsimp(dU2AB), $float(%) // jatak T ke AB
\end{eulerprompt}
\begin{eulerformula}
\[
2.135605779339061
\]
\end{eulerformula}
\eulerimg{0}{images/Vikram Zaky Ardianto_22305144028_Geometri-064-large.png}
\begin{eulercomment}
Ternyata jarak T ke C sama dengan jarak T ke AB. Coba Anda pilih titik T yang lain dan
ulangi perhitungan-perhitungan di atas untuk menunjukkan bahwa hasilnya juga sama.
\end{eulercomment}
\eulersubheading{Contoh 5: Trigonometri Rasional}
\begin{eulercomment}
Ini terinspirasi dari ceramah N.J.Wildberger. Dalam bukunya "Divine
Proportions", Wildberger mengusulkan untuk mengganti pengertian klasik
tentang jarak dan sudut dengan kuadrat dan penyebaran. Dengan
menggunakan ini, memang mungkin untuk menghindari fungsi trigonometri
dalam banyak contoh, dan tetap "rasional".

Berikut ini, saya memperkenalkan konsep, dan memecahkan beberapa
masalah. Saya menggunakan perhitungan simbolik Maxima di sini, yang
menyembunyikan keuntungan utama dari trigonometri rasional bahwa
perhitungan hanya dapat dilakukan dengan kertas dan pensil. Anda
diundang untuk memeriksa hasil tanpa komputer.

Intinya adalah bahwa perhitungan rasional simbolis sering kali
menghasilkan hasil yang sederhana. Sebaliknya, trigonometri klasik
menghasilkan hasil trigonometri yang rumit, yang hanya mengevaluasi
perkiraan numerik.
\end{eulercomment}
\begin{eulerprompt}
>load geometry;
\end{eulerprompt}
\begin{eulercomment}
Untuk pengenalan pertama, kami menggunakan segitiga persegi panjang
dengan proporsi Mesir terkenal 3, 4 dan 5. Perintah berikut adalah
perintah Euler untuk merencanakan geometri bidang yang terdapat dalam
file Euler "geometry.e".
\end{eulercomment}
\begin{eulerprompt}
>C&:=[0,0]; A&:=[4,0]; B&:=[0,3]; ...
>setPlotRange(-1,5,-1,5); ...
>plotPoint(A,"A"); plotPoint(B,"B"); plotPoint(C,"C"); ...
>plotSegment(B,A,"c"); plotSegment(A,C,"b"); plotSegment(C,B,"a"); ...
>insimg(30);
\end{eulerprompt}
\eulerimg{27}{images/Vikram Zaky Ardianto_22305144028_Geometri-065.png}
\begin{eulercomment}
Tentu saja,

\end{eulercomment}
\begin{eulerformula}
\[
\sin(w_a)=\frac{a}{c},
\]
\end{eulerformula}
\begin{eulercomment}
di mana wa adalah sudut di A. Cara yang biasa untuk menghitung sudut
ini, adalah dengan mengambil invers dari fungsi sinus. Hasilnya adalah
sudut yang tidak dapat dicerna, yang hanya dapat dicetak kira-kira.
\end{eulercomment}
\begin{eulerprompt}
>wa := arcsin(3/5); degprint(wa)
\end{eulerprompt}
\begin{euleroutput}
  36°52'11.63''
\end{euleroutput}
\begin{eulercomment}
Trigonometri rasional mencoba menghindari hal ini.

Gagasan pertama trigonometri rasional adalah kuadran, yang
menggantikan jarak. Sebenarnya, itu hanya jarak kuadrat. Berikut ini,
a, b, dan c menunjukkan kuadrat dari sisi-sisinya.

Teorema Pythogoras menjadi a+b=c.
\end{eulercomment}
\begin{eulerprompt}
>a &= 3^2; b &= 4^2; c &= 5^2; &a+b=c
\end{eulerprompt}
\begin{euleroutput}
  
                                 25 = 25
  
\end{euleroutput}
\begin{eulercomment}
Pengertian kedua dari trigonometri rasional adalah penyebaran. Spread
mengukur pembukaan antar baris. Ini adalah 0, jika garis-garisnya
sejajar, dan 1, jika garis-garisnya persegi panjang. Ini adalah
kuadrat sinus sudut antara dua garis.

Penyebaran garis AB dan AC pada gambar di atas didefinisikan sebagai:

\end{eulercomment}
\begin{eulerformula}
\[
s_a = \sin(\alpha)^2 = \frac{a}{c},
\]
\end{eulerformula}
\begin{eulercomment}
di mana a dan c adalah kuadrat dari sembarang segitiga siku-siku
dengan salah satu sudut di A.
\end{eulercomment}
\begin{eulerprompt}
>sa &= a/c; $sa
\end{eulerprompt}
\begin{eulerformula}
\[
\frac{9}{25}
\]
\end{eulerformula}
\begin{eulercomment}
Ini lebih mudah dihitung daripada sudut, tentu saja. Tetapi Anda
kehilangan properti bahwa sudut dapat ditambahkan dengan mudah.

Tentu saja, kita dapat mengonversi nilai perkiraan untuk sudut wa
menjadi sprad, dan mencetaknya sebagai pecahan.
\end{eulercomment}
\begin{eulerprompt}
>fracprint(sin(wa)^2)
\end{eulerprompt}
\begin{euleroutput}
  9/25
\end{euleroutput}
\begin{eulercomment}
Hukum kosinus trgonometri klasik diterjemahkan menjadi "hukum silang"
berikut.

\end{eulercomment}
\begin{eulerformula}
\[
(c+b-a)^2 = 4 b c \, (1-s_a)
\]
\end{eulerformula}
\begin{eulercomment}
Di sini a, b, dan c adalah kuadrat dari sisi-sisi segitiga, dan sa
adalah penyebaran sudut A. Sisi a, seperti biasa, berhadapan dengan
sudut A.

Hukum ini diimplementasikan dalam file geometri.e yang kami muat ke
Euler.
\end{eulercomment}
\begin{eulerprompt}
>$crosslaw(aa,bb,cc,saa)
\end{eulerprompt}
\begin{eulerformula}
\[
\left[ \left({\it bb}-{\it aa}+\frac{7}{6}\right)^2 , \left(  {\it bb}-{\it aa}+\frac{7}{6}\right)^2 , \left({\it bb}-{\it aa}+  \frac{5}{3\,\sqrt{2}}\right)^2 \right] =\left[ \frac{14\,{\it bb}\,  \left(1-{\it saa}\right)}{3} , \frac{14\,{\it bb}\,\left(1-{\it saa}  \right)}{3} , \frac{5\,2^{\frac{3}{2}}\,{\it bb}\,\left(1-{\it saa}  \right)}{3} \right] 
\]
\end{eulerformula}
\begin{eulercomment}
Dalam kasus kami, kami mendapatkan
\end{eulercomment}
\begin{eulerprompt}
>$crosslaw(a,b,c,sa)
\end{eulerprompt}
\begin{eulerformula}
\[
1024=1024
\]
\end{eulerformula}
\begin{eulercomment}
Mari kita gunakan crosslaw ini untuk mencari spread di A. Untuk
melakukan ini, kita buat crosslaw untuk kuadran a, b, dan c, dan
selesaikan untuk spread yang tidak diketahui sa.

Anda dapat melakukannya dengan tangan dengan mudah, tetapi saya
menggunakan Maxima. Tentu saja, kami mendapatkan hasilnya, kami sudah
memilikinya.
\end{eulercomment}
\begin{eulerprompt}
>$crosslaw(a,b,c,x), $solve(%,x)
\end{eulerprompt}
\begin{eulerformula}
\[
\left[ x=\frac{9}{25} \right] 
\]
\end{eulerformula}
\eulerimg{1}{images/Vikram Zaky Ardianto_22305144028_Geometri-070-large.png}
\begin{eulercomment}
Kita sudah tahu ini. Definisi spread adalah kasus khusus dari
crosslaw.

Kita juga dapat menyelesaikan ini untuk umum a,b,c. Hasilnya adalah
rumus yang menghitung penyebaran sudut segitiga yang diberikan kuadrat
dari ketiga sisinya.
\end{eulercomment}
\begin{eulerprompt}
>$solve(crosslaw(aa,bb,cc,x),x)
\end{eulerprompt}
\begin{eulerformula}
\[
\left[ \left[ \frac{168\,{\it bb}\,x+36\,{\it bb}^2+\left(-72\,  {\it aa}-84\right)\,{\it bb}+36\,{\it aa}^2-84\,{\it aa}+49}{36} ,   \frac{168\,{\it bb}\,x+36\,{\it bb}^2+\left(-72\,{\it aa}-84\right)  \,{\it bb}+36\,{\it aa}^2-84\,{\it aa}+49}{36} , \frac{15\,2^{\frac{  5}{2}}\,{\it bb}\,x+18\,{\it bb}^2+\left(-36\,{\it aa}-15\,2^{\frac{  3}{2}}\right)\,{\it bb}+18\,{\it aa}^2-15\,2^{\frac{3}{2}}\,{\it aa}  +25}{18} \right] =0 \right] 
\]
\end{eulerformula}
\begin{eulercomment}
Kita bisa membuat fungsi dari hasilnya. Fungsi seperti itu sudah
didefinisikan dalam file geometri.e dari Euler.
\end{eulercomment}
\begin{eulerprompt}
>$spread(a,b,c)
\end{eulerprompt}
\begin{eulerformula}
\[
\frac{9}{25}
\]
\end{eulerformula}
\begin{eulercomment}
Sebagai contoh, kita dapat menggunakannya untuk menghitung sudut
segitiga dengan sisi

\end{eulercomment}
\begin{eulerformula}
\[
a, \quad a, \quad \frac{4a}{7}
\]
\end{eulerformula}
\begin{eulercomment}
Hasilnya rasional, yang tidak begitu mudah didapat jika kita
menggunakan trigonometri klasik.
\end{eulercomment}
\begin{eulerprompt}
>$spread(a,a,4*a/7)
\end{eulerprompt}
\begin{eulerformula}
\[
\frac{6}{7}
\]
\end{eulerformula}
\begin{eulercomment}
Ini adalah sudut dalam derajat.
\end{eulercomment}
\begin{eulerprompt}
>degprint(arcsin(sqrt(6/7)))
\end{eulerprompt}
\begin{euleroutput}
  67°47'32.44''
\end{euleroutput}
\eulersubheading{Contoh lain}
\begin{eulercomment}
Sekarang, mari kita coba contoh yang lebih maju.

Kami mengatur tiga sudut segitiga sebagai berikut.
\end{eulercomment}
\begin{eulerprompt}
>A&:=[1,2]; B&:=[4,3]; C&:=[0,4]; ...
>setPlotRange(-1,5,1,7); ...
>plotPoint(A,"A"); plotPoint(B,"B"); plotPoint(C,"C"); ...
>plotSegment(B,A,"c"); plotSegment(A,C,"b"); plotSegment(C,B,"a"); ...
>insimg;
\end{eulerprompt}
\eulerimg{27}{images/Vikram Zaky Ardianto_22305144028_Geometri-074.png}
\begin{eulercomment}
Menggunakan Pythogoras, mudah untuk menghitung jarak antara dua titik.
Saya pertama kali menggunakan jarak fungsi file Euler untuk geometri.
Jarak fungsi menggunakan geometri klasik.
\end{eulercomment}
\begin{eulerprompt}
>$distance(A,B)
\end{eulerprompt}
\begin{eulerformula}
\[
\sqrt{10}
\]
\end{eulerformula}
\begin{eulercomment}
Euler juga mengandung fungsi untuk kuadran antara dua titik.

Dalam contoh berikut, karena c+b bukan a, maka segitiga itu bukan
persegi panjang.
\end{eulercomment}
\begin{eulerprompt}
>c &= quad(A,B); $c, b &= quad(A,C); $b, a &= quad(B,C); $a,
\end{eulerprompt}
\begin{eulerformula}
\[
17
\]
\end{eulerformula}
\eulerimg{0}{images/Vikram Zaky Ardianto_22305144028_Geometri-077-large.png}
\eulerimg{0}{images/Vikram Zaky Ardianto_22305144028_Geometri-078-large.png}
\begin{eulercomment}
Pertama, mari kita hitung sudut tradisional. Fungsi computeAngle
menggunakan metode biasa berdasarkan hasil kali titik dua vektor.
Hasilnya adalah beberapa pendekatan floating point.

\end{eulercomment}
\begin{eulerformula}
\[
A=<1,2>\quad B=<4,3>,\quad C=<0,4>
\]
\end{eulerformula}
\begin{eulerformula}
\[
\mathbf{a}=C-B=<-4,1>,\quad \mathbf{c}=A-B=<-3,-1>,\quad \beta=\angle ABC
\]
\end{eulerformula}
\begin{eulerformula}
\[
\mathbf{a}.\mathbf{c}=|\mathbf{a}|.|\mathbf{c}|\cos \beta
\]
\end{eulerformula}
\begin{eulerformula}
\[
\cos \angle ABC =\cos\beta=\frac{\mathbf{a}.\mathbf{c}}{|\mathbf{a}|.|\mathbf{c}|}=\frac{12-1}{\sqrt{17}\sqrt{10}}=\frac{11}{\sqrt{17}\sqrt{10}}
\]
\end{eulerformula}
\begin{eulerprompt}
>wb &= computeAngle(A,B,C); $wb, $(wb/pi*180)()
\end{eulerprompt}
\begin{eulerformula}
\[
\arccos \left(\frac{11}{\sqrt{10}\,\sqrt{17}}\right)
\]
\end{eulerformula}
\begin{euleroutput}
  32.4711922908
\end{euleroutput}
\begin{eulercomment}
Dengan menggunakan pensil dan kertas, kita dapat melakukan hal yang
sama dengan hukum silang. Kami memasukkan kuadran a, b, dan c ke dalam
hukum silang dan menyelesaikan x.
\end{eulercomment}
\begin{eulerprompt}
>$crosslaw(a,b,c,x), $solve(%,x), //(b+c-a)^=4b.c(1-x)
\end{eulerprompt}
\begin{eulerformula}
\[
\left[ x=\frac{49}{50} \right] 
\]
\end{eulerformula}
\eulerimg{1}{images/Vikram Zaky Ardianto_22305144028_Geometri-081-large.png}
\begin{eulercomment}
Yaitu, apa yang dilakukan oleh penyebaran fungsi yang didefinisikan
dalam "geometry.e".
\end{eulercomment}
\begin{eulerprompt}
>sb &= spread(b,a,c); $sb
\end{eulerprompt}
\begin{eulerformula}
\[
\frac{49}{170}
\]
\end{eulerformula}
\begin{eulercomment}
Maxima mendapatkan hasil yang sama menggunakan trigonometri biasa,
jika kita memaksanya. Itu menyelesaikan istilah sin(arccos(...))
menjadi hasil pecahan. Sebagian besar siswa tidak dapat melakukan ini.
\end{eulercomment}
\begin{eulerprompt}
>$sin(computeAngle(A,B,C))^2
\end{eulerprompt}
\begin{eulerformula}
\[
\frac{49}{170}
\]
\end{eulerformula}
\begin{eulercomment}
Setelah kita memiliki spread di B, kita dapat menghitung tinggi ha di
sisi a. Ingat bahwa

\end{eulercomment}
\begin{eulerformula}
\[
s_b=\frac{h_a}{c}
\]
\end{eulerformula}
\begin{eulercomment}
Menurut definisi.
\end{eulercomment}
\begin{eulerprompt}
>ha &= c*sb; $ha
\end{eulerprompt}
\begin{eulerformula}
\[
\frac{49}{17}
\]
\end{eulerformula}
\begin{eulercomment}
Gambar berikut telah dihasilkan dengan program geometri C.a.R., yang
dapat menggambar kuadrat dan menyebar.

image: (20) Rational\_Geometry\_CaR.png

Menurut definisi, panjang ha adalah akar kuadrat dari kuadratnya.
\end{eulercomment}
\begin{eulerprompt}
>$sqrt(ha)
\end{eulerprompt}
\begin{eulerformula}
\[
\frac{7}{\sqrt{17}}
\]
\end{eulerformula}
\begin{eulercomment}
Sekarang kita dapat menghitung luas segitiga. Jangan lupa, bahwa kita
berhadapan dengan kuadrat!
\end{eulercomment}
\begin{eulerprompt}
>$sqrt(ha)*sqrt(a)/2
\end{eulerprompt}
\begin{eulerformula}
\[
\frac{7}{2}
\]
\end{eulerformula}
\begin{eulercomment}
Rumus determinan biasa menghasilkan hasil yang sama.
\end{eulercomment}
\begin{eulerprompt}
>$areaTriangle(B,A,C)
\end{eulerprompt}
\begin{eulerformula}
\[
\frac{7}{2}
\]
\end{eulerformula}
\eulersubheading{Rumus Bangau}
\begin{eulercomment}
Sekarang, mari kita selesaikan masalah ini secara umum!
\end{eulercomment}
\begin{eulerprompt}
>&remvalue(a,b,c,sb,ha);
\end{eulerprompt}
\begin{eulercomment}
Pertama kita hitung spread di B untuk segitiga dengan sisi a, b, dan
c. Kemudian kita menghitung luas kuadrat ("quadrea"?), faktorkan
dengan Maxima, dan kita mendapatkan rumus Heron yang terkenal.

Memang, ini sulit dilakukan dengan pensil dan kertas.
\end{eulercomment}
\begin{eulerprompt}
>$spread(b^2,c^2,a^2), $factor(%*c^2*a^2/4)
\end{eulerprompt}
\begin{eulerformula}
\[
\frac{\left(-c+b+a\right)\,\left(c-b+a\right)\,\left(c+b-a\right)\,  \left(c+b+a\right)}{16}
\]
\end{eulerformula}
\eulerimg{1}{images/Vikram Zaky Ardianto_22305144028_Geometri-089-large.png}
\eulersubheading{Aturan Triple Spread}
\begin{eulercomment}
Kerugian dari spread adalah mereka tidak lagi hanya menambahkan sudut
yang sama.

Namun, tiga spread dari sebuah segitiga memenuhi aturan "triple
spread" berikut.
\end{eulercomment}
\begin{eulerprompt}
>&remvalue(sa,sb,sc); $triplespread(sa,sb,sc)
\end{eulerprompt}
\begin{eulerformula}
\[
\left({\it sc}+{\it sb}+{\it sa}\right)^2=2\,\left({\it sc}^2+  {\it sb}^2+{\it sa}^2\right)+4\,{\it sa}\,{\it sb}\,{\it sc}
\]
\end{eulerformula}
\begin{eulercomment}
Aturan ini berlaku untuk setiap tiga sudut yang menambah 180 °.

\end{eulercomment}
\begin{eulerformula}
\[
\alpha+\beta+\gamma=\pi
\]
\end{eulerformula}
\begin{eulercomment}
Sejak menyebar

\end{eulercomment}
\begin{eulerformula}
\[
\alpha, \pi-\alpha
\]
\end{eulerformula}
\begin{eulercomment}
sama, aturan triple spread juga benar, jika

\end{eulercomment}
\begin{eulerformula}
\[
\alpha+\beta=\gamma
\]
\end{eulerformula}
\begin{eulercomment}
Karena penyebaran sudut negatif adalah sama, aturan penyebaran rangkap
tiga juga berlaku, jika

\end{eulercomment}
\begin{eulerformula}
\[
\alpha+\beta+\gamma=0
\]
\end{eulerformula}
\begin{eulercomment}
Misalnya, kita dapat menghitung penyebaran sudut 60°. Ini 3/4.
Persamaan memiliki solusi kedua, bagaimanapun, di mana semua spread
adalah 0.
\end{eulercomment}
\begin{eulerprompt}
>$solve(triplespread(x,x,x),x)
\end{eulerprompt}
\begin{eulerformula}
\[
\left[ x=\frac{3}{4} , x=0 \right] 
\]
\end{eulerformula}
\begin{eulercomment}
Sebaran 90° jelas 1. Jika dua sudut dijumlahkan menjadi 90°,
sebarannya menyelesaikan persamaan sebaran rangkap tiga dengan a,b,1.
Dengan perhitungan berikut kita mendapatkan a+b=1.
\end{eulercomment}
\begin{eulerprompt}
>$triplespread(x,y,1), $solve(%,x)
\end{eulerprompt}
\begin{eulerformula}
\[
\left[ x=1-y \right] 
\]
\end{eulerformula}
\eulerimg{0}{images/Vikram Zaky Ardianto_22305144028_Geometri-093-large.png}
\begin{eulercomment}
Karena sebaran 180°-t sama dengan sebaran t, rumus sebaran rangkap
tiga juga berlaku, jika satu sudut adalah jumlah atau selisih dua
sudut lainnya.

Jadi kita dapat menemukan penyebaran sudut berlipat ganda. Perhatikan
bahwa ada dua solusi lagi. Kami membuat ini fungsi.
\end{eulercomment}
\begin{eulerprompt}
>$solve(triplespread(a,a,x),x), function doublespread(a) &= factor(rhs(%[1]))
\end{eulerprompt}
\begin{eulerformula}
\[
\left[ x=4\,a-4\,a^2 , x=0 \right] 
\]
\end{eulerformula}
\begin{euleroutput}
  
                              - 4 (a - 1) a
  
\end{euleroutput}
\eulersubheading{Pembagi Sudut}
\begin{eulercomment}
Ini situasinya, kita sudah tahu.
\end{eulercomment}
\begin{eulerprompt}
>C&:=[0,0]; A&:=[4,0]; B&:=[0,3]; ...
>setPlotRange(-1,5,-1,5); ...
>plotPoint(A,"A"); plotPoint(B,"B"); plotPoint(C,"C"); ...
>plotSegment(B,A,"c"); plotSegment(A,C,"b"); plotSegment(C,B,"a"); ...
>insimg;
\end{eulerprompt}
\eulerimg{27}{images/Vikram Zaky Ardianto_22305144028_Geometri-095.png}
\begin{eulercomment}
Mari kita hitung panjang garis bagi sudut di A. Tetapi kita ingin
menyelesaikannya untuk umum a,b,c.
\end{eulercomment}
\begin{eulerprompt}
>&remvalue(a,b,c);
\end{eulerprompt}
\begin{eulercomment}
Jadi pertama-tama kita hitung penyebaran sudut yang dibagi dua di A,
dengan menggunakan rumus sebaran rangkap tiga.

Masalah dengan rumus ini muncul lagi. Ini memiliki dua solusi. Kita
harus memilih yang benar. Solusi lainnya mengacu pada sudut terbelah
180 °-wa.
\end{eulercomment}
\begin{eulerprompt}
>$triplespread(x,x,a/(a+b)), $solve(%,x), sa2 &= rhs(%[1]); $sa2
\end{eulerprompt}
\begin{eulerformula}
\[
\frac{-\sqrt{b}\,\sqrt{b+a}+b+a}{2\,b+2\,a}
\]
\end{eulerformula}
\eulerimg{2}{images/Vikram Zaky Ardianto_22305144028_Geometri-097-large.png}
\eulerimg{1}{images/Vikram Zaky Ardianto_22305144028_Geometri-098-large.png}
\begin{eulercomment}
Mari kita periksa persegi panjang Mesir.
\end{eulercomment}
\begin{eulerprompt}
>$sa2 with [a=3^2,b=4^2]
\end{eulerprompt}
\begin{eulerformula}
\[
\frac{1}{10}
\]
\end{eulerformula}
\begin{eulercomment}
Kami dapat mencetak sudut dalam Euler, setelah mentransfer penyebaran
ke radian.
\end{eulercomment}
\begin{eulerprompt}
>wa2 := arcsin(sqrt(1/10)); degprint(wa2)
\end{eulerprompt}
\begin{euleroutput}
  18°26'5.82''
\end{euleroutput}
\begin{eulercomment}
Titik P adalah perpotongan garis bagi sudut dengan sumbu y.
\end{eulercomment}
\begin{eulerprompt}
>P := [0,tan(wa2)*4]
\end{eulerprompt}
\begin{euleroutput}
  [0,  1.33333]
\end{euleroutput}
\begin{eulerprompt}
>plotPoint(P,"P"); plotSegment(A,P):
\end{eulerprompt}
\eulerimg{27}{images/Vikram Zaky Ardianto_22305144028_Geometri-100.png}
\begin{eulercomment}
Mari kita periksa sudut dalam contoh spesifik kita.
\end{eulercomment}
\begin{eulerprompt}
>computeAngle(C,A,P), computeAngle(P,A,B)
\end{eulerprompt}
\begin{euleroutput}
  0.321750554397
  0.321750554397
\end{euleroutput}
\begin{eulercomment}
Sekarang kita hitung panjang garis bagi AP.

Kami menggunakan teorema sinus dalam segitiga APC. Teorema ini
menyatakan bahwa

\end{eulercomment}
\begin{eulerformula}
\[
\frac{BC}{\sin(w_a)} = \frac{AC}{\sin(w_b)} = \frac{AB}{\sin(w_c)}
\]
\end{eulerformula}
\begin{eulercomment}
berlaku dalam segitiga apa pun. Kuadratkan, itu diterjemahkan ke dalam
apa yang disebut "hukum penyebaran"

\end{eulercomment}
\begin{eulerformula}
\[
\frac{a}{s_a} = \frac{b}{s_b} = \frac{c}{s_b}
\]
\end{eulerformula}
\begin{eulercomment}
di mana a,b,c menunjukkan qudrances.

Karena spread CPA adalah 1-sa2, kita dapatkan darinya bisa/1=b/(1-sa2)
dan dapat menghitung bisa (kuadran dari garis-bagi sudut).
\end{eulercomment}
\begin{eulerprompt}
>&factor(ratsimp(b/(1-sa2))); bisa &= %; $bisa
\end{eulerprompt}
\begin{eulerformula}
\[
\frac{2\,b\,\left(b+a\right)}{\sqrt{b}\,\sqrt{b+a}+b+a}
\]
\end{eulerformula}
\begin{eulercomment}
Mari kita periksa rumus ini untuk nilai-nilai Mesir kita.
\end{eulercomment}
\begin{eulerprompt}
>sqrt(mxmeval("at(bisa,[a=3^2,b=4^2])")), distance(A,P)
\end{eulerprompt}
\begin{euleroutput}
  4.21637021356
  4.21637021356
\end{euleroutput}
\begin{eulercomment}
Kita juga dapat menghitung P menggunakan rumus spread.
\end{eulercomment}
\begin{eulerprompt}
>py&=factor(ratsimp(sa2*bisa)); $py
\end{eulerprompt}
\begin{eulerformula}
\[
-\frac{b\,\left(\sqrt{b}\,\sqrt{b+a}-b-a\right)}{\sqrt{b}\,\sqrt{b+  a}+b+a}
\]
\end{eulerformula}
\begin{eulercomment}
Nilainya sama dengan yang kita dapatkan dengan rumus trigonometri.
\end{eulercomment}
\begin{eulerprompt}
>sqrt(mxmeval("at(py,[a=3^2,b=4^2])"))
\end{eulerprompt}
\begin{euleroutput}
  1.33333333333
\end{euleroutput}
\eulersubheading{Sudut Akord}
\begin{eulercomment}
Perhatikan situasi berikut.
\end{eulercomment}
\begin{eulerprompt}
>setPlotRange(1.2); ...
>color(1); plotCircle(circleWithCenter([0,0],1)); ...
>A:=[cos(1),sin(1)]; B:=[cos(2),sin(2)]; C:=[cos(6),sin(6)]; ...
>plotPoint(A,"A"); plotPoint(B,"B"); plotPoint(C,"C"); ...
>color(3); plotSegment(A,B,"c"); plotSegment(A,C,"b"); plotSegment(C,B,"a"); ...
>color(1); O:=[0,0];  plotPoint(O,"0"); ...
>plotSegment(A,O); plotSegment(B,O); plotSegment(C,O,"r"); ...
>insimg;
\end{eulerprompt}
\eulerimg{27}{images/Vikram Zaky Ardianto_22305144028_Geometri-103.png}
\begin{eulercomment}
Kita dapat menggunakan Maxima untuk menyelesaikan rumus penyebaran
rangkap tiga untuk sudut-sudut di pusat O untuk r. Jadi kita
mendapatkan rumus untuk jari-jari kuadrat dari pericircle dalam hal
kuadrat dari sisi.

Kali ini, Maxima menghasilkan beberapa nol kompleks, yang kita
abaikan.
\end{eulercomment}
\begin{eulerprompt}
>&remvalue(a,b,c,r); // hapus nilai-nilai sebelumnya untuk perhitungan baru
>rabc &= rhs(solve(triplespread(spread(b,r,r),spread(a,r,r),spread(c,r,r)),r)[4]); $rabc
\end{eulerprompt}
\begin{eulerformula}
\[
-\frac{a\,b\,c}{c^2-2\,b\,c+a\,\left(-2\,c-2\,b\right)+b^2+a^2}
\]
\end{eulerformula}
\begin{eulercomment}
Kita dapat menjadikannya sebagai fungsi Euler.
\end{eulercomment}
\begin{eulerprompt}
>function periradius(a,b,c) &= rabc;
\end{eulerprompt}
\begin{eulercomment}
Mari kita periksa hasilnya untuk poin A,B,C.
\end{eulercomment}
\begin{eulerprompt}
>a:=quadrance(B,C); b:=quadrance(A,C); c:=quadrance(A,B);
\end{eulerprompt}
\begin{eulercomment}
Jari-jarinya memang 1.
\end{eulercomment}
\begin{eulerprompt}
>periradius(a,b,c)
\end{eulerprompt}
\begin{euleroutput}
  1
\end{euleroutput}
\begin{eulercomment}
Faktanya, spread CBA hanya bergantung pada b dan c. Ini adalah teorema
sudut chord.
\end{eulercomment}
\begin{eulerprompt}
>$spread(b,a,c)*rabc | ratsimp
\end{eulerprompt}
\begin{eulerformula}
\[
\frac{b}{4}
\]
\end{eulerformula}
\begin{eulercomment}
Sebenarnya spreadnya adalah b/(4r), dan kita melihat bahwa sudut chord
dari chord b adalah setengah dari sudut pusat.
\end{eulercomment}
\begin{eulerprompt}
>$doublespread(b/(4*r))-spread(b,r,r) | ratsimp
\end{eulerprompt}
\begin{eulerformula}
\[
0
\]
\end{eulerformula}
\eulersubheading{Contoh 6: Jarak Minimal pada Bidang}
\begin{eulercomment}
\end{eulercomment}
\eulersubheading{Catatan awal}
\begin{eulercomment}
Fungsi yang, ke titik M di bidang, menetapkan jarak AM antara titik
tetap A dan M, memiliki garis level yang agak sederhana: lingkaran
berpusat di A.
\end{eulercomment}
\begin{eulerprompt}
>&remvalue();
>A=[-1,-1];
>function d1(x,y):=sqrt((x-A[1])^2+(y-A[2])^2)
>fcontour("d1",xmin=-2,xmax=0,ymin=-2,ymax=0,hue=1, ...
>title="If you see ellipses, please set your window square"):
\end{eulerprompt}
\eulerimg{27}{images/Vikram Zaky Ardianto_22305144028_Geometri-107.png}
\begin{eulercomment}
dan grafiknya juga agak sederhana: bagian atas kerucut:
\end{eulercomment}
\begin{eulerprompt}
>plot3d("d1",xmin=-2,xmax=0,ymin=-2,ymax=0):
\end{eulerprompt}
\eulerimg{27}{images/Vikram Zaky Ardianto_22305144028_Geometri-108.png}
\begin{eulercomment}
Tentu saja minimal 0 dicapai di A.

\end{eulercomment}
\eulersubheading{Dua poin}
\begin{eulercomment}
Sekarang kita lihat fungsi MA+MB dimana A dan B adalah dua titik
(tetap). Ini adalah "fakta yang diketahui" bahwa kurva level adalah
elips, titik fokusnya adalah A dan B; kecuali untuk AB minimum yang
konstan pada segmen [AB]:
\end{eulercomment}
\begin{eulerprompt}
>B=[1,-1];
>function d2(x,y):=d1(x,y)+sqrt((x-B[1])^2+(y-B[2])^2)
>fcontour("d2",xmin=-2,xmax=2,ymin=-3,ymax=1,hue=1):
\end{eulerprompt}
\eulerimg{27}{images/Vikram Zaky Ardianto_22305144028_Geometri-109.png}
\begin{eulercomment}
Grafiknya lebih menarik:
\end{eulercomment}
\begin{eulerprompt}
>plot3d("d2",xmin=-2,xmax=2,ymin=-3,ymax=1):
\end{eulerprompt}
\eulerimg{27}{images/Vikram Zaky Ardianto_22305144028_Geometri-110.png}
\begin{eulercomment}
Pembatasan garis (AB) lebih terkenal:
\end{eulercomment}
\begin{eulerprompt}
>plot2d("abs(x+1)+abs(x-1)",xmin=-3,xmax=3):
\end{eulerprompt}
\eulerimg{27}{images/Vikram Zaky Ardianto_22305144028_Geometri-111.png}
\eulersubheading{Tiga poin}
\begin{eulercomment}
Sekarang hal-hal yang kurang sederhana: Ini sedikit kurang terkenal
bahwa MA+MB+MC mencapai minimum pada satu titik pesawat tetapi untuk
menentukan itu kurang sederhana:

1) Jika salah satu sudut segitiga ABC lebih dari 120° (katakanlah di
A), maka minimum dicapai pada titik ini (misalnya AB+AC).

Contoh:
\end{eulercomment}
\begin{eulerprompt}
>C=[-4,1];
>function d3(x,y):=d2(x,y)+sqrt((x-C[1])^2+(y-C[2])^2)
>plot3d("d3",xmin=-5,xmax=3,ymin=-4,ymax=4);
>insimg;
\end{eulerprompt}
\eulerimg{27}{images/Vikram Zaky Ardianto_22305144028_Geometri-112.png}
\begin{eulerprompt}
>fcontour("d3",xmin=-4,xmax=1,ymin=-2,ymax=2,hue=1,title="The minimum is on A");
>P=(A_B_C_A)'; plot2d(P[1],P[2],add=1,color=12);
>insimg;
\end{eulerprompt}
\eulerimg{27}{images/Vikram Zaky Ardianto_22305144028_Geometri-113.png}
\begin{eulercomment}
2) Tetapi jika semua sudut segitiga ABC kurang dari 120 °, minimumnya
adalah pada titik F di bagian dalam segitiga, yang merupakan
satu-satunya titik yang melihat sisi-sisi ABC dengan sudut yang sama
(maka masing-masing 120 ° ):
\end{eulercomment}
\begin{eulerprompt}
>C=[-0.5,1];
>plot3d("d3",xmin=-2,xmax=2,ymin=-2,ymax=2):
\end{eulerprompt}
\eulerimg{27}{images/Vikram Zaky Ardianto_22305144028_Geometri-114.png}
\begin{eulerprompt}
>fcontour("d3",xmin=-2,xmax=2,ymin=-2,ymax=2,hue=1,title="The Fermat point");
>P=(A_B_C_A)'; plot2d(P[1],P[2],add=1,color=12);
>insimg;
\end{eulerprompt}
\eulerimg{27}{images/Vikram Zaky Ardianto_22305144028_Geometri-115.png}
\begin{eulercomment}
Merupakan kegiatan yang menarik untuk mewujudkan gambar di atas dengan
perangkat lunak geometri; misalnya, saya tahu soft yang ditulis di
Jawa yang memiliki instruksi "garis kontur" ...

Semua ini di atas telah ditemukan oleh seorang hakim Perancis bernama
Pierre de Fermat; dia menulis surat kepada dilettants lain seperti
pendeta Marin Mersenne dan Blaise Pascal yang bekerja di pajak
penghasilan. Jadi titik unik F sedemikian rupa sehingga FA+FB+FC
minimal, disebut titik Fermat segitiga. Tetapi tampaknya beberapa
tahun sebelumnya, Torriccelli Italia telah menemukan titik ini sebelum
Fermat melakukannya! Bagaimanapun tradisinya adalah mencatat poin ini
F...

\end{eulercomment}
\eulersubheading{Empat poin}
\begin{eulercomment}
Langkah selanjutnya adalah menambahkan 4 titik D dan mencoba
meminimalkan MA+MB+MC+MD; katakan bahwa Anda adalah operator TV kabel
dan ingin mencari di bidang mana Anda harus meletakkan antena sehingga
Anda dapat memberi makan empat desa dan menggunakan panjang kabel
sesedikit mungkin!
\end{eulercomment}
\begin{eulerprompt}
>D=[1,1];
>function d4(x,y):=d3(x,y)+sqrt((x-D[1])^2+(y-D[2])^2)
>plot3d("d4",xmin=-1.5,xmax=1.5,ymin=-1.5,ymax=1.5):
\end{eulerprompt}
\eulerimg{27}{images/Vikram Zaky Ardianto_22305144028_Geometri-116.png}
\begin{eulerprompt}
>fcontour("d4",xmin=-1.5,xmax=1.5,ymin=-1.5,ymax=1.5,hue=1);
>P=(A_B_C_D)'; plot2d(P[1],P[2],points=1,add=1,color=12);
>insimg;
\end{eulerprompt}
\eulerimg{27}{images/Vikram Zaky Ardianto_22305144028_Geometri-117.png}
\begin{eulercomment}
Masih ada minimum dan tidak tercapai di salah satu simpul A, B, C atau
D:
\end{eulercomment}
\begin{eulerprompt}
>function f(x):=d4(x[1],x[2])
>neldermin("f",[0.2,0.2])
\end{eulerprompt}
\begin{euleroutput}
  [0.142858,  0.142857]
\end{euleroutput}
\begin{eulercomment}
Tampaknya dalam kasus ini, koordinat titik optimal adalah rasional
atau mendekati rasional...

Sekarang ABCD adalah persegi, kami berharap bahwa titik optimal akan
menjadi pusat ABCD:
\end{eulercomment}
\begin{eulerprompt}
>C=[-1,1];
>plot3d("d4",xmin=-1,xmax=1,ymin=-1,ymax=1):
\end{eulerprompt}
\eulerimg{27}{images/Vikram Zaky Ardianto_22305144028_Geometri-118.png}
\begin{eulerprompt}
>fcontour("d4",xmin=-1.5,xmax=1.5,ymin=-1.5,ymax=1.5,hue=1);
>P=(A_B_C_D)'; plot2d(P[1],P[2],add=1,color=12,points=1);
>insimg;
\end{eulerprompt}
\eulerimg{27}{images/Vikram Zaky Ardianto_22305144028_Geometri-119.png}
\eulersubheading{Contoh 7: Bola Dandelin dengan Povray}
\begin{eulercomment}
Anda dapat menjalankan demonstrasi ini, jika Anda telah menginstal
Povray, dan pvengine.exe di jalur program.

Pertama kita hitung jari-jari bola.

Jika Anda melihat gambar di bawah, Anda melihat bahwa kita membutuhkan
dua lingkaran yang menyentuh dua garis yang membentuk kerucut, dan
satu garis yang membentuk bidang yang memotong kerucut.

Kami menggunakan file geometri.e dari Euler untuk ini.
\end{eulercomment}
\begin{eulerprompt}
>load geometry;
\end{eulerprompt}
\begin{eulercomment}
Pertama dua garis yang membentuk kerucut.
\end{eulercomment}
\begin{eulerprompt}
>g1 &= lineThrough([0,0],[1,a])
\end{eulerprompt}
\begin{euleroutput}
  
                               [- a, 1, 0]
  
\end{euleroutput}
\begin{eulerprompt}
>g2 &= lineThrough([0,0],[-1,a])
\end{eulerprompt}
\begin{euleroutput}
  
                              [- a, - 1, 0]
  
\end{euleroutput}
\begin{eulercomment}
Kemudian saya baris ketiga.
\end{eulercomment}
\begin{eulerprompt}
>g &= lineThrough([-1,0],[1,1])
\end{eulerprompt}
\begin{euleroutput}
  
                               [- 1, 2, 1]
  
\end{euleroutput}
\begin{eulercomment}
Kami merencanakan semuanya sejauh ini.
\end{eulercomment}
\begin{eulerprompt}
>setPlotRange(-1,1,0,2);
>color(black); plotLine(g(),"")
>a:=2; color(blue); plotLine(g1(),""), plotLine(g2(),""):
\end{eulerprompt}
\eulerimg{27}{images/Vikram Zaky Ardianto_22305144028_Geometri-120.png}
\begin{eulercomment}
Sekarang kita ambil titik umum pada sumbu y.
\end{eulercomment}
\begin{eulerprompt}
>P &= [0,u]
\end{eulerprompt}
\begin{euleroutput}
  
                                  [0, u]
  
\end{euleroutput}
\begin{eulercomment}
Hitung jarak ke g1.
\end{eulercomment}
\begin{eulerprompt}
>d1 &= distance(P,projectToLine(P,g1)); $d1
\end{eulerprompt}
\begin{eulerformula}
\[
\sqrt{\left(\frac{a^2\,u}{a^2+1}-u\right)^2+\frac{a^2\,u^2}{\left(a  ^2+1\right)^2}}
\]
\end{eulerformula}
\begin{eulercomment}
Hitung jarak ke g.
\end{eulercomment}
\begin{eulerprompt}
>d &= distance(P,projectToLine(P,g)); $d
\end{eulerprompt}
\begin{eulerformula}
\[
\sqrt{\left(\frac{u+2}{5}-u\right)^2+\frac{\left(2\,u-1\right)^2}{  25}}
\]
\end{eulerformula}
\begin{eulercomment}
Dan temukan pusat kedua lingkaran yang jaraknya sama.
\end{eulercomment}
\begin{eulerprompt}
>sol &= solve(d1^2=d^2,u); $sol
\end{eulerprompt}
\begin{eulerformula}
\[
\left[ u=\frac{-\sqrt{5}\,\sqrt{a^2+1}+2\,a^2+2}{4\,a^2-1} , u=  \frac{\sqrt{5}\,\sqrt{a^2+1}+2\,a^2+2}{4\,a^2-1} \right] 
\]
\end{eulerformula}
\begin{eulercomment}
Ada dua solusi.

Kami mengevaluasi solusi simbolis, dan menemukan kedua pusat, dan
kedua jarak.
\end{eulercomment}
\begin{eulerprompt}
>u := sol()
\end{eulerprompt}
\begin{euleroutput}
  [0.333333,  1]
\end{euleroutput}
\begin{eulerprompt}
>dd := d()
\end{eulerprompt}
\begin{euleroutput}
  [0.149071,  0.447214]
\end{euleroutput}
\begin{eulercomment}
Plot lingkaran ke dalam gambar.
\end{eulercomment}
\begin{eulerprompt}
>color(red);
>plotCircle(circleWithCenter([0,u[1]],dd[1]),"");
>plotCircle(circleWithCenter([0,u[2]],dd[2]),"");
>insimg;
\end{eulerprompt}
\eulerimg{27}{images/Vikram Zaky Ardianto_22305144028_Geometri-124.png}
\eulersubheading{Plot dengan Povray}
\begin{eulercomment}
Selanjutnya kami merencanakan semuanya dengan Povray. Perhatikan bahwa
Anda mengubah perintah apa pun dalam urutan perintah Povray berikut,
dan menjalankan kembali semua perintah dengan Shift-Return.

Pertama kita memuat fungsi povray.
\end{eulercomment}
\begin{eulerprompt}
>load povray;
>defaultpovray="C:\(\backslash\)Program Files\(\backslash\)POV-Ray\(\backslash\)v3.7\(\backslash\)bin\(\backslash\)pvengine.exe"
\end{eulerprompt}
\begin{euleroutput}
  C:\(\backslash\)Program Files\(\backslash\)POV-Ray\(\backslash\)v3.7\(\backslash\)bin\(\backslash\)pvengine.exe
\end{euleroutput}
\begin{eulercomment}
Kami mengatur adegan dengan tepat.
\end{eulercomment}
\begin{eulerprompt}
>povstart(zoom=11,center=[0,0,0.5],height=10°,angle=140°);
\end{eulerprompt}
\begin{eulercomment}
Selanjutnya kita menulis dua bidang ke file Povray.
\end{eulercomment}
\begin{eulerprompt}
>writeln(povsphere([0,0,u[1]],dd[1],povlook(red)));
>writeln(povsphere([0,0,u[2]],dd[2],povlook(red)));
\end{eulerprompt}
\begin{eulercomment}
Dan kerucutnya, transparan.
\end{eulercomment}
\begin{eulerprompt}
>writeln(povcone([0,0,0],0,[0,0,a],1,povlook(lightgray,1)));
\end{eulerprompt}
\begin{eulercomment}
Kami menghasilkan bidang terbatas pada kerucut.
\end{eulercomment}
\begin{eulerprompt}
>gp=g();
>pc=povcone([0,0,0],0,[0,0,a],1,"");
>vp=[gp[1],0,gp[2]]; dp=gp[3];
>writeln(povplane(vp,dp,povlook(blue,0.5),pc));
\end{eulerprompt}
\begin{eulercomment}
Sekarang kita menghasilkan dua titik pada lingkaran, di mana bola
menyentuh kerucut.
\end{eulercomment}
\begin{eulerprompt}
>function turnz(v) := return [-v[2],v[1],v[3]]
>P1=projectToLine([0,u[1]],g1()); P1=turnz([P1[1],0,P1[2]]);
>writeln(povpoint(P1,povlook(yellow)));
>P2=projectToLine([0,u[2]],g1()); P2=turnz([P2[1],0,P2[2]]);
>writeln(povpoint(P2,povlook(yellow)));
\end{eulerprompt}
\begin{eulercomment}
Kemudian kami menghasilkan dua titik di mana bola menyentuh bidang.
Ini adalah fokus dari elips.
\end{eulercomment}
\begin{eulerprompt}
>P3=projectToLine([0,u[1]],g()); P3=[P3[1],0,P3[2]];
>writeln(povpoint(P3,povlook(yellow)));
>P4=projectToLine([0,u[2]],g()); P4=[P4[1],0,P4[2]];
>writeln(povpoint(P4,povlook(yellow)));
\end{eulerprompt}
\begin{eulercomment}
Selanjutnya kita hitung perpotongan P1P2 dengan bidang.
\end{eulercomment}
\begin{eulerprompt}
>t1=scalp(vp,P1)-dp; t2=scalp(vp,P2)-dp; P5=P1+t1/(t1-t2)*(P2-P1);
>writeln(povpoint(P5,povlook(yellow)));
\end{eulerprompt}
\begin{eulercomment}
Kami menghubungkan titik-titik dengan segmen garis.
\end{eulercomment}
\begin{eulerprompt}
>writeln(povsegment(P1,P2,povlook(yellow)));
>writeln(povsegment(P5,P3,povlook(yellow)));
>writeln(povsegment(P5,P4,povlook(yellow)));
\end{eulerprompt}
\begin{eulercomment}
Sekarang kita menghasilkan pita abu-abu, di mana bola menyentuh
kerucut.
\end{eulercomment}
\begin{eulerprompt}
>pcw=povcone([0,0,0],0,[0,0,a],1.01);
>pc1=povcylinder([0,0,P1[3]-defaultpointsize/2],[0,0,P1[3]+defaultpointsize/2],1);
>writeln(povintersection([pcw,pc1],povlook(gray)));
>pc2=povcylinder([0,0,P2[3]-defaultpointsize/2],[0,0,P2[3]+defaultpointsize/2],1);
>writeln(povintersection([pcw,pc2],povlook(gray)));
\end{eulerprompt}
\begin{eulercomment}
Mulai program Povray.
\end{eulercomment}
\begin{eulerprompt}
>povend();
\end{eulerprompt}
\begin{euleroutput}
  
\end{euleroutput}
\begin{eulercomment}
Untuk mendapatkan Anaglyph ini kita perlu memasukkan semuanya ke dalam
fungsi scene. Fungsi ini akan digunakan dua kali kemudian.
\end{eulercomment}
\begin{eulerprompt}
>function scene () ...
\end{eulerprompt}
\begin{eulerudf}
  global a,u,dd,g,g1,defaultpointsize;
  writeln(povsphere([0,0,u[1]],dd[1],povlook(red)));
  writeln(povsphere([0,0,u[2]],dd[2],povlook(red)));
  writeln(povcone([0,0,0],0,[0,0,a],1,povlook(lightgray,1)));
  gp=g();
  pc=povcone([0,0,0],0,[0,0,a],1,"");
  vp=[gp[1],0,gp[2]]; dp=gp[3];
  writeln(povplane(vp,dp,povlook(blue,0.5),pc));
  P1=projectToLine([0,u[1]],g1()); P1=turnz([P1[1],0,P1[2]]);
  writeln(povpoint(P1,povlook(yellow)));
  P2=projectToLine([0,u[2]],g1()); P2=turnz([P2[1],0,P2[2]]);
  writeln(povpoint(P2,povlook(yellow)));
  P3=projectToLine([0,u[1]],g()); P3=[P3[1],0,P3[2]];
  writeln(povpoint(P3,povlook(yellow)));
  P4=projectToLine([0,u[2]],g()); P4=[P4[1],0,P4[2]];
  writeln(povpoint(P4,povlook(yellow)));
  t1=scalp(vp,P1)-dp; t2=scalp(vp,P2)-dp; P5=P1+t1/(t1-t2)*(P2-P1);
  writeln(povpoint(P5,povlook(yellow)));
  writeln(povsegment(P1,P2,povlook(yellow)));
  writeln(povsegment(P5,P3,povlook(yellow)));
  writeln(povsegment(P5,P4,povlook(yellow)));
  pcw=povcone([0,0,0],0,[0,0,a],1.01);
  pc1=povcylinder([0,0,P1[3]-defaultpointsize/2],[0,0,P1[3]+defaultpointsize/2],1);
  writeln(povintersection([pcw,pc1],povlook(gray)));
  pc2=povcylinder([0,0,P2[3]-defaultpointsize/2],[0,0,P2[3]+defaultpointsize/2],1);
  writeln(povintersection([pcw,pc2],povlook(gray)));
  endfunction
\end{eulerudf}
\begin{eulercomment}
Anda membutuhkan kacamata merah/sian untuk menghargai efek berikut.
\end{eulercomment}
\begin{eulerprompt}
>povanaglyph("scene",zoom=11,center=[0,0,0.5],height=10°,angle=140°);
\end{eulerprompt}
\begin{euleroutput}
  exec:
      return _exec(program,param,dir,print,hidden,wait);
  povray:
      exec(program,params,defaulthome);
  Try "trace errors" to inspect local variables after errors.
  povanaglyph:
      povray(currentfile,w,h,aspect,exit); 
\end{euleroutput}
\eulersubheading{Contoh 8: Geometri Bumi}
\begin{eulercomment}
Dalam buku catatan ini, kami ingin melakukan beberapa perhitungan
sferis. Fungsi-fungsi tersebut terdapat dalam file "spherical.e" di
folder contoh. Kita perlu memuat file itu terlebih dahulu.
\end{eulercomment}
\begin{eulerprompt}
>load "spherical.e";
\end{eulerprompt}
\begin{eulercomment}
Untuk memasukkan posisi geografis, kami menggunakan vektor dengan dua
koordinat dalam radian (utara dan timur, nilai negatif untuk selatan
dan barat). Berikut koordinat Kampus FMIPA UNY.
\end{eulercomment}
\begin{eulerprompt}
>FMIPA=[rad(-7,-46.467),rad(110,23.05)]
\end{eulerprompt}
\begin{euleroutput}
  [-0.13569,  1.92657]
\end{euleroutput}
\begin{eulercomment}
Anda dapat mencetak posisi ini dengan sposprint (cetak posisi
spherical).
\end{eulercomment}
\begin{eulerprompt}
>sposprint(FMIPA) // posisi garis lintang dan garis bujur FMIPA UNY
\end{eulerprompt}
\begin{euleroutput}
  S 7°46.467' E 110°23.050'
\end{euleroutput}
\begin{eulercomment}
Mari kita tambahkan dua kota lagi, Solo dan Semarang.
\end{eulercomment}
\begin{eulerprompt}
>Solo=[rad(-7,-34.333),rad(110,49.683)]; Semarang=[rad(-6,-59.05),rad(110,24.533)];
>sposprint(Solo), sposprint(Semarang),
\end{eulerprompt}
\begin{euleroutput}
  S 7°34.333' E 110°49.683'
  S 6°59.050' E 110°24.533'
\end{euleroutput}
\begin{eulercomment}
Pertama kita menghitung vektor dari satu ke yang lain pada bola ideal.
Vektor ini [pos,jarak] dalam radian. Untuk menghitung jarak di bumi,
kita kalikan dengan jari-jari bumi pada garis lintang 7°.
\end{eulercomment}
\begin{eulerprompt}
>br=svector(FMIPA,Solo); degprint(br[1]), br[2]*rearth(7°)->km // perkiraan jarak FMIPA-Solo
\end{eulerprompt}
\begin{euleroutput}
  65°20'26.60''
  53.8945384608
\end{euleroutput}
\begin{eulercomment}
Ini adalah perkiraan yang baik. Rutinitas berikut menggunakan
perkiraan yang lebih baik. Pada jarak yang begitu pendek hasilnya
hampir sama.
\end{eulercomment}
\begin{eulerprompt}
>esdist(FMIPA,Semarang)->" km", // perkiraan jarak FMIPA-Semarang
\end{eulerprompt}
\begin{euleroutput}
  88.0114026318 km
\end{euleroutput}
\begin{eulercomment}
Ada fungsi untuk heading, dengan mempertimbangkan bentuk elips bumi.
Sekali lagi, kami mencetak dengan cara yang canggih.
\end{eulercomment}
\begin{eulerprompt}
>sdegprint(esdir(FMIPA,Solo))
\end{eulerprompt}
\begin{euleroutput}
       65.34°
\end{euleroutput}
\begin{eulercomment}
Sudut segitiga melebihi 180° pada bola.
\end{eulercomment}
\begin{eulerprompt}
>asum=sangle(Solo,FMIPA,Semarang)+sangle(FMIPA,Solo,Semarang)+sangle(FMIPA,Semarang,Solo); degprint(asum)
\end{eulerprompt}
\begin{euleroutput}
  180°0'10.77''
\end{euleroutput}
\begin{eulercomment}
Ini dapat digunakan untuk menghitung luas segitiga. Catatan: Untuk
segitiga kecil, ini tidak akurat karena kesalahan pengurangan dalam
asum-pi.
\end{eulercomment}
\begin{eulerprompt}
>(asum-pi)*rearth(48°)^2->" km^2", // perkiraan luas segitiga FMIPA-Solo-Semarang
\end{eulerprompt}
\begin{euleroutput}
  2116.02948749 km^2
\end{euleroutput}
\begin{eulercomment}
Ada fungsi untuk ini, yang menggunakan garis lintang rata-rata
segitiga untuk menghitung jari-jari bumi, dan menangani kesalahan
pembulatan untuk segitiga yang sangat kecil.
\end{eulercomment}
\begin{eulerprompt}
>esarea(Solo,FMIPA,Semarang)->" km^2", //perkiraan yang sama dengan fungsi esarea()
\end{eulerprompt}
\begin{euleroutput}
  2123.64310526 km^2
\end{euleroutput}
\begin{eulercomment}
Kita juga dapat menambahkan vektor ke posisi. Sebuah vektor berisi
heading dan jarak, keduanya dalam radian. Untuk mendapatkan vektor,
kami menggunakan vektor. Untuk menambahkan vektor ke posisi, kami
menggunakan vektor sadd.
\end{eulercomment}
\begin{eulerprompt}
>v=svector(FMIPA,Solo); sposprint(saddvector(FMIPA,v)), sposprint(Solo),
\end{eulerprompt}
\begin{euleroutput}
  S 7°34.333' E 110°49.683'
  S 7°34.333' E 110°49.683'
\end{euleroutput}
\begin{eulercomment}
Fungsi-fungsi ini mengasumsikan bola yang ideal. Hal yang sama di
bumi.
\end{eulercomment}
\begin{eulerprompt}
>sposprint(esadd(FMIPA,esdir(FMIPA,Solo),esdist(FMIPA,Solo))), sposprint(Solo),
\end{eulerprompt}
\begin{euleroutput}
  S 7°34.333' E 110°49.683'
  S 7°34.333' E 110°49.683'
\end{euleroutput}
\begin{eulercomment}
Mari kita beralih ke contoh yang lebih besar, Tugu Jogja dan Monas
Jakarta (menggunakan Google Earth untuk mencari koordinatnya).
\end{eulercomment}
\begin{eulerprompt}
>Tugu=[-7.7833°,110.3661°]; Monas=[-6.175°,106.811944°];
>sposprint(Tugu), sposprint(Monas)
\end{eulerprompt}
\begin{euleroutput}
  S 7°46.998' E 110°21.966'
  S 6°10.500' E 106°48.717'
\end{euleroutput}
\begin{eulercomment}
Menurut Google Earth, jaraknya adalah 429,66 km. Kami mendapatkan
pendekatan yang baik.
\end{eulercomment}
\begin{eulerprompt}
>esdist(Tugu,Monas)->" km", // perkiraan jarak Tugu Jogja - Monas Jakarta
\end{eulerprompt}
\begin{euleroutput}
  431.565659488 km
\end{euleroutput}
\begin{eulercomment}
Judulnya sama dengan judul yang dihitung di Google Earth.
\end{eulercomment}
\begin{eulerprompt}
>degprint(esdir(Tugu,Monas))
\end{eulerprompt}
\begin{euleroutput}
  294°17'2.85''
\end{euleroutput}
\begin{eulercomment}
Namun, kita tidak lagi mendapatkan posisi target yang tepat, jika kita
menambahkan heading dan jarak ke posisi semula. Hal ini terjadi,
karena kita tidak menghitung fungsi invers secara tepat, tetapi
mengambil perkiraan jari-jari bumi di sepanjang jalan.
\end{eulercomment}
\begin{eulerprompt}
>sposprint(esadd(Tugu,esdir(Tugu,Monas),esdist(Tugu,Monas)))
\end{eulerprompt}
\begin{euleroutput}
  S 6°10.500' E 106°48.717'
\end{euleroutput}
\begin{eulercomment}
Namun, kesalahannya tidak besar.
\end{eulercomment}
\begin{eulerprompt}
>sposprint(Monas),
\end{eulerprompt}
\begin{euleroutput}
  S 6°10.500' E 106°48.717'
\end{euleroutput}
\begin{eulercomment}
Tentu kita tidak bisa berlayar dengan tujuan yang sama dari satu
tujuan ke tujuan lainnya, jika kita ingin menempuh jalur terpendek.
Bayangkan, Anda terbang NE mulai dari titik mana pun di bumi. Kemudian
Anda akan berputar ke kutub utara. Lingkaran besar tidak mengikuti
heading yang konstan!

Perhitungan berikut menunjukkan bahwa kami jauh dari tujuan yang
benar, jika kami menggunakan pos yang sama selama perjalanan kami.
\end{eulercomment}
\begin{eulerprompt}
>dist=esdist(Tugu,Monas); hd=esdir(Tugu,Monas);
\end{eulerprompt}
\begin{eulercomment}
Sekarang kita tambahkan 10 kali sepersepuluh dari jarak, menggunakan
pos ke Monas, kita sampai di Tugu.
\end{eulercomment}
\begin{eulerprompt}
>p=Tugu; loop 1 to 10; p=esadd(p,hd,dist/10); end;
\end{eulerprompt}
\begin{eulercomment}
Hasilnya jauh.
\end{eulercomment}
\begin{eulerprompt}
>sposprint(p), skmprint(esdist(p,Monas))
\end{eulerprompt}
\begin{euleroutput}
  S 6°11.250' E 106°48.372'
       1.529km
\end{euleroutput}
\begin{eulercomment}
Sebagai contoh lain, mari kita ambil dua titik di bumi pada garis
lintang yang sama.
\end{eulercomment}
\begin{eulerprompt}
>P1=[30°,10°]; P2=[30°,50°];
\end{eulerprompt}
\begin{eulercomment}
Jalur terpendek dari P1 ke P2 bukanlah lingkaran garis lintang 30°,
melainkan jalur terpendek yang dimulai 10° lebih jauh ke utara di P1.
\end{eulercomment}
\begin{eulerprompt}
>sdegprint(esdir(P1,P2))
\end{eulerprompt}
\begin{euleroutput}
       79.69°
\end{euleroutput}
\begin{eulercomment}
Tapi, jika kita mengikuti pembacaan kompas ini, kita akan berputar ke
kutub utara! Jadi kita harus menyesuaikan arah kita di sepanjang
jalan. Untuk tujuan kasar, kami menyesuaikannya pada 1/10 dari total
jarak.
\end{eulercomment}
\begin{eulerprompt}
>p=P1;  dist=esdist(P1,P2); ...
>  loop 1 to 10; dir=esdir(p,P2); sdegprint(dir), p=esadd(p,dir,dist/10); end;
\end{eulerprompt}
\begin{euleroutput}
       79.69°
       81.67°
       83.71°
       85.78°
       87.89°
       90.00°
       92.12°
       94.22°
       96.29°
       98.33°
\end{euleroutput}
\begin{eulercomment}
Jaraknya tidak tepat, karena kita akan menambahkan sedikit kesalahan,
jika kita mengikuti heading yang sama terlalu lama.
\end{eulercomment}
\begin{eulerprompt}
>skmprint(esdist(p,P2))
\end{eulerprompt}
\begin{euleroutput}
       0.203km
\end{euleroutput}
\begin{eulercomment}
Kami mendapatkan perkiraan yang baik, jika kami menyesuaikan pos
setelah setiap 1/100 dari total jarak dari Tugu ke Monas.
\end{eulercomment}
\begin{eulerprompt}
>p=Tugu; dist=esdist(Tugu,Monas); ...
>  loop 1 to 100; p=esadd(p,esdir(p,Monas),dist/100); end;
>skmprint(esdist(p,Monas))
\end{eulerprompt}
\begin{euleroutput}
       0.000km
\end{euleroutput}
\begin{eulercomment}
Untuk keperluan navigasi, kita bisa mendapatkan urutan posisi GPS di
sepanjang lingkaran besar menuju Monas dengan fungsi navigasi.
\end{eulercomment}
\begin{eulerprompt}
>load spherical; v=navigate(Tugu,Monas,10); ...
>  loop 1 to rows(v); sposprint(v[#]), end;
\end{eulerprompt}
\begin{euleroutput}
  S 7°46.998' E 110°21.966'
  S 7°37.422' E 110°0.573'
  S 7°27.829' E 109°39.196'
  S 7°18.219' E 109°17.834'
  S 7°8.592' E 108°56.488'
  S 6°58.948' E 108°35.157'
  S 6°49.289' E 108°13.841'
  S 6°39.614' E 107°52.539'
  S 6°29.924' E 107°31.251'
  S 6°20.219' E 107°9.977'
  S 6°10.500' E 106°48.717'
\end{euleroutput}
\begin{eulercomment}
Kami menulis sebuah fungsi, yang memplot bumi, dua posisi, dan posisi
di antaranya.
\end{eulercomment}
\begin{eulerprompt}
>function testplot ...
\end{eulerprompt}
\begin{eulerudf}
  useglobal;
  plotearth;
  plotpos(Tugu,"Tugu Jogja"); plotpos(Monas,"Tugu Monas");
  plotposline(v);
  endfunction
\end{eulerudf}
\begin{eulercomment}
Sekarang rencanakan semuanya.
\end{eulercomment}
\begin{eulerprompt}
>plot3d("testplot",angle=25, height=6,>own,>user,zoom=4):
\end{eulerprompt}
\eulerimg{27}{images/Vikram Zaky Ardianto_22305144028_Geometri-125.png}
\begin{eulercomment}
Atau gunakan plot3d untuk mendapatkan tampilan anaglyph. Ini terlihat
sangat bagus dengan kacamata merah/sian.
\end{eulercomment}
\begin{eulerprompt}
>plot3d("testplot",angle=25,height=6,distance=5,own=1,anaglyph=1,zoom=4):
\end{eulerprompt}
\eulerimg{27}{images/Vikram Zaky Ardianto_22305144028_Geometri-126.png}
\eulersubheading{MENCOBA RUMUS-RUMUS PADA MATEI DI ATAS}
\eulersubheading{Geometri Simbolik}
\begin{eulerprompt}
>A &= [2,0]; B &= [0,2]; C &= [3,3]; // menentukan tiga titik A, B, C
>c &= lineThrough(B,C) // c=BC
\end{eulerprompt}
\begin{euleroutput}
  
                       lineThrough([0, 2], [3, 3])
  
\end{euleroutput}
\begin{eulerprompt}
>$getLineEquation(c,x,y), $solve(%,y) | expand // persamaan garis c
\end{eulerprompt}
\begin{eulerformula}
\[
{\it getLineEquation}\left({\it lineThrough}\left(\left[ 0 , 2
  \right]  , \left[ 3 , 3 \right] \right) , x , y\right)
\]
\end{eulerformula}
\begin{eulerformula}
\[
\left[ {\it getLineEquation}\left({\it lineThrough}\left(\left[ 0
  , 2 \right]  , \left[ 3 , 3 \right] \right) , x , y\right)=0
  \right] 
\]
\end{eulerformula}
\begin{eulerprompt}
>h &= perpendicular(A,lineThrough(B,C)) // h melalui A tegak lurus BC
\end{eulerprompt}
\begin{euleroutput}
  
            perpendicular([2, 0], lineThrough([0, 2], [3, 3]))
  
\end{euleroutput}
\begin{eulerprompt}
>Q &= lineIntersection(c,h) // Q titik potong garis c=BC dan h
\end{eulerprompt}
\begin{euleroutput}
  
         lineIntersection(lineThrough([0, 2], [3, 3]), 
                     perpendicular([2, 0], lineThrough([0, 2], [3, 3])))
  
\end{euleroutput}
\begin{eulerprompt}
>$projectToLine(A,lineThrough(B,C)) // proyeksi A pada BC
\end{eulerprompt}
\begin{eulerformula}
\[
{\it projectToLine}\left(\left[ 2 , 0 \right]  , {\it lineThrough}
 \left(\left[ 0 , 2 \right]  , \left[ 3 , 3 \right] \right)\right)
\]
\end{eulerformula}
\begin{eulerprompt}
>$distance(A,Q) // jarak AQ
\end{eulerprompt}
\begin{eulerformula}
\[
{\it distance}\left(\left[ 2 , 0 \right]  , {\it lineIntersection}
 \left({\it lineThrough}\left(\left[ 0 , 2 \right]  , \left[ 3 , 3
  \right] \right) , {\it perpendicular}\left(\left[ 2 , 0 \right]  , 
 {\it lineThrough}\left(\left[ 0 , 2 \right]  , \left[ 3 , 3 \right] 
 \right)\right)\right)\right)
\]
\end{eulerformula}
\begin{eulerprompt}
>cc &= circleThrough(A,B,C); $cc // (titik pusat dan jari-jari) lingkaran melalui A, B, C
\end{eulerprompt}
\begin{eulerformula}
\[
{\it circleThrough}\left(\left[ 2 , 0 \right]  , \left[ 0 , 2
  \right]  , \left[ 3 , 3 \right] \right)
\]
\end{eulerformula}
\begin{eulerprompt}
>r&=getCircleRadius(cc); $r , $float(r) // tampilkan nilai jari-jari
\end{eulerprompt}
\begin{eulerformula}
\[
{\it getCircleRadius}\left({\it circleThrough}\left(\left[ 2 , 0
  \right]  , \left[ 0 , 2 \right]  , \left[ 3 , 3 \right] \right)
 \right)
\]
\end{eulerformula}
\begin{eulerformula}
\[
{\it getCircleRadius}\left({\it circleThrough}\left(\left[ 2.0 , 
 0.0 \right]  , \left[ 0.0 , 2.0 \right]  , \left[ 3.0 , 3.0 \right] 
 \right)\right)
\]
\end{eulerformula}
\begin{eulerprompt}
>$computeAngle(A,C,B) // nilai <ACB
\end{eulerprompt}
\begin{eulerformula}
\[
{\it computeAngle}\left(\left[ 2 , 0 \right]  , \left[ 3 , 3
  \right]  , \left[ 0 , 2 \right] \right)
\]
\end{eulerformula}
\begin{eulerprompt}
>$solve(getLineEquation(angleBisector(A,C,B),x,y),y)[1] // persamaan garis bagi <ACB
\end{eulerprompt}
\begin{eulerformula}
\[
{\it getLineEquation}\left({\it angleBisector}\left(\left[ 2 , 0
  \right]  , \left[ 3 , 3 \right]  , \left[ 0 , 2 \right] \right) , x
  , y\right)=0
\]
\end{eulerformula}
\begin{eulerprompt}
>P &= lineIntersection(angleBisector(A,C,B),angleBisector(C,B,A)); $P // titik potong 2
\end{eulerprompt}
\begin{eulerformula}
\[
{\it lineIntersection}\left({\it angleBisector}\left(\left[ 2 , 0
  \right]  , \left[ 3 , 3 \right]  , \left[ 0 , 2 \right] \right) , 
 {\it angleBisector}\left(\left[ 3 , 3 \right]  , \left[ 0 , 2
  \right]  , \left[ 2 , 0 \right] \right)\right)
\]
\end{eulerformula}
\begin{eulerprompt}
>P() // 
\end{eulerprompt}
\begin{euleroutput}
  Function angleBisector not found.
  Try list ... to find functions!
  Error in expression: lineIntersection(angleBisector([2,0],[3,3],[0,2]),angleBisector([3,3],[0,2],[2,0]))
  Error in:
  P() //  ...
     ^
\end{euleroutput}
\eulersubheading{Garis dan Lingkaran yang berpotongan}
\begin{eulerprompt}
>A &:= [2,0]; c=circleWithCenter(A,4);
\end{eulerprompt}
\begin{euleroutput}
  Function circleWithCenter not found.
  Try list ... to find functions!
  Error in:
  A &:= [2,0]; c=circleWithCenter(A,4); ...
                                      ^
\end{euleroutput}
\begin{eulerprompt}
>B &:= [2,3]; C &:= [3,2]; l=lineThrough(B,C);
\end{eulerprompt}
\begin{euleroutput}
  Function lineThrough not found.
  Try list ... to find functions!
  Error in:
  B &:= [2,3]; C &:= [3,2]; l=lineThrough(B,C); ...
                                              ^
\end{euleroutput}
\begin{eulerprompt}
>setPlotRange(5); plotCircle(c); plotLine(l);
\end{eulerprompt}
\begin{euleroutput}
  Function setPlotRange not found.
  Try list ... to find functions!
  Error in:
  setPlotRange(5); plotCircle(c); plotLine(l); ...
                 ^
\end{euleroutput}
\begin{eulerprompt}
>\{P1,P2,f\}=lineCircleIntersections(l,c);
\end{eulerprompt}
\begin{euleroutput}
  Variable or function l not found.
  Error in:
  \{P1,P2,f\}=lineCircleIntersections(l,c); ...
                                     ^
\end{euleroutput}
\begin{eulerprompt}
>P1, P2,
\end{eulerprompt}
\begin{euleroutput}
  [0.523599,  0.174533]
  [0.523599,  0.872665]
\end{euleroutput}
\begin{eulerprompt}
>plotPoint(P1); plotPoint(P2):
\end{eulerprompt}
\begin{euleroutput}
  Function plotPoint not found.
  Try list ... to find functions!
  Error in:
  plotPoint(P1); plotPoint(P2): ...
               ^
\end{euleroutput}
\begin{eulercomment}
\end{eulercomment}
\begin{eulerprompt}
>c &= circleWithCenter(A,4) // lingkaran dengan pusat A jari-jari 4
\end{eulerprompt}
\begin{euleroutput}
  
                       circleWithCenter([2, 0], 4)
  
\end{euleroutput}
\begin{eulerprompt}
>l &= lineThrough(B,C) // garis l melalui B dan C
\end{eulerprompt}
\begin{euleroutput}
  
                       lineThrough([2, 3], [3, 2])
  
\end{euleroutput}
\begin{eulerprompt}
>$lineCircleIntersections(l,c) | radcan, // titik potong lingkaran c dan garis l
\end{eulerprompt}
\begin{eulerformula}
\[
{\it lineCircleIntersections}\left({\it lineThrough}\left(\left[ 2
  , 3 \right]  , \left[ 3 , 2 \right] \right) , 
 {\it circleWithCenter}\left(\left[ 2 , 0 \right]  , 4\right)\right)
\]
\end{eulerformula}
\begin{eulercomment}
\end{eulercomment}
\begin{eulerprompt}
>C=A+normalize([-3,-4])*4; plotPoint(C); plotSegment(P1,C); plotSegment(P2,C);
\end{eulerprompt}
\begin{euleroutput}
  Function normalize not found.
  Try list ... to find functions!
  Error in:
  C=A+normalize([-3,-4])*4; plotPoint(C); plotSegment(P1,C); plo ...
                        ^
\end{euleroutput}
\begin{eulerprompt}
>degprint(computeAngle(P1,C,P2))
\end{eulerprompt}
\begin{euleroutput}
  Function computeAngle not found.
  Try list ... to find functions!
  Error in:
  degprint(computeAngle(P1,C,P2)) ...
                                ^
\end{euleroutput}
\begin{eulerprompt}
>C=A+normalize([-4,-5])*4; plotPoint(C); plotSegment(P1,C); plotSegment(P2,C);
\end{eulerprompt}
\begin{euleroutput}
  Function normalize not found.
  Try list ... to find functions!
  Error in:
  C=A+normalize([-4,-5])*4; plotPoint(C); plotSegment(P1,C); plo ...
                        ^
\end{euleroutput}
\begin{eulerprompt}
>degprint(computeAngle(P1,C,P2))
\end{eulerprompt}
\begin{euleroutput}
  Function computeAngle not found.
  Try list ... to find functions!
  Error in:
  degprint(computeAngle(P1,C,P2)) ...
                                ^
\end{euleroutput}
\begin{eulerprompt}
>insimg;
\end{eulerprompt}
\eulerimg{27}{images/Vikram Zaky Ardianto_22305144028_Geometri-138.png}
\eulersubheading{ Garis Sumbu}
\begin{eulerprompt}
>A=[3,3]; B=[-2,-3];
>c1=circleWithCenter(A,distance(A,B));
\end{eulerprompt}
\begin{euleroutput}
  Function distance not found.
  Try list ... to find functions!
  Error in:
  c1=circleWithCenter(A,distance(A,B)); ...
                                     ^
\end{euleroutput}
\begin{eulerprompt}
>c2=circleWithCenter(B,distance(A,B));
\end{eulerprompt}
\begin{euleroutput}
  Function distance not found.
  Try list ... to find functions!
  Error in:
  c2=circleWithCenter(B,distance(A,B)); ...
                                     ^
\end{euleroutput}
\begin{eulerprompt}
>\{P1,P2,f\}=circleCircleIntersections(c1,c2);
\end{eulerprompt}
\begin{euleroutput}
  Variable or function c1 not found.
  Error in:
  \{P1,P2,f\}=circleCircleIntersections(c1,c2); ...
                                        ^
\end{euleroutput}
\begin{eulerprompt}
>l=lineThrough(P1,P2);
\end{eulerprompt}
\begin{euleroutput}
  Function lineThrough not found.
  Try list ... to find functions!
  Error in:
  l=lineThrough(P1,P2); ...
                      ^
\end{euleroutput}
\begin{eulerprompt}
>setPlotRange(5); plotCircle(c1); plotCircle(c2);
\end{eulerprompt}
\begin{euleroutput}
  Function setPlotRange not found.
  Try list ... to find functions!
  Error in:
  setPlotRange(5); plotCircle(c1); plotCircle(c2); ...
                 ^
\end{euleroutput}
\begin{eulerprompt}
>plotPoint(A); plotPoint(B); plotSegment(A,B); plotLine(l):
\end{eulerprompt}
\begin{euleroutput}
  Function plotPoint not found.
  Try list ... to find functions!
  Error in:
  plotPoint(A); plotPoint(B); plotSegment(A,B); plotLine(l): ...
              ^
\end{euleroutput}
\begin{eulerprompt}
>A &= [a1,a2]; B &= [b1,b2];
>c1 &= circleWithCenter(A,distance(A,B));
>c2 &= circleWithCenter(B,distance(A,B));
>P &= circleCircleIntersections(c1,c2); P1 &= P[1]; P2 &= P[2];
>g &= getLineEquation(lineThrough(P1,P2),x,y);
>$solve(g,y)
\end{eulerprompt}
\begin{eulerformula}
\[
\left[ {\it getLineEquation}\left({\it lineThrough}\left(\left(
 {\it circleCircleIntersections}\left({\it circleWithCenter}\left(
 \left[ {\it a_1} , {\it a_2} \right]  , {\it distance}\left(\left[ 
 {\it a_1} , {\it a_2} \right]  , \left[ {\it b_1} , {\it b_2}
  \right] \right)\right) , {\it circleWithCenter}\left(\left[ 
 {\it b_1} , {\it b_2} \right]  , {\it distance}\left(\left[ 
 {\it a_1} , {\it a_2} \right]  , \left[ {\it b_1} , {\it b_2}
  \right] \right)\right)\right)\right)_{1} , \left(
 {\it circleCircleIntersections}\left({\it circleWithCenter}\left(
 \left[ {\it a_1} , {\it a_2} \right]  , {\it distance}\left(\left[ 
 {\it a_1} , {\it a_2} \right]  , \left[ {\it b_1} , {\it b_2}
  \right] \right)\right) , {\it circleWithCenter}\left(\left[ 
 {\it b_1} , {\it b_2} \right]  , {\it distance}\left(\left[ 
 {\it a_1} , {\it a_2} \right]  , \left[ {\it b_1} , {\it b_2}
  \right] \right)\right)\right)\right)_{2}\right) , x , y\right)=0
  \right] 
\]
\end{eulerformula}
\begin{eulerprompt}
>$solve(getLineEquation(middlePerpendicular(A,B),x,y),y)
\end{eulerprompt}
\begin{eulerformula}
\[
\left[ {\it getLineEquation}\left({\it middlePerpendicular}\left(
 \left[ {\it a_1} , {\it a_2} \right]  , \left[ {\it b_1} , {\it b_2}
  \right] \right) , x , y\right)=0 \right] 
\]
\end{eulerformula}
\begin{eulerprompt}
>h &=getLineEquation(lineThrough(A,B),x,y);
>$solve(h,y)
\end{eulerprompt}
\begin{eulerformula}
\[
\left[ {\it getLineEquation}\left({\it lineThrough}\left(\left[ 
 {\it a_1} , {\it a_2} \right]  , \left[ {\it b_1} , {\it b_2}
  \right] \right) , x , y\right)=0 \right] 
\]
\end{eulerformula}
\eulersubheading{Garis Euler dan Parabola}
\begin{eulerprompt}
>A::=[-1.5,-1.5]; B::=[3,0]; C::=[1.5,3];
>setPlotRange(3); plotPoint(A,"A"); plotPoint(B,"B"); plotPoint(C,"C");
\end{eulerprompt}
\begin{euleroutput}
  Function setPlotRange not found.
  Try list ... to find functions!
  Error in:
  setPlotRange(3); plotPoint(A,"A"); plotPoint(B,"B"); plotPoint ...
                 ^
\end{euleroutput}
\begin{eulercomment}
\end{eulercomment}
\begin{eulerprompt}
>plotSegment(A,B,""); plotSegment(B,C,""); plotSegment(C,A,""):
\end{eulerprompt}
\begin{euleroutput}
  Function plotSegment not found.
  Try list ... to find functions!
  Error in:
  plotSegment(A,B,""); plotSegment(B,C,""); plotSegment(C,A,""): ...
                     ^
\end{euleroutput}
\begin{eulerprompt}
>$areaTriangle(A,B,C)
\end{eulerprompt}
\begin{eulerformula}
\[
{\it areaTriangle}\left(\left[ -\frac{3}{2} , -\frac{3}{2} \right] 
  , \left[ 3 , 0 \right]  , \left[ \frac{3}{2} , 3 \right] \right)
\]
\end{eulerformula}
\begin{eulercomment}
\end{eulercomment}
\begin{eulerprompt}
>c &= lineThrough(A,B)
\end{eulerprompt}
\begin{euleroutput}
  
                                    3    3
                     lineThrough([- -, - -], [3, 0])
                                    2    2
  
\end{euleroutput}
\begin{eulercomment}
\end{eulercomment}
\begin{eulerprompt}
>$getLineEquation(c,x,y)
\end{eulerprompt}
\begin{eulerformula}
\[
{\it getLineEquation}\left({\it lineThrough}\left(\left[ -\frac{3}{
 2} , -\frac{3}{2} \right]  , \left[ 3 , 0 \right] \right) , x , y
 \right)
\]
\end{eulerformula}
\begin{eulerprompt}
>$getHesseForm(c,x,y,C), $at(%,[x=C[1],y=C[2]])
\end{eulerprompt}
\begin{eulerformula}
\[
{\it getHesseForm}\left({\it lineThrough}\left(\left[ -\frac{3}{2}
  , -\frac{3}{2} \right]  , \left[ 3 , 0 \right] \right) , x , y , 
 \left[ \frac{3}{2} , 3 \right] \right)
\]
\end{eulerformula}
\begin{eulerformula}
\[
{\it getHesseForm}\left({\it lineThrough}\left(\left[ -\frac{3}{2}
  , -\frac{3}{2} \right]  , \left[ 3 , 0 \right] \right) , \frac{3}{2
 } , 3 , \left[ \frac{3}{2} , 3 \right] \right)
\]
\end{eulerformula}
\begin{eulercomment}
\end{eulercomment}
\begin{eulerprompt}
>LL &= circleThrough(A,B,C); $getCircleEquation(LL,x,y)
\end{eulerprompt}
\begin{eulerformula}
\[
{\it getCircleEquation}\left({\it circleThrough}\left(\left[ -
 \frac{3}{2} , -\frac{3}{2} \right]  , \left[ 3 , 0 \right]  , 
 \left[ \frac{3}{2} , 3 \right] \right) , x , y\right)
\]
\end{eulerformula}
\begin{eulerprompt}
>O &= getCircleCenter(LL); $O
\end{eulerprompt}
\begin{eulerformula}
\[
{\it getCircleCenter}\left({\it circleThrough}\left(\left[ -\frac{3
 }{2} , -\frac{3}{2} \right]  , \left[ 3 , 0 \right]  , \left[ \frac{
 3}{2} , 3 \right] \right)\right)
\]
\end{eulerformula}
\begin{eulerprompt}
>plotCircle(LL()); plotPoint(O(),"O"):
\end{eulerprompt}
\begin{euleroutput}
  Function circleThrough not found.
  Try list ... to find functions!
  Error in expression: circleThrough([-3/2,-3/2],[3,0],[3/2,3])
  Error in:
  plotCircle(LL()); plotPoint(O(),"O"): ...
                 ^
\end{euleroutput}
\begin{eulerprompt}
>H &= lineIntersection(perpendicular(A,lineThrough(C,B)),...
>  perpendicular(B,lineThrough(A,C))); $H
\end{eulerprompt}
\begin{eulerformula}
\[
{\it lineIntersection}\left({\it perpendicular}\left(\left[ -\frac{
 3}{2} , -\frac{3}{2} \right]  , {\it lineThrough}\left(\left[ \frac{
 3}{2} , 3 \right]  , \left[ 3 , 0 \right] \right)\right) , 
 {\it perpendicular}\left(\left[ 3 , 0 \right]  , {\it lineThrough}
 \left(\left[ -\frac{3}{2} , -\frac{3}{2} \right]  , \left[ \frac{3}{
 2} , 3 \right] \right)\right)\right)
\]
\end{eulerformula}
\begin{eulercomment}
\end{eulercomment}
\begin{eulerprompt}
>el &= lineThrough(H,O); $getLineEquation(el,x,y)
\end{eulerprompt}
\begin{eulerformula}
\[
{\it getLineEquation}\left({\it lineThrough}\left(
 {\it lineIntersection}\left({\it perpendicular}\left(\left[ -\frac{3
 }{2} , -\frac{3}{2} \right]  , {\it lineThrough}\left(\left[ \frac{3
 }{2} , 3 \right]  , \left[ 3 , 0 \right] \right)\right) , 
 {\it perpendicular}\left(\left[ 3 , 0 \right]  , {\it lineThrough}
 \left(\left[ -\frac{3}{2} , -\frac{3}{2} \right]  , \left[ \frac{3}{
 2} , 3 \right] \right)\right)\right) , {\it getCircleCenter}\left(
 {\it circleThrough}\left(\left[ -\frac{3}{2} , -\frac{3}{2} \right] 
  , \left[ 3 , 0 \right]  , \left[ \frac{3}{2} , 3 \right] \right)
 \right)\right) , x , y\right)
\]
\end{eulerformula}
\begin{eulercomment}
\end{eulercomment}
\begin{eulerprompt}
>plotPoint(H(),"H"); plotLine(el(),"Garis Euler"):
\end{eulerprompt}
\begin{euleroutput}
  Function lineThrough not found.
  Try list ... to find functions!
  Error in expression: lineIntersection(perpendicular([-3/2,-3/2],lineThrough([3/2,3],[3,0])),perpendicular([3,0],lineThrough([-3/2,-3/2],[3/2,3])))
  Error in:
  plotPoint(H(),"H"); plotLine(el(),"Garis Euler"): ...
               ^
\end{euleroutput}
\begin{eulercomment}
\end{eulercomment}
\begin{eulerprompt}
>M &= (A+B+C)/3; $getLineEquation(el,x,y) with [x=M[1],y=M[2]]
\end{eulerprompt}
\begin{eulerformula}
\[
{\it getLineEquation}\left({\it lineThrough}\left(
 {\it lineIntersection}\left({\it perpendicular}\left(\left[ -\frac{3
 }{2} , -\frac{3}{2} \right]  , {\it lineThrough}\left(\left[ \frac{3
 }{2} , 3 \right]  , \left[ 3 , 0 \right] \right)\right) , 
 {\it perpendicular}\left(\left[ 3 , 0 \right]  , {\it lineThrough}
 \left(\left[ -\frac{3}{2} , -\frac{3}{2} \right]  , \left[ \frac{3}{
 2} , 3 \right] \right)\right)\right) , {\it getCircleCenter}\left(
 {\it circleThrough}\left(\left[ -\frac{3}{2} , -\frac{3}{2} \right] 
  , \left[ 3 , 0 \right]  , \left[ \frac{3}{2} , 3 \right] \right)
 \right)\right) , 1 , \frac{1}{2}\right)
\]
\end{eulerformula}
\begin{eulerprompt}
>plotPoint(M(),"M"): // titik berat
\end{eulerprompt}
\begin{euleroutput}
  Function plotPoint not found.
  Try list ... to find functions!
  Error in:
  plotPoint(M(),"M"): // titik berat ...
                    ^
\end{euleroutput}
\begin{eulerprompt}
>$distance(M,H)/distance(M,O)|radcan
\end{eulerprompt}
\begin{eulerformula}
\[
\frac{{\it distance}\left(\left[ 1 , \frac{1}{2} \right]  , 
 {\it lineIntersection}\left({\it perpendicular}\left(\left[ -\frac{3
 }{2} , -\frac{3}{2} \right]  , {\it lineThrough}\left(\left[ \frac{3
 }{2} , 3 \right]  , \left[ 3 , 0 \right] \right)\right) , 
 {\it perpendicular}\left(\left[ 3 , 0 \right]  , {\it lineThrough}
 \left(\left[ -\frac{3}{2} , -\frac{3}{2} \right]  , \left[ \frac{3}{
 2} , 3 \right] \right)\right)\right)\right)}{{\it distance}\left(
 \left[ 1 , \frac{1}{2} \right]  , {\it getCircleCenter}\left(
 {\it circleThrough}\left(\left[ -\frac{3}{2} , -\frac{3}{2} \right] 
  , \left[ 3 , 0 \right]  , \left[ \frac{3}{2} , 3 \right] \right)
 \right)\right)}
\]
\end{eulerformula}
\begin{eulercomment}
\end{eulercomment}
\begin{eulerprompt}
>$computeAngle(A,C,B), degprint(%())
\end{eulerprompt}
\begin{eulerformula}
\[
{\it computeAngle}\left(\left[ -\frac{3}{2} , -\frac{3}{2} \right] 
  , \left[ \frac{3}{2} , 3 \right]  , \left[ 3 , 0 \right] \right)
\]
\end{eulerformula}
\begin{euleroutput}
  Function computeAngle not found.
  Try list ... to find functions!
  Error in expression: computeAngle([-3/2,-3/2],[3/2,3],[3,0])
  Error in:
  $computeAngle(A,C,B), degprint(%()) ...
                                    ^
\end{euleroutput}
\begin{eulerprompt}
>Q &= lineIntersection(angleBisector(A,C,B),angleBisector(C,B,A))|radcan; $Q
\end{eulerprompt}
\begin{eulerformula}
\[
{\it lineIntersection}\left({\it angleBisector}\left(\left[ -\frac{
 3}{2} , -\frac{3}{2} \right]  , \left[ \frac{3}{2} , 3 \right]  , 
 \left[ 3 , 0 \right] \right) , {\it angleBisector}\left(\left[ 
 \frac{3}{2} , 3 \right]  , \left[ 3 , 0 \right]  , \left[ -\frac{3}{
 2} , -\frac{3}{2} \right] \right)\right)
\]
\end{eulerformula}
\begin{eulerprompt}
>r &= distance(Q,projectToLine(Q,lineThrough(A,B)))|ratsimp; $r
\end{eulerprompt}
\begin{eulerformula}
\[
{\it distance}\left({\it lineIntersection}\left({\it angleBisector}
 \left(\left[ -\frac{3}{2} , -\frac{3}{2} \right]  , \left[ \frac{3}{
 2} , 3 \right]  , \left[ 3 , 0 \right] \right) , {\it angleBisector}
 \left(\left[ \frac{3}{2} , 3 \right]  , \left[ 3 , 0 \right]  , 
 \left[ -\frac{3}{2} , -\frac{3}{2} \right] \right)\right) , 
 {\it projectToLine}\left({\it lineIntersection}\left(
 {\it angleBisector}\left(\left[ -\frac{3}{2} , -\frac{3}{2} \right] 
  , \left[ \frac{3}{2} , 3 \right]  , \left[ 3 , 0 \right] \right) , 
 {\it angleBisector}\left(\left[ \frac{3}{2} , 3 \right]  , \left[ 3
  , 0 \right]  , \left[ -\frac{3}{2} , -\frac{3}{2} \right] \right)
 \right) , {\it lineThrough}\left(\left[ -\frac{3}{2} , -\frac{3}{2}
  \right]  , \left[ 3 , 0 \right] \right)\right)\right)
\]
\end{eulerformula}
\begin{eulerprompt}
>LD &=  circleWithCenter(Q,r); // Lingkaran dalam
\end{eulerprompt}
\begin{eulercomment}
\end{eulercomment}
\begin{eulerprompt}
>color(5); plotCircle(LD()):
\end{eulerprompt}
\begin{euleroutput}
  Function angleBisector not found.
  Try list ... to find functions!
  Error in expression: circleWithCenter(lineIntersection(angleBisector([-3/2,-3/2],[3/2,3],[3,0]),angleBisector([3/2,3],[3,0],[-3/2,-3/2])),distance(lineIntersection(angleBisector([-3/2,-3/2],[3/2,3],[3,0]),angleBisector([3/2,3],[3,0],[-3/2,-3/2])),projectToLine(lineIntersection(angleBisector([-3/2,-3/2],[3/2,3],[3,0]),angleBisector([3/2,3],[3,0],[-3/2,-3/2])),lineThrough([-3/2,-3/2],[3,0]))))
  Error in:
  color(5); plotCircle(LD()): ...
                           ^
\end{euleroutput}
\eulersubheading{contoh lain dari materi trigonometri rasional}
\begin{eulerprompt}
>A&:=[2,3]; B&:=[5,4]; C&:=[0,5]; ...
>setPlotRange(-1,5,1,7); ...
>plotPoint(A,"A"); plotPoint(B,"B"); plotPoint(C,"C"); ...
>plotSegment(B,A,"c"); plotSegment(A,C,"b"); plotSegment(C,B,"a"); ...
>insimg;
\end{eulerprompt}
\begin{euleroutput}
  Function setPlotRange not found.
  Try list ... to find functions!
  Error in:
  ... ,3]; B&:=[5,4]; C&:=[0,5]; setPlotRange(-1,5,1,7); plotPoint(A ...
                                                       ^
\end{euleroutput}
\begin{eulerprompt}
>$distance(A,B)
\end{eulerprompt}
\begin{eulerformula}
\[
{\it distance}\left(\left[ 2 , 3 \right]  , \left[ 5 , 4 \right] 
 \right)
\]
\end{eulerformula}
\begin{eulerprompt}
>c &= quad(A,B); $c, b &= quad(A,C); $b, a &= quad(B,C); $a,
\end{eulerprompt}
\begin{eulerformula}
\[
{\it quad}\left(\left[ 2 , 3 \right]  , \left[ 5 , 4 \right] 
 \right)
\]
\end{eulerformula}
\begin{eulerformula}
\[
{\it quad}\left(\left[ 2 , 3 \right]  , \left[ 0 , 5 \right] 
 \right)
\]
\end{eulerformula}
\begin{eulerformula}
\[
{\it quad}\left(\left[ 5 , 4 \right]  , \left[ 0 , 5 \right] 
 \right)
\]
\end{eulerformula}
\begin{eulercomment}
\end{eulercomment}
\begin{eulerprompt}
>wb &= computeAngle(A,B,C); $wb, $(wb/pi*180)()
\end{eulerprompt}
\begin{eulerformula}
\[
{\it computeAngle}\left(\left[ 2 , 3 \right]  , \left[ 5 , 4
  \right]  , \left[ 0 , 5 \right] \right)
\]
\end{eulerformula}
\begin{euleroutput}
  Function computeAngle not found.
  Try list ... to find functions!
  Error in expression: 180*computeAngle([2,3],[5,4],[0,5])/pi
  Error in:
  wb &= computeAngle(A,B,C); $wb, $(wb/pi*180)() ...
                                                ^
\end{euleroutput}
\begin{eulerprompt}
>$crosslaw(a,b,c,x), $solve(%,x), //(b+c-a)^=4b.c(1-x)
\end{eulerprompt}
\begin{eulerformula}
\[
{\it crosslaw}\left({\it quad}\left(\left[ 5 , 4 \right]  , \left[ 
 0 , 5 \right] \right) , {\it quad}\left(\left[ 2 , 3 \right]  , 
 \left[ 0 , 5 \right] \right) , {\it quad}\left(\left[ 2 , 3 \right] 
  , \left[ 5 , 4 \right] \right) , x\right)
\]
\end{eulerformula}
\begin{eulerformula}
\[
\left[ {\it crosslaw}\left({\it quad}\left(\left[ 5 , 4 \right]  , 
 \left[ 0 , 5 \right] \right) , {\it quad}\left(\left[ 2 , 3 \right] 
  , \left[ 0 , 5 \right] \right) , {\it quad}\left(\left[ 2 , 3
  \right]  , \left[ 5 , 4 \right] \right) , x\right)=0 \right] 
\]
\end{eulerformula}
\begin{eulerprompt}
>sb &= spread(b,a,c); $sb
\end{eulerprompt}
\begin{eulerformula}
\[
{\it spread}\left({\it quad}\left(\left[ 2 , 3 \right]  , \left[ 0
  , 5 \right] \right) , {\it quad}\left(\left[ 5 , 4 \right]  , 
 \left[ 0 , 5 \right] \right) , {\it quad}\left(\left[ 2 , 3 \right] 
  , \left[ 5 , 4 \right] \right)\right)
\]
\end{eulerformula}
\begin{eulerprompt}
>$sin(computeAngle(A,B,C))^2
\end{eulerprompt}
\begin{eulerformula}
\[
\sin ^2{\it computeAngle}\left(\left[ 2 , 3 \right]  , \left[ 5 , 4
  \right]  , \left[ 0 , 5 \right] \right)
\]
\end{eulerformula}
\begin{eulerprompt}
>ha &= c*sb; $ha
\end{eulerprompt}
\begin{eulerformula}
\[
{\it quad}\left(\left[ 2 , 3 \right]  , \left[ 5 , 4 \right] 
 \right)\,{\it spread}\left({\it quad}\left(\left[ 2 , 3 \right]  , 
 \left[ 0 , 5 \right] \right) , {\it quad}\left(\left[ 5 , 4 \right] 
  , \left[ 0 , 5 \right] \right) , {\it quad}\left(\left[ 2 , 3
  \right]  , \left[ 5 , 4 \right] \right)\right)
\]
\end{eulerformula}
\begin{eulerprompt}
>$sqrt(ha)
\end{eulerprompt}
\begin{eulerformula}
\[
\sqrt{{\it quad}\left(\left[ 2 , 3 \right]  , \left[ 5 , 4 \right] 
 \right)\,{\it spread}\left({\it quad}\left(\left[ 2 , 3 \right]  , 
 \left[ 0 , 5 \right] \right) , {\it quad}\left(\left[ 5 , 4 \right] 
  , \left[ 0 , 5 \right] \right) , {\it quad}\left(\left[ 2 , 3
  \right]  , \left[ 5 , 4 \right] \right)\right)}
\]
\end{eulerformula}
\begin{eulerprompt}
>$sqrt(ha)*sqrt(a)/2
\end{eulerprompt}
\begin{eulerformula}
\[
\frac{\sqrt{{\it quad}\left(\left[ 5 , 4 \right]  , \left[ 0 , 5
  \right] \right)}\,\sqrt{{\it quad}\left(\left[ 2 , 3 \right]  , 
 \left[ 5 , 4 \right] \right)\,{\it spread}\left({\it quad}\left(
 \left[ 2 , 3 \right]  , \left[ 0 , 5 \right] \right) , {\it quad}
 \left(\left[ 5 , 4 \right]  , \left[ 0 , 5 \right] \right) , 
 {\it quad}\left(\left[ 2 , 3 \right]  , \left[ 5 , 4 \right] \right)
 \right)}}{2}
\]
\end{eulerformula}
\begin{eulercomment}
\end{eulercomment}
\begin{eulerprompt}
>$areaTriangle(B,A,C)
\end{eulerprompt}
\begin{eulerformula}
\[
{\it areaTriangle}\left(\left[ 5 , 4 \right]  , \left[ 2 , 3
  \right]  , \left[ 0 , 5 \right] \right)
\]
\end{eulerformula}
\eulersubheading{Aturan penyebaran 3 kali lipat}
\begin{eulerprompt}
>setPlotRange(1); ...
>color(1); plotCircle(circleWithCenter([0,0],1)); ...
>A:=[cos(1),sin(1)]; B:=[cos(2),sin(2)]; C:=[cos(6),sin(6)]; ...
>plotPoint(A,"A"); plotPoint(B,"B"); plotPoint(C,"C"); ...
>color(3); plotSegment(A,B,"c"); plotSegment(A,C,"b"); plotSegment(C,B,"a"); ...
>color(1); O:=[0,0];  plotPoint(O,"0"); ...
>plotSegment(A,O); plotSegment(B,O); plotSegment(C,O,"r"); ...
>insimg;
\end{eulerprompt}
\begin{euleroutput}
  Function setPlotRange not found.
  Try list ... to find functions!
  Error in:
  setPlotRange(1); color(1); plotCircle(circleWithCenter([0,0],1 ...
                 ^
\end{euleroutput}
\begin{eulerprompt}
>&remvalue(a,b,c,r); // hapus nilai-nilai sebelumnya untuk perhitungan baru
>rabc &= rhs(solve(triplespread(spread(b,r,r),spread(a,r,r),spread(c,r,r)),r)[4]); $rabc
\end{eulerprompt}
\begin{euleroutput}
  Maxima said:
  part: invalid index of list or matrix.
   -- an error. To debug this try: debugmode(true);
  
  Error in:
  ... spread(b,r,r),spread(a,r,r),spread(c,r,r)),r)[4]); $rabc ...
                                                       ^
\end{euleroutput}
\begin{eulercomment}
\end{eulercomment}
\begin{eulerprompt}
>function periradius(a,b,c) &= rabc;
\end{eulerprompt}
\begin{eulercomment}
\end{eulercomment}
\begin{eulerprompt}
>a:=quadrance(B,C); b:=quadrance(A,C); c:=quadrance(A,B);
\end{eulerprompt}
\begin{euleroutput}
  Function quadrance not found.
  Try list ... to find functions!
  Error in:
  a:=quadrance(B,C); b:=quadrance(A,C); c:=quadrance(A,B); ...
                   ^
\end{euleroutput}
\begin{eulercomment}
\end{eulercomment}
\begin{eulerprompt}
>periradius(a,b,c)
\end{eulerprompt}
\begin{euleroutput}
  Variable rabc not found!
  Use global or local variables defined in function periradius.
  Try "trace errors" to inspect local variables after errors.
  periradius:
      useglobal; return rabc 
  Error in:
  periradius(a,b,c) ...
                   ^
\end{euleroutput}
\begin{eulerprompt}
>$spread(b,a,c)*rabc | ratsimp
\end{eulerprompt}
\begin{eulerformula}
\[
{\it spread}\left(b , a , c\right)\,{\it rabc}
\]
\end{eulerformula}
\begin{eulerprompt}
>$doublespread(b/(4*r))-spread(b,r,r) | ratsimp
\end{eulerprompt}
\begin{eulerformula}
\[
{\it doublespread}\left(\frac{b}{4\,r}\right)-{\it spread}\left(b
  , r , r\right)
\]
\end{eulerformula}
\eulersubheading{Contoh 6: Jarak Minimal pada Bidang}
\begin{eulercomment}
\end{eulercomment}
\eulersubheading{Catatan awal}
\begin{eulercomment}
Fungsi yang, ke titik M di bidang, menetapkan jarak AM antara titik
tetap A dan M, memiliki garis level yang agak sederhana: lingkaran
berpusat di A.
\end{eulercomment}
\begin{eulerprompt}
>&remvalue();
>A=[-2,-2];
>function d1(x,y):=sqrt((x-A[1])^2+(y-A[2])^2)
>fcontour("d1",xmin=-2,xmax=0,ymin=-2,ymax=0,hue=1, ...
>title="If you see ellipses, please set your window square"):
\end{eulerprompt}
\eulerimg{27}{images/Vikram Zaky Ardianto_22305144028_Geometri-170.png}
\begin{eulercomment}
dan grafiknya juga agak sederhana: bagian atas kerucut:
\end{eulercomment}
\begin{eulerprompt}
>plot3d("d1",xmin=-2,xmax=0,ymin=-2,ymax=0):
\end{eulerprompt}
\eulerimg{27}{images/Vikram Zaky Ardianto_22305144028_Geometri-171.png}
\begin{eulercomment}
Ternyata setelah mencoba yang bisa hanya dengan memasukkan angka 1,
karena ketika memakai angka 2, plot tidak membentuk kerucut diatas.

\end{eulercomment}
\eulersubheading{Dua poin}
\begin{eulercomment}
\end{eulercomment}
\begin{eulerprompt}
>B=[2,-2];
>function d2(x,y):=d1(x,y)+sqrt((x-B[1])^2+(y-B[2])^2)
>fcontour("d2",xmin=-2,xmax=2,ymin=-3,ymax=1,hue=1):
\end{eulerprompt}
\eulerimg{27}{images/Vikram Zaky Ardianto_22305144028_Geometri-172.png}
\begin{eulercomment}
Grafiknya lebih menarik:
\end{eulercomment}
\begin{eulerprompt}
>plot3d("d2",xmin=-2,xmax=2,ymin=-3,ymax=1):
\end{eulerprompt}
\eulerimg{27}{images/Vikram Zaky Ardianto_22305144028_Geometri-173.png}
\begin{eulercomment}
Pembatasan garis (AB) lebih terkenal:
\end{eulercomment}
\begin{eulerprompt}
>plot2d("abs(x+1)+abs(x-1)",xmin=-3,xmax=3):
\end{eulerprompt}
\eulerimg{27}{images/Vikram Zaky Ardianto_22305144028_Geometri-174.png}
\eulersubheading{Tiga poin}
\begin{eulercomment}
Contoh:
\end{eulercomment}
\begin{eulerprompt}
>C=[-3,2];
>function d3(x,y):=d2(x,y)+sqrt((x-C[1])^2+(y-C[2])^2)
>plot3d("d3",xmin=-5,xmax=3,ymin=-4,ymax=4);
>insimg;
\end{eulerprompt}
\eulerimg{27}{images/Vikram Zaky Ardianto_22305144028_Geometri-175.png}
\begin{eulerprompt}
>fcontour("d3",xmin=-4,xmax=1,ymin=-2,ymax=2,hue=1,title="The minimum is on A");
>P=(A_B_C_A)'; plot2d(P[1],P[2],add=1,color=12);
>insimg;
\end{eulerprompt}
\eulerimg{27}{images/Vikram Zaky Ardianto_22305144028_Geometri-176.png}
\begin{eulercomment}
Tetapi jika semua sudut segitiga ABC kurang dari 120 °, minimumnya
adalah pada titik F di bagian dalam segitiga, yang merupakan
satu-satunya titik yang melihat sisi-sisi ABC dengan sudut yang sama
(maka masing-masing 120 ° ):
\end{eulercomment}
\begin{eulerprompt}
>C=[-1,2];
>plot3d("d3",xmin=-2,xmax=2,ymin=-2,ymax=2):
\end{eulerprompt}
\eulerimg{27}{images/Vikram Zaky Ardianto_22305144028_Geometri-177.png}
\begin{eulerprompt}
>fcontour("d3",xmin=-2,xmax=2,ymin=-2,ymax=2,hue=1,title="The Fermat point");
>P=(A_B_C_A)'; plot2d(P[1],P[2],add=1,color=12);
>insimg;
\end{eulerprompt}
\eulerimg{27}{images/Vikram Zaky Ardianto_22305144028_Geometri-178.png}
\begin{eulercomment}
\end{eulercomment}
\eulersubheading{Empat poin}
\begin{eulercomment}
Langkah selanjutnya adalah menambahkan 4 titik D dan mencoba
meminimalkan MA+MB+MC+MD; katakan bahwa Anda adalah operator TV kabel
dan ingin mencari di bidang mana Anda harus meletakkan antena sehingga
Anda dapat memberi makan empat desa dan menggunakan panjang kabel
sesedikit mungkin!
\end{eulercomment}
\begin{eulerprompt}
>D=[2,21];
>function d4(x,y):=d3(x,y)+sqrt((x-D[1])^2+(y-D[2])^2)
>plot3d("d4",xmin=-1.5,xmax=1.5,ymin=-1.5,ymax=1.5):
\end{eulerprompt}
\eulerimg{27}{images/Vikram Zaky Ardianto_22305144028_Geometri-179.png}
\begin{eulerprompt}
>fcontour("d4",xmin=-1.5,xmax=1.5,ymin=-1.5,ymax=1.5,hue=1);
>P=(A_B_C_D)'; plot2d(P[1],P[2],points=1,add=1,color=12);
>insimg;
\end{eulerprompt}
\eulerimg{27}{images/Vikram Zaky Ardianto_22305144028_Geometri-180.png}
\eulersubheading{Contoh 7: Bola Dandelin dengan Povray}
\begin{eulercomment}
\end{eulercomment}
\begin{eulerprompt}
>load geometry;
\end{eulerprompt}
\begin{eulercomment}
Pertama dua garis yang membentuk kerucut.
\end{eulercomment}
\begin{eulerprompt}
>g1 &= lineThrough([0,0],[2,a])
\end{eulerprompt}
\begin{euleroutput}
  
                               [- a, 2, 0]
  
\end{euleroutput}
\begin{eulerprompt}
>g2 &= lineThrough([0,0],[-2,a])
\end{eulerprompt}
\begin{euleroutput}
  
                              [- a, - 2, 0]
  
\end{euleroutput}
\begin{eulercomment}
\end{eulercomment}
\begin{eulerprompt}
>g &= lineThrough([-2,0],[2,2])
\end{eulerprompt}
\begin{euleroutput}
  
                               [- 2, 4, 4]
  
\end{euleroutput}
\begin{eulercomment}
\end{eulercomment}
\begin{eulerprompt}
>setPlotRange(-2,2,0,3);
>color(black); plotLine(g(),"")
>a:=2; color(blue); plotLine(g1(),""), plotLine(g2(),""):
\end{eulerprompt}
\eulerimg{27}{images/Vikram Zaky Ardianto_22305144028_Geometri-181.png}
\begin{eulercomment}
Sekarang kita ambil titik umum pada sumbu y.
\end{eulercomment}
\begin{eulerprompt}
>P &= [0,u]
\end{eulerprompt}
\begin{euleroutput}
  
                                  [0, u]
  
\end{euleroutput}
\begin{eulercomment}
Hitung jarak ke g1.
\end{eulercomment}
\begin{eulerprompt}
>d1 &= distance(P,projectToLine(P,g1)); $d1
\end{eulerprompt}
\begin{eulerformula}
\[
\sqrt{\left(\frac{a^2\,u}{a^2+4}-u\right)^2+\frac{4\,a^2\,u^2}{
 \left(a^2+4\right)^2}}
\]
\end{eulerformula}
\begin{eulercomment}
Hitung jarak ke g.
\end{eulercomment}
\begin{eulerprompt}
>d &= distance(P,projectToLine(P,g)); $d
\end{eulerprompt}
\begin{eulerformula}
\[
\sqrt{\left(\frac{u+4}{5}-u\right)^2+\frac{\left(2\,u-2\right)^2}{
 25}}
\]
\end{eulerformula}
\begin{eulercomment}
Dan temukan pusat kedua lingkaran yang jaraknya sama.
\end{eulercomment}
\begin{eulerprompt}
>sol &= solve(d1^2=d^2,u); $sol
\end{eulerprompt}
\begin{eulerformula}
\[
\left[ u=\frac{-\sqrt{5}\,\sqrt{a^2+4}+a^2+4}{a^2-1} , u=\frac{
 \sqrt{5}\,\sqrt{a^2+4}+a^2+4}{a^2-1} \right] 
\]
\end{eulerformula}
\begin{eulercomment}
Ada dua solusi.

\end{eulercomment}
\begin{eulerprompt}
>u := sol()
\end{eulerprompt}
\begin{euleroutput}
  [0.558482,  4.77485]
\end{euleroutput}
\begin{eulerprompt}
>dd := d()
\end{eulerprompt}
\begin{euleroutput}
  [0.394906,  3.37633]
\end{euleroutput}
\begin{eulercomment}
Plot lingkaran ke dalam gambar.
\end{eulercomment}
\begin{eulerprompt}
>color(red);
>plotCircle(circleWithCenter([0,u[1]],dd[1]),"");
>plotCircle(circleWithCenter([0,u[2]],dd[2]),"");
>insimg;
\end{eulerprompt}
\eulerimg{27}{images/Vikram Zaky Ardianto_22305144028_Geometri-185.png}
\eulersubheading{Latihan}
\begin{eulercomment}
1. Gambarlah segi-n beraturan jika diketahui titik pusat O, n, dan
jarak titik pusat ke titik-titik sudut segi-n tersebut (jari-jari
lingkaran luar segi-n), r.

Petunjuk:

- Besar sudut pusat yang menghadap masing-masing sisi segi-n adalah
(360/n).\\
- Titik-titik sudut segi-n merupakan perpotongan lingkaran luar segi-n
dan garis-garis yang melalui pusat dan saling membentuk sudut sebesar
kelipatan (360/n).\\
- Untuk n ganjil, pilih salah satu titik sudut adalah di atas.\\
- Untuk n genap, pilih 2 titik di kanan dan kiri lurus dengan titik
pusat.\\
- Anda dapat menggambar segi-3, 4, 5, 6, 7, dst beraturan.

Penyelesaian :
\end{eulercomment}
\begin{eulerprompt}
>load geometry
\end{eulerprompt}
\begin{euleroutput}
  Numerical and symbolic geometry.
\end{euleroutput}
\begin{eulerprompt}
>setPlotRange(-3.5,3.5,-3.5,3.5);
>A=[-2,-2]; plotPoint(A,"A");
>B=[2,-2]; plotPoint(B,"B");
>C=[0,3]; plotPoint(C,"C");
>plotSegment(A,B,"c");
>plotSegment(B,C,"a");
>plotSegment(A,C,"b");
>aspect(1):
\end{eulerprompt}
\eulerimg{27}{images/Vikram Zaky Ardianto_22305144028_Geometri-186.png}
\begin{eulerprompt}
>c=circleThrough(A,B,C);
>R=getCircleRadius(c);
>O=getCircleCenter(c);
>plotPoint(O,"O");
>l=angleBisector(A,C,B);
>color(2); plotLine(l); color(1);
>plotCircle(c,"Lingkaran luar segitiga ABC"):
\end{eulerprompt}
\eulerimg{27}{images/Vikram Zaky Ardianto_22305144028_Geometri-187.png}
\begin{eulercomment}
2. Gambarlah suatu parabola yang melalui 3 titik yang diketahui.

Petunjuk:\\
- Misalkan persamaan parabolanya y= ax\textasciicircum{}2+bx+c.\\
- Substitusikan koordinat titik-titik yang diketahui ke persamaan
tersebut.\\
- Selesaikan SPL yang terbentuk untuk mendapatkan nilai-nilai a, b, c.

Penyelesaian :
\end{eulercomment}
\begin{eulerprompt}
>load geometry;
>setPlotRange(5); P=[2,0]; Q=[4,0]; R=[0,-4];
>plotPoint(P,"P"); plotPoint(Q,"Q"); plotPoint(R,"R"):
\end{eulerprompt}
\eulerimg{27}{images/Vikram Zaky Ardianto_22305144028_Geometri-188.png}
\begin{eulerprompt}
>sol &= solve([a+b=-c,16*a+4*b=-c,c=-4],[a,b,c])
\end{eulerprompt}
\begin{euleroutput}
  
                       [[a = - 1, b = 5, c = - 4]]
  
\end{euleroutput}
\begin{eulercomment}
Sehingga didapatkan nilai a = -1, b = 5 dan c = -4
\end{eulercomment}
\begin{eulerprompt}
>function y&=-x^2+5*x-4
\end{eulerprompt}
\begin{euleroutput}
  
                                 2
                              - x  + 5 x - 4
  
\end{euleroutput}
\begin{eulerprompt}
>plot2d("-x^2+5*x-4",-5,5,-5,5):
\end{eulerprompt}
\eulerimg{27}{images/Vikram Zaky Ardianto_22305144028_Geometri-189.png}
\begin{eulercomment}
3. Gambarlah suatu segi-4 yang diketahui keempat titik sudutnya,
misalnya A, B, C, D.\\
\end{eulercomment}
\begin{eulerttcomment}
   - Tentukan apakah segi-4 tersebut merupakan segi-4 garis singgung
\end{eulerttcomment}
\begin{eulercomment}
(sisinya-sisintya merupakan garis singgung lingkaran yang sama yakni
lingkaran dalam segi-4 tersebut).\\
\end{eulercomment}
\begin{eulerttcomment}
   - Suatu segi-4 merupakan segi-4 garis singgung apabila keempat
\end{eulerttcomment}
\begin{eulercomment}
garis bagi sudutnya bertemu di satu titik.\\
\end{eulercomment}
\begin{eulerttcomment}
   - Jika segi-4 tersebut merupakan segi-4 garis singgung, gambar
\end{eulerttcomment}
\begin{eulercomment}
lingkaran dalamnya.\\
\end{eulercomment}
\begin{eulerttcomment}
   - Tunjukkan bahwa syarat suatu segi-4 merupakan segi-4 garis
\end{eulerttcomment}
\begin{eulercomment}
singgung apabila hasil kali panjang sisi-sisi yang berhadapan sama.

Penyelesaian :
\end{eulercomment}
\begin{eulerprompt}
>load geometry
\end{eulerprompt}
\begin{euleroutput}
  Numerical and symbolic geometry.
\end{euleroutput}
\begin{eulerprompt}
>setPlotRange(-4.5,4.5,-4.5,4.5);
>A=[-3,-3]; plotPoint(A,"A");
>B=[3,-3]; plotPoint(B,"B");
>C=[3,3]; plotPoint(C,"C");
>D=[-3,3]; plotPoint(D,"D");
>plotSegment(A,B,"");
>plotSegment(B,C,"");
>plotSegment(C,D,"");
>plotSegment(A,D,"");
>aspect(1):
\end{eulerprompt}
\eulerimg{27}{images/Vikram Zaky Ardianto_22305144028_Geometri-190.png}
\begin{eulerprompt}
>l=angleBisector(A,B,C);
>m=angleBisector(B,C,D);
>P=lineIntersection(l,m);
>color(5); plotLine(l); plotLine(m); color(1);
>plotPoint(P,"P"):
\end{eulerprompt}
\eulerimg{27}{images/Vikram Zaky Ardianto_22305144028_Geometri-191.png}
\begin{eulercomment}
Dari gambar diatas terlihat bahwa keempat garis bagi sudutnya bertemu
di satu titik yaitu titik P.
\end{eulercomment}
\begin{eulerprompt}
>r=norm(P-projectToLine(P,lineThrough(A,B)));
>plotCircle(circleWithCenter(P,r),"Lingkaran dalam segiempat ABCD"):
\end{eulerprompt}
\eulerimg{27}{images/Vikram Zaky Ardianto_22305144028_Geometri-192.png}
\begin{eulercomment}
Dari gambar diatas, terlihat bahwa sisi-sisinya merupakan garis
singgung lingkaran yang sama yaitu lingkaran dalam segiempat.\\
Akan ditunjukkan bahwa hasil kali panjang sisi-sisi yang berhadapan
sama.
\end{eulercomment}
\begin{eulerprompt}
>AB=norm(A-B) //panjang sisi AB
\end{eulerprompt}
\begin{euleroutput}
  6
\end{euleroutput}
\begin{eulerprompt}
>CD=norm(C-D) //panjang sisi CD
\end{eulerprompt}
\begin{euleroutput}
  6
\end{euleroutput}
\begin{eulerprompt}
>AD=norm(A-D) //panjang sisi AD
\end{eulerprompt}
\begin{euleroutput}
  6
\end{euleroutput}
\begin{eulerprompt}
>BC=norm(B-C) //panjang sisi BC
\end{eulerprompt}
\begin{euleroutput}
  6
\end{euleroutput}
\begin{eulerprompt}
>AB.CD
\end{eulerprompt}
\begin{euleroutput}
  36
\end{euleroutput}
\begin{eulerprompt}
>AD.BC
\end{eulerprompt}
\begin{euleroutput}
  36
\end{euleroutput}
\begin{eulercomment}
Terbukti bahwa hasil kali panjang sisi-sisi yang berhadapan sama yaitu
36. Jadi dapat dipastikan bahwa segiempat tersebut merupakan segiempat
garis singgung.


4. Gambarlah suatu ellips jika diketahui kedua titik fokusnya,
misalnya P dan Q. Ingat ellips dengan fokus P dan Q adalah tempat
kedudukan titik-titik yang jumlah jarak ke P dan ke Q selalu sama
(konstan).

Penyelesaian :\\
Diketahui kedua titik fokus P = [-1,-1] dan Q = [1,-1]
\end{eulercomment}
\begin{eulerprompt}
>P=[-1,-1]; Q=[1,-1];
>function d1(x,y):=sqrt((x-P[1])^2+(y-P[2])^2)
>Q=[1,-1]; function d2(x,y):=sqrt((x-P[1])^2+(y-P[2])^2)+sqrt((x-Q[1])^2+(y-Q[2])^2)
>fcontour("d2",xmin=-2,xmax=2,ymin=-3,ymax=1,hue=1):
\end{eulerprompt}
\eulerimg{27}{images/Vikram Zaky Ardianto_22305144028_Geometri-193.png}
\begin{eulercomment}
Grafik yang lebih menarik
\end{eulercomment}
\begin{eulerprompt}
>plot3d("d2",xmin=-2,xmax=2,ymin=-3,ymax=1):
\end{eulerprompt}
\eulerimg{27}{images/Vikram Zaky Ardianto_22305144028_Geometri-194.png}
\begin{eulercomment}
Batasan ke garis PQ
\end{eulercomment}
\begin{eulerprompt}
>plot2d("abs(x+1)+abs(x-1)",xmin=-3,xmax=3):
\end{eulerprompt}
\eulerimg{27}{images/Vikram Zaky Ardianto_22305144028_Geometri-195.png}
\begin{eulercomment}
5. Gambarlah suatu hiperbola jika diketahui kedua titik fokusnya,
misalnya P dan Q. Ingat ellips dengan fokus P dan Q adalah tempat
kedudukan titik-titik yang selisih jarak ke P dan ke Q selalu sama
(konstan).

Penyelesaian :
\end{eulercomment}
\begin{eulerprompt}
>P=[-1,-1]; Q=[1,-1];
>function d1(x,y):=sqrt((x-p[1])^2+(y-p[2])^2)
>Q=[1,-1]; function d2(x,y):=sqrt((x-P[1])^2+(y-P[2])^2)+sqrt((x+Q[1])^2+(y+Q[2])^2)
>fcontour("d2",xmin=-2,xmax=2,ymin=-3,ymax=1,hue=1):
\end{eulerprompt}
\eulerimg{27}{images/Vikram Zaky Ardianto_22305144028_Geometri-196.png}
\begin{eulercomment}
Grafik yang lebih menarik
\end{eulercomment}
\begin{eulerprompt}
>plot3d("d2",xmin=-2,xmax=2,ymin=-3,ymax=1):
\end{eulerprompt}
\eulerimg{27}{images/Vikram Zaky Ardianto_22305144028_Geometri-197.png}
\begin{eulerprompt}
>plot2d("abs(x+1)+abs(x-1)",xmin=-3,xmax=3):
\end{eulerprompt}
\eulerimg{27}{images/Vikram Zaky Ardianto_22305144028_Geometri-198.png}
\end{eulernotebook}
\end{document}


\newpage
\chapter{KB Pekan 10; Menggunakan EMT untuk Statistika}
\documentclass{article}

\usepackage{eumat}

\begin{document}
\begin{eulernotebook}
\begin{eulercomment}
Nama : Vikram Zaky Ardianto\\
Kelas: Matematika E 2022\\
NIM  : 22305144028\\
\begin{eulercomment}
\eulerheading{EMT untuk Statistika}
\begin{eulercomment}
Dalam buku catatan ini, kami mendemonstrasikan plot statistik utama,
pengujian, dan distribusi di Euler.

Mari kita mulai dengan beberapa statistik deskriptif. Ini bukan
pengantar statistik. Jadi, Anda mungkin memerlukan beberapa latar
belakang untuk memahami detailnya.

Asumsikan pengukuran berikut. Kami ingin menghitung nilai rata-rata
dan standar deviasi yang diukur.
\end{eulercomment}
\begin{eulerprompt}
>M=[1000,1004,998,997,1002,1001,998,1004,998,997]; ...
>mean(M), dev(M),
\end{eulerprompt}
\begin{euleroutput}
  999.9
  2.72641400622
\end{euleroutput}
\begin{eulercomment}
Kita dapat memplot plot kotak-dan-kumis untuk data. Dalam kasus kami
tidak ada outlier.
\end{eulercomment}
\begin{eulerprompt}
>boxplot(M):
\end{eulerprompt}
\eulerimg{27}{images/Vikram Zaky Ardianto_22305144028_EMT STATISTIKA-001.png}
\begin{eulercomment}
Kami menghitung probabilitas bahwa suatu nilai lebih besar dari 1005,
dengan asumsi nilai terukur dan distribusi normal.

Semua fungsi untuk distribusi di Euler diakhiri dengan ...dis dan
menghitung distribusi probabilitas kumulatif (CPF).

lateks: \textbackslash{}text\{normaldis(x,m,d)\}=\textbackslash{}int\_\{-\textbackslash{}infty\}\textasciicircum{}x
\textbackslash{}frac\{1\}\{d\textbackslash{}sqrt\{2\textbackslash{}pi\}\}e\textasciicircum{}\{-\textbackslash{}frac\{1\}\{2 \}(\textbackslash{}frac\{t-m\}\{d\})\textasciicircum{}2\}\textbackslash{} dt.

Kami mencetak hasilnya dalam \% dengan akurasi 2 digit menggunakan
fungsi cetak.
\end{eulercomment}
\begin{eulerprompt}
>print((1-normaldis(1005,mean(M),dev(M)))*100,2,unit=" %")
\end{eulerprompt}
\begin{euleroutput}
        3.07 %
\end{euleroutput}
\begin{eulercomment}
Untuk contoh berikutnya, kami mengasumsikan jumlah pria berikut dalam
rentang ukuran yang diberikan.
\end{eulercomment}
\begin{eulerprompt}
>r=155.5:4:187.5; v=[22,71,136,169,139,71,32,8];
\end{eulerprompt}
\begin{eulercomment}
Berikut adalah plot distribusinya.
\end{eulercomment}
\begin{eulerprompt}
>plot2d(r,v,a=150,b=200,c=0,d=190,bar=1,style="\(\backslash\)/"):
\end{eulerprompt}
\eulerimg{27}{images/Vikram Zaky Ardianto_22305144028_EMT STATISTIKA-002.png}
\begin{eulercomment}
Kita bisa memasukkan data mentah tersebut ke dalam sebuah tabel.

Tabel adalah metode untuk menyimpan data statistik. Tabel kita harus
berisi tiga kolom: Awal jangkauan, akhir jangkauan, jumlah orang dalam
jangkauan.

Tabel dapat dicetak dengan header. Kami menggunakan vektor string
untuk mengatur header.
\end{eulercomment}
\begin{eulerprompt}
>T:=r[1:8]' | r[2:9]' | v'; writetable(T,labc=["from","to","count"])
\end{eulerprompt}
\begin{euleroutput}
        from        to     count
       155.5     159.5        22
       159.5     163.5        71
       163.5     167.5       136
       167.5     171.5       169
       171.5     175.5       139
       175.5     179.5        71
       179.5     183.5        32
       183.5     187.5         8
\end{euleroutput}
\begin{eulercomment}
Jika kita membutuhkan nilai rata-rata dan statistik lain dari ukuran,
kita perlu menghitung titik tengah rentang. Kita dapat menggunakan dua
kolom pertama dari tabel kita untuk ini.

Sumbul "\textbar{}" digunakan untuk memisahkan kolom, fungsi "writetable"
digunakan untuk menulis tabel, dengan opsion "labc" adalah untuk
menentukan header kolom.
\end{eulercomment}
\begin{eulerprompt}
>(T[,1]+T[,2])/2 // the midpoint of each interval
\end{eulerprompt}
\begin{euleroutput}
          157.5 
          161.5 
          165.5 
          169.5 
          173.5 
          177.5 
          181.5 
          185.5 
\end{euleroutput}
\begin{eulercomment}
Tetapi lebih mudah, untuk melipat rentang dengan vektor [1/2.1/2].
\end{eulercomment}
\begin{eulerprompt}
>M=fold(r,[0.5,0.5])
\end{eulerprompt}
\begin{euleroutput}
  [157.5,  161.5,  165.5,  169.5,  173.5,  177.5,  181.5,  185.5]
\end{euleroutput}
\begin{eulercomment}
Sekarang kita dapat menghitung mean dan deviasi sampel dengan
frekuensi yang diberikan.
\end{eulercomment}
\begin{eulerprompt}
>\{m,d\}=meandev(M,v); m, d,
\end{eulerprompt}
\begin{euleroutput}
  169.901234568
  5.98912964449
\end{euleroutput}
\begin{eulercomment}
Mari kita tambahkan distribusi normal dari nilai-nilai ke plot batang
di atas. Rumus untuk distribusi normal dengan mean m dan standar
deviasi d adalah:

lateks: y=\textbackslash{}frac\{1\}\{d\textbackslash{}sqrt\{2\textbackslash{}pi\}\}e\textasciicircum{}\{\textbackslash{}frac\{-(x-m)\textasciicircum{}2\}\{2d\textasciicircum{}2\}\}.

Karena nilainya antara 0 dan 1, untuk memplotnya pada bar plot harus
dikalikan dengan 4 kali jumlah total data.
\end{eulercomment}
\begin{eulerprompt}
>plot2d("qnormal(x,m,d)*sum(v)*4", ...
>  xmin=min(r),xmax=max(r),thickness=3,add=1):
\end{eulerprompt}
\eulerimg{27}{images/Vikram Zaky Ardianto_22305144028_EMT STATISTIKA-003.png}
\eulerheading{Meja}
\begin{eulercomment}
Di direktori notebook ini Anda menemukan file dengan tabel. Data
tersebut mewakili hasil survei. Berikut adalah empat baris pertama
dari file tersebut. Data berasal dari buku online Jerman "Einführung
in die Statistik mit R" oleh A. Handl.
\end{eulercomment}
\begin{eulerprompt}
>printfile("table.dat",4);
\end{eulerprompt}
\begin{euleroutput}
  Person Sex Age Titanic Evaluation Tip Problem
  1 m 30 n . 1.80 n
  2 f 23 y g 1.80 n
  3 f 26 y g 1.80 y
\end{euleroutput}
\begin{eulercomment}
Tabel berisi 7 kolom angka atau token (string). Kami ingin membaca
tabel dari file. Pertama, kami menggunakan terjemahan kami sendiri
untuk token.

Untuk ini, kami mendefinisikan set token. Fungsi strtokens()
mendapatkan vektor string token dari string yang diberikan.
\end{eulercomment}
\begin{eulerprompt}
>mf:=["m","f"]; yn:=["y","n"]; ev:=strtokens("g vg m b vb");
\end{eulerprompt}
\begin{eulercomment}
Sekarang kita membaca tabel dengan terjemahan ini.

Argumen tok2, tok4 dll. adalah terjemahan dari kolom tabel. Argumen
ini tidak ada dalam daftar parameter readtable(), jadi Anda harus
menyediakannya dengan ":=".
\end{eulercomment}
\begin{eulerprompt}
>\{MT,hd\}=readtable("table.dat",tok2:=mf,tok4:=yn,tok5:=ev,tok7:=yn);
>load over statistics;
\end{eulerprompt}
\begin{eulercomment}
Untuk mencetak, kita perlu menentukan set token yang sama. Kami
mencetak empat baris pertama saja.
\end{eulercomment}
\begin{eulerprompt}
>writetable(MT[1:4],labc=hd,wc=5,tok2:=mf,tok4:=yn,tok5:=ev,tok7:=yn);
\end{eulerprompt}
\begin{euleroutput}
   Person  Sex  Age Titanic Evaluation  Tip Problem
        1    m   30       n          .  1.8       n
        2    f   23       y          g  1.8       n
        3    f   26       y          g  1.8       y
        4    m   33       n          .  2.8       n
\end{euleroutput}
\begin{eulercomment}
Titik "." mewakili nilai-nilai, yang tidak tersedia.

Jika kita tidak ingin menentukan token untuk terjemahan terlebih
dahulu, kita hanya perlu menentukan, kolom mana yang berisi token dan
bukan angka.
\end{eulercomment}
\begin{eulerprompt}
>ctok=[2,4,5,7]; \{MT,hd,tok\}=readtable("table.dat",ctok=ctok);
\end{eulerprompt}
\begin{eulercomment}
Fungsi readtable() sekarang mengembalikan satu set token.
\end{eulercomment}
\begin{eulerprompt}
>tok
\end{eulerprompt}
\begin{euleroutput}
  m
  n
  f
  y
  g
  vg
\end{euleroutput}
\begin{eulercomment}
Tabel berisi entri dari file dengan token yang diterjemahkan ke angka.

String khusus NA="." ditafsirkan sebagai "Tidak Tersedia", dan
mendapatkan NAN (bukan angka) dalam tabel. Terjemahan ini dapat diubah
dengan parameter NA, dan NAval.
\end{eulercomment}
\begin{eulerprompt}
>MT[1]
\end{eulerprompt}
\begin{euleroutput}
  [1,  1,  30,  2,  NAN,  1.8,  2]
\end{euleroutput}
\begin{eulercomment}
Berikut isi tabel dengan angka yang belum diterjemahkan.
\end{eulercomment}
\begin{eulerprompt}
>writetable(MT,wc=5)
\end{eulerprompt}
\begin{euleroutput}
      1    1   30    2    .  1.8    2
      2    3   23    4    5  1.8    2
      3    3   26    4    5  1.8    4
      4    1   33    2    .  2.8    2
      5    1   37    2    .  1.8    2
      6    1   28    4    5  2.8    4
      7    3   31    4    6  2.8    2
      8    1   23    2    .  0.8    2
      9    3   24    4    6  1.8    4
     10    1   26    2    .  1.8    2
     11    3   23    4    6  1.8    4
     12    1   32    4    5  1.8    2
     13    1   29    4    6  1.8    4
     14    3   25    4    5  1.8    4
     15    3   31    4    5  0.8    2
     16    1   26    4    5  2.8    2
     17    1   37    2    .  3.8    2
     18    1   38    4    5    .    2
     19    3   29    2    .  3.8    2
     20    3   28    4    6  1.8    2
     21    3   28    4    1  2.8    4
     22    3   28    4    6  1.8    4
     23    3   38    4    5  2.8    2
     24    3   27    4    1  1.8    4
     25    1   27    2    .  2.8    4
\end{euleroutput}
\begin{eulercomment}
Untuk kenyamanan, Anda dapat memasukkan output readtable() ke dalam
daftar.
\end{eulercomment}
\begin{eulerprompt}
>Table=\{\{readtable("table.dat",ctok=ctok)\}\};
\end{eulerprompt}
\begin{eulercomment}
Menggunakan kolom token yang sama dan token yang dibaca dari file,
kita dapat mencetak tabel. Kita dapat menentukan ctok, tok, dll. Atau
menggunakan daftar Tabel.
\end{eulercomment}
\begin{eulerprompt}
>writetable(Table,ctok=ctok,wc=5);
\end{eulerprompt}
\begin{euleroutput}
   Person  Sex  Age Titanic Evaluation  Tip Problem
        1    m   30       n          .  1.8       n
        2    f   23       y          g  1.8       n
        3    f   26       y          g  1.8       y
        4    m   33       n          .  2.8       n
        5    m   37       n          .  1.8       n
        6    m   28       y          g  2.8       y
        7    f   31       y         vg  2.8       n
        8    m   23       n          .  0.8       n
        9    f   24       y         vg  1.8       y
       10    m   26       n          .  1.8       n
       11    f   23       y         vg  1.8       y
       12    m   32       y          g  1.8       n
       13    m   29       y         vg  1.8       y
       14    f   25       y          g  1.8       y
       15    f   31       y          g  0.8       n
       16    m   26       y          g  2.8       n
       17    m   37       n          .  3.8       n
       18    m   38       y          g    .       n
       19    f   29       n          .  3.8       n
       20    f   28       y         vg  1.8       n
       21    f   28       y          m  2.8       y
       22    f   28       y         vg  1.8       y
       23    f   38       y          g  2.8       n
       24    f   27       y          m  1.8       y
       25    m   27       n          .  2.8       y
\end{euleroutput}
\begin{eulercomment}
Fungsi tablecol() mengembalikan nilai kolom tabel, melewatkan setiap
baris dengan nilai NAN("." dalam file), dan indeks kolom, yang berisi
nilai-nilai ini.
\end{eulercomment}
\begin{eulerprompt}
>\{c,i\}=tablecol(MT,[5,6]);
\end{eulerprompt}
\begin{eulercomment}
Kita dapat menggunakan ini untuk mengekstrak kolom dari tabel untuk
tabel baru.
\end{eulercomment}
\begin{eulerprompt}
>j=[1,5,6]; writetable(MT[i,j],labc=hd[j],ctok=[2],tok=tok)
\end{eulerprompt}
\begin{euleroutput}
      Person Evaluation       Tip
           2          g       1.8
           3          g       1.8
           6          g       2.8
           7         vg       2.8
           9         vg       1.8
          11         vg       1.8
          12          g       1.8
          13         vg       1.8
          14          g       1.8
          15          g       0.8
          16          g       2.8
          20         vg       1.8
          21          m       2.8
          22         vg       1.8
          23          g       2.8
          24          m       1.8
\end{euleroutput}
\begin{eulercomment}
Tentu saja, kita perlu mengekstrak tabel itu sendiri dari daftar Tabel
dalam kasus ini.
\end{eulercomment}
\begin{eulerprompt}
>MT=Table[1];
\end{eulerprompt}
\begin{eulercomment}
Tentu saja, kita juga dapat menggunakannya untuk menentukan nilai
rata-rata kolom atau nilai statistik lainnya.
\end{eulercomment}
\begin{eulerprompt}
>mean(tablecol(MT,6))
\end{eulerprompt}
\begin{euleroutput}
  2.175
\end{euleroutput}
\begin{eulercomment}
Fungsi getstatistics() mengembalikan elemen dalam vektor, dan
jumlahnya. Kami menerapkannya pada nilai "m" dan "f" di kolom kedua
tabel kami.
\end{eulercomment}
\begin{eulerprompt}
>\{xu,count\}=getstatistics(tablecol(MT,2)); xu, count,
\end{eulerprompt}
\begin{euleroutput}
  [1,  3]
  [12,  13]
\end{euleroutput}
\begin{eulercomment}
Kami dapat mencetak hasilnya dalam tabel baru.
\end{eulercomment}
\begin{eulerprompt}
>writetable(count',labr=tok[xu])
\end{eulerprompt}
\begin{euleroutput}
           m        12
           f        13
\end{euleroutput}
\begin{eulercomment}
Fungsi selecttable() mengembalikan tabel baru dengan nilai dalam satu
kolom yang dipilih dari vektor indeks. Pertama kita mencari indeks
dari dua nilai kita di tabel token.
\end{eulercomment}
\begin{eulerprompt}
>v:=indexof(tok,["g","vg"])
\end{eulerprompt}
\begin{euleroutput}
  [5,  6]
\end{euleroutput}
\begin{eulercomment}
Sekarang kita dapat memilih baris tabel, yang memiliki salah satu
nilai dalam v di baris ke-5.
\end{eulercomment}
\begin{eulerprompt}
>MT1:=MT[selectrows(MT,5,v)]; i:=sortedrows(MT1,5);
\end{eulerprompt}
\begin{eulercomment}
Sekarang kita dapat mencetak tabel, dengan nilai yang diekstrak dan
diurutkan di kolom ke-5.
\end{eulercomment}
\begin{eulerprompt}
>writetable(MT1[i],labc=hd,ctok=ctok,tok=tok,wc=7);
\end{eulerprompt}
\begin{euleroutput}
   Person    Sex    Age Titanic Evaluation    Tip Problem
        2      f     23       y          g    1.8       n
        3      f     26       y          g    1.8       y
        6      m     28       y          g    2.8       y
       18      m     38       y          g      .       n
       16      m     26       y          g    2.8       n
       15      f     31       y          g    0.8       n
       12      m     32       y          g    1.8       n
       23      f     38       y          g    2.8       n
       14      f     25       y          g    1.8       y
        9      f     24       y         vg    1.8       y
        7      f     31       y         vg    2.8       n
       20      f     28       y         vg    1.8       n
       22      f     28       y         vg    1.8       y
       13      m     29       y         vg    1.8       y
       11      f     23       y         vg    1.8       y
\end{euleroutput}
\begin{eulercomment}
Untuk statistik berikutnya, kami ingin menghubungkan dua kolom tabel.
Jadi kami mengekstrak kolom 2 dan 4 dan mengurutkan tabel.
\end{eulercomment}
\begin{eulerprompt}
>i=sortedrows(MT,[2,4]);  ...
>  writetable(tablecol(MT[i],[2,4])',ctok=[1,2],tok=tok)
\end{eulerprompt}
\begin{euleroutput}
           m         n
           m         n
           m         n
           m         n
           m         n
           m         n
           m         n
           m         y
           m         y
           m         y
           m         y
           m         y
           f         n
           f         y
           f         y
           f         y
           f         y
           f         y
           f         y
           f         y
           f         y
           f         y
           f         y
           f         y
           f         y
\end{euleroutput}
\begin{eulercomment}
Dengan getstatistics(), kita juga bisa menghubungkan hitungan dalam
dua kolom tabel satu sama lain.
\end{eulercomment}
\begin{eulerprompt}
>MT24=tablecol(MT,[2,4]); ...
>\{xu1,xu2,count\}=getstatistics(MT24[1],MT24[2]); ...
>writetable(count,labr=tok[xu1],labc=tok[xu2])
\end{eulerprompt}
\begin{euleroutput}
                     n         y
           m         7         5
           f         1        12
\end{euleroutput}
\begin{eulercomment}
Tabel dapat ditulis ke file.
\end{eulercomment}
\begin{eulerprompt}
>filename="test.dat"; ...
>writetable(count,labr=tok[xu1],labc=tok[xu2],file=filename);
\end{eulerprompt}
\begin{eulercomment}
Kemudian kita bisa membaca tabel dari file tersebut.
\end{eulercomment}
\begin{eulerprompt}
>\{MT2,hd,tok2,hdr\}=readtable(filename,>clabs,>rlabs); ...
>writetable(MT2,labr=hdr,labc=hd)
\end{eulerprompt}
\begin{euleroutput}
                     n         y
           m         7         5
           f         1        12
\end{euleroutput}
\begin{eulercomment}
Dan hapus file tersebut.
\end{eulercomment}
\begin{eulerprompt}
>fileremove(filename);
\end{eulerprompt}
\eulerheading{Distribusi}
\begin{eulercomment}
Dengan plot2d, terdapat metode yang sangat mudah untuk memplot sebaran
data eksperimen.
\end{eulercomment}
\begin{eulerprompt}
>p=normal(1,1000); //1000 random normal-distributed sample p
>plot2d(p,distribution=20,style="\(\backslash\)/"); // plot the random sample p
>plot2d("qnormal(x,0,1)",add=1): // add the standard normal distribution plot
\end{eulerprompt}
\eulerimg{27}{images/Vikram Zaky Ardianto_22305144028_EMT STATISTIKA-004.png}
\begin{eulercomment}
Harap perhatikan perbedaan antara plot batang (sampel) dan kurva
normal (distribusi nyata). Masukkan kembali tiga perintah untuk
melihat hasil pengambilan sampel lainnya.
\end{eulercomment}
\begin{eulercomment}
Berikut adalah perbandingan 10 simulasi dari 1000 nilai terdistribusi
normal menggunakan apa yang disebut plot kotak. Plot ini menunjukkan
median, kuartil 25\% dan 75\%, nilai minimal dan maksimal, dan outlier.
\end{eulercomment}
\begin{eulerprompt}
>p=normal(10,1000); boxplot(p):
\end{eulerprompt}
\eulerimg{27}{images/Vikram Zaky Ardianto_22305144028_EMT STATISTIKA-005.png}
\begin{eulercomment}
Untuk membangkitkan bilangan bulat acak, Euler memiliki intrarandom.
Mari kita simulasikan lemparan dadu dan plot distribusinya.

Kita menggunakan fungsi getmultiplicities(v,x), yang menghitung
seberapa sering elemen v muncul di x. Kemudian kita memplot hasilnya
menggunakan columnplot().
\end{eulercomment}
\begin{eulerprompt}
>k=intrandom(1,6000,6);  ...
>columnsplot(getmultiplicities(1:6,k));  ...
>ygrid(1000,color=red):
\end{eulerprompt}
\eulerimg{27}{images/Vikram Zaky Ardianto_22305144028_EMT STATISTIKA-006.png}
\begin{eulercomment}
Sementara intrandom(n,m,k) mengembalikan bilangan bulat yang
terdistribusi secara seragam dari 1 ke k, dimungkinkan untuk
menggunakan distribusi bilangan bulat lain yang diberikan dengan
randpint().

Dalam contoh berikut, probabilitas untuk 1,2,3 masing-masing adalah
0,4,0,1,0,5.
\end{eulercomment}
\begin{eulerprompt}
>randpint(1,1000,[0.4,0.1,0.5]); getmultiplicities(1:3,%)
\end{eulerprompt}
\begin{euleroutput}
  [378,  102,  520]
\end{euleroutput}
\begin{eulercomment}
Euler dapat menghasilkan nilai acak dari lebih banyak distribusi. Coba
lihat referensinya.

Misalnya, kami mencoba distribusi eksponensial. Variabel acak kontinu
X dikatakan memiliki distribusi eksponensial, jika PDF-nya diberikan
oleh\\
lateks: f\_X(x)=\textbackslash{}lambda e\textasciicircum{}\{-\textbackslash{}lambda x\},\textbackslash{}quad x\textgreater{}0,\textbackslash{}quad \textbackslash{}lambda\textgreater{}0,\\
dengan parameter\\
lateks: \textbackslash{}lambda=\textbackslash{}frac\{1\}\{\textbackslash{}mu\},\textbackslash{}quad \textbackslash{}mu \textbackslash{}text\{ adalah rata-rata, dan
dilambangkan dengan \} X \textbackslash{}sim \textbackslash{}text\{Eksponensial\}(\textbackslash{}lambda).
\end{eulercomment}
\begin{eulerprompt}
>plot2d(randexponential(1,1000,2),>distribution):
\end{eulerprompt}
\eulerimg{27}{images/Vikram Zaky Ardianto_22305144028_EMT STATISTIKA-007.png}
\begin{eulercomment}
Untuk banyak distribusi, Euler dapat menghitung fungsi distribusi dan
inversnya.
\end{eulercomment}
\begin{eulerprompt}
>plot2d("normaldis",-4,4): 
\end{eulerprompt}
\eulerimg{27}{images/Vikram Zaky Ardianto_22305144028_EMT STATISTIKA-008.png}
\begin{eulercomment}
Berikut ini adalah salah satu cara untuk memplot kuantil.
\end{eulercomment}
\begin{eulerprompt}
>plot2d("qnormal(x,1,1.5)",-4,6);  ...
>plot2d("qnormal(x,1,1.5)",a=2,b=5,>add,>filled):
\end{eulerprompt}
\eulerimg{27}{images/Vikram Zaky Ardianto_22305144028_EMT STATISTIKA-009.png}
\begin{eulercomment}
lateks: \textbackslash{}text\{normaldis(x,m,d)\}=\textbackslash{}int\_\{-\textbackslash{}infty\}\textasciicircum{}x
\textbackslash{}frac\{1\}\{d\textbackslash{}sqrt\{2\textbackslash{}pi\}\}e\textasciicircum{}\{-\textbackslash{}frac\{1\}\{2 \}(\textbackslash{}frac\{t-m\}\{d\})\textasciicircum{}2\}\textbackslash{} dt.\\
Probabilitas untuk berada di area hijau adalah sebagai berikut.
\end{eulercomment}
\begin{eulerprompt}
>normaldis(5,1,1.5)-normaldis(2,1,1.5)
\end{eulerprompt}
\begin{euleroutput}
  0.248662156979
\end{euleroutput}
\begin{eulercomment}
Ini dapat dihitung secara numerik dengan integral berikut.\\
lateks: \textbackslash{}int\_2\textasciicircum{}5
\textbackslash{}frac\{1\}\{1.5\textbackslash{}sqrt\{2\textbackslash{}pi\}\}e\textasciicircum{}\{-\textbackslash{}frac\{1\}\{2\}(\textbackslash{}frac\{x-1\}\{1.5\})\textasciicircum{}2\}\textbackslash{} dx .
\end{eulercomment}
\begin{eulerprompt}
>gauss("qnormal(x,1,1.5)",2,5)
\end{eulerprompt}
\begin{euleroutput}
  0.248662156979
\end{euleroutput}
\begin{eulercomment}
Mari kita bandingkan distribusi binomial dengan distribusi normal mean
dan deviasi yang sama. Fungsi invbindis() memecahkan interpolasi
linier antara nilai bilangan bulat.
\end{eulercomment}
\begin{eulerprompt}
>invbindis(0.95,1000,0.5), invnormaldis(0.95,500,0.5*sqrt(1000))
\end{eulerprompt}
\begin{euleroutput}
  525.516721219
  526.007419394
\end{euleroutput}
\begin{eulercomment}
Fungsi qdis() adalah kepadatan distribusi chi-kuadrat. Seperti biasa,
Euler memetakan vektor ke fungsi ini. Jadi kita mendapatkan plot dari
semua distribusi chi-kuadrat dengan derajat 5 sampai 30 dengan mudah
dengan cara berikut.
\end{eulercomment}
\begin{eulerprompt}
>plot2d("qchidis(x,(5:5:50)')",0,50):
\end{eulerprompt}
\eulerimg{27}{images/Vikram Zaky Ardianto_22305144028_EMT STATISTIKA-010.png}
\begin{eulercomment}
Euler memiliki fungsi yang akurat untuk mengevaluasi distribusi. Mari
kita periksa chidis() dengan integral.

Penamaan mencoba untuk konsisten. Misalnya.,

- distribusi chi-kuadrat adalah chidis(),\\
- fungsi kebalikannya adalah invchidis(),\\
- densitasnya adalah qchidis().

Pelengkap distribusi (ekor atas) adalah chicdis().
\end{eulercomment}
\begin{eulerprompt}
>chidis(1.5,2), integrate("qchidis(x,2)",0,1.5)
\end{eulerprompt}
\begin{euleroutput}
  0.527633447259
  0.527633447259
\end{euleroutput}
\eulerheading{Distribusi Diskrit}
\begin{eulercomment}
Untuk menentukan distribusi diskrit Anda sendiri, Anda dapat
menggunakan metode berikut.

Pertama kita mengatur fungsi distribusi.
\end{eulercomment}
\begin{eulerprompt}
>wd = 0|((1:6)+[-0.01,0.01,0,0,0,0])/6
\end{eulerprompt}
\begin{euleroutput}
  [0,  0.165,  0.335,  0.5,  0.666667,  0.833333,  1]
\end{euleroutput}
\begin{eulercomment}
Artinya dengan probabilitas wd[i+1]-wd[i] kita menghasilkan nilai acak
i.

Ini hampir merupakan distribusi yang seragam. Mari kita tentukan
generator angka acak untuk ini. Fungsi find(v,x) menemukan nilai x
dalam vektor v. Fungsi ini juga berlaku untuk vektor x.
\end{eulercomment}
\begin{eulerprompt}
>function wrongdice (n,m) := find(wd,random(n,m))
\end{eulerprompt}
\begin{eulercomment}
Kesalahannya sangat halus sehingga kami melihatnya hanya dengan
iterasi yang sangat banyak.
\end{eulercomment}
\begin{eulerprompt}
>columnsplot(getmultiplicities(1:6,wrongdice(1,1000000))):
\end{eulerprompt}
\eulerimg{27}{images/Vikram Zaky Ardianto_22305144028_EMT STATISTIKA-011.png}
\begin{eulercomment}
Berikut adalah fungsi sederhana untuk memeriksa distribusi seragam
dari nilai 1...K dalam v. Kami menerima hasilnya, jika untuk semua
frekuensi

lateks: \textbackslash{}left\textbar{}f\_i-\textbackslash{}frac\{1\}\{K\}\textbackslash{}right\textbar{} \textless{} \textbackslash{}frac\{\textbackslash{}delta\}\{\textbackslash{}sqrt\{n\}\}.
\end{eulercomment}
\begin{eulerprompt}
>function checkrandom (v, delta=1) ...
\end{eulerprompt}
\begin{eulerudf}
    K=max(v); n=cols(v);
    fr=getfrequencies(v,1:K);
    return max(fr/n-1/K)<delta/sqrt(n);
    endfunction
\end{eulerudf}
\begin{eulercomment}
Memang fungsi menolak distribusi seragam.
\end{eulercomment}
\begin{eulerprompt}
>checkrandom(wrongdice(1,1000000))
\end{eulerprompt}
\begin{euleroutput}
  0
\end{euleroutput}
\begin{eulercomment}
Dan itu menerima generator acak bawaan.
\end{eulercomment}
\begin{eulerprompt}
>checkrandom(intrandom(1,1000000,6))
\end{eulerprompt}
\begin{euleroutput}
  1
\end{euleroutput}
\begin{eulercomment}
Kita dapat menghitung distribusi binomial. Pertama ada binomialsum(),
yang mengembalikan probabilitas i atau kurang hit dari n percobaan.
\end{eulercomment}
\begin{eulerprompt}
>bindis(410,1000,0.4)
\end{eulerprompt}
\begin{euleroutput}
  0.751401349654
\end{euleroutput}
\begin{eulercomment}
Fungsi Beta terbalik digunakan untuk menghitung interval kepercayaan
Clopper-Pearson untuk parameter p. Level default adalah alfa.

Arti interval ini adalah jika p berada di luar interval, hasil
pengamatan 410 dalam 1000 jarang terjadi.
\end{eulercomment}
\begin{eulerprompt}
>clopperpearson(410,1000)
\end{eulerprompt}
\begin{euleroutput}
  [0.37932,  0.441212]
\end{euleroutput}
\begin{eulercomment}
Perintah berikut adalah cara langsung untuk mendapatkan hasil di atas.
Tapi untuk n besar, penjumlahan langsungnya tidak akurat dan lambat.
\end{eulercomment}
\begin{eulerprompt}
>p=0.4; i=0:410; n=1000; sum(bin(n,i)*p^i*(1-p)^(n-i))
\end{eulerprompt}
\begin{euleroutput}
  0.751401349655
\end{euleroutput}
\begin{eulercomment}
Omong-omong, invbinsum() menghitung kebalikan dari binomialsum().
\end{eulercomment}
\begin{eulerprompt}
>invbindis(0.75,1000,0.4)
\end{eulerprompt}
\begin{euleroutput}
  409.932733047
\end{euleroutput}
\begin{eulercomment}
Di Bridge, kami mengasumsikan 5 kartu beredar (dari 52) dengan dua
tangan (26 kartu). Mari kita hitung probabilitas distribusi yang lebih
buruk dari 3:2 (mis. 0:5, 1:4, 4:1, atau 5:0).
\end{eulercomment}
\begin{eulerprompt}
>2*hypergeomsum(1,5,13,26)
\end{eulerprompt}
\begin{euleroutput}
  0.321739130435
\end{euleroutput}
\begin{eulercomment}
Ada juga simulasi distribusi multinomial.
\end{eulercomment}
\begin{eulerprompt}
>randmultinomial(10,1000,[0.4,0.1,0.5])
\end{eulerprompt}
\begin{euleroutput}
            381           100           519 
            376            91           533 
            417            80           503 
            440            94           466 
            406           112           482 
            408            94           498 
            395           107           498 
            399            96           505 
            428            87           485 
            400            99           501 
\end{euleroutput}
\eulerheading{Merencanakan Data}
\begin{eulercomment}
Untuk memplot data, kami mencoba hasil pemilu Jerman sejak 1990, yang
diukur dalam jumlah kursi.
\end{eulercomment}
\begin{eulerprompt}
>BW := [ ...
>1990,662,319,239,79,8,17; ...
>1994,672,294,252,47,49,30; ...
>1998,669,245,298,43,47,36; ...
>2002,603,248,251,47,55,2; ...
>2005,614,226,222,61,51,54; ...
>2009,622,239,146,93,68,76; ...
>2013,631,311,193,0,63,64];
\end{eulerprompt}
\begin{eulercomment}
Untuk partai, kami menggunakan rangkaian nama.
\end{eulercomment}
\begin{eulerprompt}
>P:=["CDU/CSU","SPD","FDP","Gr","Li"];
\end{eulerprompt}
\begin{eulercomment}
Mari kita cetak persentasenya dengan baik.

Pertama, kami mengekstrak kolom yang diperlukan. Kolom 3 sampai 7
adalah kursi masing-masing partai, dan kolom 2 adalah jumlah kursi.
kolom adalah tahun pemilihan.
\end{eulercomment}
\begin{eulerprompt}
>BT:=BW[,3:7]; BT:=BT/sum(BT); YT:=BW[,1]';
\end{eulerprompt}
\begin{eulercomment}
Kemudian kami mencetak statistik dalam bentuk tabel. Kami menggunakan
nama sebagai tajuk kolom, dan tahun sebagai tajuk untuk baris. Lebar
default untuk kolom adalah wc=10, tetapi kami lebih memilih hasil yang
lebih padat. Kolom akan diperluas untuk label kolom, jika perlu.
\end{eulercomment}
\begin{eulerprompt}
>writetable(BT*100,wc=6,dc=0,>fixed,labc=P,labr=YT)
\end{eulerprompt}
\begin{euleroutput}
         CDU/CSU   SPD   FDP    Gr    Li
    1990      48    36    12     1     3
    1994      44    38     7     7     4
    1998      37    45     6     7     5
    2002      41    42     8     9     0
    2005      37    36    10     8     9
    2009      38    23    15    11    12
    2013      49    31     0    10    10
\end{euleroutput}
\begin{eulercomment}
Perkalian matriks berikut mengekstrak jumlah persentase dari dua
partai besar yang menunjukkan bahwa partai-partai kecil telah
mendapatkan rekaman di parlemen hingga tahun 2009.
\end{eulercomment}
\begin{eulerprompt}
>BT1:=(BT.[1;1;0;0;0])'*100
\end{eulerprompt}
\begin{euleroutput}
  [84.29,  81.25,  81.1659,  82.7529,  72.9642,  61.8971,  79.8732]
\end{euleroutput}
\begin{eulercomment}
Ada juga plot statistik sederhana. Kami menggunakannya untuk
menampilkan garis dan titik secara bersamaan. Alternatifnya adalah
memanggil plot2d dua kali dengan \textgreater{}add.
\end{eulercomment}
\begin{eulerprompt}
>statplot(YT,BT1,"b"):
\end{eulerprompt}
\eulerimg{27}{images/Vikram Zaky Ardianto_22305144028_EMT STATISTIKA-012.png}
\begin{eulercomment}
Tentukan beberapa warna untuk masing-masing pihak.
\end{eulercomment}
\begin{eulerprompt}
>CP:=[rgb(0.5,0.5,0.5),red,yellow,green,rgb(0.8,0,0)];
\end{eulerprompt}
\begin{eulercomment}
Sekarang kita bisa memplot hasil pemilu 2009 dan mengubahnya menjadi
satu plot menggunakan gambar. Kita dapat menambahkan vektor kolom ke
setiap plot.
\end{eulercomment}
\begin{eulerprompt}
>figure(2,1);  ...
>figure(1); columnsplot(BW[6,3:7],P,color=CP); ...
>figure(2); columnsplot(BW[6,3:7]-BW[5,3:7],P,color=CP);  ...
>figure(0):
\end{eulerprompt}
\eulerimg{27}{images/Vikram Zaky Ardianto_22305144028_EMT STATISTIKA-013.png}
\begin{eulercomment}
Plot data menggabungkan deretan data statistik dalam satu plot.
\end{eulercomment}
\begin{eulerprompt}
>J:=BW[,1]'; DP:=BW[,3:7]'; ...
>dataplot(YT,BT',color=CP);  ...
>labelbox(P,colors=CP,styles="[]",>points,w=0.2,x=0.3,y=0.4):
\end{eulerprompt}
\eulerimg{27}{images/Vikram Zaky Ardianto_22305144028_EMT STATISTIKA-014.png}
\begin{eulercomment}
Plot kolom 3D menunjukkan deretan data statistik dalam bentuk kolom.
Kami menyediakan label untuk baris dan kolom. sudut adalah sudut
pandang.
\end{eulercomment}
\begin{eulerprompt}
>columnsplot3d(BT,scols=P,srows=YT, ...
>  angle=30°,ccols=CP):
\end{eulerprompt}
\eulerimg{27}{images/Vikram Zaky Ardianto_22305144028_EMT STATISTIKA-015.png}
\begin{eulercomment}
Representasi lainnya adalah plot mozaik. Perhatikan bahwa kolom plot
mewakili kolom matriks di sini. Karena panjang label CDU/CSU, kami
mengambil jendela yang lebih kecil dari biasanya.
\end{eulercomment}
\begin{eulerprompt}
>shrinkwindow(>smaller);  ...
>mosaicplot(BT',srows=YT,scols=P,color=CP,style="#"); ...
>shrinkwindow():
\end{eulerprompt}
\eulerimg{27}{images/Vikram Zaky Ardianto_22305144028_EMT STATISTIKA-016.png}
\begin{eulercomment}
Kita juga bisa membuat diagram lingkaran. Karena hitam dan kuning
membentuk koalisi, kami menyusun ulang elemennya.
\end{eulercomment}
\begin{eulerprompt}
>i=[1,3,5,4,2]; piechart(BW[6,3:7][i],color=CP[i],lab=P[i]):
\end{eulerprompt}
\eulerimg{27}{images/Vikram Zaky Ardianto_22305144028_EMT STATISTIKA-017.png}
\begin{eulercomment}
Ini jenis plot lainnya.
\end{eulercomment}
\begin{eulerprompt}
>starplot(normal(1,10)+4,lab=1:10,>rays):
\end{eulerprompt}
\eulerimg{27}{images/Vikram Zaky Ardianto_22305144028_EMT STATISTIKA-018.png}
\begin{eulercomment}
Beberapa plot di plot2d bagus untuk statika. Berikut adalah plot
impuls dari data acak, terdistribusi secara seragam di [0,1].
\end{eulercomment}
\begin{eulerprompt}
>plot2d(makeimpulse(1:10,random(1,10)),>bar):
\end{eulerprompt}
\eulerimg{27}{images/Vikram Zaky Ardianto_22305144028_EMT STATISTIKA-019.png}
\begin{eulercomment}
Tetapi untuk data yang terdistribusi secara eksponensial, kita mungkin
memerlukan plot logaritmik.
\end{eulercomment}
\begin{eulerprompt}
>logimpulseplot(1:10,-log(random(1,10))*10):
\end{eulerprompt}
\eulerimg{27}{images/Vikram Zaky Ardianto_22305144028_EMT STATISTIKA-020.png}
\begin{eulercomment}
Fungsi columnplot() lebih mudah digunakan, karena hanya membutuhkan
vektor nilai. Selain itu, ia dapat mengatur labelnya ke apa pun yang
kita inginkan, kita telah mendemonstrasikannya di tutorial ini.

Ini adalah aplikasi lain, di mana kami menghitung karakter dalam
sebuah kalimat dan memplot statistik.
\end{eulercomment}
\begin{eulerprompt}
>v=strtochar("the quick brown fox jumps over the lazy dog"); ...
>w=ascii("a"):ascii("z"); x=getmultiplicities(w,v); ...
>cw=[]; for k=w; cw=cw|char(k); end; ...
>columnsplot(x,lab=cw,width=0.05):
\end{eulerprompt}
\eulerimg{27}{images/Vikram Zaky Ardianto_22305144028_EMT STATISTIKA-021.png}
\begin{eulercomment}
Dimungkinkan juga untuk mengatur sumbu secara manual.
\end{eulercomment}
\begin{eulerprompt}
>n=10; p=0.4; i=0:n; x=bin(n,i)*p^i*(1-p)^(n-i); ...
>columnsplot(x,lab=i,width=0.05,<frame,<grid); ...
>yaxis(0,0:0.1:1,style="->",>left); xaxis(0,style="."); ...
>label("p",0,0.25), label("i",11,0); ...
>textbox(["Binomial distribution","with p=0.4"]):
\end{eulerprompt}
\eulerimg{27}{images/Vikram Zaky Ardianto_22305144028_EMT STATISTIKA-022.png}
\begin{eulercomment}
Berikut ini adalah cara memplot frekuensi bilangan dalam vektor.

Kami membuat vektor bilangan acak bilangan bulat 1 hingga 6.
\end{eulercomment}
\begin{eulerprompt}
>v:=intrandom(1,10,10)
\end{eulerprompt}
\begin{euleroutput}
  [8,  5,  8,  8,  6,  8,  8,  3,  5,  5]
\end{euleroutput}
\begin{eulercomment}
Kemudian ekstrak angka unik di v.
\end{eulercomment}
\begin{eulerprompt}
>vu:=unique(v)
\end{eulerprompt}
\begin{euleroutput}
  [3,  5,  6,  8]
\end{euleroutput}
\begin{eulercomment}
Dan plot frekuensi dalam plot kolom.
\end{eulercomment}
\begin{eulerprompt}
>columnsplot(getmultiplicities(vu,v),lab=vu,style="/"):
\end{eulerprompt}
\eulerimg{27}{images/Vikram Zaky Ardianto_22305144028_EMT STATISTIKA-023.png}
\begin{eulercomment}
Kami ingin menunjukkan fungsi untuk distribusi nilai empiris.
\end{eulercomment}
\begin{eulerprompt}
>x=normal(1,20);
\end{eulerprompt}
\begin{eulercomment}
Fungsi empdist(x,vs) membutuhkan array nilai yang diurutkan. Jadi kita
harus mengurutkan x sebelum kita dapat menggunakannya.
\end{eulercomment}
\begin{eulerprompt}
>xs=sort(x);
\end{eulerprompt}
\begin{eulercomment}
Kemudian kami memplot distribusi empiris dan beberapa batang kepadatan
menjadi satu plot. Alih-alih plot batang untuk distribusi, kali ini
kami menggunakan plot gigi gergaji.
\end{eulercomment}
\begin{eulerprompt}
>figure(2,1); ...
>figure(1); plot2d("empdist",-4,4;xs); ...
>figure(2); plot2d(histo(x,v=-4:0.2:4,<bar));  ...
>figure(0):
\end{eulerprompt}
\eulerimg{27}{images/Vikram Zaky Ardianto_22305144028_EMT STATISTIKA-024.png}
\begin{eulercomment}
Plot pencar mudah dilakukan di Euler dengan plot titik biasa. Grafik
berikut menunjukkan bahwa X dan X+Y jelas berkorelasi positif.
\end{eulercomment}
\begin{eulerprompt}
>x=normal(1,100); plot2d(x,x+rotright(x),>points,style=".."):
\end{eulerprompt}
\eulerimg{27}{images/Vikram Zaky Ardianto_22305144028_EMT STATISTIKA-025.png}
\begin{eulercomment}
Seringkali, kami ingin membandingkan dua sampel dari distribusi yang
berbeda. Ini dapat dilakukan dengan plot kuantil-kuantil.

Untuk pengujian, kami mencoba distribusi student-t dan distribusi
eksponensial.
\end{eulercomment}
\begin{eulerprompt}
>x=randt(1,1000,5); y=randnormal(1,1000,mean(x),dev(x)); ...
>plot2d("x",r=6,style="--",yl="normal",xl="student-t",>vertical); ...
>plot2d(sort(x),sort(y),>points,color=red,style="x",>add):
\end{eulerprompt}
\eulerimg{27}{images/Vikram Zaky Ardianto_22305144028_EMT STATISTIKA-026.png}
\begin{eulercomment}
Plot jelas menunjukkan bahwa nilai terdistribusi normal cenderung
lebih kecil di ujung ekstrim.

Jika kita memiliki dua distribusi dengan ukuran berbeda, kita dapat
memperluas yang lebih kecil atau mengecilkan yang lebih besar. Fungsi
berikut ini baik untuk keduanya. Dibutuhkan nilai median dengan
persentase antara 0 dan 1.
\end{eulercomment}
\begin{eulerprompt}
>function medianexpand (x,n) := median(x,p=linspace(0,1,n-1));
\end{eulerprompt}
\begin{eulercomment}
Mari kita bandingkan dua distribusi yang sama.
\end{eulercomment}
\begin{eulerprompt}
>x=random(1000); y=random(400); ...
>plot2d("x",0,1,style="--"); ...
>plot2d(sort(medianexpand(x,400)),sort(y),>points,color=red,style="x",>add):
\end{eulerprompt}
\eulerimg{27}{images/Vikram Zaky Ardianto_22305144028_EMT STATISTIKA-027.png}
\eulerheading{Regresi dan Korelasi}
\begin{eulercomment}
Regresi linier dapat dilakukan dengan fungsi polyfit() atau berbagai
fungsi fit.

Sebagai permulaan, kami menemukan garis regresi untuk data univariat
dengan polyfit(x,y,1).
\end{eulercomment}
\begin{eulerprompt}
>x=1:10; y=[2,3,1,5,6,3,7,8,9,8]; writetable(x'|y',labc=["x","y"])
\end{eulerprompt}
\begin{euleroutput}
           x         y
           1         2
           2         3
           3         1
           4         5
           5         6
           6         3
           7         7
           8         8
           9         9
          10         8
\end{euleroutput}
\begin{eulercomment}
Kami ingin membandingkan kecocokan yang tidak berbobot dan berbobot.
Pertama koefisien fit linier.
\end{eulercomment}
\begin{eulerprompt}
>p=polyfit(x,y,1)
\end{eulerprompt}
\begin{euleroutput}
  [0.733333,  0.812121]
\end{euleroutput}
\begin{eulercomment}
Sekarang koefisien dengan bobot yang menekankan nilai terakhir.
\end{eulercomment}
\begin{eulerprompt}
>w &= "exp(-(x-10)^2/10)"; pw=polyfit(x,y,1,w=w(x))
\end{eulerprompt}
\begin{euleroutput}
  [4.71566,  0.38319]
\end{euleroutput}
\begin{eulercomment}
Kami memasukkan semuanya ke dalam satu plot untuk poin dan garis
regresi, dan untuk bobot yang digunakan.
\end{eulercomment}
\begin{eulerprompt}
>figure(2,1);  ...
>figure(1); statplot(x,y,"b",xl="Regression"); ...
>  plot2d("evalpoly(x,p)",>add,color=blue,style="--"); ...
>  plot2d("evalpoly(x,pw)",5,10,>add,color=red,style="--"); ...
>figure(2); plot2d(w,1,10,>filled,style="/",fillcolor=red,xl=w); ...
>figure(0):
\end{eulerprompt}
\eulerimg{27}{images/Vikram Zaky Ardianto_22305144028_EMT STATISTIKA-028.png}
\begin{eulercomment}
Koefisien korelasi menunjukkan korelasi positif.
\end{eulercomment}
\begin{eulerprompt}
>correl(cs[1],cs[2])
\end{eulerprompt}
\begin{euleroutput}
  0.7588307236
\end{euleroutput}
\begin{eulercomment}
Korelasi peringkat adalah ukuran untuk urutan yang sama di kedua
vektor. Ini juga cukup positif.
\end{eulercomment}
\begin{eulerprompt}
>rankcorrel(cs[1],cs[2])
\end{eulerprompt}
\begin{euleroutput}
  0.758925292358
\end{euleroutput}
\eulerheading{Membuat Fungsi baru}
\begin{eulercomment}
Tentu saja, bahasa EMT dapat digunakan untuk memprogram fungsi-fungsi
baru. Misalnya, kita mendefinisikan fungsi skewness.

lateks: \textbackslash{}text\{sk\}(x) = \textbackslash{}dfrac\{\textbackslash{}sqrt\{n\} \textbackslash{}sum\_i (x\_i-m)\textasciicircum{}3\}\{\textbackslash{}left(\textbackslash{}sum\_i
(x\_i-m)\textasciicircum{}2\textbackslash{}right)\textasciicircum{}\{3/2 \}\}

di mana m adalah rata-rata dari x.
\end{eulercomment}
\begin{eulerprompt}
>function skew (x:vector) ...
\end{eulerprompt}
\begin{eulerudf}
  m=mean(x);
  return sqrt(cols(x))*sum((x-m)^3)/(sum((x-m)^2))^(3/2);
  endfunction
\end{eulerudf}
\begin{eulercomment}
Seperti yang Anda lihat, kita dapat dengan mudah menggunakan bahasa
matriks untuk mendapatkan implementasi yang sangat singkat dan
efisien. Mari kita coba fungsi ini.
\end{eulercomment}
\begin{eulerprompt}
>data=normal(20); skew(normal(10))
\end{eulerprompt}
\begin{euleroutput}
  -0.198710316203
\end{euleroutput}
\begin{eulercomment}
Ini adalah fungsi lain, yang disebut koefisien kemiringan Pearson.
\end{eulercomment}
\begin{eulerprompt}
>function skew1 (x) := 3*(mean(x)-median(x))/dev(x)
>skew1(data)
\end{eulerprompt}
\begin{euleroutput}
  -0.0801873249135
\end{euleroutput}
\eulerheading{Simulasi Monte Carlo}
\begin{eulercomment}
Euler dapat digunakan untuk mensimulasikan kejadian acak. Kita telah
melihat contoh sederhana di atas. Ini satu lagi, yang mensimulasikan
1000 kali 3 lemparan dadu, dan meminta distribusi jumlahnya.
\end{eulercomment}
\begin{eulerprompt}
>ds:=sum(intrandom(1000,3,6))';  fs=getmultiplicities(3:18,ds)
\end{eulerprompt}
\begin{euleroutput}
  [5,  17,  35,  44,  75,  97,  114,  116,  143,  116,  104,  53,  40,
  22,  13,  6]
\end{euleroutput}
\begin{eulercomment}
Kita bisa merencanakan ini sekarang.
\end{eulercomment}
\begin{eulerprompt}
>columnsplot(fs,lab=3:18):
\end{eulerprompt}
\eulerimg{27}{images/Vikram Zaky Ardianto_22305144028_EMT STATISTIKA-029.png}
\begin{eulercomment}
Untuk menentukan distribusi yang diharapkan tidak begitu mudah. Kami
menggunakan rekursi lanjutan untuk ini.

Fungsi berikut menghitung banyaknya cara bilangan k dapat dinyatakan
sebagai jumlah dari n bilangan dalam rentang 1 sampai m. Ini bekerja
secara rekursif dengan cara yang jelas.
\end{eulercomment}
\begin{eulerprompt}
>function map countways (k; n, m) ...
\end{eulerprompt}
\begin{eulerudf}
    if n==1 then return k>=1 && k<=m
    else
      sum=0; 
      loop 1 to m; sum=sum+countways(k-#,n-1,m); end;
      return sum;
    end;
  endfunction
\end{eulerudf}
\begin{eulercomment}
Inilah hasil lemparan dadu sebanyak tiga kali.
\end{eulercomment}
\begin{eulerprompt}
>cw=countways(3:18,3,6)
\end{eulerprompt}
\begin{euleroutput}
  [1,  3,  6,  10,  15,  21,  25,  27,  27,  25,  21,  15,  10,  6,  3,
  1]
\end{euleroutput}
\begin{eulercomment}
Kami menambahkan nilai yang diharapkan ke plot.
\end{eulercomment}
\begin{eulerprompt}
>plot2d(cw/6^3*1000,>add); plot2d(cw/6^3*1000,>points,>add):
\end{eulerprompt}
\eulerimg{27}{images/Vikram Zaky Ardianto_22305144028_EMT STATISTIKA-030.png}
\begin{eulercomment}
Untuk simulasi lain, penyimpangan nilai rata-rata n 0-1-variabel acak
terdistribusi normal adalah 1/sqrt(n).
\end{eulercomment}
\begin{eulerprompt}
>longformat; 1/sqrt(10)
\end{eulerprompt}
\begin{euleroutput}
  0.316227766017
\end{euleroutput}
\begin{eulercomment}
Mari kita periksa ini dengan simulasi. Kami menghasilkan 10.000 kali
10 vektor acak.
\end{eulercomment}
\begin{eulerprompt}
>M=normal(10000,10); dev(mean(M)')
\end{eulerprompt}
\begin{euleroutput}
  0.319493614817
\end{euleroutput}
\begin{eulerprompt}
>plot2d(mean(M)',>distribution):
\end{eulerprompt}
\eulerimg{27}{images/Vikram Zaky Ardianto_22305144028_EMT STATISTIKA-031.png}
\begin{eulercomment}
Median dari 10 bilangan acak terdistribusi 0-1-normal memiliki deviasi
yang lebih besar.
\end{eulercomment}
\begin{eulerprompt}
>dev(median(M)')
\end{eulerprompt}
\begin{euleroutput}
  0.374460271535
\end{euleroutput}
\begin{eulercomment}
Karena kita dapat dengan mudah membuat jalan acak, kita dapat
mensimulasikan proses Wiener. Kami mengambil 1000 langkah dari 1000
proses. Kami kemudian memplot standar deviasi dan rata-rata langkah
ke-n dari proses ini bersama dengan nilai yang diharapkan dalam warna
merah.
\end{eulercomment}
\begin{eulerprompt}
>n=1000; m=1000; M=cumsum(normal(n,m)/sqrt(m)); ...
>t=(1:n)/n; figure(2,1); ...
>figure(1); plot2d(t,mean(M')'); plot2d(t,0,color=red,>add); ...
>figure(2); plot2d(t,dev(M')'); plot2d(t,sqrt(t),color=red,>add); ...
>figure(0):
\end{eulerprompt}
\eulerimg{27}{images/Vikram Zaky Ardianto_22305144028_EMT STATISTIKA-032.png}
\eulerheading{Tes}
\begin{eulercomment}
Tes adalah alat penting dalam statistik. Di Euler, banyak tes yang
diterapkan. Semua tes ini mengembalikan kesalahan yang kami terima
jika kami menolak hipotesis nol.

Sebagai contoh, kami menguji lemparan dadu untuk distribusi seragam.
Pada 600 lemparan, kami mendapat nilai berikut, yang kami masukkan ke
uji chi-square.
\end{eulercomment}
\begin{eulerprompt}
>chitest([90,103,114,101,103,89],dup(100,6)')
\end{eulerprompt}
\begin{euleroutput}
  0.498830517952
\end{euleroutput}
\begin{eulercomment}
Tes chi-kuadrat juga memiliki mode yang menggunakan simulasi Monte
Carlo untuk menguji statistik. Hasilnya harus hampir sama. Parameter
\textgreater{}p menginterpretasikan vektor-y sebagai vektor probabilitas.
\end{eulercomment}
\begin{eulerprompt}
>chitest([90,103,114,101,103,89],dup(1/6,6)',>p,>montecarlo)
\end{eulerprompt}
\begin{euleroutput}
  0.526
\end{euleroutput}
\begin{eulercomment}
Kesalahan ini terlalu besar. Jadi kita tidak bisa menolak pemerataan
distribusi. Ini tidak membuktikan bahwa dadu kami adil. Tapi kita
tidak bisa menolak hipotesis kita.

Selanjutnya kami menghasilkan 1000 lemparan dadu menggunakan generator
angka acak, dan melakukan pengujian yang sama.
\end{eulercomment}
\begin{eulerprompt}
>n=1000; t=random([1,n*6]); chitest(count(t*6,6),dup(n,6)')
\end{eulerprompt}
\begin{euleroutput}
  0.528028118442
\end{euleroutput}
\begin{eulercomment}
Mari kita uji nilai rata-rata 100 dengan uji-t.
\end{eulercomment}
\begin{eulerprompt}
>s=200+normal([1,100])*10; ...
>ttest(mean(s),dev(s),100,200)
\end{eulerprompt}
\begin{euleroutput}
  0.0218365848476
\end{euleroutput}
\begin{eulercomment}
Fungsi ttest() membutuhkan nilai rata-rata, simpangan, jumlah data,
dan nilai rata-rata untuk diuji.

Sekarang mari kita periksa dua pengukuran untuk rata-rata yang sama.
Kami menolak hipotesis bahwa mereka memiliki rata-rata yang sama, jika
hasilnya \textless{}0,05.
\end{eulercomment}
\begin{eulerprompt}
>tcomparedata(normal(1,10),normal(1,10))
\end{eulerprompt}
\begin{euleroutput}
  0.38722000942
\end{euleroutput}
\begin{eulercomment}
Jika kami menambahkan bias ke satu distribusi, kami mendapat lebih
banyak penolakan. Ulangi simulasi ini beberapa kali untuk melihat
efeknya.
\end{eulercomment}
\begin{eulerprompt}
>tcomparedata(normal(1,10),normal(1,10)+2)
\end{eulerprompt}
\begin{euleroutput}
  5.60009101758e-07
\end{euleroutput}
\begin{eulercomment}
Dalam contoh berikutnya, kami menghasilkan 20 lemparan dadu acak 100
kali dan menghitungnya. Harus ada rata-rata 20/6=3,3.
\end{eulercomment}
\begin{eulerprompt}
>R=random(100,20); R=sum(R*6<=1)'; mean(R)
\end{eulerprompt}
\begin{euleroutput}
  3.28
\end{euleroutput}
\begin{eulercomment}
Kami sekarang membandingkan jumlah satu dengan distribusi binomial.
Pertama kita memplot distribusi satuan.
\end{eulercomment}
\begin{eulerprompt}
>plot2d(R,distribution=max(R)+1,even=1,style="\(\backslash\)/"):
\end{eulerprompt}
\eulerimg{27}{images/Vikram Zaky Ardianto_22305144028_EMT STATISTIKA-033.png}
\begin{eulerprompt}
>t=count(R,21);
\end{eulerprompt}
\begin{eulercomment}
Kemudian kami menghitung nilai yang diharapkan.
\end{eulercomment}
\begin{eulerprompt}
>n=0:20; b=bin(20,n)*(1/6)^n*(5/6)^(20-n)*100;
\end{eulerprompt}
\begin{eulercomment}
Kita harus mengumpulkan beberapa angka untuk mendapatkan kategori yang
cukup besar.
\end{eulercomment}
\begin{eulerprompt}
>t1=sum(t[1:2])|t[3:7]|sum(t[8:21]); ...
>b1=sum(b[1:2])|b[3:7]|sum(b[8:21]);
\end{eulerprompt}
\begin{eulercomment}
Uji chi-square menolak hipotesis bahwa distribusi kita adalah
distribusi binomial, jika hasilnya \textless{}0,05.
\end{eulercomment}
\begin{eulerprompt}
>chitest(t1,b1)
\end{eulerprompt}
\begin{euleroutput}
  0.53921579764
\end{euleroutput}
\begin{eulercomment}
Contoh berikut berisi hasil dari dua kelompok orang (pria dan wanita,
katakanlah) memilih satu dari enam partai.
\end{eulercomment}
\begin{eulerprompt}
>A=[23,37,43,52,64,74;27,39,41,49,63,76];  ...
>  writetable(A,wc=6,labr=["m","f"],labc=1:6)
\end{eulerprompt}
\begin{euleroutput}
             1     2     3     4     5     6
       m    23    37    43    52    64    74
       f    27    39    41    49    63    76
\end{euleroutput}
\begin{eulercomment}
Kami ingin menguji independensi suara dari jenis kelamin. Tes tabel
chi\textasciicircum{}2 melakukan ini. Hasilnya terlalu besar untuk menolak kemerdekaan.
Jadi kami tidak bisa mengatakan, jika pemungutan suara tergantung pada
jenis kelamin dari data tersebut.
\end{eulercomment}
\begin{eulerprompt}
>tabletest(A)
\end{eulerprompt}
\begin{euleroutput}
  0.990701632326
\end{euleroutput}
\begin{eulercomment}
Berikut adalah tabel yang diharapkan, jika kita mengasumsikan
frekuensi pemungutan suara yang diamati.
\end{eulercomment}
\begin{eulerprompt}
>writetable(expectedtable(A),wc=6,dc=1,labr=["m","f"],labc=1:6)
\end{eulerprompt}
\begin{euleroutput}
             1     2     3     4     5     6
       m  24.9  37.9  41.9  50.3  63.3  74.7
       f  25.1  38.1  42.1  50.7  63.7  75.3
\end{euleroutput}
\begin{eulercomment}
Kita dapat menghitung koefisien kontingensi yang dikoreksi. Karena
sangat mendekati 0, kami menyimpulkan bahwa pemungutan suara tidak
bergantung pada jenis kelamin.
\end{eulercomment}
\begin{eulerprompt}
>contingency(A)
\end{eulerprompt}
\begin{euleroutput}
  0.0427225484717
\end{euleroutput}
\eulerheading{Beberapa Tes Lagi}
\begin{eulercomment}
Selanjutnya kami menggunakan analisis varians (F-test) untuk menguji
tiga sampel data yang terdistribusi normal untuk nilai rata-rata yang
sama. Metode tersebut dinamakan ANOVA (analysis of variance). Di
Euler, fungsi varanalysis() digunakan.
\end{eulercomment}
\begin{eulerprompt}
>x1=[109,111,98,119,91,118,109,99,115,109,94]; mean(x1),
\end{eulerprompt}
\begin{euleroutput}
  106.545454545
\end{euleroutput}
\begin{eulerprompt}
>x2=[120,124,115,139,114,110,113,120,117]; mean(x2),
\end{eulerprompt}
\begin{euleroutput}
  119.111111111
\end{euleroutput}
\begin{eulerprompt}
>x3=[120,112,115,110,105,134,105,130,121,111]; mean(x3)
\end{eulerprompt}
\begin{euleroutput}
  116.3
\end{euleroutput}
\begin{eulerprompt}
>varanalysis(x1,x2,x3)
\end{eulerprompt}
\begin{euleroutput}
  0.0138048221371
\end{euleroutput}
\begin{eulercomment}
Ini berarti, kami menolak hipotesis dengan nilai rata-rata yang sama.
Kami melakukan ini dengan probabilitas kesalahan 1,3\%.

Ada juga uji median yang menolak sampel data dengan distribusi
rata-rata yang berbeda menguji median sampel bersatu.
\end{eulercomment}
\begin{eulerprompt}
>a=[56,66,68,49,61,53,45,58,54];
>b=[72,81,51,73,69,78,59,67,65,71,68,71];
>mediantest(a,b)
\end{eulerprompt}
\begin{euleroutput}
  0.0241724220052
\end{euleroutput}
\begin{eulercomment}
Tes lain tentang kesetaraan adalah tes peringkat. Ini jauh lebih tajam
daripada tes median.
\end{eulercomment}
\begin{eulerprompt}
>ranktest(a,b)
\end{eulerprompt}
\begin{euleroutput}
  0.00199969612469
\end{euleroutput}
\begin{eulercomment}
Dalam contoh berikut, kedua distribusi memiliki rata-rata yang sama.
\end{eulercomment}
\begin{eulerprompt}
>ranktest(random(1,100),random(1,50)*3-1)
\end{eulerprompt}
\begin{euleroutput}
  0.129608141484
\end{euleroutput}
\begin{eulercomment}
Mari kita coba mensimulasikan dua perlakuan a dan b yang diterapkan
pada orang yang berbeda.
\end{eulercomment}
\begin{eulerprompt}
>a=[8.0,7.4,5.9,9.4,8.6,8.2,7.6,8.1,6.2,8.9];
>b=[6.8,7.1,6.8,8.3,7.9,7.2,7.4,6.8,6.8,8.1];
\end{eulerprompt}
\begin{eulercomment}
Tes signum memutuskan, jika a lebih baik dari b.
\end{eulercomment}
\begin{eulerprompt}
>signtest(a,b)
\end{eulerprompt}
\begin{euleroutput}
  0.0546875
\end{euleroutput}
\begin{eulercomment}
Ini terlalu banyak kesalahan. Kita tidak dapat menolak bahwa a sama
baiknya dengan b.

Tes Wilcoxon lebih tajam dari tes ini, tetapi bergantung pada nilai
kuantitatif perbedaannya.
\end{eulercomment}
\begin{eulerprompt}
>wilcoxon(a,b)
\end{eulerprompt}
\begin{euleroutput}
  0.0296680599405
\end{euleroutput}
\begin{eulercomment}
Mari kita coba dua tes lagi menggunakan rangkaian yang dihasilkan.
\end{eulercomment}
\begin{eulerprompt}
>wilcoxon(normal(1,20),normal(1,20)-1)
\end{eulerprompt}
\begin{euleroutput}
  0.0068706451766
\end{euleroutput}
\begin{eulerprompt}
>wilcoxon(normal(1,20),normal(1,20))
\end{eulerprompt}
\begin{euleroutput}
  0.275145971064
\end{euleroutput}
\eulerheading{Angka Acak}
\begin{eulercomment}
Berikut ini adalah tes untuk generator angka acak. Euler menggunakan
generator yang sangat bagus, jadi kita tidak perlu berharap ada
masalah.

Pertama kami menghasilkan sepuluh juta angka acak di [0,1].
\end{eulercomment}
\begin{eulerprompt}
>n:=10000000; r:=random(1,n);
\end{eulerprompt}
\begin{eulercomment}
Selanjutnya kita menghitung jarak antara dua angka kurang dari 0,05.
\end{eulercomment}
\begin{eulerprompt}
>a:=0.05; d:=differences(nonzeros(r<a));
\end{eulerprompt}
\begin{eulercomment}
Akhirnya, kami memplot berapa kali, setiap jarak terjadi, dan
membandingkannya dengan nilai yang diharapkan.
\end{eulercomment}
\begin{eulerprompt}
>m=getmultiplicities(1:100,d); plot2d(m); ...
>  plot2d("n*(1-a)^(x-1)*a^2",color=red,>add):
\end{eulerprompt}
\eulerimg{27}{images/Vikram Zaky Ardianto_22305144028_EMT STATISTIKA-034.png}
\begin{eulercomment}
Hapus datanya.
\end{eulercomment}
\begin{eulerprompt}
>remvalue n;
\end{eulerprompt}
\eulerheading{Pengantar untuk Pengguna Proyek R}
\begin{eulercomment}
Jelas, EMT tidak bersaing dengan R sebagai paket statistik. Namun, ada
banyak prosedur dan fungsi statistik yang tersedia di EMT juga. Jadi
EMT dapat memenuhi kebutuhan dasar. Lagi pula, EMT hadir dengan paket
numerik dan sistem aljabar komputer.

Notebook ini cocok untuk Anda jika sudah familiar dengan R, namun
perlu mengetahui perbedaan sintaks EMT dan R. Kami mencoba memberikan
gambaran umum tentang hal-hal yang jelas dan kurang jelas yang perlu
Anda ketahui.

Selain itu, kami mencari cara untuk bertukar data antara kedua sistem.
\end{eulercomment}
\begin{eulercomment}
Perhatikan bahwa ini adalah pekerjaan yang sedang berjalan.
\end{eulercomment}
\eulerheading{Sintaks Dasar}
\begin{eulercomment}
Hal pertama yang Anda pelajari di R adalah membuat vektor. Di EMT,
perbedaan utamanya adalah operator : dapat mengambil ukuran langkah.
Apalagi daya ikatnya rendah.
\end{eulercomment}
\begin{eulerprompt}
>n=10; 0:n/20:n-1
\end{eulerprompt}
\begin{euleroutput}
  [0,  0.5,  1,  1.5,  2,  2.5,  3,  3.5,  4,  4.5,  5,  5.5,  6,  6.5,
  7,  7.5,  8,  8.5,  9]
\end{euleroutput}
\begin{eulercomment}
Fungsi c() tidak ada. Dimungkinkan untuk menggunakan vektor untuk
menggabungkan berbagai hal.

Contoh berikut, seperti banyak lainnya, dari "Introduction to R" yang
disertakan dengan proyek R. Jika Anda membaca PDF ini, Anda akan
menemukan bahwa saya mengikuti jalannya dalam tutorial ini.
\end{eulercomment}
\begin{eulerprompt}
>x=[10.4, 5.6, 3.1, 6.4, 21.7]; [x,0,x]
\end{eulerprompt}
\begin{euleroutput}
  [10.4,  5.6,  3.1,  6.4,  21.7,  0,  10.4,  5.6,  3.1,  6.4,  21.7]
\end{euleroutput}
\begin{eulercomment}
Operator titik dua dengan ukuran langkah EMT diganti dengan fungsi
seq() di R. Kita bisa menulis fungsi ini di EMT.
\end{eulercomment}
\begin{eulerprompt}
>function seq(a,b,c) := a:b:c; ...
>seq(0,-0.1,-1)
\end{eulerprompt}
\begin{euleroutput}
  [0,  -0.1,  -0.2,  -0.3,  -0.4,  -0.5,  -0.6,  -0.7,  -0.8,  -0.9,  -1]
\end{euleroutput}
\begin{eulercomment}
Fungsi rep() dari R tidak ada di EMT. Untuk input vektor, dapat
ditulis sebagai berikut.
\end{eulercomment}
\begin{eulerprompt}
>function rep(x:vector,n:index) := flatten(dup(x,n)); ...
>rep(x,2)
\end{eulerprompt}
\begin{euleroutput}
  [10.4,  5.6,  3.1,  6.4,  21.7,  10.4,  5.6,  3.1,  6.4,  21.7]
\end{euleroutput}
\begin{eulercomment}
Perhatikan bahwa "=" atau ":=" digunakan untuk tugas. Operator "-\textgreater{}"
digunakan untuk unit di EMT.
\end{eulercomment}
\begin{eulerprompt}
>125km -> " miles"
\end{eulerprompt}
\begin{euleroutput}
  77.6713990297 miles
\end{euleroutput}
\begin{eulercomment}
The "\textless{}-" operator for assignment is misleading anyway, and not a good
idea of R. The following will compare a and -4 in EMT.
\end{eulercomment}
\begin{eulerprompt}
>a=2; a<-4
\end{eulerprompt}
\begin{euleroutput}
  0
\end{euleroutput}
\begin{eulercomment}
Di R, "a\textless{}-4\textless{}3" berfungsi, tetapi "a\textless{}-4\textless{}-3" tidak. Saya juga memiliki
ambiguitas serupa di EMT, tetapi mencoba menghilangkannya sedikit demi
sedikit.

EMT dan R memiliki vektor tipe boolean. Namun dalam EMT, angka 0 dan 1
digunakan untuk mewakili salah dan benar. Di R, nilai benar dan salah
tetap bisa digunakan dalam aritmatika biasa seperti di EMT.
\end{eulercomment}
\begin{eulerprompt}
>x<5, %*x
\end{eulerprompt}
\begin{euleroutput}
  [0,  0,  1,  0,  0]
  [0,  0,  3.1,  0,  0]
\end{euleroutput}
\begin{eulercomment}
EMT melempar kesalahan atau menghasilkan NAN tergantung pada bendera
"kesalahan".
\end{eulercomment}
\begin{eulerprompt}
>errors off; 0/0, isNAN(sqrt(-1)), errors on;
\end{eulerprompt}
\begin{euleroutput}
  NAN
  1
\end{euleroutput}
\begin{eulercomment}
String sama di R dan EMT. Keduanya berada di lokal saat ini, bukan di
Unicode.

Di R ada paket untuk Unicode. Di EMT, sebuah string dapat berupa
string Unicode. String unicode dapat diterjemahkan ke pengkodean lokal
dan sebaliknya. Selain itu, u"..." dapat berisi entitas HTML.
\end{eulercomment}
\begin{eulerprompt}
>u"&#169; Ren&eacut; Grothmann"
\end{eulerprompt}
\begin{euleroutput}
  © René Grothmann
\end{euleroutput}
\begin{eulercomment}
Berikut ini mungkin atau mungkin tidak ditampilkan dengan benar di
sistem Anda sebagai A dengan titik dan garis di atasnya. Itu
tergantung pada font yang Anda gunakan.
\end{eulercomment}
\begin{eulerprompt}
>chartoutf([480])
\end{eulerprompt}
\begin{euleroutput}
  Ǡ
\end{euleroutput}
\begin{eulercomment}
Penggabungan string dilakukan dengan "+" atau "\textbar{}". Itu bisa termasuk
angka, yang akan dicetak dalam format saat ini.
\end{eulercomment}
\begin{eulerprompt}
>"pi = "+pi
\end{eulerprompt}
\begin{euleroutput}
  pi = 3.14159265359
\end{euleroutput}
\eulerheading{Pengindeksan}
\begin{eulercomment}
Sebagian besar waktu, ini akan berfungsi seperti di R.

Tetapi EMT akan menginterpretasikan indeks negatif dari belakang
vektor, sedangkan R menginterpretasikan x[n] sebagai x tanpa elemen
ke-n.
\end{eulercomment}
\begin{eulerprompt}
>x, x[1:3], x[-2]
\end{eulerprompt}
\begin{euleroutput}
  [10.4,  5.6,  3.1,  6.4,  21.7]
  [10.4,  5.6,  3.1]
  6.4
\end{euleroutput}
\begin{eulercomment}
Perilaku R dapat dicapai dalam EMT dengan drop().
\end{eulercomment}
\begin{eulerprompt}
>drop(x,2)
\end{eulerprompt}
\begin{euleroutput}
  [10.4,  3.1,  6.4,  21.7]
\end{euleroutput}
\begin{eulercomment}
Vektor logis tidak diperlakukan berbeda sebagai indeks di EMT, berbeda
dengan R. Anda perlu mengekstraksi elemen bukan nol terlebih dahulu di
EMT.
\end{eulercomment}
\begin{eulerprompt}
>x, x>5, x[nonzeros(x>5)]
\end{eulerprompt}
\begin{euleroutput}
  [10.4,  5.6,  3.1,  6.4,  21.7]
  [1,  1,  0,  1,  1]
  [10.4,  5.6,  6.4,  21.7]
\end{euleroutput}
\begin{eulercomment}
Sama seperti di R, vektor indeks dapat berisi pengulangan.
\end{eulercomment}
\begin{eulerprompt}
>x[[1,2,2,1]]
\end{eulerprompt}
\begin{euleroutput}
  [10.4,  5.6,  5.6,  10.4]
\end{euleroutput}
\begin{eulercomment}
Tetapi nama untuk indeks tidak dimungkinkan di EMT. Untuk paket
statistik, hal ini sering diperlukan untuk memudahkan akses ke elemen
vektor.

Untuk meniru perilaku ini, kita dapat mendefinisikan fungsi sebagai
berikut.
\end{eulercomment}
\begin{eulerprompt}
>function sel (v,i,s) := v[indexof(s,i)]; ...
>s=["first","second","third","fourth"]; sel(x,["first","third"],s)
\end{eulerprompt}
\begin{euleroutput}
  
  Trying to overwrite protected function sel!
  Error in:
  function sel (v,i,s) := v[indexof(s,i)]; ... ...
               ^
  
  Trying to overwrite protected function sel!
  Error in:
  function sel (v,i,s) := v[indexof(s,i)]; ... ...
               ^
  [10.4,  3.1]
\end{euleroutput}
\eulerheading{Tipe Data}
\begin{eulercomment}
EMT memiliki lebih banyak tipe data tetap daripada R. Jelas, di R
terdapat vektor yang tumbuh. Anda dapat menyetel vektor numerik kosong
v dan menetapkan nilai ke elemen v[17]. Ini tidak mungkin di EMT.

Berikut ini agak tidak efisien.
\end{eulercomment}
\begin{eulerprompt}
>v=[]; for i=1 to 10000; v=v|i; end;
\end{eulerprompt}
\begin{eulercomment}
EMT sekarang akan membuat vektor dengan v dan i ditambahkan pada
tumpukan dan menyalin vektor itu kembali ke variabel global v.

Semakin efisien pra-mendefinisikan vektor.
\end{eulercomment}
\begin{eulerprompt}
>v=zeros(10000); for i=1 to 10000; v[i]=i; end;
\end{eulerprompt}
\begin{eulercomment}
Untuk mengubah jenis tanggal di EMT, Anda dapat menggunakan fungsi
seperti complex().
\end{eulercomment}
\begin{eulerprompt}
>complex(1:4)
\end{eulerprompt}
\begin{euleroutput}
  [ 1+0i ,  2+0i ,  3+0i ,  4+0i  ]
\end{euleroutput}
\begin{eulercomment}
Konversi ke string hanya dimungkinkan untuk tipe data dasar. Format
saat ini digunakan untuk penggabungan string sederhana. Tapi ada
fungsi seperti print() atau frac().

Untuk vektor, Anda dapat dengan mudah menulis fungsi Anda sendiri.
\end{eulercomment}
\begin{eulerprompt}
>function tostr (v) ...
\end{eulerprompt}
\begin{eulerudf}
  s="[";
  loop 1 to length(v);
     s=s+print(v[#],2,0);
     if #<length(v) then s=s+","; endif;
  end;
  return s+"]";
  endfunction
\end{eulerudf}
\begin{eulerprompt}
>tostr(linspace(0,1,10))
\end{eulerprompt}
\begin{euleroutput}
  [0.00,0.10,0.20,0.30,0.40,0.50,0.60,0.70,0.80,0.90,1.00]
\end{euleroutput}
\begin{eulercomment}
Untuk komunikasi dengan Maxima, terdapat fungsi convertmxm(), yang
juga dapat digunakan untuk memformat vektor untuk output.
\end{eulercomment}
\begin{eulerprompt}
>convertmxm(1:10)
\end{eulerprompt}
\begin{euleroutput}
  [1,2,3,4,5,6,7,8,9,10]
\end{euleroutput}
\begin{eulercomment}
Untuk Lateks, perintah tex dapat digunakan untuk mendapatkan perintah
Lateks.
\end{eulercomment}
\begin{eulerprompt}
>tex(&[1,2,3])
\end{eulerprompt}
\begin{euleroutput}
  \(\backslash\)left[ 1 , 2 , 3 \(\backslash\)right] 
\end{euleroutput}
\eulerheading{Faktor dan Tabel}
\begin{eulercomment}
Dalam pengantar R ada contoh dengan apa yang disebut faktor.

Berikut ini adalah daftar wilayah dari 30 negara bagian.
\end{eulercomment}
\begin{eulerprompt}
>austates = ["tas", "sa", "qld", "nsw", "nsw", "nt", "wa", "wa", ...
>"qld", "vic", "nsw", "vic", "qld", "qld", "sa", "tas", ...
>"sa", "nt", "wa", "vic", "qld", "nsw", "nsw", "wa", ...
>"sa", "act", "nsw", "vic", "vic", "act"];
\end{eulerprompt}
\begin{eulercomment}
Asumsikan, kita memiliki pendapatan yang sesuai di setiap negara
bagian.
\end{eulercomment}
\begin{eulerprompt}
>incomes = [60, 49, 40, 61, 64, 60, 59, 54, 62, 69, 70, 42, 56, ...
>61, 61, 61, 58, 51, 48, 65, 49, 49, 41, 48, 52, 46, ...
>59, 46, 58, 43];
\end{eulerprompt}
\begin{eulercomment}
Sekarang, kami ingin menghitung rata-rata pendapatan di wilayah
tersebut. Menjadi program statistik, R memiliki factor() dan tappy()
untuk ini.

EMT dapat melakukannya dengan menemukan indeks wilayah di daftar unik
wilayah.
\end{eulercomment}
\begin{eulerprompt}
>auterr=sort(unique(austates)); f=indexofsorted(auterr,austates)
\end{eulerprompt}
\begin{euleroutput}
  [6,  5,  4,  2,  2,  3,  8,  8,  4,  7,  2,  7,  4,  4,  5,  6,  5,  3,
  8,  7,  4,  2,  2,  8,  5,  1,  2,  7,  7,  1]
\end{euleroutput}
\begin{eulercomment}
Pada saat itu, kita dapat menulis fungsi loop kita sendiri untuk
melakukan sesuatu hanya untuk satu faktor.

Atau kita bisa meniru fungsi tapply() dengan cara berikut.
\end{eulercomment}
\begin{eulerprompt}
>function map tappl (i; f$:call, cat, x) ...
\end{eulerprompt}
\begin{eulerudf}
  u=sort(unique(cat));
  f=indexof(u,cat);
  return f$(x[nonzeros(f==indexof(u,i))]);
  endfunction
\end{eulerudf}
\begin{eulercomment}
Ini sedikit tidak efisien, karena menghitung wilayah unik untuk setiap
i, tetapi berhasil.
\end{eulercomment}
\begin{eulerprompt}
>tappl(auterr,"mean",austates,incomes)
\end{eulerprompt}
\begin{euleroutput}
  [44.5,  57.3333333333,  55.5,  53.6,  55,  60.5,  56,  52.25]
\end{euleroutput}
\begin{eulercomment}
Perhatikan bahwa ini berfungsi untuk setiap vektor wilayah.
\end{eulercomment}
\begin{eulerprompt}
>tappl(["act","nsw"],"mean",austates,incomes)
\end{eulerprompt}
\begin{euleroutput}
  [44.5,  57.3333333333]
\end{euleroutput}
\begin{eulercomment}
Sekarang, paket statistik EMT mendefinisikan tabel seperti pada R.
Fungsi readtable() dan writetable() dapat digunakan untuk input dan
output.

Sehingga kita bisa mencetak rata-rata pendapatan negara di daerah
dengan cara yang bersahabat.
\end{eulercomment}
\begin{eulerprompt}
>writetable(tappl(auterr,"mean",austates,incomes),labc=auterr,wc=7)
\end{eulerprompt}
\begin{euleroutput}
      act    nsw     nt    qld     sa    tas    vic     wa
     44.5  57.33   55.5   53.6     55   60.5     56  52.25
\end{euleroutput}
\begin{eulercomment}
Kami juga dapat mencoba meniru perilaku R sepenuhnya.

Faktor jelas harus disimpan dalam kumpulan dengan jenis dan kategori
(negara bagian dan teritori dalam contoh kita). Untuk EMT, kami
menambahkan indeks yang telah dihitung sebelumnya.
\end{eulercomment}
\begin{eulerprompt}
>function makef (t) ...
\end{eulerprompt}
\begin{eulerudf}
  ## Factor data
  ## Returns a collection with data t, unique data, indices.
  ## See: tapply
  u=sort(unique(t));
  return \{\{t,u,indexofsorted(u,t)\}\};
  endfunction
\end{eulerudf}
\begin{eulerprompt}
>statef=makef(austates);
\end{eulerprompt}
\begin{eulercomment}
Sekarang elemen ketiga dari koleksi akan berisi indeks.
\end{eulercomment}
\begin{eulerprompt}
>statef[3]
\end{eulerprompt}
\begin{euleroutput}
  [6,  5,  4,  2,  2,  3,  8,  8,  4,  7,  2,  7,  4,  4,  5,  6,  5,  3,
  8,  7,  4,  2,  2,  8,  5,  1,  2,  7,  7,  1]
\end{euleroutput}
\begin{eulercomment}
Sekarang kita bisa meniru tapply() dengan cara berikut. Ini akan
mengembalikan tabel sebagai kumpulan data tabel dan judul kolom.
\end{eulercomment}
\begin{eulerprompt}
>function tapply (t:vector,tf,f$:call) ...
\end{eulerprompt}
\begin{eulerudf}
  ## Makes a table of data and factors
  ## tf : output of makef()
  ## See: makef
  uf=tf[2]; f=tf[3]; x=zeros(length(uf));
  for i=1 to length(uf);
     ind=nonzeros(f==i);
     if length(ind)==0 then x[i]=NAN;
     else x[i]=f$(t[ind]);
     endif;
  end;
  return \{\{x,uf\}\};
  endfunction
\end{eulerudf}
\begin{eulercomment}
Kami tidak menambahkan banyak pengecekan tipe di sini. Satu-satunya
tindakan pencegahan menyangkut kategori (faktor) tanpa data. Tetapi
orang harus memeriksa panjang t yang benar dan kebenaran koleksi tf.

Tabel ini dapat dicetak sebagai tabel dengan writetable().
\end{eulercomment}
\begin{eulerprompt}
>writetable(tapply(incomes,statef,"mean"),wc=7)
\end{eulerprompt}
\begin{euleroutput}
      act    nsw     nt    qld     sa    tas    vic     wa
     44.5  57.33   55.5   53.6     55   60.5     56  52.25
\end{euleroutput}
\eulerheading{Array}
\begin{eulercomment}
EMT hanya memiliki dua dimensi untuk array. Tipe datanya disebut
matriks. Namun, akan mudah untuk menulis fungsi untuk dimensi yang
lebih tinggi atau pustaka C untuk ini.

R memiliki lebih dari dua dimensi. Di R array adalah vektor dengan
bidang dimensi.

Dalam EMT, vektor adalah matriks dengan satu baris. Itu dapat dibuat
menjadi matriks dengan redim().
\end{eulercomment}
\begin{eulerprompt}
>shortformat; X=redim(1:20,4,5)
\end{eulerprompt}
\begin{euleroutput}
          1         2         3         4         5 
          6         7         8         9        10 
         11        12        13        14        15 
         16        17        18        19        20 
\end{euleroutput}
\begin{eulercomment}
Ekstraksi baris dan kolom, atau sub-matriks, sangat mirip dengan R.
\end{eulercomment}
\begin{eulerprompt}
>X[,2:3]
\end{eulerprompt}
\begin{euleroutput}
          2         3 
          7         8 
         12        13 
         17        18 
\end{euleroutput}
\begin{eulercomment}
Namun, dalam R dimungkinkan untuk menetapkan daftar indeks spesifik
vektor ke suatu nilai. Hal yang sama dimungkinkan di EMT hanya dengan
satu putaran.
\end{eulercomment}
\begin{eulerprompt}
>function setmatrixvalue (M, i, j, v) ...
\end{eulerprompt}
\begin{eulerudf}
  loop 1 to max(length(i),length(j),length(v))
     M[i\{#\},j\{#\}] = v\{#\};
  end;
  endfunction
\end{eulerudf}
\begin{eulercomment}
Kami mendemonstrasikan ini untuk menunjukkan bahwa matriks dilewatkan
dengan referensi di EMT. Jika Anda tidak ingin mengubah matriks asli
M, Anda perlu menyalinnya ke dalam fungsi.
\end{eulercomment}
\begin{eulerprompt}
>setmatrixvalue(X,1:3,3:-1:1,0); X,
\end{eulerprompt}
\begin{euleroutput}
          1         2         0         4         5 
          6         0         8         9        10 
          0        12        13        14        15 
         16        17        18        19        20 
\end{euleroutput}
\begin{eulercomment}
Produk luar di EMT hanya dapat dilakukan di antara vektor. Ini
otomatis karena bahasa matriks. Satu vektor harus berupa vektor kolom
dan yang lainnya vektor baris.
\end{eulercomment}
\begin{eulerprompt}
>(1:5)*(1:5)'
\end{eulerprompt}
\begin{euleroutput}
          1         2         3         4         5 
          2         4         6         8        10 
          3         6         9        12        15 
          4         8        12        16        20 
          5        10        15        20        25 
\end{euleroutput}
\begin{eulercomment}
Dalam pengantar PDF untuk R ada contoh, yang menghitung distribusi
ab-cd untuk a,b,c,d dipilih dari 0 sampai n secara acak. Solusi dalam
R adalah membentuk matriks 4 dimensi dan menjalankan table() di
atasnya.

Tentu saja, ini bisa dicapai dengan satu putaran. Tapi loop tidak
efektif di EMT atau R. Di EMT, kita bisa menulis loop di C dan itu
akan menjadi solusi tercepat.

Tapi kami ingin meniru perilaku R. Untuk ini, kami perlu meratakan
perkalian ab dan membuat matriks ab-cd.
\end{eulercomment}
\begin{eulerprompt}
>a=0:6; b=a'; p=flatten(a*b); q=flatten(p-p'); ...
>u=sort(unique(q)); f=getmultiplicities(u,q); ...
>statplot(u,f,"h"):
\end{eulerprompt}
\eulerimg{27}{images/Vikram Zaky Ardianto_22305144028_EMT STATISTIKA-035.png}
\begin{eulercomment}
Selain perkalian yang tepat, EMT dapat menghitung frekuensi dalam
vektor.
\end{eulercomment}
\begin{eulerprompt}
>getfrequencies(q,-50:10:50)
\end{eulerprompt}
\begin{euleroutput}
  [0,  23,  132,  316,  602,  801,  333,  141,  53,  0]
\end{euleroutput}
\begin{eulercomment}
Cara paling mudah untuk memplot ini sebagai distribusi adalah sebagai
berikut.
\end{eulercomment}
\begin{eulerprompt}
>plot2d(q,distribution=11):
\end{eulerprompt}
\eulerimg{27}{images/Vikram Zaky Ardianto_22305144028_EMT STATISTIKA-036.png}
\begin{eulercomment}
Tetapi juga memungkinkan untuk melakukan pra-perhitungan hitungan
dalam interval yang dipilih sebelumnya. Tentu saja, berikut ini
menggunakan getfrequencies() secara internal.

Karena fungsi histo() mengembalikan frekuensi, kita perlu
menskalakannya sehingga integral di bawah grafik batang adalah 1.
\end{eulercomment}
\begin{eulerprompt}
>\{x,y\}=histo(q,v=-55:10:55); y=y/sum(y)/differences(x); ...
>plot2d(x,y,>bar,style="/"):
\end{eulerprompt}
\eulerimg{27}{images/Vikram Zaky Ardianto_22305144028_EMT STATISTIKA-037.png}
\eulerheading{Daftar}
\begin{eulercomment}
EMT memiliki dua jenis daftar. Salah satunya adalah daftar global yang
bisa berubah, dan yang lainnya adalah tipe daftar yang tidak bisa
diubah. Kami tidak peduli dengan daftar global di sini.

Jenis daftar yang tidak dapat diubah disebut koleksi di EMT. Ini
berperilaku seperti struktur di C, tetapi elemennya hanya diberi nomor
dan tidak diberi nama.
\end{eulercomment}
\begin{eulerprompt}
>L=\{\{"Fred","Flintstone",40,[1990,1992]\}\}
\end{eulerprompt}
\begin{euleroutput}
  Fred
  Flintstone
  40
  [1990,  1992]
\end{euleroutput}
\begin{eulercomment}
Saat ini elemen tidak memiliki nama, meskipun nama dapat diatur untuk
tujuan khusus. Mereka diakses oleh nomor.
\end{eulercomment}
\begin{eulerprompt}
>(L[4])[2]
\end{eulerprompt}
\begin{euleroutput}
  1992
\end{euleroutput}
\eulerheading{File Input dan Output (Membaca dan Menulis Data)}
\begin{eulercomment}
Anda sering ingin mengimpor matriks data dari sumber lain ke EMT.
Tutorial ini memberitahu Anda tentang banyak cara untuk mencapai hal
ini. Fungsi sederhana adalah writematrix() dan readmatrix().

Mari kita tunjukkan cara membaca dan menulis vektor real ke file.
\end{eulercomment}
\begin{eulerprompt}
>a=random(1,100); mean(a), dev(a),
\end{eulerprompt}
\begin{euleroutput}
  0.49815
  0.28037
\end{euleroutput}
\begin{eulercomment}
Untuk menulis data ke file, kami menggunakan fungsi writematrix().

Karena pengantar ini kemungkinan besar ada di direktori, di mana
pengguna tidak memiliki akses tulis, kami menulis data ke direktori
home pengguna. Untuk buku catatan sendiri, hal ini tidak diperlukan,
karena file data akan ditulis ke dalam direktori yang sama.
\end{eulercomment}
\begin{eulerprompt}
>filename="test.dat";
\end{eulerprompt}
\begin{eulercomment}
Sekarang kita menulis vektor kolom a' ke file. Ini menghasilkan satu
nomor di setiap baris file.
\end{eulercomment}
\begin{eulerprompt}
>writematrix(a',filename);
\end{eulerprompt}
\begin{eulercomment}
Untuk membaca data, kami menggunakan readmatrix().
\end{eulercomment}
\begin{eulerprompt}
>a=readmatrix(filename)';
\end{eulerprompt}
\begin{eulercomment}
Dan hapus file tersebut.
\end{eulercomment}
\begin{eulerprompt}
>fileremove(filename);
>mean(a), dev(a),
\end{eulerprompt}
\begin{euleroutput}
  0.49815
  0.28037
\end{euleroutput}
\begin{eulercomment}
Fungsi writematrix() atau writetable() dapat dikonfigurasi untuk
bahasa lain.

Misalnya, jika Anda memiliki sistem bahasa Indonesia (titik desimal
dengan koma), Excel Anda memerlukan nilai dengan koma desimal yang
dipisahkan oleh titik koma dalam file csv (defaultnya adalah nilai
yang dipisahkan koma). File berikut "test.csv" akan muncul di folder
cuurent Anda.
\end{eulercomment}
\begin{eulerprompt}
>filename="test.csv"; ...
>writematrix(random(5,3),file=filename,separator=",");
\end{eulerprompt}
\begin{eulercomment}
Anda sekarang dapat membuka file ini dengan Excel bahasa Indonesia
secara langsung.
\end{eulercomment}
\begin{eulerprompt}
>fileremove(filename);
\end{eulerprompt}
\begin{eulercomment}
Terkadang kami memiliki string dengan token seperti berikut ini.
\end{eulercomment}
\begin{eulerprompt}
>s1:="f m m f m m m f f f m m f";  ...
>s2:="f f f m m f f";
\end{eulerprompt}
\begin{eulercomment}
Untuk menandai ini, kami mendefinisikan vektor token.
\end{eulercomment}
\begin{eulerprompt}
>tok:=["f","m"]
\end{eulerprompt}
\begin{euleroutput}
  f
  m
\end{euleroutput}
\begin{eulercomment}
Kemudian kita dapat menghitung berapa kali setiap token muncul dalam
string, dan memasukkan hasilnya ke dalam tabel.
\end{eulercomment}
\begin{eulerprompt}
>M:=getmultiplicities(tok,strtokens(s1))_ ...
>  getmultiplicities(tok,strtokens(s2));
\end{eulerprompt}
\begin{eulercomment}
Tulis tabel dengan header token.
\end{eulercomment}
\begin{eulerprompt}
>writetable(M,labc=tok,labr=1:2,wc=8)
\end{eulerprompt}
\begin{euleroutput}
                 f       m
         1       6       7
         2       5       2
\end{euleroutput}
\begin{eulercomment}
Untuk statika, EMT dapat membaca dan menulis tabel.
\end{eulercomment}
\begin{eulerprompt}
>file="test.dat"; open(file,"w"); ...
>writeln("A,B,C"); writematrix(random(3,3)); ...
>close();
\end{eulerprompt}
\begin{eulercomment}
The file looks like this.
\end{eulercomment}
\begin{eulerprompt}
>printfile(file)
\end{eulerprompt}
\begin{euleroutput}
  A,B,C
  0.7003664386138074,0.1875530821001213,0.3262339279660414
  0.5926249243193858,0.1522927283984059,0.368140583062521
  0.8065535209872989,0.7265910840408142,0.7332619844597152
  
\end{euleroutput}
\begin{eulercomment}
Fungsi readtable() dalam bentuknya yang paling sederhana dapat membaca
ini dan mengembalikan kumpulan nilai dan baris heading.
\end{eulercomment}
\begin{eulerprompt}
>L=readtable(file,>list);
\end{eulerprompt}
\begin{eulercomment}
Koleksi ini dapat dicetak dengan writetable() ke notebook, atau ke
file.
\end{eulercomment}
\begin{eulerprompt}
>writetable(L,wc=10,dc=5)
\end{eulerprompt}
\begin{euleroutput}
           A         B         C
     0.70037   0.18755   0.32623
     0.59262   0.15229   0.36814
     0.80655   0.72659   0.73326
\end{euleroutput}
\begin{eulercomment}
Matriks nilai adalah elemen pertama dari L. Perhatikan bahwa mean()
dalam EMT menghitung nilai rata-rata dari baris matriks.
\end{eulercomment}
\begin{eulerprompt}
>mean(L[1])
\end{eulerprompt}
\begin{euleroutput}
    0.40472 
    0.37102 
    0.75547 
\end{euleroutput}
\eulerheading{File CSV}
\begin{eulercomment}
Pertama, mari kita menulis matriks ke dalam file. Untuk hasilnya, kami
membuat file di direktori kerja saat ini.
\end{eulercomment}
\begin{eulerprompt}
>file="test.csv";  ...
>M=random(3,3); writematrix(M,file);
\end{eulerprompt}
\begin{eulercomment}
Berikut adalah isi dari file ini.
\end{eulercomment}
\begin{eulerprompt}
>printfile(file)
\end{eulerprompt}
\begin{euleroutput}
  0.8221197733097619,0.821531098722547,0.7771240608094004
  0.8482947121863489,0.3237767724883862,0.6501422353377985
  0.1482301827518109,0.3297459716109594,0.6261901074210923
  
\end{euleroutput}
\begin{eulercomment}
CVS ini dapat dibuka pada sistem bahasa Inggris ke dalam Excel dengan
klik dua kali. Jika Anda mendapatkan file seperti itu di sistem
Jerman, Anda perlu mengimpor data ke Excel dengan memperhatikan titik
desimal.

Tetapi titik desimal juga merupakan format default untuk EMT. Anda
dapat membaca matriks dari file dengan readmatrix().
\end{eulercomment}
\begin{eulerprompt}
>readmatrix(file)
\end{eulerprompt}
\begin{euleroutput}
    0.82212   0.82153   0.77712 
    0.84829   0.32378   0.65014 
    0.14823   0.32975   0.62619 
\end{euleroutput}
\begin{eulercomment}
Dimungkinkan untuk menulis beberapa matriks ke satu file. Perintah
open() dapat membuka file untuk ditulis dengan parameter "w".
Standarnya adalah "r" untuk membaca.
\end{eulercomment}
\begin{eulerprompt}
>open(file,"w"); writematrix(M); writematrix(M'); close();
\end{eulerprompt}
\begin{eulercomment}
Matriks dipisahkan oleh garis kosong. Untuk membaca matriks, buka file
dan panggil readmatrix() beberapa kali.
\end{eulercomment}
\begin{eulerprompt}
>open(file); A=readmatrix(); B=readmatrix(); A==B, close();
\end{eulerprompt}
\begin{euleroutput}
          1         0         0 
          0         1         0 
          0         0         1 
\end{euleroutput}
\begin{eulercomment}
Di Excel atau spreadsheet serupa, Anda dapat mengekspor matriks
sebagai CSV (nilai yang dipisahkan koma). Di Excel 2007, gunakan
"simpan sebagai" dan "format lain", lalu pilih "CSV". Pastikan, tabel
saat ini hanya berisi data yang ingin Anda ekspor.

Ini sebuah contoh.
\end{eulercomment}
\begin{eulerprompt}
>printfile("excel-data.csv")
\end{eulerprompt}
\begin{euleroutput}
  0;1000;1000
  1;1051,271096;1072,508181
  2;1105,170918;1150,273799
  3;1161,834243;1233,67806
  4;1221,402758;1323,129812
  5;1284,025417;1419,067549
  6;1349,858808;1521,961556
  7;1419,067549;1632,31622
  8;1491,824698;1750,6725
  9;1568,312185;1877,610579
  10;1648,721271;2013,752707
\end{euleroutput}
\begin{eulercomment}
Seperti yang Anda lihat, sistem Jerman saya menggunakan titik koma
sebagai pemisah dan koma desimal. Anda dapat mengubahnya di pengaturan
sistem atau di Excel, tetapi tidak perlu membaca matriks ke dalam EMT.

Cara termudah untuk membaca ini ke Euler adalah readmatrix(). Semua
koma diganti dengan titik dengan parameter \textgreater{}koma. Untuk CSV bahasa
Inggris, hilangkan saja parameter ini.
\end{eulercomment}
\begin{eulerprompt}
>M=readmatrix("excel-data.csv",>comma)
\end{eulerprompt}
\begin{euleroutput}
          0      1000      1000 
          1    1051.3    1072.5 
          2    1105.2    1150.3 
          3    1161.8    1233.7 
          4    1221.4    1323.1 
          5      1284    1419.1 
          6    1349.9      1522 
          7    1419.1    1632.3 
          8    1491.8    1750.7 
          9    1568.3    1877.6 
         10    1648.7    2013.8 
\end{euleroutput}
\begin{eulercomment}
Let us plot this.
\end{eulercomment}
\begin{eulerprompt}
>plot2d(M'[1],M'[2:3],>points,color=[red,green]'):
\end{eulerprompt}
\eulerimg{27}{images/Vikram Zaky Ardianto_22305144028_EMT STATISTIKA-038.png}
\begin{eulercomment}
Ada cara yang lebih mendasar untuk membaca data dari file. Anda dapat
membuka file dan membaca angka baris demi baris. Fungsi
getvectorline() akan membaca angka dari baris data. Secara default,
ini mengharapkan titik desimal. Tapi itu juga bisa menggunakan koma
desimal, jika Anda memanggil setdecimaldot(",") sebelum Anda
menggunakan fungsi ini.

Fungsi berikut adalah contoh untuk ini. Itu akan berhenti di akhir
file atau baris kosong.
\end{eulercomment}
\begin{eulerprompt}
>function myload (file) ...
\end{eulerprompt}
\begin{eulerudf}
  open(file);
  M=[];
  repeat
     until eof();
     v=getvectorline(3);
     if length(v)>0 then M=M_v; else break; endif;
  end;
  return M;
  close(file);
  endfunction
\end{eulerudf}
\begin{eulerprompt}
>myload(file)
\end{eulerprompt}
\begin{euleroutput}
    0.82212   0.82153   0.77712 
    0.84829   0.32378   0.65014 
    0.14823   0.32975   0.62619 
\end{euleroutput}
\begin{eulercomment}
Dimungkinkan juga untuk membaca semua angka dalam file itu dengan
getvector().
\end{eulercomment}
\begin{eulerprompt}
>open(file); v=getvector(10000); close(); redim(v[1:9],3,3)
\end{eulerprompt}
\begin{euleroutput}
    0.82212   0.82153   0.77712 
    0.84829   0.32378   0.65014 
    0.14823   0.32975   0.62619 
\end{euleroutput}
\begin{eulercomment}
Thus it is very easy to save a vector of values, one value in each
line and read back this vector.
\end{eulercomment}
\begin{eulerprompt}
>v=random(1000); mean(v)
\end{eulerprompt}
\begin{euleroutput}
  0.50303
\end{euleroutput}
\begin{eulerprompt}
>writematrix(v',file); mean(readmatrix(file)')
\end{eulerprompt}
\begin{euleroutput}
  0.50303
\end{euleroutput}
\eulerheading{Menggunakan Tabel}
\begin{eulercomment}
Tabel dapat digunakan untuk membaca atau menulis data numerik. Sebagai
contoh, kami menulis tabel dengan tajuk baris dan kolom ke file.
\end{eulercomment}
\begin{eulerprompt}
>file="test.tab"; M=random(3,3);  ...
>open(file,"w");  ...
>writetable(M,separator=",",labc=["one","two","three"]);  ...
>close(); ...
>printfile(file)
\end{eulerprompt}
\begin{euleroutput}
  one,two,three
        0.09,      0.39,      0.86
        0.39,      0.86,      0.71
         0.2,      0.02,      0.83
\end{euleroutput}
\begin{eulercomment}
Ini dapat diimpor ke Excel.

Untuk membaca file di EMT, kami menggunakan readtable().
\end{eulercomment}
\begin{eulerprompt}
>\{M,headings\}=readtable(file,>clabs); ...
>writetable(M,labc=headings)
\end{eulerprompt}
\begin{euleroutput}
         one       two     three
        0.09      0.39      0.86
        0.39      0.86      0.71
         0.2      0.02      0.83
\end{euleroutput}
\eulerheading{Menganalisis Garis}
\begin{eulercomment}
Anda bahkan dapat mengevaluasi setiap baris dengan tangan. Misalkan,
kita memiliki garis dengan format berikut.
\end{eulercomment}
\begin{eulerprompt}
>line="2020-11-03,Tue,1'114.05"
\end{eulerprompt}
\begin{euleroutput}
  2020-11-03,Tue,1'114.05
\end{euleroutput}
\begin{eulercomment}
Pertama kita dapat menandai garis.
\end{eulercomment}
\begin{eulerprompt}
>vt=strtokens(line)
\end{eulerprompt}
\begin{euleroutput}
  2020-11-03
  Tue
  1'114.05
\end{euleroutput}
\begin{eulercomment}
Kemudian kita dapat mengevaluasi setiap elemen garis menggunakan
evaluasi yang sesuai.
\end{eulercomment}
\begin{eulerprompt}
>day(vt[1]),  ...
>indexof(["mon","tue","wed","thu","fri","sat","sun"],tolower(vt[2])),  ...
>strrepl(vt[3],"'","")()
\end{eulerprompt}
\begin{euleroutput}
  7.3816e+05
  2
  1114
\end{euleroutput}
\begin{eulercomment}
Menggunakan ekspresi reguler, dimungkinkan untuk mengekstraksi hampir
semua informasi dari sebaris data.

Asumsikan kita memiliki baris berikut sebuah dokumen HTML.
\end{eulercomment}
\begin{eulerprompt}
>line="<tr><td>1145.45</td><td>5.6</td><td>-4.5</td><tr>"
\end{eulerprompt}
\begin{euleroutput}
  <tr><td>1145.45</td><td>5.6</td><td>-4.5</td><tr>
\end{euleroutput}
\begin{eulercomment}
Untuk mengekstrak ini, kami menggunakan ekspresi reguler, yang mencari

\end{eulercomment}
\begin{eulerttcomment}
 - tanda kurung tutup >,
 - string apa pun yang tidak mengandung tanda kurung dengan
\end{eulerttcomment}
\begin{eulercomment}
sub-pertandingan "(...)",\\
\end{eulercomment}
\begin{eulerttcomment}
 - braket pembuka dan penutup menggunakan solusi terpendek,
 - sekali lagi string apa pun yang tidak mengandung tanda kurung,
 - dan tanda kurung buka <.
\end{eulerttcomment}
\begin{eulercomment}

Ekspresi reguler agak sulit dipelajari tetapi sangat kuat.
\end{eulercomment}
\begin{eulerprompt}
>\{pos,s,vt\}=strxfind(line,">([^<>]+)<.+?>([^<>]+)<");
\end{eulerprompt}
\begin{eulercomment}
Hasilnya adalah posisi kecocokan, string yang cocok, dan vektor string
untuk sub-kecocokan.
\end{eulercomment}
\begin{eulerprompt}
>for k=1:length(vt); vt[k](), end;
\end{eulerprompt}
\begin{euleroutput}
  1145.5
  5.6
\end{euleroutput}
\begin{eulercomment}
Ini adalah fungsi yang membaca semua item numerik antara \textless{}td\textgreater{} dan
\textless{}/td\textgreater{}.
\end{eulercomment}
\begin{eulerprompt}
>function readtd (line) ...
\end{eulerprompt}
\begin{eulerudf}
  v=[]; cp=0;
  repeat
     \{pos,s,vt\}=strxfind(line,"<td.*?>(.+?)</td>",cp);
     until pos==0;
     if length(vt)>0 then v=v|vt[1]; endif;
     cp=pos+strlen(s);
  end;
  return v;
  endfunction
\end{eulerudf}
\begin{eulerprompt}
>readtd(line+"<td>non-numerical</td>")
\end{eulerprompt}
\begin{euleroutput}
  1145.45
  5.6
  -4.5
  non-numerical
\end{euleroutput}
\eulerheading{Membaca dari Web}
\begin{eulercomment}
Situs web atau file dengan URL dapat dibuka di EMT dan dapat dibaca
baris demi baris.

Dalam contoh, kami membaca versi terkini dari situs EMT. Kami
menggunakan ekspresi reguler untuk memindai "Versi ..." dalam judul.
\end{eulercomment}
\begin{eulerprompt}
>function readversion () ...
\end{eulerprompt}
\begin{eulerudf}
  urlopen("http://www.euler-math-toolbox.de/Programs/Changes.html");
  repeat
    until urleof();
    s=urlgetline();
    k=strfind(s,"Version ",1);
    if k>0 then substring(s,k,strfind(s,"<",k)-1), break; endif;
  end;
  urlclose();
  endfunction
\end{eulerudf}
\begin{eulerprompt}
>readversion
\end{eulerprompt}
\begin{euleroutput}
  
\end{euleroutput}
\eulerheading{Input dan Output Variabel}
\begin{eulercomment}
Anda dapat menulis variabel dalam bentuk definisi Euler ke file atau
ke baris perintah.
\end{eulercomment}
\begin{eulerprompt}
>writevar(pi,"mypi");
\end{eulerprompt}
\begin{euleroutput}
  mypi = 3.141592653589793;
\end{euleroutput}
\begin{eulercomment}
Untuk pengujian, kami membuat file Euler di direktori kerja EMT.
\end{eulercomment}
\begin{eulerprompt}
>file="test.e"; ...
>writevar(random(2,2),"M",file); ...
>printfile(file,3)
\end{eulerprompt}
\begin{euleroutput}
  M = [ ..
  0.5991820585590205, 0.7960280262224293;
  0.5167243983231363, 0.2996684599070898];
\end{euleroutput}
\begin{eulercomment}
We can now load the file. It will define the matrix M.
\end{eulercomment}
\begin{eulerprompt}
>load(file); show M,
\end{eulerprompt}
\begin{euleroutput}
  M = 
    0.59918   0.79603 
    0.51672   0.29967 
\end{euleroutput}
\begin{eulercomment}
By the way, jika writevar() digunakan pada variabel, itu akan mencetak
definisi variabel dengan nama variabel ini.
\end{eulercomment}
\begin{eulerprompt}
>writevar(M); writevar(inch$)
\end{eulerprompt}
\begin{euleroutput}
  M = [ ..
  0.5991820585590205, 0.7960280262224293;
  0.5167243983231363, 0.2996684599070898];
  inch$ = 0.0254;
\end{euleroutput}
\begin{eulercomment}
Kami juga dapat membuka file baru atau menambahkan file yang sudah
ada. Dalam contoh kami menambahkan file yang dihasilkan sebelumnya.
\end{eulercomment}
\begin{eulerprompt}
>open(file,"a"); ...
>writevar(random(2,2),"M1"); ...
>writevar(random(3,1),"M2"); ...
>close();
>load(file); show M1; show M2;
\end{eulerprompt}
\begin{euleroutput}
  M1 = 
    0.30287   0.15372 
     0.7504   0.75401 
  M2 = 
    0.27213 
   0.053211 
    0.70249 
\end{euleroutput}
\begin{eulercomment}
Untuk menghapus file apa pun gunakan fileremove().
\end{eulercomment}
\begin{eulerprompt}
>fileremove(file);
\end{eulerprompt}
\begin{eulercomment}
Vektor baris dalam file tidak memerlukan koma, jika setiap angka
berada di baris baru. Mari kita buat file seperti itu, menulis setiap
baris satu per satu dengan writeln().
\end{eulercomment}
\begin{eulerprompt}
>open(file,"w"); writeln("M = ["); ...
>for i=1 to 5; writeln(""+random()); end; ...
>writeln("];"); close(); ...
>printfile(file)
\end{eulerprompt}
\begin{euleroutput}
  M = [
  0.344851384551
  0.0807510017715
  0.876519562911
  0.754157709472
  0.688392638934
  ];
\end{euleroutput}
\begin{eulerprompt}
>load(file); M
\end{eulerprompt}
\begin{euleroutput}
  [0.34485,  0.080751,  0.87652,  0.75416,  0.68839]
\end{euleroutput}
\end{eulernotebook}
\end{document}


\end{document}