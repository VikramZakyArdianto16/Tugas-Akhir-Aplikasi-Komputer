\documentclass{article}

\usepackage{eumat}

\begin{document}
\begin{eulernotebook}
\eulersubheading{Vikram Zaky Ardianto}
\eulersubheading{22305144028}
\eulersubheading{Matematika E}
\begin{eulercomment}
\begin{eulercomment}
\eulerheading{Menggambar Grafik Fungsi Satu Variabel}
\begin{eulercomment}
\begin{eulercomment}
\eulerheading{dalam Bentuk Ekspresi Langsung Ekspresi tunggal}
\begin{eulercomment}
Di dalam program numerik EMT, ekspresi adalah string. Jika ditandai
sebagai simbolis, mereka akan mencetak melalui Maxima, jika tidak
melalui EMT. Ekspresi dalam string digunakan untuk membuat plot dan
banyak fungsi numerik. Untuk ini, variabel dalam ekspresi harus "x".

expresi dalam string
\end{eulercomment}
\begin{eulerprompt}
>expr := "x^5-x^2-3"
\end{eulerprompt}
\begin{euleroutput}
  x^5-x^2-3
\end{euleroutput}
\begin{eulercomment}
plot ekspresi
\end{eulercomment}
\begin{eulerprompt}
>plot2d(expr,-2,2) :
\end{eulerprompt}
\eulerimg{27}{images/Vikram Zaky Ardianto_22305144028_Plot 2d-001.png}
\begin{eulercomment}
contoh 1
\end{eulercomment}
\begin{eulerprompt}
>expr := "sin (x-5)"
\end{eulerprompt}
\begin{euleroutput}
  sin (x-5)
\end{euleroutput}
\begin{eulerprompt}
>aspect (1) ; plot2d(expr,-2,2):
\end{eulerprompt}
\eulerimg{27}{images/Vikram Zaky Ardianto_22305144028_Plot 2d-002.png}
\begin{eulercomment}
contoh 2 dan penggunaan grid
\end{eulercomment}
\begin{eulerprompt}
>aspect(1)plot2d("log(x) + 3",-0.1,2, grid=6):
\end{eulerprompt}
\begin{euleroutput}
  Commands must be separated by semicolon or comma!
  Found: plot2d("log(x) + 3",-0.1,2, grid=6): (character 112)
  You can disable this in the Options menu.
  Error in:
  aspect(1)plot2d("log(x) + 3",-0.1,2, grid=6): ...
           ^
\end{euleroutput}
\begin{eulercomment}
contoh 3 dan penggunaan parameter square (atau \textgreater{}square) untuk memilih
y-range secara otomatis 
\end{eulercomment}
\begin{eulerprompt}
>aspect(1,1) ; plot2d("x^4-2",-5,5, >square); insimg(15)
\end{eulerprompt}
\eulerimg{14}{images/Vikram Zaky Ardianto_22305144028_Plot 2d-003.png}
\begin{eulerprompt}
>aspect(2) ; plot2d("x^4-2", -5,5 ):
\end{eulerprompt}
\eulerimg{13}{images/Vikram Zaky Ardianto_22305144028_Plot 2d-004.png}
\begin{eulercomment}
contoh 4 dan memberikan nama atau label pada garis sumbu
\end{eulercomment}
\begin{eulerprompt}
>plot2d("cos(x)", -4, 6, xl="x",yl="y") :
\end{eulerprompt}
\eulerimg{13}{images/Vikram Zaky Ardianto_22305144028_Plot 2d-005.png}
\eulerheading{Menggambar Grafik Fungsi Satu Variabel}
\begin{eulercomment}
* yang Rumusnya Disimpan dalam Variabel Ekspresi

ekspresi
\end{eulercomment}
\begin{eulerprompt}
>expr &= x^5-1
\end{eulerprompt}
\begin{euleroutput}
  
                                   5
                                  x  - 1
  
\end{euleroutput}
\begin{eulercomment}
plot dari ekspresi diatas 
\end{eulercomment}
\begin{eulerprompt}
>aspect(2); plot2d(expr,-1,1):
\end{eulerprompt}
\eulerimg{13}{images/Vikram Zaky Ardianto_22305144028_Plot 2d-006.png}
\begin{eulercomment}
contoh 1
\end{eulercomment}
\begin{eulerprompt}
>expr := "x^10-x-5"
\end{eulerprompt}
\begin{euleroutput}
  x^10-x-5
\end{euleroutput}
\begin{eulerprompt}
>aspect(2) ; plot2d(expr,-1,1):
\end{eulerprompt}
\eulerimg{13}{images/Vikram Zaky Ardianto_22305144028_Plot 2d-007.png}
\begin{eulercomment}
menggunakan variabel lokal
\end{eulercomment}
\begin{eulercomment}
Ekspresi dapat dievaluasi secara numerik. Variabel x,y,z ditetapkan
secara otomatis. Variabel lain dapat ditetapkan berdasarkan parameter
yang ditetapkan( variabel lokal ) atau melalui variabel global.
variabel global adalah variabel yang selalu bisa diakses kapan pun dan
di mana pun.
\end{eulercomment}
\begin{eulerprompt}
>expr &= a*x^5
\end{eulerprompt}
\begin{euleroutput}
  
                                      5
                                   a x
  
\end{euleroutput}
\begin{eulercomment}
menggunakan variabel global 
\end{eulercomment}
\begin{eulerprompt}
>a=6; expr(2.5)
\end{eulerprompt}
\begin{euleroutput}
  585.9375
\end{euleroutput}
\begin{eulercomment}
menggunakan variabel lokal
\end{eulercomment}
\begin{eulerprompt}
>expr(2.5,a=6)
\end{eulerprompt}
\begin{euleroutput}
  585.9375
\end{euleroutput}
\begin{eulercomment}
evaluasi langsung
\end{eulercomment}
\begin{eulerprompt}
>"a*x^5"(3,4)
\end{eulerprompt}
\begin{euleroutput}
  1458
\end{euleroutput}
\begin{eulercomment}
Oleh karena itu, banyak algoritma EMT yang dapat menggunakan ekspresi
dalam x, bukan fungsi. Namun jika parameter tambahan yang tidak
bersifat global dilibatkan, fungsi harus diutamakan.

menggunakan variabel  global "a"
\end{eulercomment}
\begin{eulerprompt}
>a=5; plot2d("a*x^3-x",0,1):
\end{eulerprompt}
\eulerimg{13}{images/Vikram Zaky Ardianto_22305144028_Plot 2d-008.png}
\begin{eulerprompt}
>function f(x,a) := a*x^3-x
\end{eulerprompt}
\begin{eulercomment}
gunakan "a=6" sebagai parameter
\end{eulercomment}
\begin{eulerprompt}
>plot2d("f",0,1;6):
\end{eulerprompt}
\eulerimg{13}{images/Vikram Zaky Ardianto_22305144028_Plot 2d-009.png}
\begin{eulercomment}
alternatif lain
\end{eulercomment}
\begin{eulerprompt}
>plot2d(\{\{"f",6\}\},0,1):
\end{eulerprompt}
\eulerimg{13}{images/Vikram Zaky Ardianto_22305144028_Plot 2d-010.png}
\begin{eulercomment}
alternatif lain 
\end{eulercomment}
\begin{eulerprompt}
>plot2d("f(x,6)",0,1):
\end{eulerprompt}
\eulerimg{13}{images/Vikram Zaky Ardianto_22305144028_Plot 2d-011.png}
\eulerheading{Menggambar Fungsi Simbolik}
\begin{eulercomment}
Fungsi Plot yang paling penting untuk plot planar adalah plot2d().
Fungsi ini diimplementasikan dalam bahasa Euler dalam file "plot.e",
yang dimuat diawal program.

plot2d() menerima ekspresi, fungsi, dan data.

Rentang plot diatur dengan parameter yang ditetapkan ssbagai berikut\\
- a,b: rentang x (default -2,2)\\
- -c,d: rentang y (default: skala dengan nilai)\\
- r: alternatifnya radius di sekitar pusat plot\\
- cx,cy: koordinat pusat plot (default 0,0)

Keterangan:(menggambar grafik fungsi satu variabel yang fungsinya
didefinisikan sebagai fungsi simbolik)\\
- \&: untuk menampilkan variabel pada teks

Berikut adalah beberapa contoh menggunakan fungsi. Seperti biasa di
EMT, fungsi yang berfungsi untuk fungsi atau ekspresi lain, jadi kita
dapat meneruskan parameter tambahan (selain x) yang bukan variabel
global ke fungsi dengan parameter titik koma atau dengan koleksi
panggilan.
\end{eulercomment}
\begin{eulerprompt}
>plot2d("f",0,1;0.4): // plot with a=0.4
\end{eulerprompt}
\eulerimg{13}{images/Vikram Zaky Ardianto_22305144028_Plot 2d-012.png}
\begin{eulerprompt}
>plot2d(\{\{"f",0.2\}\},0,1); 
>plot2d(\{\{"f(x,b)",b=0.1\}\},0,1):
\end{eulerprompt}
\eulerimg{13}{images/Vikram Zaky Ardianto_22305144028_Plot 2d-013.png}
\begin{eulerprompt}
>function f(x) := x^3-x;...
>plot2d("f",r=1):
\end{eulerprompt}
\eulerimg{13}{images/Vikram Zaky Ardianto_22305144028_Plot 2d-014.png}
\begin{eulerprompt}
>plot2d("exp(-a*x^2)/a"):
\end{eulerprompt}
\eulerimg{13}{images/Vikram Zaky Ardianto_22305144028_Plot 2d-015.png}
\begin{eulercomment}
Berikut merupakan ringkasan dari fungsi yang diterima\\
- ekspresi atau ekspresi simbolik dalam x\\
- fungsi atau fungsi simbolis dengan nama sebagai "f"\\
- fungsi simbolis hanya dengan nama f\\
\end{eulercomment}
\begin{eulerttcomment}
 
\end{eulerttcomment}
\begin{eulercomment}
Fungsi plot2d() juga menerima fungsi simbolis. Untuk fungsi simbolis,
hanya nama saja yang berfungsi.
\end{eulercomment}
\begin{eulerprompt}
>function f(x) &= diff(x^x,x)
\end{eulerprompt}
\begin{euleroutput}
  
                              x
                             x  (log(x) + 1)
  
\end{euleroutput}
\begin{eulerprompt}
>plot2d(f,0,2):
\end{eulerprompt}
\eulerimg{13}{images/Vikram Zaky Ardianto_22305144028_Plot 2d-016.png}
\begin{eulerprompt}
>$&expr = sin (x)*exp(-x)
\end{eulerprompt}
\begin{eulerformula}
\[
a\,x^5=e^ {- x }\,\sin x
\]
\end{eulerformula}
\begin{eulerprompt}
>plot2d(expr,0,3pi):
\end{eulerprompt}
\eulerimg{13}{images/Vikram Zaky Ardianto_22305144028_Plot 2d-018.png}
\begin{eulerprompt}
>plot2d("cos(x)","sin(3*x)"):
\end{eulerprompt}
\eulerimg{13}{images/Vikram Zaky Ardianto_22305144028_Plot 2d-019.png}
\eulerheading{Menggambar Fungsi Numerik}
\begin{eulercomment}
Fungsi Numerik adalah sebuah fungsi dengan himpunan bilangan cacah
sebagai domain dan himpunan mendasar yang melibatkan hubungan
matematis antara bilangan yang menjadi domain dan bilangan sebagai
kodomain.
\end{eulercomment}
\begin{eulerprompt}
> 
\end{eulerprompt}
\begin{eulercomment}
Fungsi numerik  memiliki  1  atau  lebih  variabel  independen, yang
sering dilambangkan sebagai "X". Variabel X adalah nilai atau
parameter yang dapat berubah, dan fungsi numerik menggambarkan
bagaimana variabel ini memengaruhi variabel dependen. Variabel
dependen adalah hasil perhitungan atau keluaran dari fungsi numerik
yang bergantung pada nilai atau perubahan dalam variabel independen.

\end{eulercomment}
\begin{eulercomment}
Dalam EMT cara mendefinisikan fungsi menggunakan syntak function.
untuk mendefinisikan fungsi numerik menggunakan tanda ":="

Fungsi  numerik  menjelaskan bagaimana bilangan  dalam  domain
berhubungan dengan bilangan sebagai kodomain, biasanya diberikan dalam
bentuk rumus matematik(persamaan) atau aturan yang memetakan setiap
domain kedalam kodomain yang sesuai. contoh:

f(x)=2x+3
\end{eulercomment}
\begin{eulerprompt}
> 
\end{eulerprompt}
\begin{eulercomment}
(x)(variabel dependen) adalah fungsi yang memetakan setiap nilai
x(variabel independen)kedalam nilai 2x+3. Terdapat berbagai jenis
fungsi yang termasuk ke dalam fungsi numerik, diantaranya:

Fungsi linier dengan bentuk umum\\
f (x) = ax + b
\end{eulercomment}
\begin{eulercomment}
Fungsi kuadrat dengan bentuk umum

f (x) = ax2 + bx + c
\end{eulercomment}
\begin{eulercomment}
Fungsi eksponensial dengan bentuk umum

f (x) = ax
\end{eulercomment}
\begin{eulercomment}
Fungsi logaritma dengan bentuk umum

f (x) = log a(x)

\end{eulercomment}
\begin{eulercomment}
Fungsi trigonometri dengan bentuk umum

f (x) = sin(x), f (x) = cos(x)

\end{eulercomment}
\begin{eulercomment}
Salah satu  cara  yang  umum  digunakan  untuk  memvisualisasikan
fungsi numerik adalah dengan menggambar grafiknya. Grafik ini
menggambarkan bagaimana variabel dependen berubah seiring perubahan
variabel independen dan membantu dalam memahami sifat-sifat fungsi,
seperti titik ekstrim
\end{eulercomment}
\eulersubheading{Contoh soal}
\begin{eulerprompt}
>function r(x):= abs(x-10)
>function s(x):= abs(sin(x))
>r(-5)
\end{eulerprompt}
\begin{euleroutput}
  15
\end{euleroutput}
\begin{eulerprompt}
>function t(x):=log(x*(2+sin(x/1000)))
>function u(x):=integrate("(sin(x)*exp(-x^2)"0,x)
>function v(x):=logbase((x^2),2)
>plot2d("v"):
\end{eulerprompt}
\eulerimg{13}{images/Vikram Zaky Ardianto_22305144028_Plot 2d-020.png}
\begin{eulerprompt}
>plot2d("s"):
\end{eulerprompt}
\eulerimg{13}{images/Vikram Zaky Ardianto_22305144028_Plot 2d-021.png}
\begin{eulerprompt}
>plot2d("t",-2,2):
\end{eulerprompt}
\eulerimg{13}{images/Vikram Zaky Ardianto_22305144028_Plot 2d-022.png}
\begin{eulerprompt}
>function P(x):=x*cos(x)
>plot2d("P",-2*pi,2*pi):
\end{eulerprompt}
\eulerimg{13}{images/Vikram Zaky Ardianto_22305144028_Plot 2d-023.png}
\begin{eulercomment}
Fungsi plot2d() adalah fungsi serbaguna untuk membuat grafik dalam
bidang (grafik 2D). Fungsi ini dapat digunakan untuk membuat grafik
fungsi-fungsi satu variabel, grafik data,  kurva-kurva  dalam  bidang,
grafik batang (bar plots), grid dari bilangan kompleks, dan grafik
implisit dari fungsi dua variabel.

Parameter\\
x,y : persamaan, fungsi, atau vektor data a,b,c,d : area plot (default
a=-2, b=2)\\
r  :  jika  r  diatur,  maka  a=cx-r,  b=cx+r,  c=cy-r,  d=cy+r r bisa
berupa vektor [rx,ry] atau vektor [rx1,rx2,ry1,ry2]. xmin,xmax :
rentang parameter untuk kurva\\
auto : tentukan rentang y secara otomatis (default)\\
square : jika benar, mencoba menjaga rentang x-y tetap persegi n :
jumlah interval (default adalah adaptif)\\
grid : 0 = tanpa grid dan label, 1 = hanya sumbu,\\
2 = grid normal (lihat di bawah untuk jumlah garis grid) 3 = di dalam
sumbu\\
4 = tanpa grid\\
5 = grid penuh termasuk margin 6 = tanda di pinggiran\\
7 = hanya sumbu\\
8 = hanya sumbu, sub-ticks frame : 0 = tanpa bingkai\\
framecolor: warna bingkai dan grid\\
margin : angka antara 0 dan 0,4 untuk margin di sekitar plot color :
Warna kurva. Jika ini adalah vektor warna,akan digunakan untuk setiap
baris matriks plot. Dalam  hal grafik titik, harus berupa vektor
kolom. Jika vektor baris atau matriks penuh warna digunakan untuk
grafik titik, akan digunakan untuk setiap titik data.\\
thickness : ketebalan garis untuk kurva

Nilai ini dapat lebih kecil dari 1 untuk garis yang sangat tipis. \\
style: Gaya plot untuk garis, penanda, dan isian.

Untuk titik gunakan\\
"[]", "\textless{}\textgreater{}", ".", "..", "...", "*", "+", " ", "-", "o"\\
"[]", "\textless{}\textgreater{}", "o" (bentuk terisi)\\
"[]w", "\textless{}\textgreater{}w", "ow" (tidak transparan)

Untuk garis gunakan\\
"-", "-", "-.", ".", ".-.", "-.-", "-\textgreater{}"

Untuk poligon terisi atau plot batang gunakan\\
"", "O", "O", "/", "", "/","+", " ", "-", "t"

points : plot titik tunggal sebagai gantinya garis segmen addpoints :
jika benar, plot segmen garis dan titik\\
add : tambahkan plot ke plot yang ada\\
user : aktifkan interaksi pengguna untuk fungsi delta : ukuran langkah
untuk interaksi pengguna\\
bar : plot batang (x adalah batas interval, y adalah nilai interval)
histogram : plot frekuensi x dalam n subinterval\\
distribusi=n : plot distribusi x dengan n subinterval even : gunakan
nilai antar untuk histogram otomatis. steps : plot fungsi sebagai
fungsi langkah (steps=1,2)\\
adaptive : gunakan plot adaptif (n adalah jumlah minimal langkah)
level : plot garis level dari fungsi implisit dua variabel\\
outline : menggambar batas rentang level.
\end{eulercomment}
\begin{eulerprompt}
>function s(x):=(x-10)
>function r(x):=abs(sin(x))
>s(-5)
\end{eulerprompt}
\begin{euleroutput}
  -15
\end{euleroutput}
\begin{eulerprompt}
>function t(x):=log(x*(2+sin(x/1000)))
>function u(x):=integrate("(sin(x)*exp(-x^2)"),0,x)
>function v(x):=logbase((x^2),2)
>plot2d("v"):
\end{eulerprompt}
\eulerimg{13}{images/Vikram Zaky Ardianto_22305144028_Plot 2d-024.png}
\begin{eulerprompt}
>plot2d("s"):
\end{eulerprompt}
\eulerimg{13}{images/Vikram Zaky Ardianto_22305144028_Plot 2d-025.png}
\begin{eulerprompt}
>function P(x):=x*cos(x)
>plot2d("P", -2*pi,2*pi):
\end{eulerprompt}
\eulerimg{13}{images/Vikram Zaky Ardianto_22305144028_Plot 2d-026.png}
\eulerheading{Menggambar Beberapa Kurva Sekaligus}
\begin{eulercomment}
Dalam subtopik ini, kita akan membahas mengenai cara menggambar
beberapa kurva sekaligus. Dalam hal ini kita dapat menggambar beberapa
kurva dalam jendela grafik yang berbeda secara bersama-sama. Untuk
membuat ini kita dapat menggunakan perintah figure(). Berikut contoh
dari menggambar beberapa kurva sekaligus

Menggambar plot fungsi\\
\end{eulercomment}
\begin{eulerformula}
\[
x^n, 1 \leq n \leq 4
\]
\end{eulerformula}
\begin{eulerprompt}
>reset;
>figure(2,2);...
>for n=1 to 4; figure(n); plot2d("x^"+n); end;...
>figure(0):
\end{eulerprompt}
\eulerimg{27}{images/Vikram Zaky Ardianto_22305144028_Plot 2d-028.png}
\begin{eulercomment}
Penjelasan sintaks dari plot fungsi

\end{eulercomment}
\begin{eulerformula}
\[
x^n,  1 \leq n \leq 4
\]
\end{eulerformula}
\begin{eulercomment}
- reset;\\
Perintah ini berguna untuk menghapus grafik yang telah ada sebelumnya,
sehingga kita dapat memulai dari awal untuk menggambar grafik\\
- figure(2x2);\\
Perintah figure() digunakan untuk membuat jendela grafik dengan ukuran\\
axb. Dalam kasus ini perintah figure(2,2) memiliki makna bahwa jendela
grafik yang dibuat berukuran 2x2. Artinya, akan ada empat jendela
grafik yang akan ditampilkan dengan tata letak 2 baris dan 2 kolom.\\
- for n=1 to 4;\\
Perintah ini digunakan untuk melakukan pengulangan (looping) perintah
sebanyak empat kali, yaitu dari 1 hingga 4.\\
- figure(n);\\
Perintah ini digunakan untuk beralih dari jendela grafik satu ke
jendela grafik lainnya (jendela grafik ke-n).\\
- plot2d("x\textasciicircum{}"+n);\\
Perintah plot2d() digunakan untuk membuat plot fungsi matematika.\\
Dalam hal ini fungsi yang diplot adalah x\textasciicircum{}n, di mana n adalah nilai
dari variabel yang sedang diulang. Dengan kata lain, ini akan membuat\\
plot dari x\textasciicircum{}1, x\textasciicircum{}2, x\textasciicircum{}3, dan x\textasciicircum{}4 dalam jendela grafik yang sesuai\\
- end;\\
Perintah ini menandakan akhir dari looping.\\
- figure(0);\\
Perintah ini digunakan untuk beralih kembali ke jendela grafik utama.
\end{eulercomment}
\begin{eulercomment}
Dari sini dapat kita perhatikan untuk membuat kurva fungsi x\textasciicircum{}n (x
pangkat n) perintahnya tidak ditulis dengan (x\textasciicircum{}n) melainkan ditulis
dengan ("x\textasciicircum{}"+n). Tanda petik dua ("...") digunakan untuk
mengidentifikasi bahwa teks tersebut merupakan ekspresi matematika.\\
Sedangkan tanda (+) digunakan untuk menggabungkan string dengan nilai
yang berubah-ubah atau variabel.

Contoh lain:\\
Menggambar plot fungsi\\
\end{eulercomment}
\begin{eulerformula}
\[
f(x)=x^3-x, -2<x<2
\]
\end{eulerformula}
\begin{eulerprompt}
>reset;
>figure(3,3);...
>for k=1:9; figure(k); plot2d("x^3-x",-2,2,grid=k); end;...
>figure(0):
\end{eulerprompt}
\eulerimg{27}{images/Vikram Zaky Ardianto_22305144028_Plot 2d-029.png}
\begin{eulerttcomment}
 Penjelasan sintaks dari plot fungsi
\end{eulerttcomment}
\begin{eulerformula}
\[
f(x)=x^3-x, -2<x<2
\]
\end{eulerformula}
\begin{eulercomment}
- reset;\\
Perintah ini berguna untuk menghapus grafik yang telah ada sebelumnya,
sehingga kita dapat memulai dari awal untuk menggambar grafik\\
- figure (3,3);\\
Perintah ini digunakan untuk membuat jendela grafik dengan ukuran 3x3.
Artinya, akan ada empat jendela grafik yang akan ditampilkan dengan
tata letak 3 baris dan 3 kolom.\\
- for k=1:9;\\
Perintah ini digunakan untuk melakukan pengulangan (looping) perintah
sebanyak sembilan kali.\\
- figure(n);\\
Perintah ini digunakan untuk beralih dari jendela grafik satu ke\\
\end{eulercomment}
\begin{eulerttcomment}
 jendela grafik lainnya (jendela grafik ke-n).
\end{eulerttcomment}
\begin{eulercomment}
- plot2d("x\textasciicircum{}3-x",-2,2,grid=k);\\
Perintah plot2d() digunakan untuk membuat plot fungsi matematika.\\
Dalam hal ini fungsi yang diplot adalah x\textasciicircum{}3-x, dengan batas sumbu x
dari -2 hingga 2. Argumen grid=k digunakan untuk mengaktifkan grid
pada jendela grafik ke-k.\\
- end;\\
Perintah ini menandakan akhir dari looping.\\
- figure(0);\\
Perintah ini digunakan untuk beralih kembali ke jendela grafik utama.

Dari contoh diatas dapat kita perhatikan bahwa tampilan plot dari yang
ke-1 hingga ke-9 memiliki tampilan yang berbeda-beda. Dalam EMT
memiliki berbagai gaya plot 2D yang dapat dijalankan menggunakan
perintah grid=n dimana n adalah jumlah langkah minimal. Setiap nilai n
memiliki tampilan plot adaptif yang berbeda dalam plot 2D, diantaranya
yaitu:\\
0 : tidak ada grid (kisi), frame, sumbu, dan label, hanya kurva saja\\
1 : dengan sumbu, label-label sumbu di luar frame jendela grafik\\
2 : tampilan default\\
3 : dengan grid pada sumbu x dan y, label-label sumbu berada di dalam
jendela grafik\\
4 : tidak ada grid (kisi), sumbu x dan y, dan label berada di luar
frame jendela grafik\\
5 : tampilan default tanpa margin di sekitar plot\\
6 : hanya dengan sumbu x y dan label, tanpa grid\\
7 : hanya dengan sumbu x y dan tanda-tanda pada sumbu.\\
8 : hanya dengan sumbu dan tanda-tanda pada sumbu, dengan tanda-tanda
yang lebih halus pada sumbu.\\
9 : tampilan default dengan tanda-tanda kecil di dalam jendela\\
10: hanya dengan sumbu-sumbu, tanpa tanda

Contoh lain:\\
Menggambar plot fungsi\\
\end{eulercomment}
\begin{eulerformula}
\[
g(x)=2x^3-x
\]
\end{eulerformula}
\begin{eulerprompt}
>reset;
>aspect(1.2);
>figure(3,4); ...
> figure(2); plot2d("2x^3-x",grid=1); ... // x-y-axis
> figure(3); plot2d("2x^3-x",grid=2); ... // default ticks
>figure(4); plot2d("2x^3-x",grid=3); ... // x-y- axis with labels inside
> figure(5); plot2d("2x^3-x",grid=4); ... // no ticks, only labels
>figure(6); plot2d("2x^3-x",grid=5); ... // default, but no margin
>figure(7); plot2d("2x^3-x",grid=6); ... // axes only
>figure(8); plot2d("2x^3-x",grid=7); ... // axes only, ticks at axis
>figure(9); plot2d("2x^3-x",grid=8); ... // axes only, finer ticks at axis
>figure(10); plot2d("2x^3-x",grid=9); ... // default, small ticks inside
>figure(11); plot2d("2x^3-x",grid=10); ...// no ticks, axes only
>figure(0):
\end{eulerprompt}
\eulerimg{22}{images/Vikram Zaky Ardianto_22305144028_Plot 2d-030.png}
\begin{eulercomment}
Penjelasan sintaks dari plot fungsi\\
\end{eulercomment}
\begin{eulerformula}
\[
g(x)=2x^3-x
\]
\end{eulerformula}
\begin{eulercomment}
- aspect(1.2);\\
Perintah aspect() digunakan untuk mengatur rasio aspek dari jendela
grafik. Hal ini berarti perintah aspect(1.2); akan menghasilkan plot
dengan perbandingan rasio panjang dan lebar 2:1.\\
- figure(3,4);\\
Perintah ini digunakan untuk membuat jendela grafik dengan ukuran 3x4.\\
Jadi, akan ada total 12 jendela grafik yang akan ditampilkan dalam
tata letak 3 baris dan 4 kolom.\\
- figure(1); plot2d("x\textasciicircum{}3-x",grid=0); ...\\
Adalah perintah untuk beralih ke jendela grafik pertama dan menggambar
plot dari fungsi x\textasciicircum{}3 - x tanpa grid, frame, atau sumbu.\\
- figure(2); plot2d("x\textasciicircum{}3-x",grid=1); ...\\
Adalah perintah untuk beralih ke jendela grafik kedua dan menggambar
plot dari fungsi x\textasciicircum{}3 - x dengan grid hanya pada sumbu x dan y.\\
- figure(3); plot2d("x\textasciicircum{}3-x",grid=2); ...\\
Adalah perintah untuk beralih ke jendela grafik ketiga dan menggambar
plot dari fungsi x\textasciicircum{}3 - x dengan tampilan default, termasuk tanda-tanda
default pada sumbu.\\
- figure(4); plot2d("x\textasciicircum{}3-x",grid=3); ...\\
Adalah perintah untuk beralih ke jendela grafik keempat dan menggambar
plot dari fungsi x\textasciicircum{}3 - x dengan grid pada sumbu x dan y, serta
label-label sumbu yang ada di dalam jendela.\\
- figure(5); plot2d("x\textasciicircum{}3-x",grid=4); ...\\
Adalah perintah untuk beralih ke jendela grafik kelima dan menggambar
plot dari fungsi x\textasciicircum{}3 - x tanpa tanda-tanda sumbu, hanya label-label
yang ada.\\
- figure(6); plot2d("x\textasciicircum{}3-x",grid=5); ...\\
Adalah perintah untuk beralih ke jendela grafik keenam dan menggambar
plot dari fungsi x\textasciicircum{}3 - x dengan tampilan default, tetapi tanpa margin
di sekitar plot.\\
- figure(7); plot2d("x\textasciicircum{}3-x",grid=6); ...\\
Adalah perintah untuk beralih ke jendela grafik ketujuh dan menggambar
plot dari fungsi x\textasciicircum{}3 - x hanya dengan sumbu-sumbu (tanpa grid atau
label).\\
- figure(8); plot2d("x\textasciicircum{}3-x",grid=7); ...\\
Adalah perintah untuk beralih ke jendela grafik kedelapan dan
menggambar plot dari fungsi x\textasciicircum{}3 - x hanya dengan sumbu-sumbu dan
tanda-tanda pada sumbu.\\
- figure(9); plot2d("x\textasciicircum{}3-x",grid=8); ...\\
Adalah perintah untuk beralih ke jendela grafik kesembilan dan
menggambar plot dari fungsi x\textasciicircum{}3 - x hanya dengan sumbu-sumbu dan
tanda-tanda pada sumbu, dengan tanda-tanda yang lebih halus pada
sumbu.\\
- figure(10); plot2d("x\textasciicircum{}3-x",grid=9); ...\\
Adalah perintah untuk beralih ke jendela grafik kesepuluh dan
menggambar plot dari fungsi x\textasciicircum{}3 - x dengan tanda-tanda default kecil
di dalam jendela.\\
- figure(11); plot2d("x\textasciicircum{}3-x",grid=10); ...\\
Adalah perintah untuk beralih ke jendela grafik kesebelas dan
menggambar plot dari fungsi x\textasciicircum{}3 - x hanya dengan sumbu-sumbu, tanpa
tanda-tanda.\\
- figure(0);\\
Adalah perintah untuk beralih kembali ke jendela grafik utama atau
jendela grafik dengan nomor 0 setelah semua perintah dalam urutan
selesai dieksekusi.

Dari ketiga contoh di atas, dapat kita katakan bahwa untuk menggambar
beberapa kurva sekaligus itu dapat dilakukan dengan satu baris
perintah ataupun dengan cara mendefinisikannya 1 per 1.

Terlihat beberapa jenis grid memiliki tampilan yang mirip atau sama,
seperti 1 dan 2, 2 dan 5, 4 dan 9, 7 dan 8, untuk dapat membedakannya
secara lebih jelas, ubah grid dari contoh di bawah ini.
\end{eulercomment}
\begin{eulerprompt}
>reset;
>aspect(1.3);
>figure(1,3);...
>figure (1); plot2d("x^2*exp(-x)",0,10);...
>figure (2); plot2d("2*exp(x)",-5,5);...
>figure (3); plot2d("exp(x^2)",-2,2);...
>figure (0):
\end{eulerprompt}
\eulerimg{20}{images/Vikram Zaky Ardianto_22305144028_Plot 2d-031.png}
\begin{eulercomment}
Contoh lain:
\end{eulercomment}
\begin{eulerprompt}
>reset;
>aspect(3/4);
>figure(2,1);...
>for a=1:2; figure(a); plot2d("2*x*log(x^2)",0,3,grid=a); end;...
>figure(0):
\end{eulerprompt}
\eulerimg{34}{images/Vikram Zaky Ardianto_22305144028_Plot 2d-032.png}
\eulerheading{Menggambar Beberapa Kurva pada bidang }
\begin{eulercomment}
* koordinat yang sama 

Plot lebih dari satu fungsi (multiple function) ke dalam satu jendela
dapat dilakukan dengan berbagai cara. Salah satu caranya adalah
menggunakan \textgreater{}add untuk beberapa panggilan ke plot2d secara
keseluruhan, kecuali panggilan pertama.

Berikut contohnya:\\
menggambar kurva\\
\end{eulercomment}
\begin{eulerformula}
\[
 f(x)=cos(x)
\]
\end{eulerformula}
\begin{eulerformula}
\[
f(x)= x^2
\]
\end{eulerformula}
\begin{eulerprompt}
>aspect(); plot2d("cos(x)",r=3); plot2d("x^2",style=".",>add):
\end{eulerprompt}
\eulerimg{27}{images/Vikram Zaky Ardianto_22305144028_Plot 2d-035.png}
\begin{eulerformula}
\[
f(x)=cos(x)-1
\]
\end{eulerformula}
\begin{eulerformula}
\[
f(x)= sin(x)-1
\]
\end{eulerformula}
\begin{eulerprompt}
>aspect(2); plot2d("cos(x)-1",-1,6); plot2d("sin(x)-1",style="--",>add):
\end{eulerprompt}
\eulerimg{13}{images/Vikram Zaky Ardianto_22305144028_Plot 2d-036.png}
\begin{eulercomment}
Selain menggunakan \textgreater{}add kita juga bisa menambahkannya secara langsung

Berikut contohnya:\\
Menggambar kurva\\
\end{eulercomment}
\begin{eulerformula}
\[
f(x)= 2x+1
\]
\end{eulerformula}
\begin{eulerformula}
\[
f(x)= -2x+1
\]
\end{eulerformula}
\begin{eulerprompt}
>plot2d(["2x+1","x"],0,8):
\end{eulerprompt}
\eulerimg{13}{images/Vikram Zaky Ardianto_22305144028_Plot 2d-037.png}
\begin{eulerformula}
\[
f(x)=sin(2x)
\]
\end{eulerformula}
\begin{eulerformula}
\[
f(x)=cos(3x)
\]
\end{eulerformula}
\begin{eulerprompt}
>aspect(1.5); plot2d(["sin(2x)","cos(3x)"],0,8):
\end{eulerprompt}
\eulerimg{17}{images/Vikram Zaky Ardianto_22305144028_Plot 2d-038.png}
\begin{eulercomment}
Kegunaan \textgreater{}add yang lain juga bisa untuk menambahkan titik pada kurva.

Berikut contohnya:\\
Menambahkan sebuah titik di\\
\end{eulercomment}
\begin{eulerformula}
\[
f(x)= x+4
\]
\end{eulerformula}
\begin{eulerprompt}
>aspect(); plot2d("x+4",-2,5,); plot2d(2,6,>points,>add):
\end{eulerprompt}
\eulerimg{27}{images/Vikram Zaky Ardianto_22305144028_Plot 2d-039.png}
\begin{eulercomment}
Kita juga bisa mencari titik perpotongan dengan cara berikut:

\end{eulercomment}
\begin{eulerformula}
\[
sin(x)=2x
\]
\end{eulerformula}
\begin{eulerprompt}
>plot2d(["sin(x)","2x"],r=2,cx=1,cy=1, ...
>  color=[black,blue],style=["-","."], ...
>  grid=1);
>x0=solve("sin(x)-2x",1);  ...
>  plot2d(x0,x0,>points,>add);  ...
>  label("sin(x) = 2x",x0,x0,pos="cl",offset=20):
\end{eulerprompt}
\eulerimg{27}{images/Vikram Zaky Ardianto_22305144028_Plot 2d-040.png}
\begin{eulerprompt}
>function f(x,a) := x^2+a*x-x/a; ...
>plot2d("f",-10,10;1,title="a=1"):
\end{eulerprompt}
\eulerimg{27}{images/Vikram Zaky Ardianto_22305144028_Plot 2d-041.png}
\begin{eulerprompt}
> plot2d(\{\{"f",1\}\},-10,10); ...
>for a=1:10; plot2d(\{\{"f",a\}\},>add); end:
\end{eulerprompt}
\eulerimg{27}{images/Vikram Zaky Ardianto_22305144028_Plot 2d-042.png}
\begin{eulerprompt}
>function f(x,a) := x^2*exp(-x^2/a); ...
>plot2d("f",-10,10;5,thickness=2,title="a=5"):
\end{eulerprompt}
\eulerimg{27}{images/Vikram Zaky Ardianto_22305144028_Plot 2d-043.png}
\begin{eulerprompt}
>plot2d(\{\{"f",1\}\},-8,8); ...
>for a=2:5; plot2d(\{\{"f",a\}\},>add,thickness=2); end:
\end{eulerprompt}
\eulerimg{27}{images/Vikram Zaky Ardianto_22305144028_Plot 2d-044.png}
\begin{eulerprompt}
>aspect(2.1); &plot2d(1/x,[x,-1,1]):
\end{eulerprompt}
\eulerimg{27}{images/Vikram Zaky Ardianto_22305144028_Plot 2d-045.png}
\begin{eulerprompt}
>x=linspace(-1,1,50);...
>plot2d("1/x"):
\end{eulerprompt}
\eulerimg{12}{images/Vikram Zaky Ardianto_22305144028_Plot 2d-046.png}
\eulerheading{Menuliskan Label koordinat,label kurva,}
\begin{eulercomment}
* dan keterangan kurva(legend)

Dalam EMT, untuk menambahkan judul dapat dilakukan dengan title="..."\\
untuk menambahkan sumbu x dan sumbu y dapat dilakukan dengan x1="...",
y1="..."\\
sebagai contoh:
\end{eulercomment}
\begin{eulerprompt}
>plot2d("x^2-4*x"):
\end{eulerprompt}
\eulerimg{12}{images/Vikram Zaky Ardianto_22305144028_Plot 2d-047.png}
\begin{eulercomment}
untuk menambahkan judul dapat dilakukan dengan title="..."\\
untuk menambahkan sumbu x dan sumbu y dapat dilakukan dengan x1="...",
y1="..."
\end{eulercomment}
\begin{eulerprompt}
>plot2d("x^2-4*x",title="FUNGSI y=x^2-4*x",yl="Sumbu y",xl="Sumbu x"):
\end{eulerprompt}
\eulerimg{12}{images/Vikram Zaky Ardianto_22305144028_Plot 2d-048.png}
\begin{eulercomment}
Selain itu juga dapat dengan cara lain seperti contoh berikut:
\end{eulercomment}
\begin{eulerprompt}
>expr := "x^3-x"; ...
>  plot2d(expr,title="y="+expr,xl="Sumbu x",yl="Sumbu y"); ...
>  label("(1,0)",1,0);  label("Max",E,expr(E),pos="lc"): 
\end{eulerprompt}
\eulerimg{12}{images/Vikram Zaky Ardianto_22305144028_Plot 2d-049.png}
\eulerheading{Mengatur ukuran gambar,format(style),dan warna kurva}
\begin{eulercomment}
Untuk mengubah ukuran, dapat dilakukan dengan menggunakan
aspect="...", semakin besar nilai aspect, maka ukuran kurva akan
semakin kecil, begitupun sebaliknya

untuk mengganti style, dapat dipilih dengan berbagai pilihan\\
style="...", dapat dipilih dari, misal : "-","\_',"-.",".-.","-.-".

untuk warna dapat dipilih sebagai salah satu warna default\\
color="...", warna default= red,green,blue,yellow, dll

sebagai contoh:
\end{eulercomment}
\begin{eulerprompt}
>aspect(1); plot2d("exp(x^2-3)"):
\end{eulerprompt}
\eulerimg{27}{images/Vikram Zaky Ardianto_22305144028_Plot 2d-050.png}
\begin{eulercomment}
ukuran kurva dapat diganti dengan mengganti nilai aspect="...",
semakin besar nilai aspect, maka ukuran kurva akan semakin kecil Untuk
mengganti warna dapat ditambahkan dengan color="...", sedangkan untuk
mengganti format(style) dapat dilakukan dengan menambahkan style="..."
\end{eulercomment}
\begin{eulerprompt}
>aspect(2); plot2d("exp(x^2-3)", color=red, style="--"):
\end{eulerprompt}
\eulerimg{13}{images/Vikram Zaky Ardianto_22305144028_Plot 2d-051.png}
\begin{eulercomment}
Berikut adalah tampilan warna EMT yang telah ditentukan
\end{eulercomment}
\begin{eulerprompt}
>aspect (1) ; columnsplot (ones(1,16),lab=0:15,grid=0, color=0:15) :
\end{eulerprompt}
\eulerimg{27}{images/Vikram Zaky Ardianto_22305144028_Plot 2d-052.png}
\begin{eulercomment}
selain menggunakan warna default, untuk mengubah warna dapat juga
dengan menggunakan kode warna di atas\\
sebagai contoh:
\end{eulercomment}
\begin{eulerprompt}
>aspect(1); plot2d("exp(x^3+2*x)",r=3, color=1, style="--"):
\end{eulerprompt}
\eulerimg{27}{images/Vikram Zaky Ardianto_22305144028_Plot 2d-053.png}
\begin{euleroutput}
  
\end{euleroutput}
\eulerheading{Membuat Gambar Kurva yang Bersifat Interaktif}
\begin{eulercomment}
Saat memplot fungsi atau ekspresi, parameter \textgreater{}user memungkinkan
pengguna untuk memperbesar dan menggeser plot dengan tombol kursor
atau mouse. Pengguna dapat

- perbesar dengan + atau -\\
- pindahkan plot dengan tombol kursor\\
- pilih jendela plot dengan mouse\\
- atur ulang tampilan dengan spasi\\
- keluar dengan kembali

Tombol spasi akan mengatur ulang plot ke jendela plot asli.

Saat memplot data, flag \textgreater{}user hanya akan menunggu penekanan tombol.
\end{eulercomment}
\begin{eulerprompt}
>plot2d(\{\{"x^3-a*x",a=1\}\},>user,title="Press any key!"); ...
>insimg;  
\end{eulerprompt}
\eulerimg{27}{images/Vikram Zaky Ardianto_22305144028_Plot 2d-054.png}
\begin{eulerprompt}
>plot2d("exp(x)*sin(x)",user=true, ...
>  title="+/- or cursor keys (return to exit)"):
\end{eulerprompt}
\eulerimg{27}{images/Vikram Zaky Ardianto_22305144028_Plot 2d-055.png}
\begin{eulercomment}
Berikut ini menunjukkan cara interaksi pengguna tingkat lanjut (lihat
tutorial tentang pemrograman untuk detailnya).

Fungsi bawaan mousedrag() menunggu event mouse atau keyboard. Ini
melaporkan mouse ke bawah, mouse dipindahkan atau mouse ke atas, dan
penekanan tombol. Fungsi dragpoints() memanfaatkan ini, dan
memungkinkan pengguna menyeret titik mana pun dalam plot.

Kita membutuhkan fungsi plot terlebih dahulu. Sebagai contoh, kita
interpolasi dalam 5 titik dengan polinomial. Fungsi harus diplot ke
area plot tetap.
\end{eulercomment}
\begin{eulerprompt}
>function plotf(xp,yp,select) ...
\end{eulerprompt}
\begin{eulerudf}
    d=interp(xp,yp);
    plot2d("interpval(xp,d,x)";d,xp,r=2);
    plot2d(xp,yp,>points,>add);
    if select>0 then
      plot2d(xp[select],yp[select],color=red,>points,>add);
    endif;
    title("Drag one point, or press space or return!");
  endfunction
\end{eulerudf}
\begin{eulercomment}
Perhatikan parameter titik koma di plot2d (d dan xp), yang diteruskan
ke evaluasi fungsi interp(). Tanpa ini, kita harus menulis fungsi
plotinterp() terlebih dahulu, mengakses nilai secara global.

Sekarang kita menghasilkan beberapa nilai acak, dan membiarkan
pengguna menyeret poin.
\end{eulercomment}
\begin{eulerprompt}
>t=-1:0.5:1; dragpoints("plotf",t,random(size(t))-0.5):
\end{eulerprompt}
\eulerimg{27}{images/Vikram Zaky Ardianto_22305144028_Plot 2d-056.png}
\begin{eulercomment}
Ada juga fungsi, yang memplot fungsi lain tergantung pada vektor
parameter, dan memungkinkan pengguna menyesuaikan parameter ini.

Pertama kita membutuhkan fungsi plot.
\end{eulercomment}
\begin{eulerprompt}
>function plotf([a,b]) := plot2d("exp(a*x)*cos(2pi*b*x)",0,2pi;a,b);
\end{eulerprompt}
\begin{eulercomment}
Kemudian kita membutuhkan nama untuk parameter, nilai awal dan matriks
rentang nx2, opsional baris judul.\\
Ada slider interaktif, yang dapat mengatur nilai oleh pengguna. Fungsi
dragvalues() menyediakan ini.
\end{eulercomment}
\begin{eulerprompt}
>dragvalues("plotf",["a","b"],[-1,2],[[-2,2];[1,10]], ...
>  heading="Drag these values:",hcolor=black):
\end{eulerprompt}
\eulerimg{27}{images/Vikram Zaky Ardianto_22305144028_Plot 2d-057.png}
\begin{eulercomment}
Dimungkinkan untuk membatasi nilai yang diseret ke bilangan bulat.
Sebagai contoh, kita menulis fungsi plot, yang memplot polinomial
Taylor derajat n ke fungsi kosinus.
\end{eulercomment}
\begin{eulerprompt}
>function plotf(n) ...
\end{eulerprompt}
\begin{eulerudf}
  plot2d("cos(x)",0,2pi,>square,grid=6);
  plot2d(&"taylor(cos(x),x,0,@n)",color=blue,>add);
  textbox("Taylor polynomial of degree "+n,0.1,0.02,style="t",>left);
  endfunction
\end{eulerudf}
\begin{eulercomment}
Sekarang kami mengizinkan derajat n bervariasi dari 0 hingga 20 dalam
20 pemberhentian. Hasil dragvalues() digunakan untuk memplot sketsa
dengan n ini, dan untuk memasukkan plot ke dalam buku catatan.
\end{eulercomment}
\begin{eulerprompt}
>nd=dragvalues("plotf","degree",2,[0,20],20,y=0.8, ...
>   heading="Drag the value:"); ...
>plotf(nd):
\end{eulerprompt}
\eulerimg{27}{images/Vikram Zaky Ardianto_22305144028_Plot 2d-058.png}
\begin{eulercomment}
Berikut ini adalah demonstrasi sederhana dari fungsi tersebut.
Pengguna dapat menggambar di atas jendela plot, meninggalkan jejak
poin.
\end{eulercomment}
\begin{eulerprompt}
>function dragtest ...
\end{eulerprompt}
\begin{eulerudf}
    plot2d(none,r=1,title="Drag with the mouse, or press any key!");
    start=0;
    repeat
      \{flag,m,time\}=mousedrag();
      if flag==0 then return; endif;
      if flag==2 then
        hold on; mark(m[1],m[2]); hold off;
      endif;
    end
  endfunction
\end{eulerudf}
\begin{eulerprompt}
>dragtest // lihat hasilnya dan cobalah lakukan!
\end{eulerprompt}
\eulerheading{Menggambar Sekumpulan Kurva dengan Satu Perintah}
\begin{eulercomment}
Secara default, EMT menghitung tick sumbu otomatis dan menambahkan
label ke setiap tick. Ini dapat diubah dengan parameter grid. Gaya
default sumbu dan label dapat dimodifikasi. Selain itu, label dan
judul dapat ditambahkan secara manual. Untuk mengatur ulang ke gaya
default, gunakan reset().
\end{eulercomment}
\begin{eulerprompt}
>aspect();
>figure(3,4); ...
> figure(1); plot2d("x^3-x",grid=0); ... // no grid, frame or axis
> figure(2); plot2d("x^3-x",grid=1); ... // x-y-axis
> figure(3); plot2d("x^3-x",grid=2); ... // default ticks
> figure(4); plot2d("x^3-x",grid=3); ... // x-y- axis with labels inside
> figure(5); plot2d("x^3-x",grid=4); ... // no ticks, only labels
> figure(6); plot2d("x^3-x",grid=5); ... // default, but no margin
> figure(7); plot2d("x^3-x",grid=6); ... // axes only
> figure(8); plot2d("x^3-x",grid=7); ... // axes only, ticks at axis
> figure(9); plot2d("x^3-x",grid=8); ... // axes only, finer ticks at axis
> figure(10); plot2d("x^3-x",grid=9); ... // default, small ticks inside
> figure(11); plot2d("x^3-x",grid=10); ...// no ticks, axes only
> figure(0):
\end{eulerprompt}
\eulerimg{27}{images/Vikram Zaky Ardianto_22305144028_Plot 2d-059.png}
\begin{eulercomment}
Parameter \textless{}frame mematikan frame, dan framecolor=blue mengatur frame
ke warna biru.

Jika Anda ingin centang sendiri, Anda dapat menggunakan style=0, dan
menambahkan semuanya nanti.
\end{eulercomment}
\begin{eulerprompt}
>aspect(1.5); 
>plot2d("x^3-x",grid=0); // plot
>frame; xgrid([-1,0,1]); ygrid(0): // add frame and grid
\end{eulerprompt}
\eulerimg{17}{images/Vikram Zaky Ardianto_22305144028_Plot 2d-060.png}
\begin{eulercomment}
Untuk judul plot dan label sumbu, lihat contoh berikut.
\end{eulercomment}
\begin{eulerprompt}
>plot2d("exp(x)",-1,1);
>textcolor(black); // set the text color to black
>title(latex("y=e^x")); // title above the plot
>xlabel(latex("x")); // "x" for x-axis
>ylabel(latex("y"),>vertical); // vertical "y" for y-axis
>label(latex("(0,1)"),0,1,color=blue): // label a point
\end{eulerprompt}
\eulerimg{17}{images/Vikram Zaky Ardianto_22305144028_Plot 2d-061.png}
\begin{eulercomment}
Sumbu dapat digambar secara terpisah dengan xaxis() dan yaxis().
\end{eulercomment}
\begin{eulerprompt}
>plot2d("x^3-x",<grid,<frame);
>xaxis(0,xx=-2:1,style="->"); yaxis(0,yy=-5:5,style="->"):
\end{eulerprompt}
\eulerimg{17}{images/Vikram Zaky Ardianto_22305144028_Plot 2d-062.png}
\begin{eulercomment}
Teks pada plot dapat diatur dengan label(). Dalam contoh berikut, "lc"
berarti tengah bawah. Ini mengatur posisi label relatif terhadap
koordinat plot.
\end{eulercomment}
\begin{eulerprompt}
>function f(x) &= x^3-x
\end{eulerprompt}
\begin{euleroutput}
  
                                   3
                                  x  - x
  
\end{euleroutput}
\begin{eulerprompt}
>plot2d(f,-1,1,>square);
>x0=fmin(f,0,1); // compute point of minimum
>label("Rel. Min.",x0,f(x0),pos="lc"): // add a label there
\end{eulerprompt}
\eulerimg{17}{images/Vikram Zaky Ardianto_22305144028_Plot 2d-063.png}
\begin{eulercomment}
Ada juga kotak teks.
\end{eulercomment}
\begin{eulerprompt}
>plot2d(&f(x),-1,1,-2,2); // function
>plot2d(&diff(f(x),x),>add,style="--",color=red); // derivative
>labelbox(["f","f'"],["-","--"],[black,red]): // label box
\end{eulerprompt}
\eulerimg{17}{images/Vikram Zaky Ardianto_22305144028_Plot 2d-064.png}
\begin{eulerprompt}
>plot2d(["exp(x)","1+x"],color=[black,blue],style=["-","-.-"]):
\end{eulerprompt}
\eulerimg{17}{images/Vikram Zaky Ardianto_22305144028_Plot 2d-065.png}
\begin{eulerprompt}
>gridstyle("->",color=gray,textcolor=gray,framecolor=gray);  ...
> plot2d("x^3-x",grid=1);   ...
> settitle("y=x^3-x",color=black); ...
> label("x",2,0,pos="bc",color=gray);  ...
> label("y",0,6,pos="cl",color=gray); ...
> reset():
\end{eulerprompt}
\eulerimg{27}{images/Vikram Zaky Ardianto_22305144028_Plot 2d-066.png}
\begin{eulercomment}
Untuk kontrol lebih, sumbu x dan sumbu y dapat dilakukan secara
manual.

Perintah fullwindow() memperluas jendela plot karena kita tidak lagi
membutuhkan tempat untuk label di luar jendela plot. Gunakan
shrinkwindow() atau reset() untuk mengatur ulang ke default.
\end{eulercomment}
\begin{eulerprompt}
>fullwindow; ...
> gridstyle(color=darkgray,textcolor=darkgray); ...
> plot2d(["2^x","1","2^(-x)"],a=-2,b=2,c=0,d=4,<grid,color=4:6,<frame); ...
> xaxis(0,-2:1,style="->"); xaxis(0,2,"x",<axis); ...
> yaxis(0,4,"y",style="->"); ...
> yaxis(-2,1:4,>left); ...
> yaxis(2,2^(-2:2),style=".",<left); ...
> labelbox(["2^x","1","2^-x"],colors=4:6,x=0.8,y=0.2); ...
> reset:
\end{eulerprompt}
\eulerimg{27}{images/Vikram Zaky Ardianto_22305144028_Plot 2d-067.png}
\begin{eulercomment}
Berikut adalah contoh lain, di mana string Unicode digunakan dan sumbu
di luar area plot.
\end{eulercomment}
\begin{eulerprompt}
>aspect(1.5); 
>plot2d(["sin(x)","cos(x)"],0,2pi,color=[red,green],<grid,<frame); ...
> xaxis(-1.1,(0:2)*pi,xt=["0",u"&pi;",u"2&pi;"],style="-",>ticks,>zero);  ...
> xgrid((0:0.5:2)*pi,<ticks); ...
> yaxis(-0.1*pi,-1:0.2:1,style="-",>zero,>grid); ...
> labelbox(["sin","cos"],colors=[red,green],x=0.5,y=0.2,>left); ...
> xlabel(u"&phi;"); ylabel(u"f(&phi;)"):
\end{eulerprompt}
\eulerimg{17}{images/Vikram Zaky Ardianto_22305144028_Plot 2d-068.png}
\eulerheading{Merencanakan Data 2D}
\begin{eulercomment}
Jika x dan y adalah vektor data, data ini akan digunakan sebagai
koordinat x dan y dari suatu kurva. Dalam hal ini, a, b, c, dan d,
atau radius r dapat ditentukan, atau jendela plot akan menyesuaikan
secara otomatis dengan data. Atau, \textgreater{}persegi dapat diatur untuk menjaga
rasio aspek persegi.

Memplot ekspresi hanyalah singkatan untuk plot data. Untuk plot data,
Anda memerlukan satu atau beberapa baris nilai x, dan satu atau
beberapa baris nilai y. Dari rentang dan nilai-x, fungsi plot2d akan
menghitung data yang akan diplot, secara default dengan evaluasi
fungsi yang adaptif. Untuk plot titik gunakan "\textgreater{}points", untuk garis
campuran dan titik gunakan "\textgreater{}addpoints".

Tapi Anda bisa memasukkan data secara langsung.

- Gunakan vektor baris untuk x dan y untuk satu fungsi.\\
- Matriks untuk x dan y diplot baris demi baris.

Berikut adalah contoh dengan satu baris untuk x dan y.
\end{eulercomment}
\begin{eulerprompt}
>x=-10:0.1:10; y=exp(-x^2)*x; plot2d(x,y):
\end{eulerprompt}
\eulerimg{17}{images/Vikram Zaky Ardianto_22305144028_Plot 2d-069.png}
\begin{eulercomment}
Data juga dapat diplot sebagai titik. Gunakan poin=true untuk ini.
Plotnya bekerja seperti poligon, tetapi hanya menggambar
sudut-sudutnya.

- style="...": Pilih dari "[]", "\textless{}\textgreater{}", "o", ".", "..", "+", "*", "[]#",
"\textless{} \textgreater{}#", "o#", "..#", "#", "\textbar{}".

Untuk memplot set poin gunakan \textgreater{}points. Jika warna adalah vektor
warna, setiap titik\\
mendapat warna yang berbeda. Untuk matriks koordinat dan vektor kolom,
warna berlaku untuk baris matriks.\\
Parameter \textgreater{}addpoints menambahkan titik ke segmen garis untuk plot
data.
\end{eulercomment}
\begin{eulerprompt}
>xdata=[1,1.5,2.5,3,4]; ydata=[3,3.1,2.8,2.9,2.7]; // data
>plot2d(xdata,ydata,a=0.5,b=4.5,c=2.5,d=3.5,style="."); // lines
>plot2d(xdata,ydata,>points,>add,style="o"): // add points
\end{eulerprompt}
\eulerimg{17}{images/Vikram Zaky Ardianto_22305144028_Plot 2d-070.png}
\begin{eulerprompt}
>p=polyfit(xdata,ydata,1); // get regression line
>plot2d("polyval(p,x)",>add,color=red): // add plot of line
\end{eulerprompt}
\eulerimg{17}{images/Vikram Zaky Ardianto_22305144028_Plot 2d-071.png}
\eulerheading{Menggambar Daerah Yang Dibatasi Kurva}
\begin{eulercomment}
Plot data benar-benar poligon. Kita juga dapat memplot kurva atau
kurva terisi.

- terisi=benar mengisi plot.\\
- style="...": Pilih dari "#", "/", "\textbackslash{}", "\textbackslash{}/".\\
- fillcolor: Lihat di atas untuk warna yang tersedia.

Warna isian ditentukan oleh argumen "fillcolor", dan pada \textless{}outline
opsional mencegah menggambar batas untuk semua gaya kecuali yang
default.
\end{eulercomment}
\begin{eulerprompt}
>t=linspace(0,2pi,1000); // parameter for curve
>x=sin(t)*exp(t/pi); y=cos(t)*exp(t/pi); // x(t) and y(t)
>figure(1,2); aspect(16/9)
>figure(1); plot2d(x,y,r=10); // plot curve
>figure(2); plot2d(x,y,r=10,>filled,style="/",fillcolor=red); // fill curve
>figure(0):
\end{eulerprompt}
\eulerimg{14}{images/Vikram Zaky Ardianto_22305144028_Plot 2d-072.png}
\begin{eulercomment}
Dalam contoh berikut kami memplot elips terisi dan dua segi enam
terisi menggunakan kurva tertutup dengan 6 titik dengan gaya isian
berbeda.
\end{eulercomment}
\begin{eulerprompt}
>x=linspace(0,2pi,1000); plot2d(sin(x),cos(x)*0.5,r=1,>filled,style="/"):
\end{eulerprompt}
\eulerimg{14}{images/Vikram Zaky Ardianto_22305144028_Plot 2d-073.png}
\begin{eulerprompt}
>t=linspace(0,2pi,6); ...
>plot2d(cos(t),sin(t),>filled,style="/",fillcolor=red,r=1.2):
\end{eulerprompt}
\eulerimg{14}{images/Vikram Zaky Ardianto_22305144028_Plot 2d-074.png}
\begin{eulerprompt}
>t=linspace(0,2pi,6); plot2d(cos(t),sin(t),>filled,style="#"):
\end{eulerprompt}
\eulerimg{14}{images/Vikram Zaky Ardianto_22305144028_Plot 2d-075.png}
\begin{eulercomment}
Contoh lainnya adalah segi tujuh, yang kita buat dengan 7 titik pada
lingkaran satuan.
\end{eulercomment}
\begin{eulerprompt}
>t=linspace(0,2pi,7);  ...
> plot2d(cos(t),sin(t),r=1,>filled,style="/",fillcolor=red):
\end{eulerprompt}
\eulerimg{14}{images/Vikram Zaky Ardianto_22305144028_Plot 2d-076.png}
\begin{eulercomment}
Berikut ini adalah himpunan nilai maksimal dari empat kondisi linier
yang kurang dari atau sama dengan 3. Ini adalah A[k].v\textless{}=3 untuk semua
baris A. Untuk mendapatkan sudut yang bagus, kita menggunakan n yang
relatif besar.
\end{eulercomment}
\begin{eulerprompt}
>A=[2,1;1,2;-1,0;0,-1];
>function f(x,y) := max([x,y].A');
>plot2d("f",r=4,level=[0;3],color=green,n=111):
\end{eulerprompt}
\eulerimg{14}{images/Vikram Zaky Ardianto_22305144028_Plot 2d-077.png}
\begin{eulercomment}
Poin utama dari bahasa matriks adalah memungkinkan untuk menghasilkan
tabel fungsi dengan mudah.
\end{eulercomment}
\begin{eulerprompt}
>t=linspace(0,2pi,1000); x=cos(3*t); y=sin(4*t);
\end{eulerprompt}
\begin{eulercomment}
Kami sekarang memiliki vektor x dan y nilai. plot2d() dapat memplot
nilai-nilai ini\\
sebagai kurva yang menghubungkan titik-titik. Plotnya bisa diisi. Pada
kasus ini\\
ini menghasilkan hasil yang bagus karena aturan lilitan, yang
digunakan untuk\\
isi.
\end{eulercomment}
\begin{eulerprompt}
>plot2d(x,y,<grid,<frame,>filled):
\end{eulerprompt}
\eulerimg{14}{images/Vikram Zaky Ardianto_22305144028_Plot 2d-078.png}
\begin{eulercomment}
Sebuah vektor interval diplot terhadap nilai x sebagai daerah terisi\\
antara nilai interval bawah dan atas.

Hal ini dapat berguna untuk memplot kesalahan perhitungan. Tapi itu
bisa\\
juga digunakan untuk memplot kesalahan statistik.
\end{eulercomment}
\begin{eulerprompt}
>t=0:0.1:1; ...
> plot2d(t,interval(t-random(size(t)),t+random(size(t))),style="|");  ...
> plot2d(t,t,add=true):
\end{eulerprompt}
\eulerimg{14}{images/Vikram Zaky Ardianto_22305144028_Plot 2d-079.png}
\begin{eulercomment}
Jika x adalah vektor yang diurutkan, dan y adalah vektor interval,
maka plot2d akan memplot rentang interval yang terisi dalam bidang.
Gaya isian sama dengan gaya poligon.
\end{eulercomment}
\begin{eulerprompt}
>t=-1:0.01:1; x=~t-0.01,t+0.01~; y=x^3-x;
>plot2d(t,y):
\end{eulerprompt}
\eulerimg{14}{images/Vikram Zaky Ardianto_22305144028_Plot 2d-080.png}
\begin{eulercomment}
Dimungkinkan untuk mengisi wilayah nilai untuk fungsi tertentu. Untuk\\
ini, level harus berupa matriks 2xn. Baris pertama adalah batas bawah\\
dan baris kedua berisi batas atas.
\end{eulercomment}
\begin{eulerprompt}
>expr := "2*x^2+x*y+3*y^4+y"; // define an expression f(x,y)
>plot2d(expr,level=[0;1],style="-",color=blue): // 0 <= f(x,y) <= 1
\end{eulerprompt}
\eulerimg{14}{images/Vikram Zaky Ardianto_22305144028_Plot 2d-081.png}
\begin{eulercomment}
Kami juga dapat mengisi rentang nilai seperti

\end{eulercomment}
\begin{eulerformula}
\[
-1 \le (x^2+y^2)^2-x^2+y^2 \le 0.
\]
\end{eulerformula}
\begin{eulercomment}
\end{eulercomment}
\begin{eulerprompt}
>plot2d("(x^2+y^2)^2-x^2+y^2",r=1.2,level=[-1;0],style="/"):
\end{eulerprompt}
\eulerimg{14}{images/Vikram Zaky Ardianto_22305144028_Plot 2d-082.png}
\begin{eulerprompt}
>plot2d("cos(x)","sin(x)^3",xmin=0,xmax=2pi,>filled,style="/"):
\end{eulerprompt}
\eulerimg{14}{images/Vikram Zaky Ardianto_22305144028_Plot 2d-083.png}
\eulerheading{Grafik Fungsi Parametrik}
\begin{eulercomment}
Nilai-x tidak perlu diurutkan. (x,y) hanya menggambarkan kurva. Jika x
diurutkan, kurva tersebut merupakan grafik fungsi.

Dalam contoh berikut, kami memplot spiral

\end{eulercomment}
\begin{eulerformula}
\[
\gamma(t) = t \cdot (\cos(2\pi t),\sin(2\pi t))
\]
\end{eulerformula}
\begin{eulercomment}
Kita perlu menggunakan banyak titik untuk tampilan yang halus atau
fungsi adaptive() untuk mengevaluasi ekspresi (lihat fungsi adaptive()
untuk lebih jelasnya).
\end{eulercomment}
\begin{eulerprompt}
>t=linspace(0,1,1000); ...
>plot2d(t*cos(2*pi*t),t*sin(2*pi*t),r=1):
\end{eulerprompt}
\eulerimg{14}{images/Vikram Zaky Ardianto_22305144028_Plot 2d-084.png}
\begin{eulercomment}
Atau, dimungkinkan untuk menggunakan dua ekspresi untuk kurva. Berikut
ini plot kurva yang sama seperti di atas.
\end{eulercomment}
\begin{eulerprompt}
>plot2d("x*cos(2*pi*x)","x*sin(2*pi*x)",xmin=0,xmax=1,r=1):
\end{eulerprompt}
\eulerimg{14}{images/Vikram Zaky Ardianto_22305144028_Plot 2d-085.png}
\begin{eulerprompt}
>t=linspace(0,1,1000); r=exp(-t); x=r*cos(2pi*t); y=r*sin(2pi*t);
>plot2d(x,y,r=1):
\end{eulerprompt}
\eulerimg{14}{images/Vikram Zaky Ardianto_22305144028_Plot 2d-086.png}
\begin{eulercomment}
Dalam contoh berikutnya, kami memplot kurva

\end{eulercomment}
\begin{eulerformula}
\[
\gamma(t) = (r(t) \cos(t), r(t) \sin(t))
\]
\end{eulerformula}
\begin{eulercomment}
dengan

\end{eulercomment}
\begin{eulerformula}
\[
r(t) = 1 + \dfrac{\sin(3t)}{2}.
\]
\end{eulerformula}
\begin{eulerprompt}
>t=linspace(0,2pi,1000); r=1+sin(3*t)/2; x=r*cos(t); y=r*sin(t); ...
>plot2d(x,y,>filled,fillcolor=red,style="/",r=1.5):
\end{eulerprompt}
\eulerimg{14}{images/Vikram Zaky Ardianto_22305144028_Plot 2d-087.png}
\eulerheading{Menggambar Grafik Bilangan Kompleks}
\begin{eulercomment}
Array bilangan kompleks juga dapat diplot. Kemudian titik-titik grid
akan terhubung. Jika sejumlah garis kisi ditentukan (atau vektor garis
kisi 1x2) dalam argumen cgrid, hanya garis kisi tersebut yang
terlihat.

Matriks bilangan kompleks akan secara otomatis diplot sebagai kisi di
bidang kompleks.

Dalam contoh berikut, kami memplot gambar lingkaran satuan di bawah
fungsi eksponensial. Parameter cgrid menyembunyikan beberapa kurva
grid.
\end{eulercomment}
\begin{eulerprompt}
>aspect(); r=linspace(0,1,50); a=linspace(0,2pi,80)'; z=r*exp(I*a);...
>plot2d(z,a=-1.25,b=1.25,c=-1.25,d=1.25,cgrid=10):
\end{eulerprompt}
\eulerimg{27}{images/Vikram Zaky Ardianto_22305144028_Plot 2d-088.png}
\begin{eulerprompt}
>aspect(1.25); r=linspace(0,1,50); a=linspace(0,2pi,200)'; z=r*exp(I*a);
>plot2d(exp(z),cgrid=[40,10]):
\end{eulerprompt}
\eulerimg{21}{images/Vikram Zaky Ardianto_22305144028_Plot 2d-089.png}
\begin{eulerprompt}
>r=linspace(0,1,10); a=linspace(0,2pi,40)'; z=r*exp(I*a);
>plot2d(exp(z),>points,>add):
\end{eulerprompt}
\eulerimg{21}{images/Vikram Zaky Ardianto_22305144028_Plot 2d-090.png}
\begin{eulercomment}
Sebuah vektor bilangan kompleks secara otomatis diplot sebagai kurva
pada bidang kompleks dengan bagian real dan bagian imajiner.

Dalam contoh, kami memplot lingkaran satuan dengan

\end{eulercomment}
\begin{eulerformula}
\[
\gamma(t) = e^{it}
\]
\end{eulerformula}
\begin{eulerprompt}
>t=linspace(0,2pi,1000); ...
>plot2d(exp(I*t)+exp(4*I*t),r=2):
\end{eulerprompt}
\eulerimg{21}{images/Vikram Zaky Ardianto_22305144028_Plot 2d-091.png}
\eulerheading{Plot Statistik}
\begin{eulercomment}
Ada banyak fungsi yang dikhususkan pada plot statistik. Salah satu
plot yang sering digunakan adalah plot kolom.

Jumlah kumulatif dari nilai terdistribusi 0-1-normal menghasilkan
jalan acak.
\end{eulercomment}
\begin{eulerprompt}
>plot2d(cumsum(randnormal(1,1000))):
\end{eulerprompt}
\eulerimg{21}{images/Vikram Zaky Ardianto_22305144028_Plot 2d-092.png}
\begin{eulercomment}
Menggunakan dua baris menunjukkan jalan dalam dua dimensi.
\end{eulercomment}
\begin{eulerprompt}
>X=cumsum(randnormal(2,1000)); plot2d(X[1],X[2]):
\end{eulerprompt}
\eulerimg{21}{images/Vikram Zaky Ardianto_22305144028_Plot 2d-093.png}
\begin{eulerprompt}
>columnsplot(cumsum(random(10)),style="/",color=blue):
\end{eulerprompt}
\eulerimg{21}{images/Vikram Zaky Ardianto_22305144028_Plot 2d-094.png}
\begin{eulercomment}
Itu juga dapat menampilkan string sebagai label.
\end{eulercomment}
\begin{eulerprompt}
>months=["Jan","Feb","Mar","Apr","May","Jun", ...
>  "Jul","Aug","Sep","Oct","Nov","Dec"];
>values=[10,12,12,18,22,28,30,26,22,18,12,8];
>columnsplot(values,lab=months,color=red,style="-");
>title("Temperature"):
\end{eulerprompt}
\eulerimg{21}{images/Vikram Zaky Ardianto_22305144028_Plot 2d-095.png}
\begin{eulerprompt}
>k=0:10;
>plot2d(k,bin(10,k),>bar):
\end{eulerprompt}
\eulerimg{21}{images/Vikram Zaky Ardianto_22305144028_Plot 2d-096.png}
\begin{eulerprompt}
>plot2d(k,bin(10,k)); plot2d(k,bin(10,k),>points,>add):
\end{eulerprompt}
\eulerimg{21}{images/Vikram Zaky Ardianto_22305144028_Plot 2d-097.png}
\begin{eulerprompt}
>plot2d(normal(1000),normal(1000),>points,grid=6,style=".."):
\end{eulerprompt}
\eulerimg{21}{images/Vikram Zaky Ardianto_22305144028_Plot 2d-098.png}
\begin{eulerprompt}
>plot2d(normal(1,1000),>distribution,style="O"):
\end{eulerprompt}
\eulerimg{21}{images/Vikram Zaky Ardianto_22305144028_Plot 2d-099.png}
\begin{eulerprompt}
>plot2d("qnormal",0,5;2.5,0.5,>filled):
\end{eulerprompt}
\eulerimg{21}{images/Vikram Zaky Ardianto_22305144028_Plot 2d-100.png}
\begin{eulercomment}
Untuk memplot distribusi statistik eksperimental, Anda dapat
menggunakan distribution=n dengan plot2d.
\end{eulercomment}
\begin{eulerprompt}
>w=randexponential(1,1000); // exponential distribution
>plot2d(w,>distribution): // or distribution=n with n intervals
\end{eulerprompt}
\eulerimg{21}{images/Vikram Zaky Ardianto_22305144028_Plot 2d-101.png}
\begin{eulercomment}
Atau Anda dapat menghitung distribusi dari data dan memplot hasilnya
dengan \textgreater{}bar di plot3d, atau dengan plot kolom.
\end{eulercomment}
\begin{eulerprompt}
>w=normal(1000); // 0-1-normal distribution
>\{x,y\}=histo(w,10,v=[-6,-4,-2,-1,0,1,2,4,6]); // interval bounds v
>plot2d(x,y,>bar):
\end{eulerprompt}
\eulerimg{21}{images/Vikram Zaky Ardianto_22305144028_Plot 2d-102.png}
\begin{eulercomment}
Fungsi statplot() menyetel gaya dengan string sederhana.
\end{eulercomment}
\begin{eulerprompt}
>statplot(1:10,cumsum(random(10)),"b"):
\end{eulerprompt}
\eulerimg{21}{images/Vikram Zaky Ardianto_22305144028_Plot 2d-103.png}
\begin{eulerprompt}
>n=10; i=0:n; ...
>plot2d(i,bin(n,i)/2^n,a=0,b=10,c=0,d=0.3); ...
>plot2d(i,bin(n,i)/2^n,points=true,style="ow",add=true,color=blue):
\end{eulerprompt}
\eulerimg{21}{images/Vikram Zaky Ardianto_22305144028_Plot 2d-104.png}
\begin{eulercomment}
Selain itu, data dapat diplot sebagai batang. Dalam hal ini, x harus
diurutkan dan satu elemen lebih panjang dari y. Bilah akan memanjang
dari x[i] ke x[i+1] dengan nilai y[i]. Jika x memiliki ukuran yang
sama dengan y, maka akan diperpanjang satu elemen dengan spasi
terakhir.

Gaya isian dapat digunakan seperti di atas.
\end{eulercomment}
\begin{eulerprompt}
>n=10; k=bin(n,0:n); ...
>plot2d(-0.5:n+0.5,k,bar=true,fillcolor=lightgray):
\end{eulerprompt}
\eulerimg{21}{images/Vikram Zaky Ardianto_22305144028_Plot 2d-105.png}
\begin{eulercomment}
Data untuk plot batang (bar=1) dan histogram (histogram=1) dapat
dinyatakan secara eksplisit dalam xv dan yv, atau dapat dihitung dari
distribusi empiris dalam xv dengan \textgreater{}distribution (atau
distribution=n). Histogram nilai xv akan dihitung secara otomatis
dengan \textgreater{}histogram. Jika \textgreater{}even ditentukan, nilai xv akan dihitung dalam
interval bilangan bulat.
\end{eulercomment}
\begin{eulerprompt}
>plot2d(normal(10000),distribution=50):
\end{eulerprompt}
\eulerimg{21}{images/Vikram Zaky Ardianto_22305144028_Plot 2d-106.png}
\begin{eulerprompt}
>k=0:10; m=bin(10,k); x=(0:11)-0.5; plot2d(x,m,>bar):
\end{eulerprompt}
\eulerimg{21}{images/Vikram Zaky Ardianto_22305144028_Plot 2d-107.png}
\begin{eulerprompt}
>columnsplot(m,k):
\end{eulerprompt}
\eulerimg{21}{images/Vikram Zaky Ardianto_22305144028_Plot 2d-108.png}
\begin{eulerprompt}
>plot2d(random(600)*6,histogram=6):
\end{eulerprompt}
\eulerimg{21}{images/Vikram Zaky Ardianto_22305144028_Plot 2d-109.png}
\begin{eulercomment}
Untuk distribusi, ada parameter distribution=n, yang menghitung nilai
secara otomatis dan mencetak distribusi relatif dengan n sub-interval.
\end{eulercomment}
\begin{eulerprompt}
>plot2d(normal(1,1000),distribution=10,style="\(\backslash\)/"):
\end{eulerprompt}
\eulerimg{21}{images/Vikram Zaky Ardianto_22305144028_Plot 2d-110.png}
\begin{eulercomment}
Dengan parameter even=true, ini akan menggunakan interval integer.
\end{eulercomment}
\begin{eulerprompt}
>plot2d(intrandom(1,1000,10),distribution=10,even=true):
\end{eulerprompt}
\eulerimg{21}{images/Vikram Zaky Ardianto_22305144028_Plot 2d-111.png}
\begin{eulercomment}
Perhatikan bahwa ada banyak plot statistik, yang mungkin berguna.
Silahkan lihat tutorial tentang statistik.
\end{eulercomment}
\begin{eulerprompt}
>columnsplot(getmultiplicities(1:6,intrandom(1,6000,6))):
\end{eulerprompt}
\eulerimg{21}{images/Vikram Zaky Ardianto_22305144028_Plot 2d-112.png}
\begin{eulerprompt}
>plot2d(normal(1,1000),>distribution); ...
>  plot2d("qnormal(x)",color=red,thickness=2,>add):
\end{eulerprompt}
\eulerimg{21}{images/Vikram Zaky Ardianto_22305144028_Plot 2d-113.png}
\begin{eulercomment}
Ada juga banyak plot khusus untuk statistik. Boxplot menunjukkan
kuartil dari distribusi ini dan banyak outlier. Menurut definisi,
outlier dalam boxplot adalah data yang melebihi 1,5 kali kisaran 50\%
tengah plot.
\end{eulercomment}
\begin{eulerprompt}
>M=normal(5,1000); boxplot(quartiles(M)):
\end{eulerprompt}
\eulerimg{21}{images/Vikram Zaky Ardianto_22305144028_Plot 2d-114.png}
\eulerheading{Fungsi Implisit}
\begin{eulercomment}
Plot implisit menunjukkan garis level yang menyelesaikan f(x,y)=level,
di mana "level" dapat berupa nilai tunggal atau vektor nilai. Jika
level="auto", akan ada garis level nc, yang akan menyebar antara
fungsi minimum dan maksimum secara merata. Warna yang lebih gelap atau
lebih terang dapat ditambahkan dengan \textgreater{}hue untuk menunjukkan nilai
fungsi. Untuk fungsi implisit, xv harus berupa fungsi atau ekspresi
dari parameter x dan y, atau, sebagai alternatif, xv dapat berupa
matriks nilai.

Euler dapat menandai garis level

\end{eulercomment}
\begin{eulerformula}
\[
f(x,y) = c
\]
\end{eulerformula}
\begin{eulercomment}
dari fungsi apapun.

Untuk menggambar himpunan f(x,y)=c untuk satu atau lebih konstanta c,
Anda dapat menggunakan plot2d() dengan plot implisitnya di dalam
bidang. Parameter untuk c adalah level=c, di mana c dapat berupa
vektor garis level. Selain itu, skema warna dapat digambar di latar
belakang untuk menunjukkan nilai fungsi untuk setiap titik dalam plot.
Parameter "n" menentukan kehalusan plot.
\end{eulercomment}
\begin{eulerprompt}
>aspect(1.5); 
>plot2d("x^2+y^2-x*y-x",r=1.5,level=0,contourcolor=red):
\end{eulerprompt}
\eulerimg{17}{images/Vikram Zaky Ardianto_22305144028_Plot 2d-115.png}
\begin{eulerprompt}
>expr := "2*x^2+x*y+3*y^4+y"; // define an expression f(x,y)
>plot2d(expr,level=0): // Solutions of f(x,y)=0
\end{eulerprompt}
\eulerimg{17}{images/Vikram Zaky Ardianto_22305144028_Plot 2d-116.png}
\begin{eulerprompt}
>plot2d(expr,level=0:0.5:20,>hue,contourcolor=white,n=200): // nice
\end{eulerprompt}
\eulerimg{17}{images/Vikram Zaky Ardianto_22305144028_Plot 2d-117.png}
\begin{eulerprompt}
>plot2d(expr,level=0:0.5:20,>hue,>spectral,n=200,grid=4): // nicer
\end{eulerprompt}
\eulerimg{17}{images/Vikram Zaky Ardianto_22305144028_Plot 2d-118.png}
\begin{eulercomment}
Ini berfungsi untuk plot data juga. Tetapi Anda harus menentukan
rentangnya\\
untuk label sumbu.
\end{eulercomment}
\begin{eulerprompt}
>x=-2:0.05:1; y=x'; z=expr(x,y);
>plot2d(z,level=0,a=-1,b=2,c=-2,d=1,>hue):
\end{eulerprompt}
\eulerimg{17}{images/Vikram Zaky Ardianto_22305144028_Plot 2d-119.png}
\begin{eulerprompt}
>plot2d("x^3-y^2",>contour,>hue,>spectral):
\end{eulerprompt}
\eulerimg{17}{images/Vikram Zaky Ardianto_22305144028_Plot 2d-120.png}
\begin{eulerprompt}
>plot2d("x^3-y^2",level=0,contourwidth=3,>add,contourcolor=red):
\end{eulerprompt}
\eulerimg{17}{images/Vikram Zaky Ardianto_22305144028_Plot 2d-121.png}
\begin{eulerprompt}
>z=z+normal(size(z))*0.2;
>plot2d(z,level=0.5,a=-1,b=2,c=-2,d=1):
\end{eulerprompt}
\eulerimg{17}{images/Vikram Zaky Ardianto_22305144028_Plot 2d-122.png}
\begin{eulerprompt}
>plot2d(expr,level=[0:0.2:5;0.05:0.2:5.05],color=lightgray):
\end{eulerprompt}
\eulerimg{17}{images/Vikram Zaky Ardianto_22305144028_Plot 2d-123.png}
\begin{eulerprompt}
>plot2d("x^2+y^3+x*y",level=1,r=4,n=100):
\end{eulerprompt}
\eulerimg{17}{images/Vikram Zaky Ardianto_22305144028_Plot 2d-124.png}
\begin{eulerprompt}
>plot2d("x^2+2*y^2-x*y",level=0:0.1:10,n=100,contourcolor=white,>hue):
\end{eulerprompt}
\eulerimg{17}{images/Vikram Zaky Ardianto_22305144028_Plot 2d-125.png}
\begin{eulercomment}
Juga dimungkinkan untuk mengisi set

\end{eulercomment}
\begin{eulerformula}
\[
a \le f(x,y) \le b
\]
\end{eulerformula}
\begin{eulercomment}
dengan rentang tingkat.

Dimungkinkan untuk mengisi wilayah nilai untuk fungsi tertentu. Untuk
ini, level harus berupa matriks 2xn. Baris pertama adalah batas bawah
dan baris kedua berisi batas atas.
\end{eulercomment}
\begin{eulerprompt}
>plot2d(expr,level=[0;1],style="-",color=blue): // 0 <= f(x,y) <= 1
\end{eulerprompt}
\eulerimg{17}{images/Vikram Zaky Ardianto_22305144028_Plot 2d-126.png}
\begin{eulercomment}
Plot implisit juga dapat menunjukkan rentang level. Kemudian level
harus berupa matriks 2xn dari interval level, di mana baris pertama
berisi awal dan baris kedua adalah akhir dari setiap interval. Atau,
vektor baris sederhana dapat digunakan untuk level, dan parameter dl
memperluas nilai level ke interval.
\end{eulercomment}
\begin{eulerprompt}
>plot2d("x^4+y^4",r=1.5,level=[0;1],color=blue,style="/"):
\end{eulerprompt}
\eulerimg{17}{images/Vikram Zaky Ardianto_22305144028_Plot 2d-127.png}
\begin{eulerprompt}
>plot2d("x^2+y^3+x*y",level=[0,2,4;1,3,5],style="/",r=2,n=100):
\end{eulerprompt}
\eulerimg{17}{images/Vikram Zaky Ardianto_22305144028_Plot 2d-128.png}
\begin{eulerprompt}
>plot2d("x^2+y^3+x*y",level=-10:20,r=2,style="-",dl=0.1,n=100):
\end{eulerprompt}
\eulerimg{17}{images/Vikram Zaky Ardianto_22305144028_Plot 2d-129.png}
\begin{eulerprompt}
>plot2d("sin(x)*cos(y)",r=pi,>hue,>levels,n=100):
\end{eulerprompt}
\eulerimg{17}{images/Vikram Zaky Ardianto_22305144028_Plot 2d-130.png}
\begin{eulercomment}
Dimungkinkan juga untuk menandai suatu wilayah

\end{eulercomment}
\begin{eulerformula}
\[
a \le f(x,y) \le b.
\]
\end{eulerformula}
\begin{eulercomment}
Ini dilakukan dengan menambahkan level dengan dua baris.
\end{eulercomment}
\begin{eulerprompt}
>plot2d("(x^2+y^2-1)^3-x^2*y^3",r=1.3, ...
>  style="#",color=red,<outline, ...
>  level=[-2;0],n=100):
\end{eulerprompt}
\eulerimg{17}{images/Vikram Zaky Ardianto_22305144028_Plot 2d-131.png}
\begin{eulercomment}
Dimungkinkan untuk menentukan level tertentu. Misalnya, kita dapat
memplot solusi persamaan seperti

\end{eulercomment}
\begin{eulerformula}
\[
x^3-xy+x^2y^2=6
\]
\end{eulerformula}
\begin{eulerprompt}
>plot2d("x^3-x*y+x^2*y^2",r=6,level=1,n=100):
\end{eulerprompt}
\eulerimg{17}{images/Vikram Zaky Ardianto_22305144028_Plot 2d-132.png}
\begin{eulerprompt}
>function starplot1 (v, style="/", color=green, lab=none) ...
\end{eulerprompt}
\begin{eulerudf}
    if !holding() then clg; endif;
    w=window(); window(0,0,1024,1024);
    h=holding(1);
    r=max(abs(v))*1.2;
    setplot(-r,r,-r,r);
    n=cols(v); t=linspace(0,2pi,n);
    v=v|v[1]; c=v*cos(t); s=v*sin(t);
    cl=barcolor(color); st=barstyle(style);
    loop 1 to n
      polygon([0,c[#],c[#+1]],[0,s[#],s[#+1]],1);
      if lab!=none then
        rlab=v[#]+r*0.1;
        \{col,row\}=toscreen(cos(t[#])*rlab,sin(t[#])*rlab);
        ctext(""+lab[#],col,row-textheight()/2);
      endif;
    end;
    barcolor(cl); barstyle(st);
    holding(h);
    window(w);
  endfunction
\end{eulerudf}
\begin{eulercomment}
Tidak ada kotak atau sumbu kutu di sini. Selain itu, kami menggunakan
jendela penuh untuk plot.

Kami memanggil reset sebelum kami menguji plot ini untuk mengembalikan
default grafis. Ini tidak perlu, jika Anda yakin plot Anda berhasil.
\end{eulercomment}
\begin{eulerprompt}
>reset; starplot1(normal(1,10)+5,color=red,lab=1:10):
\end{eulerprompt}
\eulerimg{27}{images/Vikram Zaky Ardianto_22305144028_Plot 2d-133.png}
\begin{eulercomment}
Terkadang, Anda mungkin ingin merencanakan sesuatu yang tidak dapat
dilakukan plot2d, tetapi hampir.

Dalam fungsi berikut, kami melakukan plot impuls logaritmik. plot2d
dapat melakukan plot logaritmik, tetapi tidak untuk batang impuls.
\end{eulercomment}
\begin{eulerprompt}
>function logimpulseplot1 (x,y) ...
\end{eulerprompt}
\begin{eulerudf}
    \{x0,y0\}=makeimpulse(x,log(y)/log(10));
    plot2d(x0,y0,>bar,grid=0);
    h=holding(1);
    frame();
    xgrid(ticks(x));
    p=plot();
    for i=-10 to 10;
      if i<=p[4] and i>=p[3] then
         ygrid(i,yt="10^"+i);
      endif;
    end;
    holding(h);
  endfunction
\end{eulerudf}
\begin{eulercomment}
Mari kita uji dengan nilai yang terdistribusi secara eksponensial.
\end{eulercomment}
\begin{eulerprompt}
>aspect(1.5); x=1:10; y=-log(random(size(x)))*200; ...
>logimpulseplot1(x,y):
\end{eulerprompt}
\eulerimg{17}{images/Vikram Zaky Ardianto_22305144028_Plot 2d-134.png}
\begin{eulercomment}
Mari kita menganimasikan kurva 2D menggunakan plot langsung. Perintah
plot(x,y) hanya memplot kurva ke jendela plot. setplot(a,b,c,d)
mengatur jendela ini.

Fungsi wait(0) memaksa plot untuk muncul di jendela grafik. Jika
tidak, menggambar ulang terjadi dalam interval waktu yang jarang.
\end{eulercomment}
\begin{eulerprompt}
>function animliss (n,m) ...
\end{eulerprompt}
\begin{eulerudf}
  t=linspace(0,2pi,500);
  f=0;
  c=framecolor(0);
  l=linewidth(2);
  setplot(-1,1,-1,1);
  repeat
    clg;
    plot(sin(n*t),cos(m*t+f));
    wait(0);
    if testkey() then break; endif;
    f=f+0.02;
  end;
  framecolor(c);
  linewidth(l);
  endfunction
\end{eulerudf}
\begin{eulercomment}
Tekan sembarang tombol untuk menghentikan animasi ini.
\end{eulercomment}
\begin{eulerprompt}
>animliss(2,3); // lihat hasilnya, jika sudah puas, tekan ENTER
\end{eulerprompt}
\eulerheading{Plot Logaritmik}
\begin{eulercomment}
EMT menggunakan parameter "logplot" untuk skala logaritmik.\\
Plot logaritma dapat diplot baik menggunakan skala logaritma dalam y
dengan logplot=1, atau menggunakan skala logaritma dalam x dan y
dengan logplot=2, atau dalam x dengan logplot=3.

\end{eulercomment}
\begin{eulerttcomment}
 - logplot=1: y-logaritma
 - logplot=2: x-y-logaritma
 - logplot=3: x-logaritma
\end{eulerttcomment}
\begin{eulerprompt}
>plot2d("exp(x^3-x)*x^2",1,5,logplot=1):
\end{eulerprompt}
\eulerimg{17}{images/Vikram Zaky Ardianto_22305144028_Plot 2d-135.png}
\begin{eulerprompt}
>plot2d("exp(x+sin(x))",0,100,logplot=1):
\end{eulerprompt}
\eulerimg{17}{images/Vikram Zaky Ardianto_22305144028_Plot 2d-136.png}
\begin{eulerprompt}
>plot2d("exp(x+sin(x))",10,100,logplot=2):
\end{eulerprompt}
\eulerimg{17}{images/Vikram Zaky Ardianto_22305144028_Plot 2d-137.png}
\begin{eulerprompt}
>plot2d("gamma(x)",1,10,logplot=1):
\end{eulerprompt}
\eulerimg{17}{images/Vikram Zaky Ardianto_22305144028_Plot 2d-138.png}
\begin{eulerprompt}
>plot2d("log(x*(2+sin(x/100)))",10,1000,logplot=3):
\end{eulerprompt}
\eulerimg{17}{images/Vikram Zaky Ardianto_22305144028_Plot 2d-139.png}
\begin{eulercomment}
Ini juga berfungsi dengan plot data.
\end{eulercomment}
\begin{eulerprompt}
>x=10^(1:20); y=x^2-x;
>plot2d(x,y,logplot=2):
\end{eulerprompt}
\eulerimg{17}{images/Vikram Zaky Ardianto_22305144028_Plot 2d-140.png}
\end{eulernotebook}
\end{document}
